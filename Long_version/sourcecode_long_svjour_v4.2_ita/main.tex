%%%%%%%%%%%%%%%%%%%%%%% file template.tex %%%%%%%%%%%%%%%%%%%%%%%%%
%
% This is a template file for ITA 
%
% Copy it to a new file with a new name and use it as the basis
% for your article
%
%%%%%%%%%%%%%%%%%%%%%%%%   EDP Sciences  %%%%%%%%%%%%%%%%%%%%%%%%%%
%
\documentclass{ita}
%
%%%%%%%%%%%%%--PREAMBLE--%%%%%%%%%%%%%%%%%%
%%-----------------------------
%%         ...........
%%         your macros
%%         ...........
%definitions
%---------------------------------------------------------------------
%the definitions used
%---------------------------------------------------------------------

%
%---------------------------packages begin---------------------------------------------------------------
\usepackage{etex}
\usepackage[usenames]{color}
%\usepackage{xcolor}

%\usepackage{inference}
%\usepackage{vmacros,cmacros}
\usepackage{makeidx}
\usepackage{verbatim}
\usepackage{fancyhdr}
\usepackage{fancybox}
\usepackage{stmaryrd}
\usepackage[reqno]{amsmath}
%\usepackage[notref,notcite]{showkeys}
%\usepackage{layout}
%\usepackage{graphicx}
%\usepackage{amsthm}
\usepackage{mathtools}
\usepackage{mathpartir}
\usepackage{extarrows}
\usepackage{amssymb}
\usepackage{amsfonts}
%\usepackage{MnSymbol}
\usepackage{xypic}\xyoption{all}
\usepackage{tikz}

\usepackage {indentfirst}
\usepackage{graphicx}
\DeclareGraphicsRule{.jpg}{eps}{.bb}{}
\DeclareGraphicsRule{.bmp}{eps}{.bb}{}
%\usepackage{palatino}

\usepackage[hang]{subfig}
%\usepackage[font=small,labelfont=bf,tableposition=top]{caption}
%\usepackage[font=footnotesize]{subcaption}

\usepackage{cite}

\usepackage{proof}

\usepackage{bbding}
%\usepackage{mathabx}

%\usepackage{hyperref}

%\usepackage[left=1in,right=1in,top=1in,bottom=1in,bindingoffset=1cm,a4paper]{geometry}
%\usepackage[a4paper,tmargin=0.8in,bmargin=0.85in,lmargin=1.4in,rmargin=1.4in]{geometry}

%\usepackage{hyperref} % ��ñ�֤ hyperref �������صĺ��
%\hypersetup{%
%  dvipdfmx,% �趨Ҫʹ�õ� driver Ϊ dvipdfmx
%  unicode={true},% ʹ�� unicode ������ PDF �ַ���
%  pdfstartview={FitH},% �ĵ���ʼ��ͼΪƥ�����
%  bookmarksnumbered={true},% ��ǩ�����½ڱ��
%  bookmarksopen={true},% չ����ǩ
%  pdfborder={0 0 0},% �����޿�
%  pdftitle={����},
%  pdfauthor={����},
%  pdfsubject={����},
%  pdfkeywords={�ؼ���},
%  pdfcreator={Ӧ�ó���},
%  pdfproducer={PDF ��������},% �������û�����ã�
%}

%\usepackage{hyperref} % ��ñ�֤ hyperref �������صĺ��
%\hypersetup{%
%  dvipdfmx,% �趨Ҫʹ�õ� driver Ϊ dvipdfmx
%  unicode={true},% ʹ�� unicode ������ PDF �ַ���
%  colorlinks={true},
%  linkcolor={red},
%  anchorcolor={green},
%  citecolor={blue},
%  pdfstartview={FitH},% �ĵ���ʼ��ͼΪƥ�����
%  bookmarksnumbered={true},% ��ǩ�����½ڱ��
%  bookmarksopen={true},% չ����ǩ
%  pdfborder={0 0 0},% �����޿�
%  pdftitle={����},
%  pdfauthor={����},
%  pdfsubject={����},
%  pdfkeywords={�ؼ���},
%  pdfcreator={Ӧ�ó���},
%  pdfproducer={PDF ��������},% �������û�����ã�
%}
\usepackage{hyperref} % ��ñ�֤ hyperref �������صĺ��
\hypersetup{
    bookmarks={true},         % show bookmarks bar?
    bookmarksnumbered={true},
    bookmarksopen={true},
    unicode={true},         % non-Latin characters in Acrobat��s bookmarks
    pdftoolbar={true},        % show Acrobat��s toolbar?
    pdfmenubar={true},        % show Acrobat��s menu?
    pdffitwindow={false},     % window fit to page when opened
    pdfstartview={FitH},    % fits the width of the page to the window
    pdftitle={My title},    % title
    pdfauthor={Author},     % author
    pdfsubject={Subject},   % subject of the document
    pdfcreator={Creator},   % creator of the document
    pdfproducer={Producer}, % producer of the document
    pdfkeywords={keyword1, key2, key3}, % list of keywords
    pdfnewwindow={true},      % links in new PDF window
    colorlinks={true},       % false: boxed links; true: colored links
    linkcolor={blue},          % color of internal links (change box color with linkbordercolor)
    citecolor={cyan},        % color of links to bibliography
    anchorcolor={green},
    filecolor={magenta},      % color of file links
    urlcolor={red}           % color of external links
}
%---------------------------packages end----------------------------------------------------------------

%
%---------------------------thms begin--------------------------------------------------------------------
%%\newtheorem{theorem}{Theorem}
%%\newtheorem{proposition}[theorem]{Proposition}
%%\newtheorem{lemma}[theorem]{Lemma}
%%\newtheorem{corollary}[theorem]{Corollary}
%%\newtheorem{conjecture}[theorem]{Conjecture}
%%
%%\theoremstyle{definition}
%%\newtheorem{definition}[theorem]{Definition}
%%\newtheorem{example}[theorem]{Example}
%%\newtheorem{exercise}[theorem]{Exercise}
%%
%%\theoremstyle{remark}
%%\newtheorem{remark}[theorem]{Remark}
%%\newtheorem{notes}[theorem]{Notes}
%%\newtheorem{convention}[theorem]{Convention}



%\newtheorem{theorem}{Theorem}%[chapter]
%\newtheorem{proposition}[theorem]{Proposition}
%\newtheorem{lemma}[theorem]{Lemma}
%\newtheorem{corollary}[theorem]{Corollary}
%\newtheorem{conjecture}[theorem]{Conjecture}

%\theoremstyle{definition}
%\newtheorem{definition}[theorem]{Definition}
%\newtheorem{example}[theorem]{Example}
%\newtheorem{exercise}[theorem]{Exercise}

%\theoremstyle{remark}
%\newtheorem{remark}[theorem]{Remark}
%\newtheorem{notes}[theorem]{Notes}
%\newtheorem{convention}[theorem]{Convention}
%---------------------------thms end---------------------------------------------------------------------



%
%---------------------------common 1 begin-------------------------------------------------------------------
%\newcommand{\HOPi}{HOPi }
%\newcommand{\LHOPi}{LHOPi }
%\newcommand{\FOPi}{FOPi }
%\newcommand{\HOCCS}{HOCCS }
%\newcommand{\CHOCS}{CHOCS }
%\newcommand{\PC}{Plain CHOCS }

\newcommand{\HOPi}{HOPi}
\newcommand{\LHOPi}{LHOPi}
\newcommand{\FOPi}{FOPi}
%\newcommand{\FOPi}{$\pi$ }
\newcommand{\HOCCS}{HOCCS}
\newcommand{\CHOCS}{CHOCS}
\newcommand{\PC}{Plain CHOCS}

%\newcommand{\HOPi}{HOPi~}
%\newcommand{\LHOPi}{LHOPi~}
%\newcommand{\FOPi}{FOPi~}
%\newcommand{\HOCCS}{HOCCS~}
%\newcommand{\CHOCS}{CHOCS~}
%\newcommand{\PC}{Plain CHOCS~}

%
\newcommand{\NAT}{\mathcal{N}}

%
\newcommand{\para}{|\,}
%\newcommand{\para}{|}
\newcommand{\cho}{\;{+}\;}
%\newcommand{\mat}{{=}}
\newcommand{\mmat}{{\neq}}

%
\newcommand{\encode}[3]{ \lds #1 \rds^{#2}_{#3}}
\newcommand{\dpara}{\;|\!|\, }
\newcommand{\nber}[1]{|#1|}
%\newcommand{\lrangle}[1]{\langle #1 \rangle}

%
\newcommand{\fname}[1]{\mbox{\rm fn($#1$)}}   %free names
\newcommand{\bname}[1]{\mbox{\rm bn($#1$)}}   %bound names
\newcommand{\aname}[1]{\mbox{\rm names($#1$)}}     %names
\newcommand{\subjname}[1]{\mbox{\rm subj($#1$)}}     %subject of an action (prefix)

%
\newcommand{\fosub}[2]{\{#1/#2\} }
%\newcommand{\hosub}[2]{[#1/#2] }
\newcommand{\hosub}[2]{\{#1/#2\} }
\newcommand{\hosubd}[2]{[#1/#2] } %dynamic substitution

%
\newcommand{\vect}[1]{{#1_1,#1_2,...,#1_n} }
%\newcommand{\vec}[1]{\widetilde{#1}}
\newcommand{\ve}[1]{\widetilde{#1}}
%\newcommand{\seq}[1]{\overrightharpoon{#1}}
%\newcommand{\seq}[1]{\rc{\large \vec{#1}}}  % can be replaced by "\ve" ("\vec" is different). before replacing, better locate all the "\seq" to be sure
%\newcommand{\seq}[1]{\rc{\large \ve{#1}}}  % 
\newcommand{\seq}[1]{\ve{#1}}  % 

%
\newcommand{\size}[1]{|#1|}

%strong transition
%\newcommand{\st}[1]{{\xrightarrow{#1}} }
\newcommand{\stm}[1]{{\,\xrightarrow{#1}} }
%weak transition
%\newcommand{\wt}[1]{{\stackrel{#1}{\Longrightarrow}} }
%\newcommand{\wt}[1]{{\xRightarrow{#1}} }
\newcommand{\wt}[1]{{\,\xLongrightarrow{#1}} }

\newcommand{\stmh}[1]{{\xrightarrow{\widehat{#1}}} }
\newcommand{\wth}[1]{{\xLongrightarrow{\widehat{#1}}} }


%
%\newcommand{\rc}[1]{{\color{Red} #1}}
\newcommand{\rc}[1]{{\color{red} #1}}
%\newcommand{\bc}[1]{{\color{Blue} #1}}
\newcommand{\bc}[1]{{\color{blue} #1}}
%
\newcommand{\stress}[1]{{#1}}
\newcommand{\stre}[1]{\bc{#1}}
\newcommand{\stressed}[1]{\bc{#1}}


%
\newcommand{\da}{\!\!\downarrow}
\newcommand{\Da}{\!\!\Downarrow}

%
\newcommand{\sep}{\vspace*{0.7cm}}
\newcommand{\sepp}{\vspace*{0.3cm}}
%
\newcommand{\itsep}{\itemsep -0.1em}
%
%\newcommand{\nsep}{\vspace{-0.13cm}}
%\newcommand{\nsepv}[1]{\vspace{-#1cm}}
\newcommand{\nsep}{\vspace{0mm}}
\newcommand{\nsepv}[1]{\vspace{0mm}}


%raise box in tables
\newcommand{\rb}[1]{\raisebox{3.2ex}[0pt]{#1}}



%
\newcommand{\hfilldots}{$\cdots\cdots\cdots\cdots\cdots\cdots\cdots\cdots\cdots\cdots\cdots\cdots\cdots\cdots\cdots\cdots\cdots\cdots$ }

%
%\usepackage{mathabx}
\newcommand{\FROMHERE}{$\blacktriangleright\blacktriangleright\blacktriangleright\blacktriangleright\blacktriangleright\blacktriangleright\blacktriangleright\blacktriangleright\blacktriangleright\blacktriangleright\blacktriangleright\blacktriangleright\blacktriangleright\blacktriangleright\blacktriangleright\blacktriangleright\blacktriangleright\blacktriangleright\blacktriangleright\blacktriangleright\blacktriangleright\blacktriangleright$ \textsc{\large from here!!!!!!....} } 
%
\newcommand{\TODO}{$\blacktriangleright\blacktriangleright\blacktriangleright\blacktriangleright\blacktriangleright\blacktriangleright\blacktriangleright\blacktriangleright\blacktriangleright\blacktriangleright\blacktriangleright\blacktriangleright\blacktriangleright\blacktriangleright\blacktriangleright\blacktriangleright\blacktriangleright\blacktriangleright\blacktriangleright\blacktriangleright\blacktriangleright\blacktriangleright$ \textsc{\large TODO!!!!!!....} } 

%---------------------------common 1 end-------------------------------------------------------------------




%---------------------------common 2(bisimulation) 
%begin-------------------------------------------------------------------

%CCS
\newcommand{\CCSSB}{\sim }
\newcommand{\CCSWB}{\approx }
\newcommand{\CCSWC}{= }
\newcommand{\CCSIMP}{\vdash }





%first-order barbed bisimulation
\newcommand{\FOBB}{\approx_{bf}}
%higher-order barbed bisimulation
\newcommand{\HOBB}{\approx_{bh}}

%first-order barbed congruence
\newcommand{\FOBC}{\approx_{bf}^c}
%higher-order barbed congruence
\newcommand{\HOBC}{\approx_{bh}^c}

%first-order barbed equivalence
\newcommand{\FOBE}{\approx_{bf}^e}
%higher-order barbed equivalence
\newcommand{\HOBE}{\approx_{bh}^e}

%strong higher-order context bisimulation
\newcommand{\SHOCB}{\sim_{Ct}}
%higher-order context bisimulation
\newcommand{\HOCB}{\approx_{Ct}}
%higher-order context congruence
\newcommand{\HOCC}{\approx_{Ct}^c}
%higher-order normal bisimulation
\newcommand{\HONB}{\approx_{Nr}}
%higher-order triggered bisimulation
\newcommand{\HOTB}{\approx_{Tr}}


%index operation in quasi open bisimulations
\newcommand{\qidx}[2]{{#1}^{#2} }


%higher-order bibimulation defined in [San92]
\newcommand{\HB}{\approx_H }
%
\newcommand{\CHB}{\approx_H^{CHOCS} }
%a symbol for barbed bisimulation in [San92]
\newcommand{\BB}{\approx }
%a symbol for early bisimulation in [San92]
\newcommand{\EB}{\approx_e }


%local bisimilarity
\newcommand{\LB}{\approx_l }

%structural equivalence
%\newcommand{\SE}{\sim_s }
%\newcommand{\SE}{\sim }
\newcommand{\SE}{\equiv }
%structural equivalence
\newcommand{\HOSE}{\sim }



%---------------------------common 2(bisimulation) end-------------------------------------------------------------------



%---------------------------mHO begin-------------------------------------------------------------------
%higher-order pi-calculus with mismatch
%\newcommand{\mHOPi}{mHOPi }
\newcommand{\mHOPi}{mHOPi}
%\newcommand{\mHOPi}{mHOPi~}

%encoding higher-order to first-order
\newcommand{\hofo}[1]{{\mathcal{C}[\![ #1]\!]} }

%index operation in quasi open bisimulations
%\newcommand{\qidx}[2]{{#1}^{#2} }


%first-order barbed bisimulation
%\newcommand{\FOBB}{\approx_{bf}}
%higher-order barbed bisimulation
%\newcommand{\HOBB}{\approx_{bh}}

%first-order barbed congruence
%\newcommand{\FOBC}{\approx_{bf}^c}
%higher-order barbed congruence
%\newcommand{\HOBC}{\approx_{bh}^c}

%first-order barbed equivalence
%\newcommand{\FOBE}{\approx_{bf}^e}
%higher-order barbed equivalence
%\newcommand{\HOBE}{\approx_{bh}^e}
%indexed higher-order barbed equivalence
%\newcommand{\IHOBE}{^e\approx_{bh}^S}

%strong higher-order context bisimulation
%\newcommand{\SHOCB}{\sim_{Ct}}
%higher-order context bisimulation
%\newcommand{\HOCB}{\approx_{Ct}}
%higher-order context congruence
%\newcommand{\HOCC}{\approx_{Ct}^c}
%higher-order normal bisimulation
%\newcommand{\HONB}{\approx_{Nr}}
%higher-order triggered bisimulation
%\newcommand{\HOTB}{\approx_{Tr}}

%
%\newcommand{\sep}{\vspace*{0.7cm}}
%
%\newcommand{\itsep}{\itemsep -0.1em}
%
%\newcommand{\HOPi}{HOPi }

%higher-order bibimulation defined in [San92]
%\newcommand{\HB}{\approx_H }
%
%\newcommand{\CHB}{\approx_H^{CHOCS} }
%a symbol for barbed bisimulation in [San92]
%\newcommand{\BB}{\approx }
%a symbol for early bisimulation in [San92]
%\newcommand{\EB}{\approx_e }

%local bisimilarity
%\newcommand{\LB}{\approx_l }

%structural equivalence
%\newcommand{\SE}{\sim_s }


%open strong HO bisimilarity
\newcommand{\HOOSB}{\sim_{oh} }
%open weak HO bisimilarity
\newcommand{\HOOWB}{\approx_{oh} }
%open weak HO congruence
\newcommand{\HOOWC}{\simeq_{oh} }

%quasi open (weak) higher-order bisimilarity
\newcommand{\QOHOB}[1]{\approx_{qh}^{#1} }
%quasi open (weak) higher-order congruence
\newcommand{\QOHOC}{\simeq_{qh} }
%local higher-order bisimilarity
\newcommand{\LHOB}{\approx_{lh} }
%local higher-order congruence
\newcommand{\LHOC}{\simeq_{lh} }

%the axiom system for open weak HO bisimilarity
\newcommand{\ASLM}{\mathcal{AS}_{LM} }
%the axiom system for quasi open weak HO bisimilarity or local HO bisimilarity
\newcommand{\ASLMQ}{\mathcal{AS}_{LM_q} }
%the axiom system for quasi open weak HO bisimilarity or local HO bisimilarity
\newcommand{\ASLML}{\mathcal{AS}_{LM_l} }

%---------------------------mHO end-------------------------------------------------------------------



%---------------------------encoding begin-------------------------------------------------------------------
%
\newcommand{\lds}{[\![}
\newcommand{\rds}{]\!]}
%\newcommand{\para}{|}
%\newcommand{\cho}{{+}}

%
%\newcommand{\encoding}[3]{ \lds #1 \rds^{#2}_{#3}}
\newcommand{\encodingm}[3]{ \lds #1 \rds^{#2}_{#3}}
\newcommand{\iencoding}[3]{ \lds #1 \rds^{#2}_{#3}}

%
\newcommand{\wm}[1]{ \mathcal{W}[\![ #1 ]\!]}
\newcommand{\iwm}[1]{ \mathcal{W}^n[\![ #1 ]\!]}
\newcommand{\chan}[1]{#1{-}chan}
\newcommand{\ichan}[1]{#1{-}ichan}
\newcommand{\backs}[1]{\!\setminus\!#1\!}

\newcommand{\dmd}[2]{\{\diamond_{#1}\; #2\} }
%\newcommand{\dmd}[2]{\{\diamond\; #2\} }
\newcommand{\idmd}[2]{\{\diamond_{#1}^i\; #2\} }
%\newcommand{\idmd}[2]{\{\diamond^i\; #2\} }

%\newcommand{\fosub}[2]{\{#1/#2\} }
%\newcommand{\hosub}[2]{[#1/#2] }

%
%\newcommand{\vect}[1]{{#1_1,#1_2,...,#1_n} }
%\newcommand{\ve}[1]{\widetilde{#1}}

%
%\newcommand{\rc}[1]{{\color{Red} #1}}
%\newcommand{\bc}[1]{{\color{Blue} #1}}


%ground bisimulation
\newcommand{\GB}{\stackrel{.}{\approx}}
%strong applicative higher-order bisimulation
\newcommand{\SAHOB}{\stackrel{:}{\sim}}
%applicative higher-order bisimulation
\newcommand{\AHOB}{\stackrel{:}{\approx}}
%indexed applicative higher-order bisimulation
\newcommand{\IAHOB}{\stackrel{:}{\approx}_S}

%indexed higher-order barbed bisimulation
\newcommand{\IHOBB}{\approx_{bh}^S}

%indexed higher-order barbed congruence
\newcommand{\IHOBC}{^c\approx_{bh}^S}

%indexed higher-order barbed equivalence
\newcommand{\IHOBE}{^e\approx_{bh}^S}

%indexed strong higher-order context bisimulation
\newcommand{\ISHOCB}{\sim_{Ct}^S}


%higher-order context bisimulation
%\newcommand{\HOCB}{\approx_{Ct}}
%higher-order context congruence
%\newcommand{\HOCC}{\approx_{Ct}^c}
%higher-order context congruence
%\newcommand{\HOCC}{\simeq_{Ct}}


%indexed higher-order context bisimulation
\newcommand{\IHOCB}{\approx_{Ct}^S}
%indexed higher-order context congruence
%\newcommand{\IHOCC}{~^c\!\!\approx_{Ct}^S}
%indexed higher-order context congruence
\newcommand{\IHOCC}{\simeq_{Ct}^S}

%higher-order normal bisimulation
%\newcommand{\HONB}{\approx_{Nr}}
%indexed higher-order normal bisimulation
\newcommand{\IHONB}{\approx_{Nr}^S}

%wired bisimulation
\newcommand{\WB}{\approx_{Wr}}
%wired congruence
\newcommand{\WC}{\simeq_{Wr}}
%indexed wired bisimulation
\newcommand{\IWB}{\approx_{Wr}^S}
%indexed wired congruence
\newcommand{\IWC}{\simeq_{Wr}^S}

%
\newcommand{\OHOCB}{\approx_{Ct}}

%quasi open bisimulation
\newcommand{\QOB}[1]{\approx_q^{#1} }
%local open bisimilarity
\newcommand{\LOB}{\approx_{lo}}
%local wired bisimulation
\newcommand{\LWB}{\approx_{lw}}

%---------------------------encoding end-------------------------------------------------------------------



%---------------------------logic begin-------------------------------------------------------------------

%first-order q-open bisimilarity
\newcommand{\FOQOB}{\approx_{qo}}
%higher-order q-open bisimilarity
\newcommand{\HOQOB}{\approx_{qoh}}
%higher-order q-open linear bisimilarity
\newcommand{\HOQOLB}{\approx_{qoh}^{l}}


%first-order barbed bisimulation
%\newcommand{\FOBB}{\approx_{bf}}
%higher-order barbed bisimulation
%\newcommand{\HOBB}{\approx_{bh}}

%first-order barbed congruence
%\newcommand{\FOBC}{\approx_{bf}^c}
%higher-order barbed congruence
%\newcommand{\HOBC}{\approx_{bh}^c}

%first-order barbed equivalence
%\newcommand{\FOBE}{\approx_{bf}^e}
%higher-order barbed equivalence
%\newcommand{\HOBE}{\approx_{bh}^e}
%indexed higher-order barbed equivalence
%\newcommand{\IHOBE}{^e\approx_{bh}^S}


%strong higher-order context bisimulation
%\newcommand{\SHOCB}{\sim_{Ct}}
%higher-order context bisimulation
%\newcommand{\HOCB}{\approx_{Ct}}
%higher-order context congruence
%\newcommand{\HOCC}{\approx_{Ct}^c}
%higher-order normal bisimulation
%\newcommand{\HONB}{\approx_{Nr}}
%higher-order triggered bisimulation
%\newcommand{\HOTB}{\approx_{Tr}}

%
%\newcommand{\sep}{\vspace*{0.7cm}}
%
%\newcommand{\itsep}{\itemsep -0.1em}

%
%\newcommand{\HOPi}{HOPi }

%higher-order bibimulation defined in [San92]
%\newcommand{\HB}{\approx_H }
%
%\newcommand{\CHB}{\approx_H^{CHOCS} }
%a symbol for barbed bisimulation in [San92]
%\newcommand{\BB}{\approx }
%a symbol for early bisimulation in [San92]
%\newcommand{\EB}{\approx_e }

%local bisimilarity
%\newcommand{\LB}{\approx_l }

%higher-order local bisimilarity
\newcommand{\HOLB}{\approx_{l} }
%higher-order local congruence
\newcommand{\HOLC}{\simeq_{l} }

%higher-order local linear bisimilarity
\newcommand{\HOLLB}{\approx_{ll} }
%higher-order local linear congruence
\newcommand{\HOLLC}{\simeq_{ll} }
%higher-order local linear variant bisimilarity
\newcommand{\HOLLVB}{\approx_{ll}^{v} }
%higher-order local linear variant bisimilarity: hierarchy
\newcommand{\HOLLVBH}[1]{\approx^{#1} }

%higher-order general bisimilarity
\newcommand{\HOGNLB}{\approx }


%higher-order pi-calculus with mismatch
%\newcommand{\mHOPi}{mHOPi }

%structural equivalence
%\newcommand{\SE}{\sim_s }
%structural equivalence
%\newcommand{\HOSE}{\sim }

%open strong HO bisimilarity
%\newcommand{\HOOSB}{\sim_{oh} }
%open weak HO bisimilarity
%\newcommand{\HOOWB}{\approx_{oh} }
%open weak HO congruence
%\newcommand{\HOOWC}{\simeq_{oh} }


%linear higher-order logic
\newcommand{\LHOL}{LHOL }

%on logic characterization
\newcommand{\dias}[1]{\langle {#1} \rangle }
\newcommand{\boxs}[1]{[ {#1} ] }
\newcommand{\conjuct}{\wedge }
\newcommand{\disjuct}{\vee }
%semantically imply or not imply
\newcommand{\semimply}{\vDash }
\newcommand{\nsemimply}{\not\vDash }

%syntactically imply or not imply
\newcommand{\synimply}{\vdash }
\newcommand{\nsynimply}{\not\vdash }

%(): [] or <>
\newcommand{\diaobox}[1]{( {#1} ) }
%constructive implication
\newcommand{\ci}{\Rightarrow }
%depth of a formula
\newcommand{\depth}[1]{| {#1} | }
%characteristic formula
\newcommand{\cformula}[2]{C^{#1}(#2) }
%\newcommand{\cformula}[3]{{#1}^{#2}_{#3} }

%bisimulations in \cite{AD95} and \cite{BD97}, use as needed
%\newcommand{\ee}[2]{\;{#1}^\oslash_{#2}\; }%existence extension
%\newcommand{\ehc}[2]{\;{#1}^\otimes_{#2}\; }%existence hoare closure

%\newcommand{\HOSB}{\sim } %higher-order strong context bisimilarity
%\newcommand{\HOSBH}[1]{\sim^{#1} } %higher-order strong bisimulation in hierarchy
%\newcommand{\HOSBHS}[1]{\sim^{#1}_{\#} } %higher-order strong bisimulation in hierarchy and sharpened
%\newcommand{\HOLE}{\sim_{L} } %higher-order logical equivalence in the strong case
%\newcommand{\HOLEH}[1]{\sim^{#1}_{L} } %higher-order logical equivalence in hierarchy in the strong case

%\newcommand{\CB}{\approx } %higher-order weak context bisimilarity
%\newcommand{\CBOE}{\approx^\circ } %higher-order weak context bisimilarity open extension
%\newcommand{\HOEB}{\approx_{\exists} } %higher-order existential bisimilarity
%\newcommand{\HOEBEE}{\approx_{\exists}^\oslash } %higher-order existential bisimilarity open extension
%\newcommand{\HOEBH}[1]{\approx_{\exists}^{#1} } %higher-order existential bisimulation in hierarchy
%\newcommand{\HOEBHEE}[1]{\approx_{\exists}^{#1 \oslash} } %higher-order existential bisimulation in hierarchy open extension
\newcommand{\HOLE}{\approx_{L} } %higher-order logical equivalence in the weak case
\newcommand{\HOLEH}[1]{\approx^{#1}_{L} } %higher-order logical equivalence in hierarchy  in the weak case
\newcommand{\NHOLEH}[1]{\not\approx^{#1}_{L} } %negation of higher-order logical equivalence in hierarchy  in the weak case

%---------------------------logic end-------------------------------------------------------------------





%---------------------------common 3() begin-------------------------------------------------------------------

%more
\newcommand{\DEF}{\stackrel{{\rm{def}}}{=}} %definition

%add to original head_pc
\newcommand{\lrangle}[1]{\langle #1 \rangle} % for abstraction and etc.

\newcommand{\CCSSLE}{\sim_{L} } %higher-order logical equivalence in the weak case
\newcommand{\CCSSLEH}[1]{\sim_{#1} } %higher-order logical equivalence in hierarchy  in the weak case

%consider redefine \encoding{}{}{} as ``\llbracket #1 \rrbracket^{#2}_{#3}"
\newcommand{\piBiCo}{\approx_\downarrow^\pi} 

%pipe definition
%\newcommand{\PIPE}[4]{\mathfrak{#1}{-}\mathfrak{pipe}(#2,#3,#4) }
%pipe definition
\newcommand{\PIPE}[4]{\mathfrak{#1}{-}\mathfrak{pipe}(#2,#3,#4) }
%\newcommand{\PIPEA}[1]{\mathbf{#1}{-}\mathfrak{pipe} }
\newcommand{\PIPEA}[1]{\mathbf{#1}{\textendash}\mathfrak{pipe} }
%\newcommand{\PIPEA}[1]{\mathbf{#1}{\relbar}\mathfrak{pipe} }

%---------------------------common 3() end-------------------------------------------------------------------





%---------------------------expressiveness begin-------------------------------------------------------------------

%\makeindex
\newcommand{\AEHOPi}{=_{\Pi} } %absolute equality of HOPi
\newcommand{\AEFOPi}{=_{\pi} } %absolute equality of FOPi

%
\newcommand{\EWLB}{\approx_{l}^{\pi} } %external weak local bisimilarity
\newcommand{\EWBFOPi}{\approx_{w}^{\pi} }%external weak bisimilarity of FOPi
\newcommand{\CBHOPi}{\approx_{Ct}^{\Pi} }%context bisimilarity of HOPi



%---------------------------expressiveness end-------------------------------------------------------------------



%--------------------------lambda-calculus (encoding) related  begin-------------------------------------------------------------------

\newcommand{\shr}{\mathrel{\stm{}_{{\rm h}}}} %  head reduction
\newcommand{\whr}{\mathrel{\wt{}_{{\rm h}}}} % multiple head reduction

%\makeindex
\newcommand{\ABLAM}{\simeq } %applicative bisimulation of lambda-calculus
\newcommand{\OABLAM}{\simeq^\circ  } %open applicative bisimulation of lambda-calculus

\newcommand{\BISI}{\approx  } %synchronous weak bisimulation
\newcommand{\BISIA}{\approx^{asy}  } %asynchronous weak bisimulation
\newcommand{\SBISIA}{\sim^{asy}  } %asynchronous strong bisimulation
\newcommand{\MUST}{\sim_{must}  } %synchronous must equivalence
\newcommand{\MUSTA}{\sim_{must}^{asy}  } %asynchronous must equivalence

\newcommand{\LMUSTA}{\leq_{must}^{asy}  } %asynchronous must preorder (less)
\newcommand{\RMUSTA}{\geq_{must}^{asy}  } %asynchronous must preorder (greater)

\newcommand{\BSTA}{\approx_{\rm{bb}}^{\rm{lin,asy}} } %asynchronous typed barbed bisimilarity
\newcommand{\BCTA}{\approx_{\rm{bc}}^{\rm{lin,asy}} } %asynchronous typed barbed congruence
\newcommand{\CTTA}{\approx_{\rm{ct}}^{\rm{lin,asy}} } %contextual asynchronous typed bisimulation


\newcommand{\contrasy}{\succcurlyeq^{asy} } % asynchronous expansion (greater)
\newcommand{\expaasy}{\preccurlyeq^{asy} } % asynchronous expansion (less)

\newcommand{\expamay}{\expa_{may} }  %synchronous may expansion (less)
\newcommand{\contrmay}{\contr_{may} }  %synchronous may expansion (greater)


\newcommand{\tbisiasy}[2]{\approx^{#1,#2}_{a} }  %(typed) asynchronous weak \ground i/o bisimulation
\newcommand{\tbisi}[2]{\approx^{#1,#2} }  %(typed) synchronous weak \ground i/o bisimulation



\newcommand{\ECONG}{\approx^c  } %synchronous weak early congruence
\newcommand{\MAYC}{\sim_{may}^c  } %synchronous may congruence
\newcommand{\MUSTC}{\sim_{must}^c  } %synchronous must congruence

\newcommand{\EEXPCG}{\succcurlyeq^c } % synchronous weak early expansion precongruence (greater)
\newcommand{\EEXPC}{\preccurlyeq^c  } %synchronous weak early expansion precongruence (less)
\newcommand{\MAYEXPC}{\preccurlyeq_{may}^c  } %synchronous may expansion precongruence
\newcommand{\MUSTEXPC}{\preccurlyeq_{must}^c  } %synchronous must expansion precongruence



\newcommand{\subtp}{\trianglelefteq }  %subtype relation
\newcommand{\typedres}[1]{({\boldsymbol  \nu} #1)\: } % more general notation of restriction
\newcommand{\typecf}{{\rightslice} }  %\leftslice
\newcommand{\TPEQ}{{\sim_{type}} } %equality on (recursive) types
\newcommand{\TYBISI}{{\approx_t} } %typed bisimilarity
\newcommand{\ATYBISI}{{\approx^a_t} } %typed asynchronous bisimilarity


\newcommand{\contrdiv}{^{\Uparrow}\!\!\!\contr} % divergence-sensitive expansion (greater)
%\newcommand{\conref}[1]{\rc{(\ref{#1})}} % reference number for condition (for merging)
\newcommand{\conref}[1]{{(\ref{#1})}} % reference number for condition (for merging)
%\newcommand{\conref}[1]{{#1}} % reference number for condition (for merging)
%\newcommand{\defref}[1]{\rc{(\ref{#1})}} % reference number for definition (for merging)
\newcommand{\defref}[1]{{\ref{#1}}} % reference number for definition (for merging)
%\newcommand{\defref}[1]{} % reference number for definition (for merging)
%\newcommand{\hardcode}[1]{#1} % hardcoded reference number
\newcommand{\hardcode}[1]{} % hardcoded reference number
%!!!!!!!!!!!!!!!!!!!!!!!!!!!ADD \usepackage{etex} to the head file
\newcommand{\supp}[1]{\mathbf{supp}(#1) }
\newcommand{\afig}{Fig. } %sometimes Figure is abbraviated as Fig., but sometimes not
%\newcommand{\afig}{Figure }

%---------------------------lambda-calculus  (encoding)  related end-------------------------------------------------------------------




%---------------------------others begin-------------------------------------------------------------------

%\makeindex
%%color for comments
%\newcommand{\lh}[1]{{\textcolor[RGB]{178,58,238}{#1}}}
%\newcommand{\yq}[1]{{\textcolor[RGB]{255,215,0}{#1}}}
%\newcommand{\xx}[1]{{\textcolor[RGB]{64,224,208}{#1}}}
%\newcommand{\xxstress}[1]{{\textcolor[RGB]{64,224,208}{#1}}}
\newcommand{\xxstress}[1]{{#1}}
%\newcommand{\xx}[1]{{\textcolor[RGB]{50,205,50}{#1}}}
\newcommand{\xx}[1]{}
\newcommand{\oo}[1]{{\color{red} #1}}

%
%\newcommand{\xxx}[1]{
%	{\large
%		\textcolor[RGB]{64,224,78}{
%		\emph{\color{red} $\displaystyle \clubsuit$ XX-BEGIN} %
%		#1
%		\emph{\color{red} XX-END $\clubsuit$}
%		}
%	}
%}

%\newenvironment{xxenv}{\large \color{red}$\clubsuit$ XX-BEGIN}{XX-END $\clubsuit$}
%\newcommand{\xxx}[1]{
%	\textcolor[RGB]{64,224,78}{
%		#1
%	}
%}
\newenvironment{xxenv}{\fbox{\color{black}$\clubsuit$ XX-BEGIN}}{\fbox{XX-END $\clubsuit$}}
\newcommand{\xxx}[1]{
	\textcolor[RGB]{0,0,0}{
		#1
	}
}



\newcommand{\remxx}[1]{{\Large \xx{#1}}}
\newif\ifxxremark \xxremarkfalse %switch of showing xx's remark

%---------------------------others end-------------------------------------------------------------------

%more
























%temp
%>>>>>>>>>>>>>>>>>>
%encoding higher-order to first-order
%\newcommand{\hofo}[1]{{\mathcal{C}[\![ #1]\!]}}
%
%%first-order barbed bisimulation
%\newcommand{\FOBB}{\approx_{bf}}
%%higher-order barbed bisimulation
%\newcommand{\HOBB}{\approx_{bh}}
%
%%first-order barbed congruence
%\newcommand{\FOBC}{\approx_{bf}^c}
%%higher-order barbed congruence
%\newcommand{\HOBC}{\approx_{bh}^c}
%
%%first-order barbed equivalence
%\newcommand{\FOBE}{\approx_{bf}^e}
%%higher-order barbed equivalence
%\newcommand{\HOBE}{\approx_{bh}^e}
%%indexed higher-order barbed equivalence
%\newcommand{\IHOBE}{^e\approx_{bh}^S}
%
%
%%strong higher-order context bisimulation
%\newcommand{\SHOCB}{\sim_{Ct}}
%%higher-order context bisimulation
%\newcommand{\HOCB}{\approx_{Ct}}
%%higher-order context congruence
%\newcommand{\HOCC}{\approx_{Ct}^c}
%%higher-order normal bisimulation
%\newcommand{\HONB}{\approx_{Nr}}
%%higher-order triggered bisimulation
%\newcommand{\HOTB}{\approx_{Tr}}


%
%\newcommand{\HOPi}{HOPi }


%>>>>>>>>>>>>>>>>>>
%higher-order bibimulation defined in [San92]
%\newcommand{\HB}{\approx_H }
%%
%\newcommand{\CHB}{\approx_H^{CHOCS} }
%%a symbol for barbed bisimulation in [San92]
%\newcommand{\BB}{\approx }
%%a symbol for early bisimulation in [San92]
%\newcommand{\EB}{\approx_e }
%%local bisimilarity
%\newcommand{\LB}{\approx_l }
%
%%higher-order local bisimilarity
%\newcommand{\HOLB}{\approx_{l} }
%%higher-order local congruence
%\newcommand{\HOLC}{\simeq_{l} }
%
%%higher-order local linear bisimilarity
%\newcommand{\HOLLB}{\approx_{ll} }
%%higher-order local linear congruence
%\newcommand{\HOLLC}{\simeq_{ll} }
%%higher-order local linear variant bisimilarity
%\newcommand{\HOLLVB}{\approx_{ll}^{v} }
%%higher-order local linear variant bisimilarity: hierarchy
%\newcommand{\HOLLVBH}[1]{\approx^{#1} }
%
%%higher-order general bisimilarity
%\newcommand{\HOGNLB}{\approx }
%
%%higher-order pi-calculus with mismatch
%\newcommand{\mHOPi}{mHOPi }
%
%%structural equivalence
%\newcommand{\SE}{\sim_s }
%%structural equivalence
%\newcommand{\HOSE}{\sim }
%
%%open strong HO bisimilarity
%\newcommand{\HOOSB}{\sim_{oh} }
%%open weak HO bisimilarity
%\newcommand{\HOOWB}{\approx_{oh} }
%%open weak HO congruence
%\newcommand{\HOOWC}{\simeq_{oh} }
%
%
%
%%>>>>>>>>>>>>>>>>>>
%
%%linear higher-order logic
%\newcommand{\LHOL}{LHOL }
%
%%on logic characterization
%\newcommand{\dias}[1]{\langle {#1} \rangle }
%\newcommand{\boxs}[1]{[ {#1} ] }
%\newcommand{\conj}{\wedge }
%\newcommand{\disj}{\vee }
%%semantically imply or not imply
%\newcommand{\semimply}{\vDash }
%\newcommand{\nsemimply}{\not\vDash }
%
%%syntactically imply or not imply
%\newcommand{\synimply}{\vdash }
%\newcommand{\nsynimply}{\not\vdash }
%
%%(): [] or <>
%\newcommand{\diaobox}[1]{( {#1} ) }
%%constructive implication
%\newcommand{\ci}{\Rightarrow }
%%depth of a formula
%\newcommand{\depth}[1]{| {#1} | }
%%characteristic formula
%%\newcommand{\cformula}[2]{C^{#1}(#2) }
%\newcommand{\cformula}[3]{{#1}^{#2}_{#3} }
%
%
%
%%index operation in quasi open bisimulations
%\newcommand{\qidx}[2]{{#1}^{#2} }
%
%%first-order q-open bisimilarity
%\newcommand{\FOQOB}{\approx_{qo}}
%%higher-order q-open bisimilarity
%\newcommand{\HOQOB}{\approx_{qoh}}


%bisimulations in \cite{AD95} and \cite{BD97}, use as needed
%\newcommand{\ee}[2]{\;{#1}^\oslash_{#2}\; }%existence extension
%\newcommand{\ehc}[2]{\;{#1}^\otimes_{#2}\; }%existence hoare closure

%\newcommand{\HOSB}{\sim } %higher-order strong context bisimilarity
%\newcommand{\HOSBH}[1]{\sim^{#1} } %higher-order strong bisimulation in hierarchy
%\newcommand{\HOSBHS}[1]{\sim^{#1}_{\#} } %higher-order strong bisimulation in hierarchy and sharpened
%\newcommand{\HOLE}{\sim_{L} } %higher-order logical equivalence in the strong case
%\newcommand{\HOLEH}[1]{\sim^{#1}_{L} } %higher-order logical equivalence in hierarchy in the strong case

%\newcommand{\CB}{\approx } %higher-order weak context bisimilarity
%\newcommand{\CBOE}{\approx^\circ } %higher-order weak context bisimilarity open extension
%\newcommand{\HOEB}{\approx_{\exists} } %higher-order existential bisimilarity
%\newcommand{\HOEBEE}{\approx_{\exists}^\oslash } %higher-order existential bisimilarity open extension
%\newcommand{\HOEBH}[1]{\approx_{\exists}^{#1} } %higher-order existential bisimulation in hierarchy
%\newcommand{\HOEBHEE}[1]{\approx_{\exists}^{#1 \oslash} } %higher-order existential bisimulation in hierarchy open extension
%\newcommand{\HOLE}{\approx_{L} } %higher-order logical equivalence in the weak case
%\newcommand{\HOLEH}[1]{\approx^{#1}_{L} } %higher-order logical equivalence in hierarchy  in the weak case
%\newcommand{\NHOLEH}[1]{\not\approx^{#1}_{L} } %negation of higher-order logical equivalence in hierarchy  in the weak case









%---------------------------
% Local Variables:
% mode: LaTeX
% TeX-master: "main.tex"
% End:
% End:



%%-------------------------%%----
%%%%%%%%%%%%%%%--BODY--%%%%%%%%%%%%%%%%%%
\begin{document}
%%-----------------------------
%%      the top matter
%%-----------------------------
\title{Parameterizing Higher-order Processes on Names and Processes}
%\thanks{...}%\thanks{...}\thanks{...}% At most 5 thanks
% \thanks{A preliminary version of this work was presented at EXPRESS/SOS 2016. %Comparatively, 
% This paper revises and extends that version with full-fledged proofs and more refined discussions (at least more than half new materials), and moreover, the detailed analysis of a variant encoding of interest. This encoding, mentioned only as a further direction in the preliminary version, is given thorough examination in this work.}
% \thanks{This work has been supported by project ANR 12IS02001 PACE and NSF of China (61872142, 61261130589, 61472239, 61572318).}
\thanks{A preliminary version of this work was presented at EXPRESS/SOS 2016. %Comparatively, 
This paper revises and extends that version with full-fledged proofs and more refined discussions (at least more than half new materials), and moreover, the detailed analysis of a variant encoding of interest. This encoding, mentioned only as a further direction in the preliminary version, is given thorough examination in this work.}
\thanks{This work has been supported by project ANR 12IS02001 PACE and NSF of China (61872142, 61772336, 61572318, 61472239, 61261130589).}

%
\author{Xian Xu}\address{East China University of Science and Technology, Shanghai, China (200237).\\ Email: xuxian@ecust.edu.cn}
%\author{...}\address{...}
%\author{...}\address{...}
%
\date{}
%
\begin{abstract} 
% Parameterization extends higher-order processes with the capability of abstraction and application (like those in lambda-calculus). As is well-known, this extension is strict, 
% %i.e., higher-order processes equipped with parameterization are computationally more expressive. 
% \xxxx{meaning that higher-order processes equipped with parameterization are strictly more expressive than those without parameterization. 
% }
% This paper studies strictly higher-order processes (i.e., no name-passing) with two kinds of parameterization: one on names and the other on processes themselves. We present two main results. One is that in presence of parameterization, higher-order processes can interpret first-order (name-passing) processes in a quite elegant fashion, in contrast to the fact that higher-order processes without parameterization cannot encode first-order processes at all. We present two such encodings and analyze their properties in depth, particularly full abstraction. In the other result, we provide a simpler characterization of the standard context \xxxx{bisimilarity} for higher-order processes with parameterization, in terms of the normal \xxxx{bisimilarity} that stems from the well-known normal characterization for higher-order calculus. 
% \xxxx{As a spinoff, we show that the bisimulation up-to context technique is sound in the higher-order setting with parameterization.}
% %These results demonstrate more essence of the parameterization method in the higher-order paradigm toward expressiveness and behavioural equivalence.
Parameterization extends higher-order processes with the capability of abstraction and application (like those in lambda-calculus). As is well-known, this extension is strict, 
%i.e., higher-order processes equipped with parameterization are computationally more expressive. 
\xxxx{meaning that higher-order processes equipped with parameterization are strictly more expressive than those without parameterization. 
}
This paper studies strictly higher-order processes (i.e., no name-passing) with two kinds of parameterization: one on names and the other on processes themselves. We present two main results. One is that in presence of parameterization, higher-order processes can interpret first-order (name-passing) processes in a quite elegant fashion, in contrast to the fact that higher-order processes without parameterization cannot encode first-order processes at all. We present two such encodings and analyze their properties in depth, particularly full abstraction. In the other result, we provide a simpler characterization of the standard context \xxxx{bisimilarity} for higher-order processes with parameterization, in terms of the normal \xxxx{bisimilarity} that stems from the well-known normal characterization for higher-order calculus. 
\xxxx{As a spinoff, we show that the bisimulation up-to context technique is sound in the higher-order setting with parameterization.}
%These results demonstrate more essence of the parameterization method in the higher-order paradigm toward expressiveness and behavioural equivalence.


\noindent\textbf{Keywords}: {Parameterization, Encoding, Context bisimulation, Normal bisimulation, Higher-order, First-order, Processes}
\end{abstract}
%
\subjclass{68Q85, 89.20.Ff, ???}
%
%\keywords{...}
%
\maketitle
%%-----------------------------
%%      your text
%%-----------------------------
% \section*{Introduction}
% \section{...}
% \subsection{...}
% \subsection{...}
% ...
% \subsection{...}
% %
% \section{...}
% \subsection{...}
% \subsection{...}
% ...
% \subsection{...}
% ...
\section{Introduction}\label{s:introduction}


In concurrent systems, higher-order means that processes communicate by means of process-passing (i.e., program-passing), whereas first-order means that processes communicate through name-passing (i.e., reference-passing). Parameterization originates from lambda-calculus (which is itself of higher-order nature), and enables processes, in a concurrent setting, to do abstraction and application in a way similar to that of lambda-calculus. Say $P$ is a higher-order process, then an abstraction $\lrangle{U}P$ means abstracting the variable $U$ in $P$ to obtain somewhat a function (like $\lambda U.P$ in terms of lambda-calculus), and correspondingly an application $(\lrangle{U}P)\lrangle{K}$ means applying process $K$ to the abstraction and obtaining an instantiation $P\hosub{K}{U}$ (i.e., replacing each variable $U$ in $P$ with $K$, like $(\lambda U.P)K$ in terms of lambda-calculus). There are basically two kinds of parameterization: parameterization on names and parameterization on processes. In the former, $U$ is a name variable and $K$ is a concrete name. In the latter, $U$ is a process variable and $K$ is a concrete process.
Parameterization is a natural way to extend the capacity of higher-order processes and this extension is strict, that is, the computational power strictly increases with the help of parameterization \cite{LPSS10}. In this paper, we study higher-order processes in presence of parameterization.

%\sepp

Comparison between higher-order and first-order processes is a frequent topic in concurrency theory. Such comparison, for example, asks whether higher-order processes can correctly express first-order processes, or vice versa. It is well known that first-order processes can elegantly encode higher-order processes \cite{San92,SW01a}; the converse is however not quite the case. As the first issue, this paper addresses how to encode first-order processes with higher-order processes (equipped with parameterization).

The very early work on using higher-order process to interpret first-order ones is contributed by Thomsen \cite{Tho93}, who proposed a prototype encoding of first-order processes with higher-order processes with the relabelling operator (like that in CCS \cite{Mil89}). This encoding uses a gadget called wire to mimic the function of a name in the higher-order setting, and essentially employs the relabelling to make the wires work properly so as to fulfill the role of names. Due to the arbitrary ability of changing names (e.g., from global to local), the encoding has a correct operational correspondence (i.e., the correspondence between the processes before and after the encoding), but is very hard to analyze for full abstraction (i.e., the first-order processes are equivalent if and only if their encodings are; the `if only' direction is called soundness and the other direction is called completeness). Unfortunately, without the relabelling operator, the basic higher-order process (which has the elementary operators including input, output, parallel composition and restriction) is not capable of encoding first-order processes \cite{Xu12}. In the literature, several variants of higher-order processes are exploited to encode first-order processes. In \cite{SW01a}, an asynchronous higher-order calculus with parameterization on names is used to compile the asynchronous localized $\pi$-calculus (a variant of the first-order $\pi$-calculus \cite{MPW92}). This encoding depends heavily on the notions of `localized' which means only the output capability of a name can be communicated during interactions, and `asynchronous' which means the output is non-blocking. Though technically a nice reference, intuitively because this variant of $\pi$-calculus is less expressive than the full $\pi$-calculus, it is not very surprising that the higher-order processes with parameterization on names can interpret it faithfully, i.e., fully abstract with respect to barbed congruence. Then in \cite{XYL15}, we explore the encoding of the full $\pi$-calculus using higher-order processes with parameterization on names. In that effort, we construct an encoding that harnesses the idea of Thomsen's encoding and show that it is complete. In \cite{BHG06}, Bundgaard et al. use the HOMER to translate the name-passing $\pi$-calculus. This translation is possible because a HOMER process can, in a way quite different from parameterization, operates names in the continuation processes (resources), and this allows flexibility so that names can be communicated in an intermediate fashion. 
In \cite{KPY16}, Kouzapas et al. propose fully abstract encodings concerning first-order processes and session typed higher-order processes. Their encodings use session types to govern communications and show that in the context of session types, first-order and higher-order processes are equally expressive. This work is well related to those mentioned above (and that in this paper), though the context is quite different (i.e., session typed processes).
%In \cite{Fu07}, another variant  ...

Despite the extensive research on encoding first-order processes with (variant) higher-order processes, the following question has remained open: \emph{Is there an encoding of first-order processes by the higher-order processes with the capability of parameterization?}
This question is important in two aspects. One is that parameterization brings about the core of lambda-calculus to higher-order concurrency, so it appears reasonable for such an extension to be able to express first-order processes which has long been shown to be capable of expressing the lambda-calculus. Knowing how this can be achieved would be interesting. The other is that the converse has a almost standard encoding method, i.e., encoding variants of higher-order processes with first-order processes. Yet higher-order processes are still short of an effective way to express first-order ones. Resolving this can also provide (technical) reference for practical work beyond the encoding itself.


%In ...
%In ...
%In ...
%
%\stress{Related work: }
%\begin{itemize}
%\item \greycolor{\cite{Tho93}, (maybe \cite{Xu09x})}
%\item \greycolor{\cite{SW01a}, section 13.3 (page 408)}
%\item \greycolor{\cite{BHG06}}
%\item \greycolor{\cite{Fu07} (not used)}
%\item \greycolor{\cite{XYL15}}
%\item more? (to ask ds maybe)
%\end{itemize}
%

%\sep


Closely related with the first issue on expressiveness, the second issue this paper deals with is the characterization of bisimulation on higher-order processes. Bisimulation theory is a pivotal part of a process model, including the higher-order models, concerning which the almost standard behavioral equivalence is the context bisimulation \cite{San92}. The central idea of context bisimulation is that when comparing output actions, the transmitted process and the residual process (i.e., the process obtained after sending a process) are considered at the same time, rather than separately (like in the applicative higher-order bisimulation proposed by Thomsen \cite{Tho90, Tho93}). For example (for simplicity we do not consider local names), if $P$ and $Q$ are context bisimilar and $P\st{\overline{a}A} P'$ (i.e., $P$ outputs $A$ on $a$ and becomes $P'$), then $Q\wt{\overline{a}B} Q'$ (i.e., $Q$ outputs $B$ on $a$ possibly involving some internal actions and becomes $Q'$), and for every (receiving) environment $E[\cdot]$, $P'\para E[A]$ and $Q'\para E[B]$ are still context bisimilar (here $\para$ denotes concurrency, and $E[A]$ means putting $A$ in the environment $E$). However, in its original form, context bisimulation suffers from inconvenience to use, because it calls for checking with regard to every possible receiving environment. This leads to works on the simpler characterization, called normal bisimulation, of the context bisimulation. The central idea of normal bisimulation, proposed by Sangiorgi \cite{San92, SW01a}, is that instead of checking with a general process in input and a general context in output, one only needs to comply with the matching of some special process or context, specifically a class of terms called triggers. To meet this challenge, a crucial so-called factorization theorem is used to circumvent technical difficulty. We briefly explain how normal bisimulation is designed in the basic higher-order processes. In particular, the factorization states the following property, where $\WCB$ denotes context bisimulation, and $\overline{m}.P$ and $m.P$ are CCS-like prefixes in which the communicated contents are not important \cite{SW01a}. \nsepvs{.3}
\[ E[A] \WCB (m)(E[\overline{m}.0] \para !m.A)  \nsepvs{.3} 
\] 
One can clearly identify the reposition of the process $A$ of interest, which in fact captures the core of the property: move $A$ to a new position as a repository, which in turn can be retrieved as many times as needed in the original environment $E$, with the help of the pointer undertaken by the fresh channel $m$ (called trigger). Inspired by the factorization, normal bisimulation can be developed. We take the output as an example (input is similar), and restriction operation in output is omitted for the sake of simplicity. As stated above, context bisimulation requires the following chasing diagram, which is now extended with an application of the factorization. \nsepvs{.3} 
% \[
% \xymatrix{
%  &  & P \ar@{.}[rr]|-{\WCB}\ar@{->}[d]_{\overline{a}A}  &  & Q \ar@{=>}[d]^{\overline{a}B}  &  & \\
% P'\para (m)(E[\overline{m}.0] \para !m.A) \ar@{}[r]|-{\WCB} & P'\para E[A] \ar@/_1.6pc/@{.}[0,4]|{\WCB} & P'  & &  Q'  & Q'\para E[B] \ar@{}[r]|-{\WCB}  &  Q'\para (m)(E[\overline{m}.0] \para !m.B)
% }
% \]
\[ \nsepvs{.2} 
\xymatrix@C=20pt{
 &  & P \ar@{.}[rr]|-{\WCB}\ar@{->}[d]_{\overline{a}A}  &  & Q \ar@{=>}[d]^{\overline{a}B}  &  & \\
P'\para (m)(E[\overline{m}.0] \para !m.A) \ar@{}[r]|-{\WCB} & P'\para E[A] \ar@/_1.6pc/@{.}[0,4]|{\WCB} & P'  & &  Q'  & Q'\para E[B] \ar@{}[r]|-{\WCB}  &  Q'\para (m)(E[\overline{m}.0] \para !m.B)
}
\]
Since context bisimulation $\WCB$ is a congruence, one can cancel the common part of (the leftmost) $P'\para (m)(E[\overline{m}.0] \para !m.A)$ and (the rightmost) $Q'\para (m)(E[\overline{m}.0] \para !m.B) $, and simply requires that $P'\para !m.A$ and $Q'\para !m.B$ are related, without fearing losing any discriminating power. This in turn leads to the following requirement in normal bisimulation (assuming $\mathcal{R}$ is a normal bisimulation). \nsepvs{.3} 
\[\nsepvs{.2} 
\xymatrix{
 & P \ar@{.}[rr]|-{\mathcal{R}}\ar@{->}[d]_{\overline{a}A}  &  & Q \ar@{=>}[d]^{\overline{a}B}  &   \\
 P'\para !m.A \ar@/_1.6pc/@{.}[0,4]|{\mathcal{R}} & P'  & &  Q'  & Q'\para !m.B
}
\]

Subsequent works attempt to extend the normal bisimulation to variants of higher-order processes. In Sangiorgi's initial work \cite{San92}, normal bisimulation is also obtained for higher-order processes with parameterization. That characterization , however, is made in the presence of first-order processes (i.e., name-passing), and thus not very convincing with regard to the inner complexity of context bisimulation in presence of parameterization. In \cite{Xu13}, we revisited this issue and show that in a purely higher-order setting (viz., no name-passing at all), parameterization on processes does not deprive one of the convenience of normal bisimulation. Although the idea is inspired by the original work of Sangiorgi, the proof approach is more direct. In \cite{LSS09, LSS11}, Lenglet et al. study higher-order processes with passivation (i.e., the process in the output position may evolve), and report a normal bisimulation for a sub-calculus without the restriction operator, but that characterization has somewhat a different flavor, since the higher-order bisimulation \cite{Tho93} rather than the context bisimulation is taken. Though these works carry out insightful research and give meaningful references, it is currently still not clear how to construct a simple characterization of context bisimulation based on parameterization over names, and this raises the following fundamental question: \emph{Does higher-order processes with parameterization on names have a normal bisimulation?}
In the second part of this paper, we move further from \cite{Xu13,San92}, and offer a normal bisimulation for higher-order processes in the setting of parameterization over both names and processes.  

%(\stress{explain the basic idea of normal bisimulation without technical details and then state the results in the field})

%\[
%\xymatrix{
%  T \ar@{}[r]|-{\equiv} & G[F[\enc{R}\fosub{b}{x}]] \ar@{.}[rr]|-{?}\ar@{.}[d]|-{?}  &  & H[F[\enc{R'}\fosub{b}{x}]] \ar@{.}[d]|-{?} \ar@{}[r]|-{\equiv}  & T'  \\
%  T_1 \ar@{}[r]|-{\equiv} & G[\enc{R}\fosub{b}{x}] \ar@{}[rr]|-{ \SCB\, \enc{P_1}\,\mathcal{R}\,\enc{Q_1}\,\SCB}  & &  H[\enc{R'}\fosub{b}{x}] \ar@{}[r]|-{\equiv} & T_2
%}
%\]


%In ...
%In ...
%In ...
%
%\stress{Related work: }
%\begin{itemize}
%\item \greycolor{\cite{Tho93}}
%\item \greycolor{\cite{San92, SW01a}}
%\item \greycolor{\cite{Xu13} (notice maybe ahe full version that includes some discussion concerning Schmitt's work \cite{LSS09, LSS11})}
%\item more? (to ask ds maybe)
%\end{itemize}

\psepvs{.2}
%\paragraph{Contribution}
\noindent\textbf{Contribution}~ 
In summary, our contribution of this work is as follows.
\begin{itemize}
\item %(\bc{Expressiveness.})
We show that the extension with parameterization (on both names and processes) allows higher-order processes to interpret first-order processes in a surprisingly concise yet elegant manner. Such kind of encoding is of a somewhat dissimilar flavor, and moreover not possible in absence of parameterization. We give the detailed encoding strategy, and prove that it satisfies a number of desired properties well-known in the field. \\%, including soundness and completeness.
%Then in, we explore the encoding of the full $\pi$-calculus using higher-order processes with parameterization on names.
The idea of the encoding in this paper is quite different from our abovementioned work in \cite{XYL15}, where we build an encoding that allows parameterization merely on names (i.e., no parameterization on processes). The soundness of that encoding is not very satisfying, which in a sense defeats some purpose of the encoding, and this actually precipitates the work here.

\item %(\bc{Normal bisimulation.})
We establish the normal bisimulation, as an effectively simpler characterization of context bisimulation, for higher-order processes with both kinds of parameterization. This normal bisimulation extends those for higher-order processes without parameterization, particularly in the manipulation of abstractions on names. As far as we are concerned, similar characterization has not been reported before.\\
That the processes are purely higher-order (that is, without name-passing) improves the result in \cite{San92}, and articulates that the characterization based on normal bisimulation is a property independent of first-order name-passing. Moreover, this does not contradict the argument in \cite{Xu13} that there is little hope that normal bisimulation exists in higher-order processes with (only) parameterization on names, because here the processes are capable of parameterization on processes as well (though still higher-order).
\end{itemize}


%\paragraph{Paper organization}
\noindent\textbf{Organization}
The remainder of this paper is organized as below. In Section \ref{s:preliminary}, we introduce the calculi and a notion of encoding used in this paper. In Section \ref{s:encoding}, we present the encoding from first-order processes to higher-order processes with parameterization, and discuss its properties. In Section \ref{s:normal}, we define the normal bisimulation for higher-order processes with parameterization, and prove that it truly characterizes context bisimulation. Section \ref{s:conclusion} concludes this work and point out some further directions.






%---------------------------
% Local Variables:
% mode: LaTeX
% TeX-master: "main.tex"
% End:

\section{Preliminary}\label{s:preliminary}  %\nts{------FROM HERE------}\\
In this section, we give the basic definitions and notations used in this work. \psepvs{.2}

\subsection{Calculus \FOPi}
%\noindent\textbf{\large 2.1~ Calculus \FOPi}
%Use name dichotomy \cite{EN86a, EN00, Fu15}. \\
%\stress{\scriptsize In output: Is bound output sufficient? (i.e., remove free output? seems unnecessary now); (maybe helpful in discussing the encoding).}

%\stress{\large see ``[ref]pi\_PiDs\_defs.tex" in the root folder; or REFER TO``GIT/hp\_comcan\_gl (\cite{XYL15})"; }

The first-order (name-passing) pi-calculus, \FOPi, is proposed by Milner et al. \cite{MPW92}. For the sake of simplicity, throughout the paper, names (ranged over by $m,n,u,v,w$) are divided into two classes: name constants (ranged over by $a,b,c,d,e,f,g,h$) and name variables (ranged over by $x,y,z$) \cite{EN86a,EN00,Fu15}. The grammar is as below with the constructs having their standard meaning. We note that guarded input replication is used instead of general replication, and this does not decrease the  expressiveness \cite{San98,FL09a}. \nsepvs{.2}
\[P,Q := 0 \,\Big{|}\, m(x).P \,\Big{|}\, \overline{m}n.P \,\Big{|}\,  (c)P \,\Big{|}\, P\para Q \,\Big{|}\, !m(x).P %\,\Big{|}\, !\overline{m}n.P
\]

A name constant $a$ is bound (or local) in $(a)P$ and free (or global) otherwise. A name variable $x$ is bound in $m(x).P$ and free otherwise.
Respectively $\fn{\cdot}, \bn{\cdot}, \n{\cdot}, \fnv{\cdot}, \bnv{\cdot}, \nv{\cdot}$ denote free name constants, bound name constants, names, free name variables, bound name variables, and name variables in a set of processes. A name is fresh if it does not appear in any process under discussion. By default, closed processes are considered, i.e., those having no free variables.
As usual, here are a few derived operators: $\overline{m}(d).P \DEF (d)\overline{m}d.P$, $a.P \DEF  m(x).P \;(x\notin \fnv{P})$, $\overline{m}.P \DEF \overline{m}(d).P\;(d\notin \fn{P})$; $\tau.P\DEF (a)(a.P\para \overline{a}.0)$ ($a$ fresh). A trailing $0$ process is usually omitted.
We denote tuples by a tilde. For tuple $\ve{n}$: $\size{\ve{n}}$ denotes its length; %$m\in \ve{n}$ means $m$ is its element;
$m\ve{n}$ denotes incorporating $m$. Multiple restriction $(c_1)(c_2)\cdots (c_k)E$ is abbreviated as $(\ve{c})E$.
Substitution $\fosub{n}{m}$, ranged over by $\sigma$ and used in the form $P\fosub{n}{m}$, is a mapping that replaces free $m$ with $n$ (in $P$) while keeping the rest unchanged. %are ranged over by $\sigma$.
%Sometimes substituting a constant for another is called renaming, and assignment if for a variable.
A context $C$ is a process with some subprocess replaced by the hole $[\cdot]$, and $C[A]$ is the process obtained by filling in the hole with $A$.
%The LTS is recalled in appendix \ref{appendix:semantics}.

The semantics of \FOPi\ is defined by the LTS (Labelled Transition System) below. \nsepvs{.2} %(symmetric rules are skipped).
\[
\begin{array}{lllll}
\infer{a(x).P\st{a(b)} P\fosub{b}{x}}{} \quad &
\infer{\overline{a}b.P\st{\overline{a}b} P}{} \quad &
\infer{!a(x).P \st{a(b)} P\fosub{b}{x}\para !a(x).P}{}\quad  & %& \infer{!\overline{a}b.P \st{\overline{a}b} P\para !\overline{a}b.P}{}
\infer[{\scriptstyle c\not\in n(\lambda)}]{(c)P\st{\lambda} P'}{P\st{\lambda} P'} \quad
\end{array}
\]
\[
\begin{array}{llll}
\infer[{\scriptstyle c\neq a}]{(c)P\st{\overline{a}(c)} P'}{P\st{\overline{a}c} P'} \quad &
\infer[{\scriptstyle bn(\lambda)\cap fn(Q)=\emptyset}]{P\para Q\st{\lambda} P'\para Q}{P\st{\lambda} P'} \quad &
\infer{P\para Q\st{\tau}P'\para Q'}{P\st{a(b)}P' \quad Q\st{\overline{a}b} Q'}\quad &
\infer{P\para Q\st{\tau}(b)(P'\para Q')}{P\st{a(b)} P'\quad Q\st{\overline{a}(b)} Q'} %\quad&
\end{array}
\]
%\[
%\begin{array}{ll}
%\end{array}
%\]
%\[Structural:
%\begin{array}{c}
%\frac{\displaystyle Q\equiv P,\; P\st{\lambda} P', \; P'\equiv Q'}{\displaystyle Q\st{\lambda} Q'}
%\end{array}
%\]
Actions, ranged over by $\lambda,\alpha$, comprise internal move $\tau$, and visible ones: input ($a(b)$), output ($\overline{a}b$) and bound output ($\overline{a}(c)$). We note that actions occur only on name constants, and a communicated name is also a constant. % (i.e. no variable can be transmitted).
%Define $\overline{\lambda}$ as $\overline{a}b$ if $\lambda$ is $a(b)$, $a(b)$ if $\lambda$ is $\overline{a}b \mbox{ or } \overline{a}(b)$, and $\tau$ if $\lambda$ is $\tau$.
We denote by $\equiv$ the structural congruence \cite{MPW92}\cite{SW01a}, which is the smallest relation satisfying $\alpha$-convertibility, the monoid laws for parallel composition, commutative laws for both composition and restriction, and a distributive law $(c)(P\para Q) \SE (c)P\para Q$ (if $c\notin \fn{Q}$).
We assume no name capture (i.e., a free name falling into the scope of a same bound name) subject to $\alpha$-conversion all the time.
We use $\wt{}$ for the reflexive transitive closure of $\st{\tau}$, $\wt{\lambda}$ for $\wt{}\st{\lambda}\wt{}$, and $\wt{\widehat{\lambda}}$ for $\wt{\lambda}$ if $\lambda$ is not $\tau$, and $\wt{}$ otherwise. A process $P$ is divergent, written $\diverge{P}$, if it has an infinite sequence of $\tau$ actions.
%\strs{A relation $\R$ is divergence-sensitive if whenever $P\,\R\, Q$ then $\diverge{Q}$ implies $\diverge{P}$}.
%Let $\equiv$ denote the standard structural congruence, as defined below.
%\[
%\begin{array}{l}
%P\equiv Q, \mbox{ if they are $\alpha$-convertible to each other} \\
%P\para 0 \equiv P, \, P\para (Q\para R) \equiv (P\para Q)\para R, \, P \para Q  \equiv Q\para P \\
%(c)(d)P \equiv (d)(c)P, \,(c)0\equiv 0 \\
% (c)(P\para Q) \equiv (c)P\para Q \mbox{ whenever } c\notin fn(Q) %\\
%\end{array}
%\]

Throughout the paper, we use the following standard notion of bisimulation \cite{MPW92,EN00,SW01a}. %; it coincides with the ground bisimulation (hence the name) \cite{MPW92,EN00,SW01a}.  %In the standard way, the bisimulation on $\pi$ is defined as below and is a congruence relation \cite{MPW92}\cite{EN00}. %\cite{FZ09}.
\begin{definition}
A \ground bisimulation is a symmetric relation $\mathcal{R}$ on \FOPi\ processes s.t. whenever $P\,\mathcal{R}\, Q$ the following property holds:
%\begin{itemize}
If $P\st{\alpha} P'$ where $\alpha$ is $a(b)$, $\overline{a}b$, $\overline{a}(b)$, or $\tau$, then $Q\wt{\hat{\alpha}} Q'$ for some $Q'$ and $P'\,\mathcal{R}\, Q'$.
%\end{itemize}
\Ground Bisimilarity, $\WGB$, is the largest \ground bisimulation.
\end{definition}
%\begin{definition}[Bisimulation]\label{ex-w-bisi}
%Weak bisimilarity $\WGB$ is the largest symmetric bisimulation relation $\mathcal{R}$ on $\pi$ processes such that whenever $P \,\mathcal{R}\, Q$ and $P\st{\lambda} P'$ then $Q\wt{\widehat{\lambda}} Q'$ and $P' \,\mathcal{R}\, Q'$.
%\end{definition}
The subscript `g' in $\WGB$ is used to differentiate from other notions of bisimulation in this work and also stands for ``ground" (it actually coincides with the standard notion of ground bisimulation \cite{EN00,SW01a,Fu05b}).
We denote by $\SGB$ the strong \ground bisimilarity (i.e., replacing $\wt{\widehat{\alpha}}$ with  $\st{\alpha}$ in the definition). It is well-known that $\WGB$ is a congruence \cite{EN00,SW01a}, and coincides with the so-called local bisimilarity as defined below \cite{Fu05b,Xu12}.
%Notice that an alternative way to define weak bisimulation %(similar for other bisimulation relations)
%is to use $\wt{\lambda}$ instead of $\st{\lambda}$ in Definition \ref{ex-w-bisi} (see \cite{Mil89, SW01a}).
%The local bisimilarity ($\EWLB$) characterizes weak bisimilarity, i.e. $\EWLB$ coincides with $\WGB$ (see \cite{Fu05b} for a proof), and will be used when discussing the encodings.
\begin{definition}\label{ext-local-bisi}
Local bisimilarity $\EWLB$ is the largest symmetric local bisimulation relation $\mathcal{R}$ on $\pi$ processes such that:
%\begin{itemize}
%\item
(1) if $P\st{\lambda} P'$, $\lambda$ is not bound output, then $Q\wt{\widehat{\lambda}} Q'$ and $P'\,\mathcal{R}\, Q'$;
%\item
(2) if $P\st{\overline{a}(b)} P'$, then $Q\wt{\overline{a}(b)} Q'$, and for every $R$, $(b)(P'\para R)\,\mathcal{R}\, (b)(Q'\para R)$.
%\end{itemize}
\end{definition}
% $\EWLB$ indeed characterizes $\AEFOPi$ (see \cite{Fu05b} for a proof).
% \begin{lemma}\label{ext-local-bisi-prop}
% $\EWLB$ coincides with $\AEFOPi$.
% \end{lemma}

%\subsubsection{Ground bisimulation}
%\begin{definition}
%A \ground bisimulation is a symmetric relation $\mathcal{R}$ on \FOPi\ processes s.t. whenever $P\,\mathcal{R}\, Q$ the following property holds.
%\begin{itemize}
%\item If $P\st{\alpha}$ where $\alpha$ is $a(b)$, $\overline{a}b$, $\overline{a}(b)$, or $\tau$, then $Q\wt{\hat{\alpha}} Q'$ for some $Q'$ and $P'\,\mathcal{R}\, Q'$.
%\end{itemize}
%Ground bisimilarity, $\WGB$, is the largest \ground bisimulation.
%\end{definition}




\subsection{Calculus \HOPiDd} %\HOPiD, \HOPid,
%\psepvs{.2}
%\noindent\textbf{\large 2.2 ~Calculus \HOPiDd}
%\stress{\large see ``[ref]pi\_PiDs\_defs.tex" in the root folder; or REFER TO ``GIT/xx (\cite{XYL13})" ; }

%For the sake of conciseness,
We first define the basic higher-order calculus and then the extension with parameterization.

\subsubsection{Calculus \HOPi}
%\psepvs{.2}
%\noindent\textbf{2.2.1 ~ Calculus \HOPi}

The basic higher-order (process-passing) calculus, \HOPi, is defined by the following grammar in which the operators have their standard meaning.
%We use processes, denoted by uppercase letters ($T,P,Q,R,A,B,E,F...$), are defined by the following grammar. Lowercase letters stand for channel names.
We denote by $X,Y,Z$ process variables.
%We use $\Pi_{seg}$ for convenience.
\[
\begin{array}{l}
T,T' ::= 0 \,\Big{|}\, X \,\Big{|}\,  u(X).T \,\Big{|}\, \overline{u}T'.T \,\Big{|}\, T\para T' \,\Big{|}\, (c)T \,\Big{|}\,  !u(X).T \,\Big{|}\, !\overline{u}T'.T
\end{array}
\]
%The operators are: prefix:$u(X).T, \overline{u}T'.T$; composition: $T\para T'$; restriction: $(c)T$ in which $c$ is bound. Parallel composition has the least precedence.
We use $a.0$ for $a(X).0$, $\overline{a}.0$ for $\overline{a}0.0$, %$\overline{m}A$ for $\overline{m}A.0$;
$\tau.P$ for $(a)(a.P\para \overline{a}.0)$, and sometimes $\overline{a}[A].T$ for $\overline{a}A.T$. Like \FOPi, a tilde represents a tuple.
We reuse the notations for names in \FOPi\, and additionally use $\fpv{\cdot}$, $\bpv{\cdot}$, $\pv{\cdot}$ respectively to denote free process variables, bound process variables and process variables in a set of processes. Closed processes are those having no free variables and are considered by default.
A higher-order substitution $T\hosub{A}{X}$ replaces variable $X$ with $A$ and can be extended to tuples in the usual way.
$E[\ve{X}]$ denotes $E$ with (possibly) variables $\ve{X}$, and $E[\ve{A}]$ stands for $E\hosub{\ve{A}}{\ve{X}}$.
%We work up-to $\alpha$-conversion and always assume no capture.
%We use input-guarded replication as a derived operator \cite{Tho93,LPSS08}:
%$ !\phi.P \DEF (c)(Q_{\scriptscriptstyle c,\phi,P} \para \overline{c}Q_{\scriptscriptstyle c,\phi,P})$,
%  $Q_{\scriptscriptstyle c,\phi,P} \DEF c(X).(\phi.(X\para P) \para \overline{c}X)$, where $\phi$ is a prefix.
The guarded replications used in the grammar can actually be derived \cite{Tho93,LPSS08}, and we make them primitive for convenience. %the sake of convenience.
The semantics of \HOPi\ is as below. \nsepvs{.2}%standard    Symmetric rules are omitted.
%\begin{figure}[htbp]
\[
\begin{array}{lllll}
\frac{\displaystyle }{\displaystyle a(X).T\st{a(A)} T\hosub{A}{X}}  &
 \frac{}{\displaystyle \overline{a}A.T\st{\overline{a}A} T}  &
\frac{\displaystyle T\st{\lambda} T'}{\displaystyle (c)T\st{\lambda} (c)T'}{\scriptstyle c\not\in n(\lambda)} &
\frac{\displaystyle T\st{\lambda} T'}{\displaystyle T\para T_1\st{\lambda} T'\para T_1} & %{\scriptstyle bn(\lambda)\cap fn(T_1)=\emptyset} &
 \frac{}{\displaystyle !\overline{a}A.T\st{\overline{a}A} T \para !\overline{a}A.T}
\end{array}
\]
\[\begin{array}{lll}
\frac{\displaystyle T\st{(\ve{c})\overline{a}[A]} T'}{\displaystyle (d)T\st{(d)(\ve{c})\overline{a}[A]} T'} {\scriptstyle d \in fn(A){-}\{\ve{c},a\}}  &
 \frac{\displaystyle T_1\st{a(A)} T_1', T_2\st{(\ve{c})\overline{a}[A]} T_2'}{\displaystyle T_1\para T_2 \st{\tau}(\ve{c})(T_1'\,|\,T_2')} & % {\scriptstyle \ve{c}\cap fn(T_1) = \emptyset} &
\frac{\displaystyle }{\displaystyle !a(X).T\st{a(A)} T\hosub{A}{X}\para !a(X).T }
\end{array}
\]
%*************************
%\vspace*{-.5cm}
%*************************
%\caption{Semantics of $\Pi$}\label{lts-Pi}
%\end{figure}
%The rules are mostly self-explanatory. For instance, in higher-order input $a(A)$, the received process $A$ becomes part of the receiving environment through a substitution.
We denote by $\alpha,\lambda$ the actions: internal move ($\tau$), input ($a(A)$), output ($(\ve{c})\overline{a}A$) in which $\ve{c}$ is some local names carried by $A$ during the output. We always assume no name capture with resort to $\alpha$-conversion.
%Operations $fn(), bn(), n()$ can be similarly defined on actions.
%$\wt{}$ is the reflexive transitive closure of internal actions $\tau$, and $\wt{\lambda}$ is $\wt{}\xrightarrow{\lambda}\wt{}$.  $\wt{\hat{\lambda}}$ is $\wt{}$ when $\lambda$ is $\tau$ and $\wt{\lambda}$ otherwise. $\st{\tau}_k$ means $k$ consecutive $\tau$'s.
The notations $\wt{}$, $\wt{\lambda}$ and $\wt{\widehat{\lambda}}$ are similar to those in \FOPi. We also reuse $\equiv$ for the structural congruence in \HOPi\ (and also \HOPiDd\ to be defined shortly) \cite{SW01a}, and this shall not raise confusion under specific context.
%``$P\wt{\hat{\lambda}} P'$" abbreviates ``there is a process $P'$ such that $P\wt{\hat{\lambda}} P'$".
 %$P\wt{}\cdot \mathcal{R}\, Q$ means $P\wt{} Q'$ and $Q' \,\mathcal{R}\, Q$ (i.e. $(Q',Q)\in \mathcal{R}$), where $\mathcal{R}$ is a binary relation.
%We say relation $\mathcal{R}$ is closed under (variable) substitution if $(E\hosub{A}{X},F\hosub{A}{X})\in \mathcal{R}$ for any $A$ whenever $(E,F)\in \mathcal{R}$.
%A process diverges if it can perform an infinite $\tau$ sequence.
%*************************
%\vspace*{-.9cm}
%*************************


\subsubsection{Calculus \HOPiDd}
%\psepvs{.2}
%\noindent\textbf{2.2.2 ~ Calculus \HOPiDd}

Parameterization extends \HOPi\ with the syntax and semantics below. %in Figure \ref{PiDn}.
Symbol $U_i$ (respectively, $K_i$) ($i=1,...,n$) is used as a meta-parameter of an abstraction (respectively, meta-instance of an application), and stands for a process variable or name variable (respectively, a process or a name). \nsepvs{.2} %, as will be clear shortly when defining \HOPiDd.
%\begin{figure}[htbp]
\[
\begin{array}{ll}
\mbox{\small Extension of syntax: } & \lrangle{\vect{U}} T \;\Big{|}\;  T'\lrangle{\vect{K}}\\
\mbox{\small Extension of semantics: } & \frac{\displaystyle Q\equiv P\quad P\st{\lambda} P'\quad P'\equiv Q'}{\displaystyle Q\st{\lambda} Q'} \\
\mbox{\small Extension of structural congruence ($\equiv$): } & F\lrangle{\ve{K}} \equiv T\hosub{\ve{K}}{\ve{U}} \quad \mbox{ where } F\DEF \lrangle{\ve{U}}T \mbox{ and } \size{\ve{U}}{=}\size{\ve{K}} %{=}n
\end{array}
\]
%*************************
%\vspace*{-.5cm}
%*************************
%\caption{$\Pi$ with parameterization}\label{PiDn}
%\end{figure}

We denote by $\lrangle{\vect{U}} T$ an n-ary abstraction in which $\vect{U}$ are the parameters to be instantiated during the application $T'\lrangle{\vect{K}}$ in which the parameters are replaced by instances $\vect{K}$. This application is modelled by an extensional rule for structural congruence as above, in combination with the usual LTS rule for structural congruence as well, so as to make the process engaged in application evolve effectively.
The condition $\size{\ve{U}}{=}\size{\ve{K}}$ requires that the parameters and the instantiating objects should be equal in length.
%It also expresses that the parameterized process can do an action only after the application happens.
%Calculus $\Pi^D_n$, which has process parameterization (or higher-order abstraction),
%  is defined by taking $\ve{U},\ve{K}$ as $\ve{X},\ve{T'}$ respectively.
%Calculus $\Pi^d_n$, which has name parameterization (or first-order abstraction),  is defined by taking $\ve{U},\ve{K}$ as $\ve{x},\ve{u}$ respectively.
%For convenience, names (ranged over by $u,v,w$) are handled dichotomically: name constants (ranged over by $a,b,c,...,m,n$); name variables (ranged over by $x,y,z$).

Now parameterization on process is obtained by taking $\ve{U},\ve{K}$ as $\ve{X},\ve{T'}$ respectively, and parameterization on names is obtained by taking $\ve{U},\ve{K}$ as $\ve{x},\ve{u}$ respectively. The corresponding abstractions are sometimes called process abstraction and name abstraction respectively, and we may use abstraction and parameterization interchangeably. For convenience, names are sometimes handled in a similar way as that in \FOPi\ (so are the related notations). We denote by \HOPiDd\ the calculus \HOPi\ extended with both kinds of parameterizations.
Calculus \HOPiDd\ can be made more precise with the help of a type system \cite{San92} which however is not important for this work and not presented, and we always assume type consistency. We note that in $\lrangle{\vect{U}} T$, variables $\vect{U}$ are bound.

%*************************
%\vspace*{-.4cm}
%*************************
%*************************
%\vspace*{-.3cm}
%*************************

%remark  about the calculi

%Process expressions (or terms) of the form $\lrangle{\ve{X}} P$ or $\lrangle{\ve{x}} P$, in which $\ve{X}$ or $\ve{x}$ is not empty, are \emph{parameterized processes}.
%Terms without outmost parameterization are \emph{non-parameterized processes}, or simply processes.
%We mainly focus on processes in the encodings to be presented. Sometimes related notations are slightly abused if no confusion is caused.
%Only free  variables can be effectively parameterized; they become \emph{bound} after parameterization. In the syntax, null-parameterizations, for example $\lrangle{X_1,X_2}P$ in which $X_2\notin fv(P)$, are allowed.
%A $\Pi^D_n$($n\geq 1$) process is definable in $\Pi^D_{n+1}$ (similar for $\Pi^d_n$) by making use of fresh dummy variable.
%The semantics of parameterization renders it somewhat natural to deem application as extra rule of structural congruence,
%denoted by $\equiv$, which is defined in the standard way \cite{Mil92,San94}. In addition to the standard rules (e.g. monoid rules for composition), it includes the application rules:
%$(\lrangle{\ve{X}}P)\lrangle{\ve{A}}\equiv P\hosub{\ve{A}}{\ve{X}}$ and $(\lrangle{\ve{x}}P)\lrangle{\ve{u}}\equiv P\fosub{\ve{u}}{\ve{x}}$.

%Type systems for $\Pi^D_n$ and $\Pi^d_n$ (also $\Pi^{D,d}_n$) can be routinely defined in a similar way to that in \cite{San92}, where the concept of \emph{order} is also formalized.
%As mentioned, here we require that only second-order processes be admitted.
%So parameterizing does not occur on parameterized processes, and application does not use parameterized processes as instance.
%That is, we stipulate that in $\lrangle{\ve{U}}T'$ and $T\lrangle{\ve{T'}}$, $T'$ are not parameterized processes.
%This prohibits processes such as $\lrangle{X}(\lrangle{Y}T)$ and $(\lrangle{X}(X\lrangle{A}))\lrangle{B}$ in which $B\equiv \lrangle{Y}Y$; instead, in the former one could use $\lrangle{X,Y}T$, and in the latter the instance $A$ could be sent to a service containing $B$ before retrieving the result.
%The main reason we restrict to second-order processes is that it serves well in related study (e.g. the encodings in this paper and work from \cite{San92}) and can also simplify type consistency in programming. Technically this rules out a trivial encoding in section \ref{hierarchy} (some more remark is given in its end), simplifies the encoding in section \ref{relationship-ho-fo-abstraction} because it is currently not obvious how to extend to the general case.



%\subsubsection{Context bisimulation}
%\stress{Notice that in context bisimulation the matching action (i.e., input or output) should have the same type of communicated processes as to the original action. E.g., $Q\WCB P\st{\overline{a}A} P'$ matched by $Q\wt{\overline{a}B} Q'$ in which $B$ has the same type as $A$.}

Throughout the paper, we reply on the following notion of context bisimulation \cite{San92,San94}. %The definition is the same for calculi \HOPiD, \HOPid, and \HOPiDd.
\begin{definition}\label{context-bisimulation}
A symmetric relation $\mathcal{R}$ on \HOPiDd\ processes is a context bisimulation, if $P\,\mathcal{R}\, Q$ implies the following properties:
\begin{itemize}
\item[(1)] if $P \st{\alpha} P'$ and $\alpha$ is $a(A)$ or $\tau$, then $Q \wt{\widehat{\alpha}} Q'$ for some $Q'$ and $P'\,\mathcal{R}\, Q'$;
\item[(2)] if $P \st{(\ve{c})\overline{a}A} P'$ and $A$ is a process abstraction or name abstraction or not an abstraction, then $Q \wt{(\ve{d})\overline{a}B} Q'$ for some $B$ that is accordingly a process abstraction or name abstraction or not an abstraction, and moreover for every $E[X]$ s.t. $\{\ve{c},\ve{d}\}\cap \fn{E}=\emptyset$ it holds that $(\ve{c})(E[A]\para P') \; \mathcal{R}\;  (\ve{d})(E[B]\para Q')$.
\end{itemize}
Context bisimilarity, written $\WCB$, is the largest context bisimulation.
\end{definition}
We note that the matching for output in context bisimulation is required to bear the same kind of communicated process as compared to the simulated action. Relation $\SCB$ denotes the strong context bisimilarity. As is well-known, $\WCB$ is a congruence \cite{San92,San94}.



\subsection{A notion of encoding}\label{s:criteria}
%\psepvs{.2}
%\noindent\textbf{\large 2.3 ~A notion of encoding}

We define a notion of encoding in this section. %First, we define what
We assume that a process model $\mathcal{L}$ is a triplet $(\mathcal{P}, \st{}, \approx)$, where $\mathcal{P}$ is the set of processes, $\st{}$ is the  LTS with a set $\mathcal{A}$ of actions, and $\approx$ is a behavioral equivalence {(with structural congruence implicit)}.
Given $\mathcal{L}_i\DEF (\mathcal{P}_i, \st{}_i, \approx_i)$ ($i{=}1,2$), an encoding from  $\mathcal{L}_1$ to $\mathcal{L}_2$ is a function $\encoding{\cdot}{}{}: \mathcal{P}_1 \longrightarrow \mathcal{P}_2$ that satisfies some set of criteria.
Notation $\encode{\mathcal{P}_1}{}{}$ stands for the image of the $\mathcal{L}_1$-processes inside $\mathcal{L}_2$ under the encoding. It should be clear that  $\encode{\mathcal{P}_1}{}{}\subseteq \mathcal{P}_2$. We use $\dot\approx_2$ to denote the behavioural equivalence $\approx_2$ restricted to $\encode{\mathcal{P}_1}{}{}$ {(up-to structural congruence)} \cite{Gor09}.
The following criteria set (Definition \ref{gorla-like-cond}) used in this paper, as a benchmark for encodings, stems from \cite{LPSS10} (the version provided in \cite{LPSS10} is based on \cite{Gor08a}).
%Notation $\mathcal{L}_1 \sqsubseteq \mathcal{L}_2$ means there is an encoding from $\mathcal{L}_1$ to $\mathcal{L}_2$. We know that $\sqsubseteq$ enjoys transitivity \cite{LPSS10}.
As is known, encodability enjoys transitivity \cite{LPSS10}.
%We will show that the encoding in Section \ref{s:encoding} satisfies all the criteria in Definition \ref{gorla-like-cond} except adequacy (1a).


\begin{definition}[Criteria for encodings]\label{gorla-like-cond}
%\begin{enumerate}
%\item[]
\noindent\textbf{Static criteria}:
(1) Compositionality. \emph{For any $k$-ary operator $op$ of $\mathcal{L}_1$, and all $P_1,...,P_k\in \mathcal{P}_1$,  $\encoding{op(P_1,...,P_k)}{}{}$ $=\, C_{op}[\encoding{P_1}{}{},...,\encoding{P_k}{}{}]$ for some (multihole) context $C_{op}[\cdots]\in \mathcal{P}_2$;}
%(2) [\stress{DROP THIS?}]~ Name invariance. \emph{For any injective substitution $\sigma$ of names, $\encoding{P\sigma}{}{} \,=\, \encoding{P}{}{}\sigma$. }

%\item[]
\noindent
\textbf{Dynamic criteria}:
%(1) [\stress{DROP THIS?}]~ Forth operational correspondence. %Action completeness.
%\emph{Whenever $P\wt{\lambda} P'$, it holds $\encoding{P}{}{} \wt{\lambda'} \approx_2 \encoding{P'}{}{}$, for some action $\lambda'$ with the same \emph{subject}  as that of $\lambda$ (the subject of an action (e.g., $a(A)$) is the name on which the action happens (e.g., $a$)) ;} \\
%(2) [\stress{DROP THIS?}]~ Back operational correspondence. %Action soundness.
%\emph{Whenever $\encoding{P}{}{} \wt{\lambda'} T$, there exist $P'$ and $\lambda$ with the same \emph{subject} as that of $\lambda'$ s.t. $P\wt{\lambda} P'$ and $T\wt{} \approx_2 \encoding{P'}{}{}$;}\\
(1a) Adequacy. \emph{$P \approx_1 P'$ implies $\encoding{P}{}{} \approx_2 \encoding{P'}{}{}$. This is also known as \emph{soundness}. The converse is known as \emph{completeness};}
(1b) Weak adequacy (or weak soundness). \emph{$P \approx_1 P'$ implies $\encoding{P}{}{} \dot\approx_2 \encoding{P'}{}{}$;} ~
(2) Divergence-reflecting. \emph{If $\encoding{P}{}{}$ diverges, so does $P$.}
%\end{enumerate}
\end{definition}

Adequacy (1a) obviously entails weak adequacy (1b), since $\approx_2$ allows more processes in the target model $\mathcal{L}_2$ (thus more variety of contexts). Yet weak adequacy is still useful because it may be too strong if one requires the encoding process to be compatible with all kinds of contexts in the target model. For instance, in order to achieve first-order interactions in a higher-order target model, it appears quite demanding to require equivalence under all kinds of input because the target higher-order model may have more powerful computation ability (so it can feed a much involved input). So sometimes using limited contexts in the target model, e.g., the encoding processes themselves, may be sufficient to meet the goal of the encoding.

It is worthwhile to note that the criteria are short of those for operational correspondence. Although generally soundness and completeness may appear not very informative in absence of operational correspondence \cite{GN16,Par16}, we make this choice in this work out of the following consideration. The criteria for operational correspondence used in \cite{Gor08a}, though proven useful in many models, appear not quite convenient when discussing encodings into higher-order models \cite{LPSS10}, since (for example) the case of input can be hard to comply with the criteria due to the increased complexity in the environment (namely in the context of the target higher-order model). After all, here it seems more important to have the soundness and completeness properties eventually (w.r.t. the canonical bisimulation equivalences in the source and target models, not arbitrary ones as discussed in \cite{Par16}), likely based on a different manner of operational correspondence. Notwithstanding, we will discuss the operational correspondence of the encoding in this work, and moreover, as will be seen, the concrete operational correspondence therein somehow strengthens the criteria of operational correspondence (and related concepts) used in \cite{Gor08a,LPSS10} (in \cite{Gor08a} the criteria are not action-labelled and thus the notion of success sensitiveness is contrived; in \cite{LPSS10} a labelled variant criteria is posited to its purpose). %success sensitiveness
Beyond the scope of this paper, it would be intriguing to examine the possibility of formally pinning down some variant criteria of operational correspondence having vantage for higher-order (process) models.

%Three possible motiv: 1) encoding used to achieve correct FO interactions in HO, so limited contexts may be ok; 2) too demanding to require all kinds of input because the target (HO) model has somewhat more powerful computation ability;  3) in practice usually not all possible objects are communicated (rather some typical ones).
















%---------------------------
% Local Variables:
% mode: LaTeX
% TeX-master: "main.tex"
% End:

\section{Encoding \FOPi\ into \HOPiDd}\label{s:encoding}
We show that \FOPi\ can be encoded in \HOPiDd.
%\bc{\scriptsize We first recall the notations:
%\[
%\begin{array}{ll}
%\lrangle{X}P \,,\, A\lrangle{Q} \quad & \mbox{abstraction and application on processes} \\
%\lrangle{x}P \,,\, A\lrangle{b} \quad & \mbox{abstraction and application on names}
%\end{array}
%\]
%}

\subsection{The encoding}
%In light of related work in the field \cite{San92, Tho93, LPSS10, XYL14},
%It is still not clear whether the encoding can be adapted (or rewritten) into one that only uses name parameterization.
We have the encoding defined as below (being homomorphism on the other operators, except that the encoding of input guarded replication is defined as $\enc{!m(x).P}\DEF !\enc{m(x).P}$).
\[
\begin{array} {lrcll}
 & \enc{m(x).P} & \DEF & m(Y).Y\lrangle{\lrangle{x}\enc{P}} & \\ %\quad \quad (Y \mbox{ is fresh}) \\
 & \enc{\overline{m}n.Q} &\DEF & \overline{m}[\lrangle{Z}(Z\lrangle{n})].\enc{Q} &  \\ %\quad \quad (Z \mbox{ is fresh}) %\\
%(\mbox{\rc{asynchronous}}) & \enc{\overline{m}n} &\DEF & \overline{m}[\lrangle{Z}(Z\lrangle{n})] & \quad \quad (Z \mbox{ is fresh})
\end{array}
\]
The encoding above uses both name parameterization and process parameterization. %, and seems correct. 
%We note that for input guarded replication, the encoding is. 
Typically one can assume that $Y$ and $Z$ are fresh for simplicity, but this is not essential, because these variables are bound and can be $\alpha$-converted whenever necessary, and moreover the encoded first-order process does not have higher-order variables.
Specifically, the encoding of an output `transmits' the name to be sent (i.e., $n$) in terms of a process parameterization (i.e., $\lrangle{Z}(Z\lrangle{n})$) that, once being received by the encoding of an input, is instantiated by a name-parameterized term (i.e., $\lrangle{x}\enc{P}$), which then can apply $n$ on $x$ in the encoding of $P$, thus fulfilling `name-passing'. Below we give an example. Suppose $P\DEF (c)(a(x).\overline{x}c.P_1)$ and $Q\DEF (d)(\overline{a}d.d(y).Q_1)$. So
\[
\begin{array}{lcl}
P\para Q &\st{\tau}& (d)((c)(\overline{d}c.P_1\fosub{d}{x})\para d(y).Q_1) \\
&\st{\tau}& (dc)(P_1\fosub{d}{x}\para Q_1\fosub{c}{y})
\end{array}
\] The encoding and interactions of $\enc{P\para Q}$ are as below. For clarity, we use \textbf{bold font} to indicate the evolving part during a communication.
\[
\begin{array}{lcl}
\enc{P\para Q} &\equiv& (c)(a(Y).Y\lrangle{\lrangle{x}\enc{\overline{x}c.P_1}}) \,\para\, (d)(\overline{a}[\lrangle{Z}(Z\lrangle{d})].\enc{d(y).Q_1}) \\
 &\st{\tau}& (d)\big((c)(\bm{(\lrangle{Z}(Z\lrangle{d}))\lrangle{\lrangle{x}\enc{\overline{x}c.P_1}}}) \,\para\, \enc{d(y).Q_1} \big) \\
 &\equiv& (d)\big((c)( \bm{\enc{\overline{x}c.P_1}\fosub{d}{x}}) \,\para\, \enc{d(y).Q_1} \big) \\
 &\equiv& (d)\big((c)( \bm{ (\overline{x}[\lrangle{Z}(Z\lrangle{c})].\enc{P_1}) \fosub{d}{x}}) \,\para\, d(Y).Y\lrangle{\lrangle{y}\enc{Q_1}}  \big) \\
 &\equiv& (d)\big((c)( \bm{ (\overline{d}[\lrangle{Z}(Z\lrangle{c})].\enc{P_1}\fosub{d}{x}) }) \,\para\, d(Y).Y\lrangle{\lrangle{y}\enc{Q_1}}  \big) \\
 &\st{\tau}&  (dc)\big(\enc{P_1}\fosub{d}{x} \,\para\, \bm{(\lrangle{Z}(Z\lrangle{c}))(\lrangle{\lrangle{y}\enc{Q_1}})} \big) \\
 &\equiv&  (dc)\big(\enc{P_1}\fosub{d}{x} \,\para\, \enc{Q_1}\fosub{c}{y} \big) \\
 &\equiv&  (dc)\big(\enc{P_1\fosub{d}{x}} \,\para\, \enc{Q_1\fosub{c}{y}} \big)
\end{array}
\]


Apparently the encoding is compositional, preserves the (free) names, and moreover divergence-reflecting (since the encoding does not introduce any extra internal action), as stated in the follow-up lemma whose proof is a standard induction. %We stress that the side conditions in the encoding is not essential.
\begin{lemma}
Assume $P$ is a \FOPi\ process. The encoding above from \FOPi\ to \HOPiDd\ is compositional and divergence-reflecting; moreover $\enc{P}\fosub{n}{m} \equiv \enc{P\fosub{n}{m}}$.
\end{lemma}
%\sep

\tdup{
\bc{TODO (across \HOPid\ and \FOPi):}
\begin{itemize}
\item \bc{Operational correspondence}. \bc{$\checkmark$}

\item \bc{Soundness/weak adequacy}.  \bc{$\checkmark$}
(Ground bisimilarity: $\WGB$; Strong ground bisimilarity: $\SGB$; Context bisimilarity: $\WCB$; Strong context bisimilarity: $\SCB$)
\[
P\WGB Q \mbox{~~ implies ~~} \enc{P} \WCB \enc{Q}
\]
\stress{(Possibly use the result in Section \ref{s:normal} on normal bisimulation.)}
\end{itemize}
}

\subsection{Operational correspondence}
We have the following properties clarifying the correspondence of actions before and after the encoding.
To delineate some case of the operational correspondence in terms of certain special input, i.e., a trigger, we define $\triggerD \DEF \lrangle{Z}\overline{m}Z$ in which $m$ is assumed to be fresh (it will also be used in Section \ref{s:normal}, but here simply allows for more flexible characterization of the operational correspondence). We note that sometimes existential quantification is omitted when it is clear from context.
\begin{lemma}\label{l:opcor}
Suppose $P$ is a \FOPi\ process.
%\begin{enumerate}
%\item 
(1) If $P \st{a(b)} P'$, then $\enc{P} \st{a(\lrangle{Z}(Z\lrangle{b}))} T$ and $T\SCB \enc{P'}$; ~
\tdup{
\item (\rc{useful?seems not! to remove!}) If $P \st{a(b)} P'$, then for some fresh $m$, $\enc{P} \st{a(\lrangle{Z}\overline{m}[Z\lrangle{b}])} T$ and $(m)(T \para !m(Y).Y) \WCB \enc{P'}$;
}
%\rc{Use equation (\ref{eqn-fact-hopiDd}) to deal with $Z\lrangle{b}$ in $\lrangle{Z}\overline{m}[Z\lrangle{b}]$ ??}
%\item 
(2) If $P \st{a(b)} P'$, then $\enc{P} \st{a(\triggerD)} T$ and $(m)(T \para !m(Y).Y\lrangle{b}) \WCB \enc{P'}$; ~
%\stress{ (relating $T$ and $\enc{P'}$) };\\
%\item 
(3) If $P \st{\overline{a}b} P'$, then $\enc{P} \st{\overline{a}[\lrangle{Z}(Z\lrangle{b})]} T$ and $T\SCB \enc{P'}$; ~
%\item 
(4) If $P \st{\overline{a}(b)} P'$, then $\enc{P} \st{(b)\overline{a}[\lrangle{Z}(Z\lrangle{b})]} T$ and $T\SCB \enc{P'}$; ~
%\item 
(5) If $P \st{\tau} P'$, then $\enc{P} \st{\tau} T$ and $T\SCB \enc{P'}$.
%\end{enumerate}
\end{lemma}

The converse is as below.
\begin{lemma}\label{l:opcor-conv}
Suppose $P$ is a \FOPi\ process.
%\begin{enumerate}
%\item 
(1) If $\enc{P} \st{a(\lrangle{Z}(Z\lrangle{b}))} T$, then $P \st{a(b)} P'$ and $T\SCB \enc{P'}$; ~
\tdup{
\item (\rc{useful? seems not! to remove!}) If for some fresh $m$, $\enc{P} \st{a(\lrangle{Z}\overline{m}[Z\lrangle{b}])} T$, then $P \st{a(b)} P'$ and $(m)(T \para !m(Y).Y) \WCB \enc{P'}$;
}
%\item 
(2) If $\enc{P} \st{a(\triggerD)} T$, then $P \st{a(b)} P'$ and $(m)(T \para !m(Y).Y\lrangle{b}) \WCB \enc{P'}$; ~
%\stress{ (relating $T$ and $\enc{P'}$) };\\
%\item 
(3) If $\enc{P} \st{\overline{a}[\lrangle{Z}(Z\lrangle{b})]} T$, then $P \st{\overline{a}b} P'$ and $T\SCB \enc{P'}$; ~
%\item 
(4) If $\enc{P} \st{(b)\overline{a}[\lrangle{Z}(Z\lrangle{b})]} T$, then $P \st{\overline{a}(b)} P'$ and $T\SCB \enc{P'}$; ~
%\item 
(5) If $\enc{P} \st{\tau} T$, then $P \st{\tau} P'$ and $T\SCB \enc{P'}$.
%\end{enumerate}
\end{lemma}

Lemma \ref{l:opcor} and Lemma \ref{l:opcor-conv} can be proven in a similar fashion 
\iftoggle{appendixing}{%
  %using appendixing
 (we give details in Appendix \ref{a:proofs-encoding}),
}{%
  %no appendixing
 (details can be found in \cite{Xu16app}),
}
and moreover be lifted to the weak situation. That is, if one replaces strong transitions (single arrows) with weak transitions (double arrows), % and $\SCB$ with $\WCB$,
the results still hold ($\SCB$ retains because the encoding does not bring any extra internal action); see \cite{San92,SW01a} for a reference.
We will however simply refer to these two lemmas in related discussions.



\subsection{Soundness}
In this section, we discuss the soundness of the encoding.
First of all, it is unfortunate that the soundness of the encoding is not true. To see this, take the processes $R_1$ and $R_2$ below. We recall that the CCS-like prefixes are defined as usual, i.e., $a.P\DEF a(x).P$ ($x\notin \n{P}$), $\overline{a}.P\DEF (c)\overline{a}c.P$ ($c\notin \n{P}$); sometimes we trim the trailing $0$, e.g., $a$ stands for $a.0$ and $\overline{a}$ for $\overline{a}.0$.
\[
\begin{array}{lcllcl}
R_1 &\DEF& (b)(a.\overline{b} \para b.\overline{c}) &\qquad\quad R_2 &\DEF& (b)(a.\overline{b} \para b.\overline{c} \para b.\overline{c})
\end{array}
\]
Obviously, $R_1$ and $R_2$ are ground bisimilar.
%(The synchronizations are defined as usual. )
Now we examine their encodings. %m(Y).Y\lrangle{\lrangle{x}\enc{P}}   ;      \overline{m}[\lrangle{Z}(Z\lrangle{n})].\enc{Q}
\[
\begin{array}{lcl}
\enc{R_1} &\equiv& (b)(a(Y).Y\lrangle{\lrangle{x}\enc{\overline{b}}} \para b(Y).Y\lrangle{\lrangle{x}\enc{\overline{c}}}) \\
\enc{R_2} &\equiv& (b)(a(Y).Y\lrangle{\lrangle{x}\enc{\overline{b}}} \para b(Y).Y\lrangle{\lrangle{x}\enc{\overline{c}}} \para b(Y).Y\lrangle{\lrangle{x}\enc{\overline{c}}})
\end{array}
\]

We show that $\enc{R_1}$ and $\enc{R_2}$ are not context bisimilar. Define $T\DEF (m)(\overline{a}[\lrangle{Z}\overline{m}Z] \para m(X).(X\lrangle{d} \para X\lrangle{d})$.
Then $(a)(\enc{R_1}\para T)$ and $(a)(\enc{R_2}\para T)$ can be distinguished. The latter can fire two output on $c$, whereas the former cannot, as shown below.
\[
\begin{array}{lrl}
 &(a)(\enc{R_1}\para T) \quad \st{\tau}\SCB& (m)((b)(\overline{m}[\lrangle{x}\enc{\overline{b}}] \para b(Y).Y\lrangle{\lrangle{x}\enc{\overline{c}}}) \para m(X).(X\lrangle{d} \para X\lrangle{d})) \\
&\st{\tau}\SCB& (b)(b(Y).Y\lrangle{\lrangle{x}\enc{\overline{c}}} \para \enc{\overline{b}} \para \enc{\overline{b}}) \\
&\equiv& (b)(b(Y).Y\lrangle{\lrangle{x}\enc{\overline{c}}} \para (e)\overline{b}[\lrangle{Z}(Z\lrangle{e})] \para \enc{\overline{b}}) \\
&\st{\tau}\SCB& (b)(\enc{\overline{c}} \para \enc{\overline{b}}) \\
&\equiv& (b)((f)\overline{c}[\lrangle{Z}(Z\lrangle{f})] \para \enc{\overline{b}}) \\
&\st{(f)\overline{c}[\lrangle{Z}(Z\lrangle{f})]}\SCB& 0 %\\\\
\end{array}
\]
\[
\begin{array}{lrl}
 &(a)(\enc{R_2}\para T) \quad \st{\tau}\SCB& (m)((b)(\overline{m}[\lrangle{x}\enc{\overline{b}}] \para b(Y).Y\lrangle{\lrangle{x}\enc{\overline{c}}} \para b(Y).Y\lrangle{\lrangle{x}\enc{\overline{c}}}) \para m(X).(X\lrangle{d} \para X\lrangle{d})) \\
&\st{\tau}\SCB& (b)(b(Y).Y\lrangle{\lrangle{x}\enc{\overline{c}}}\para b(Y).Y\lrangle{\lrangle{x}\enc{\overline{c}}} \para \enc{\overline{b}} \para \enc{\overline{b}}) \\
&\equiv& (b)(b(Y).Y\lrangle{\lrangle{x}\enc{\overline{c}}}\para b(Y).Y\lrangle{\lrangle{x}\enc{\overline{c}}} \para (e)\overline{b}[\lrangle{Z}(Z\lrangle{e})] \para (e)\overline{b}[\lrangle{Z}(Z\lrangle{e})]) \\
&\st{\tau}\st{\tau}\SCB& \enc{\overline{c}} \para \enc{\overline{c}} \\
&\equiv& (f)\overline{c}[\lrangle{Z}(Z\lrangle{f})] \para (f)\overline{c}[\lrangle{Z}(Z\lrangle{f})] \\
&\st{(f)\overline{c}[\lrangle{Z}(Z\lrangle{f})]}\SCB& (f)\overline{c}[\lrangle{Z}(Z\lrangle{f})] \\
&\st{(f)\overline{c}[\lrangle{Z}(Z\lrangle{f})]}\SCB& 0
\end{array}
\]


Intuitively, the reason general soundness does not hold is that context bisimulation is somewhat more discriminating in the target higher-order calculus, which can have more flexibility when dealing with blocks of processes in presence of parameterization (e.g., some subprocess can be sent as needed). This is however beyond the capability of a first-order process.

In spite of the falsity of soundness in general, we can have a somewhat weaker yet still sensible soundness. Remember that our main goal is to achieve first-order concurrency in the higher-order model, so maybe we do not need to be so demanding when coping with the encodings of  first-order processes, that is, when testing an encoding process with an input, one can focus on those representing a name instead of a general one. Then it is expected that soundness will hold under this assumption. Fortunately, this is indeed true.

%The choice of this variant soundness is further motivated below.(opt.) .

%\sepp

%The soundness of the encoding is stated in the followig lemma.
We have the following lemma stating the weak soundness of the encoding. Recall that $\WWCB$ is the $\WCB$ restricted to the image of the encoding (i.e., the processes in the target model that have reverse-image w.r.t. the encoding).
\begin{lemma}\label{l:soundness}
Suppose $P$ is a \FOPi\ process. Then $P\WGB Q$ implies $\enc{P} \,\WWCB\, \enc{Q}$.
\end{lemma}

\tdup{
$\myxcancel{
\fbox{
\begin{minipage}{8cm}
\rc{Use normal bisimulation for \HOPiDd\ to prove soundness, and original context bisimulation for completeness (?); Notice normal bisimulation for \HOPiDd\ inherits that for \HOPiD\ (type of input can be name-parameterization or process-parameterization.)}
\end{minipage}
}}$
}

\begin{proof}
\tdup{
\stress{ \scriptsize DONE! $\bcancel{\mbox{TODO: ,}}$
to deal with input and output, USE ``up-to context" technique (\cite{SW01a}, page 80-92; to confirm that it can be extended to higher-order paradigm (e.g., the case a context hole appears beneath an input or name-abstraction (seems ok)) !!; maybe also notice (e.g.) \cite{BPPR15}). }

\stress{ \scriptsize NOTICE that in input/output bisimulation (to prove $\enc{P}\WNB \enc{Q}$ or $\enc{P}\WCB \enc{Q}$):
\begin{itemize}
\item (by going through the FO processes $P,Q$) $\enc{P}\st{a(\triggerD)}$ must be able to be matched by $\enc{Q}\st{a(\triggerD)}$;
\item (by going through the FO processes $P,Q$)  $\enc{P}\st{\overline{a}[\lrangle{Z}(Z\lrangle{b})]}$ must be able to be matched by $\enc{Q}\st{\overline{a}[\lrangle{Z}(Z\lrangle{b})]}$.
\end{itemize}}
}

%\sep

We show that $\mathcal{R}\DEF \{(\enc{P},\enc{Q}) \,|\, P\WGB Q\} \cup \WWCB$ is a context bisimulation up-to context and $\SCB$ (we refer the reader to, for example,  \cite{SW01a,BPPR15} and the references therein for the up-to proof technique for establishing bisimulations; we note that using $\SCB$ here is sufficient since it is stronger than $\WSCB$, i.e., $\SCB$ restricted to the image of the encoding). %\stress{\large ***About the ``up-to context" technique (\cite{SW01a}, page 80-92; to confirm that it can be extended to higher-order paradigm (e.g., the case a context hole appears beneath an input or name-abstraction. (seems ok)) !!; maybe also notice (e.g.) \cite{BPPR15}).***}

Suppose $\enc{P}\,\mathcal{R}\, \enc{Q}$. There are several cases, where  Lemma \ref{l:opcor} and Lemma \ref{l:opcor-conv} %(as well as their weak versions)
play an important part.
\begin{itemize}
\item $\enc{P}\wt{a(\lrangle{Z}(Z\lrangle{b}))} T$.
By Lemma \ref{l:opcor-conv}, $P \wt{a(b)} P'$ and $T\SCB \enc{P'}$. Because $P\WGB Q$, we know that $Q \wt{a(b)} Q'$ ~ $\WGB P'$ and thus $\enc{P'} \,\mathcal{R}\, \enc{Q'}$. Then by Lemma \ref{l:opcor}, $\enc{Q} \wt{a(\lrangle{Z}(Z\lrangle{b}))} T'$ and $T'\SCB \enc{Q'}$. So we have $T \SCB \enc{P'} \,\mathcal{R}\, \enc{Q'} \SCB T'$.

\tdup{
$\myxcancel{
\fbox{
\begin{minipage}{14cm}
$\enc{P}\wt{a(\triggerD)} T$. \stress{todo ~~ (\& see NOTICE just above: use Lemma \ref{l:opcor},~\ref{l:opcor-conv}(3) and up-to context}) \\
By Lemma \ref{l:opcor-conv}, $P \wt{a(b)} P'$ and $(m)(T \para !m(Y).Y\lrangle{b}) \WCB \enc{P'}$. Because $P\WGB Q$, we know that $Q \wt{a(b)} Q' \WGB P'$ and thus $\enc{P'} \,\mathcal{R}\, \enc{Q'}$. Then by Lemma \ref{l:opcor}, $\enc{Q} \wt{a(\triggerD)} T'$ and $(m)(T' \para !m(Y).Y\lrangle{b}) \WCB \enc{Q'}$. So  {\large \stress{??? (some informal discussion in ``q.txt")}} \\ %$\alpha$
\stress{Input is the crux!! Try ...
A compromise is to confine to $\enc{}(\FOPi)$
(i.e., the image of the encoding that communicate only $\lrangle{Z}(Z\lrangle{b})$)}; \\
\stress{under this constraint the argument for input would be straightforward (see the box on the right).}\\
Three possible motiv: 1) encoding used to achieve correct FO interactions in HO, so limited contexts may be ok; 2) too demanding to require all kinds of input because the target (HO) model has somewhat more powerful computation ability;  3) in practice usually not all possible objects are communicated (rather some typical ones).
\end{minipage}
}}$
}%\tdup
%\fbox{
%\begin{minipage}{7cm}
%\begin{itemize}
%\item $\enc{P}\wt{a(\lrangle{Z}(Z\lrangle{b}))} T$.
%By Lemma \ref{l:opcor-conv}, $P \wt{a(b)} P'$ and $T\SCB \enc{P'}$. Because $P\WGB Q$, we know that $Q \wt{a(b)} Q' \WGB P'$ and thus $\enc{P'} \,\mathcal{R}\, \enc{Q'}$. Then by Lemma \ref{l:opcor}, $\enc{Q} \wt{a(\lrangle{Z}(Z\lrangle{b}))} T'$ and $T'\SCB \enc{Q'}$. So we have $T \SCB \enc{P'} \,\mathcal{R}\, \enc{Q'} \SCB T'$.
%\end{itemize}
%\end{minipage}
%}
%\sep\sep

\tdup{
$\myxcancel{
\fbox{
\begin{minipage}{14cm}
\rc{Some more discussion: tackle general input directly?}\\
{First let us consider a special case, i.e., the input is $\lrangle{Z}Z\lrangle{b}$, somewhat the encoding of a transmitted name.
%(replace the current discussion with one on this special case)
By Lemma \ref{l:opcor-conv}, that $\enc{P}\wt{a(\lrangle{Z}(Z\lrangle{b}))} T_1$ implies that $P \wt{a(b)} P_1$ and $T_1\SCB \enc{P_1}$. Because $P\WGB Q$, we know that $Q \wt{a(b)} Q_1 \WGB P_1$ and thus $\enc{P_1} \,\mathcal{R}\, \enc{Q_1}$. Then by Lemma \ref{l:opcor}, $\enc{Q} \wt{a(\lrangle{Z}(Z\lrangle{b}))} T_2$ and $T_2\SCB \enc{Q_1}$. So we have $T_1 \SCB \enc{P_1} \,\mathcal{R}\, \enc{Q_1} \SCB T_2$. \\
Now consider the general input $A$, which basically should take the form $\lrangle{Z}F[Z\lrangle{b}]$ for some context $F$, so as to make the applications happen in a correct manner (otherwise the discussion would be similar, e.g., $Z$ does not appear in $F$ or is not fed with a name).
%(now do the input and use the special case and up-to context to finish the simulation)
Say $\enc{P}\wt{a(\lrangle{Z}F[Z\lrangle{b}])} T$. Then we know from the special case above that $\enc{Q}\wt{a(\lrangle{Z}F[Z\lrangle{b}])} T'$.
\bc{The problem here is how to relate $T$ with $T_1$ (and $T'$ with $T_2$)}.
Specifically, we know $T_1\equiv G[(\lrangle{x}\enc{R})\lrangle{b}] \equiv G[\enc{R}\fosub{b}{x}]$ for some context $G$ and $\lrangle{x}R$. Then $T\equiv G[F[(\lrangle{x}\enc{R})\lrangle{b}]] \equiv G[F[\enc{R}\fosub{b}{x}]]$. Similarly, we have $T_2\equiv H[(\lrangle{x}\enc{R'})\lrangle{b}] \equiv H[\enc{R'}\fosub{b}{x}]$ for some context $H$ and $\lrangle{x}R'$, and $T'\equiv H[F[(\lrangle{x}\enc{R'})\lrangle{b}]] \equiv H[F[\enc{R'}\fosub{b}{x}]]$. The situation is depicted below.
\[
\xymatrix{
  T \ar@{}[r]|-{\equiv} & G[F[\enc{R}\fosub{b}{x}]] \ar@{.}[rr]|-{?}\ar@{.}[d]|-{?}  &  & H[F[\enc{R'}\fosub{b}{x}]] \ar@{.}[d]|-{?} \ar@{}[r]|-{\equiv}  & T'  \\
  T_1 \ar@{}[r]|-{\equiv} & G[\enc{R}\fosub{b}{x}] \ar@{}[rr]|-{ \SCB\, \enc{P_1}\,\mathcal{R}\,\enc{Q_1}\,\SCB}  & &  H[\enc{R'}\fosub{b}{x}] \ar@{}[r]|-{\equiv} & T_2
}
\]
\rc{How can we proceed? Use normal bisimulation, we can set $F$ as $\overline{m}[\cdot]$. But then how? }
}
\end{minipage}
}}$
}%\tdup


%\sepp\sepp

\item $\enc{P}\wt{\overline{a}[(b)\lrangle{Z}(Z\lrangle{b})]} T$. %\stress{\scriptsize done! ~~(\& note NOTICE just above: use Lemma \ref{l:opcor},~\ref{l:opcor-conv}(5) and up-to context}) \\
By Lemma \ref{l:opcor-conv}, $P \wt{\overline{a}(b)} P'$ and $T\SCB \enc{P'}$. Because $P\WGB Q$, we know that $Q \wt{\overline{a}(b)} Q' \WGB P'$ and thus $\enc{P'} \,\mathcal{R}\, \enc{Q'}$. Then by Lemma \ref{l:opcor}, $\enc{Q} \st{(b)\overline{a}[\lrangle{Z}(Z\lrangle{b})]} T'$ and $T'\SCB \enc{Q'}$. Consider the following pair
\[
(b)(T\para E[A]) \;\quad,\quad\;  (b)(T'\para E[A])
\] in which $b\notin \fn{E[X]}$ %$E[X]\DEF !\bc{m(Z).X\lrangle{Z}}$ (\stress{\small do not set this if using context bisimulation instead of normal bisimulation})
and $A\DEF \lrangle{Z}(Z\lrangle{b})$. So
\[
(b)(T\para E[A]) \SCB (b)(\enc{P'}\para E[A]) \;\quad,\quad\; (b)(\enc{Q'}\para E[A]) \SCB (b)(T'\para E[A])
\] By setting a context $C\DEF (b)([\cdot]\para E[A])$, we have the following pair in which $\enc{P'} \,\mathcal{R}\, \enc{Q'}$.
\[
C[\enc{P'}] \;\quad,\quad\; C[\enc{Q'}]
\] This suffices to close this case in terms of the up-to context requirement.

\item $\enc{P}\wt{\overline{a}[\lrangle{Z}(Z\lrangle{b})]} T$. %\stress{\scriptsize done! ~~ (\& note NOTICE just above: use Lemma \ref{l:opcor},~\ref{l:opcor-conv}(4) and up-to context}) \\
This case is similar to the last case.

\item $\enc{P}\wt{\tau} T$. By Lemma \ref{l:opcor-conv}, $P \wt{\tau} P'$ and $T\SCB \enc{P'}$. From $P\WGB Q$, we know $Q \wt{} Q'\WGB P'$ and thus $\enc{P'} \,\mathcal{R}\, \enc{Q'}$. Then by Lemma \ref{l:opcor}, $\enc{Q} \wt{} T'$ and $T'\SCB \enc{Q'}$. So we have $T\SCB \enc{P'} \,\mathcal{R}\, \enc{Q'}\SCB T'$.
\end{itemize}

\end{proof}


\subsection{Completeness}
The completeness of the encoding is stated in the lemma below. {We note that completeness is true even if we do not constrain the domain to be the image of the encoded \FOPi\ processes. }
\begin{lemma}\label{l:completeness}
Suppose $P$ is a \FOPi\ process. Then $\enc{P} \WCB \enc{Q}$ implies $P\WGB Q$.
\end{lemma}
\begin{proof}
\tdup{
\stress{ \scriptsize DONE! $\bcancel{\mbox{TODO: if necessary,}}$ \\
(1) (for bound output in particular) maybe Use ``local bisimulation" on the \FOPi\ side and context bisimulation on \HOPiDd\ side, to analyze using ``context surjection" (e.g., \cite{Fu05b,Xu12}). \\
(2) maybe use certain up-to technique on the FO side.}
}

\tdup{
\stress{\scriptsize NOTICE that in input/output bisimulation (to prove $P\WGB Q$):
\begin{itemize}
%\item to prove $P$ and $Q$ bisimulate on input: (by going through the HO processes $\enc{P},\enc{Q}$) $\enc{P}\st{a(\triggerD)}$ must be able to be matched by $\enc{Q}\st{a(\triggerD)}$;
\item to prove $P$ and $Q$ bisimulate on input: (by going through the HO processes $\enc{P},\enc{Q}$) $\enc{P}\st{a(\lrangle{Z}(Z\lrangle{b}))}$ must be able to be matched by $\enc{Q}\st{a(\lrangle{Z}(Z\lrangle{b}))}$;
\item to prove $P$ and $Q$ bisimulate on free output:(by going through the HO processes $\enc{P},\enc{Q}$)  $\enc{P}\st{\overline{a}[\lrangle{Z}(Z\lrangle{b})]}$ must be able to be matched by $\enc{Q}\st{\overline{a}[\lrangle{Z}(Z\lrangle{b})]}$, because $\enc{Q}$ can only emit such form of processes, and moreover if the matching is $\enc{Q}\st{\overline{a}[\lrangle{Z}(Z\lrangle{c})]}$ then one can design a context to distinguish $\enc{P}$ and $\enc{Q}$;
\item to prove $P$ and $Q$ bisimulate on bound output:(by going through the HO processes $\enc{P},\enc{Q}$)  $\enc{P}\st{\overline{a}[(b)\lrangle{Z}(Z\lrangle{b})]}$ must be able to be matched by $\enc{Q}\st{(b)\overline{a}[\lrangle{Z}(Z\lrangle{b})]}$ (apply $\alpha$-conversion if needed), because $\enc{Q}$ can only emit such form of processes, and moreover if the matching is $\enc{Q}\st{(c)\overline{a}[\lrangle{Z}(Z\lrangle{c})]}$ (or $\enc{Q}\st{\overline{a}[\lrangle{Z}(Z\lrangle{c})]}$) then one can design a context to distinguish between $\enc{P}$ and $\enc{Q}$;
\end{itemize}}
}%\tdup
%\sep

We show that $\mathcal{R}\DEF \{(P,Q) \,|\, \enc{P} \WCB \enc{Q}\} \cup \WGB$ is a local bisimulation. %a (ground)/(local) bisimulation up-to $\SGB$ and \stress{??}. 
Suppose $P\,\mathcal{R}\, Q$. There are several cases. %, where again Lemma \ref{l:opcor} and Lemma \ref{l:opcor-conv} are frequently used.
\begin{itemize}
\item $P\wt{a(b)} P'$. %\stress{see NOTICE just above \& todo: use input clause of context bisimulation, apply Lemma \ref{l:opcor},~\ref{l:opcor-conv}(3)} \\
By Lemma \ref{l:opcor}, $\enc{P} \wt{a(\lrangle{Z}(Z\lrangle{b}))} T$ and $T\SCB \enc{P'}$. Because $\enc{P} \WCB \enc{Q}$, we know that $\enc{Q} \wt{a(\lrangle{Z}(Z\lrangle{b}))} T'\WCB T$. By Lemma \ref{l:opcor-conv}, $Q \wt{a(b)} Q'$ and $T'\SCB \enc{Q'}$. Thus we have $\enc{P'} \SCB T \WCB T'\SCB \enc{Q'}$, so $P'\,\mathcal{R}\, Q'$, which fulfills this case.

\item $P\wt{\overline{a}b} P'$. %\stress{see NOTICE just above \& todo: use free output clause of context bisimulation, apply Lemma \ref{l:opcor},~\ref{l:opcor-conv}(3)} \\
By Lemma \ref{l:opcor}, $\enc{P} \wt{\overline{a}[\lrangle{Z}(Z\lrangle{b})]} T$ and $T\SCB \enc{P'}$. Since $\enc{P} \WCB \enc{Q}$, we know that $\enc{P}$ must be able to be matched by $\enc{Q}\wt{\overline{a}[\lrangle{Z}(Z\lrangle{b})]} T'$, because $\enc{Q}$ can only output such shape of processes, and if the matching is, e.g., $\enc{Q}\wt{\overline{a}[\lrangle{Z}(Z\lrangle{c})]} T''$ then a context can be designed to distinguish between $\enc{P}$ and $\enc{Q}$. So for every $E[X]$, we have $T \para E[\lrangle{Z}(Z\lrangle{b})] \WCB T' \para E[\lrangle{Z}(Z\lrangle{b})]$. By Lemma \ref{l:opcor-conv}, $Q \wt{\overline{a}b} Q'$ and $T'\SCB \enc{Q'}$. So we know
\begin{equation}\label{eq:comp4}
\enc{P'} \para E[\lrangle{Z}(Z\lrangle{b})) \WCB \enc{Q'} \para E[\lrangle{Z}(Z\lrangle{b})]
\end{equation}
We want to show
\begin{equation}\label{eq:comp5}
P' \mathcal{R} Q' \quad \mbox{ that is, }\; \enc{P'} \WCB \enc{Q'}
\end{equation}
By setting $E$ to be $0$ in (\ref{eq:comp4}), we obtain (\ref{eq:comp5}), and thus close this case.

\item $P\wt{\overline{a}(b)} P'$. %\stress{see NOTICE just above \& todo: use bound output clause of context bisimulation, and ``context surjection", apply Lemma \ref{l:opcor},~\ref{l:opcor-conv}(3)} \\
By Lemma \ref{l:opcor}, $\enc{P} \wt{(b)\overline{a}[\lrangle{Z}(Z\lrangle{b})]} T$ and $T\SCB \enc{P'}$. Since $\enc{P} \WCB \enc{Q}$, we know that $\enc{P}$  must be able to be matched by $\enc{Q}\wt{(b)\overline{a}[\lrangle{Z}(Z\lrangle{b})]} T'$ (apply $\alpha$-conversion if needed). This is because $\enc{Q}$ can only emit such form of processes, and moreover if the matching does not have a bound name (e.g., $\enc{Q}\wt{\overline{a}[\lrangle{Z}(Z\lrangle{c})]} T''$) then one can design a context to distinguish $\enc{P}$ and $\enc{Q}$. So for every $E[X]$ s.t. $b\notin \mbox{fn}(E)$, we have $(b)(T \para E[\lrangle{Z}(Z\lrangle{b})]) \WCB (b)(T' \para E[\lrangle{Z}(Z\lrangle{b})])$. By Lemma \ref{l:opcor-conv}, $Q \wt{\overline{a}(b)} Q'$ and $T'\SCB \enc{Q'}$. So we know
\begin{equation}\label{eq:comp1}
(b)(\enc{P'} \para E[\lrangle{Z}(Z\lrangle{b})]) \WCB (b)(\enc{Q'} \para E[\lrangle{Z}(Z\lrangle{b})])
\end{equation}
In terms of local bisimulation \cite{Fu05b,Xu12}, for every \FOPi\ process $R$, we need to show
\begin{equation}\label{eq:comp2}
(b)(P'\para R) \,\mathcal{R}\, (b)(Q'\para R) \quad\mbox{ i.e., }\quad (b)(\enc{P'}\para \enc{R}) \WCB (b)(\enc{Q'} \para \enc{R})
\end{equation}
%That is,
%\begin{equation}\label{eq:comp3}
%(b)(\enc{P'}\para \enc{R}) \WCB (b)(\enc{Q'} \para \enc{R})
%\end{equation}
Comparing equations (\ref{eq:comp1}) and (\ref{eq:comp2}), one can see that the different part is $E[\lrangle{Z}(Z\lrangle{b})]$ and $\enc{R}$. Since the inverse of the encoding is a surjection, if all possible forms of $E$ is itinerated, $\enc{R}$ must be hit somewhere (i.e., some choice of $E$ makes $E[\lrangle{Z}(Z\lrangle{b})]$ and $\enc{R}$ equal). Therefore we infer that (\ref{eq:comp2}) %(and thus (\ref{eq:comp2})) 
is true and thus complete this case.


\item $P\wt{\tau} P'$. By Lemma \ref{l:opcor}, $\enc{P} \wt{\tau} T$ and $T\SCB \enc{P'}$. Because $\enc{P} \WCB \enc{Q}$, we know $\enc{Q} \wt{} T' \WCB T$. Then by Lemma \ref{l:opcor-conv}, $Q \wt{} Q'$ and $T'\SCB \enc{Q'}$. % (notice that the case $\enc{Q}$ is exactly $T'$ is trivial).
So we have $P'\,\mathcal{R}\, Q'$ because $\enc{P'}\SCB T \WCB T' \SCB \enc{Q'}$.
\end{itemize}


\end{proof}






%---------------------------
% Local Variables:
% mode: LaTeX
% TeX-master: "main.tex"
% End:

\section{Another approach of encoding \FOPi\ into \HOPiDd}\label{s:encoding_variant}
In this section, we present another way of encoding \FOPi\ with \HOPiDd. This encoding, from which the encoding in the previous section borrows some inspiration, appears more natural in some sense, though its properties are not clear before this work \footnote{This encoding was suggested by Alan Schmitt during the communication concerning another work.}. 
We note that the discussion of this section, among others, extends the preliminary work \cite{Xu16} and offers more insight into the `first-order' programming capacity of \HOPiDd. 

%\nts{\fbox{move the discussion from section conclusion to here}}
%\nts{\fbox{refactor the following and the appendix it points to (move here)}}

%\rc{
%The encoding of this work is inspired by a seemingly similar one proposed by . 
The encoding, as given below skipping the homomorphic parts, somewhat swaps the roles of input and output and treats $a(x).P$ somehow as $a.\lrangle{x}P$ (like those calculi admitting abstractions and concretions \cite{San92}). For convenience, we reuse the encoding notation $\enc{\,}$ and there should be no confusion in context. 
% \[
% \begin{array} {rcl}
% \enc{a(x).P} & \DEF & \overline{a}[\lrangle{x}\enc{P}] \\
% \enc{\overline{a}b.Q} &\DEF & a(Y).(Y\lrangle{b}\para \enc{Q}) \\%\quad \quad (Y \mbox{ is fresh})
% \end{array}
% \]
\[
\begin{array} {rcl}
\enc{m(x).P} & \DEF & \overline{m}[\lrangle{x}\enc{P}] \\
\enc{\overline{m}n.Q} &\DEF & m(Y).(Y\lrangle{n}\para \enc{Q}) \\%\quad \quad (Y \mbox{ is fresh})
\end{array}
\]

%\[
%\mbox{ \begin{tabular}{l} 
%\rc{[Optional] EXTEND the discussion of this encoding (to some extent) ? in Appendix \ref{appendix:variant_encoding}! } \\
%\rc{(maybe without much proof) (1) operational correspodence: apparently not satisfied directly,}\\
%\rc{(2) (counter)-example: try the original counterexample (Section \ref{s:encoding_soundness}) or its variant, } \\
%\rc{(3) (partial) soundness: if the counterexample still applies (\bc{which seems the case}),} \\
%\rc{then soundness is (similarly) compromised to the image,} \\
%\rc{(4) completeness: may be still depend on the sujection of the reverse mapping of the encoding.}
%\end{tabular}
%}
%\]
\xxrmcolor{The encoding strategy above only employs parameterization on name. One might wonder why not use \HOPiDd\ with process parameterization eliminated. We will comment on this at the end of this section. 
From the angle of achieving first-order interaction, the encoding makes full use of the name-abstraction mechanism and is truly appealing. 
Based on the results in Section \ref{s:encoding}, it is tempting to expect that this encoding have (nearly) the  same properties. %, and 
%}
However at first sight, it appears not to satisfy some usual operational correspondence (say, in \cite{Gor08a} or \cite{LPSS10}), and full abstraction is thus not quite clear. % with similar effort.
This is indeed worthwhile for more investigation. 
}
\xxrmcolor{Below first give an example of the encoding by reusing the instances in the last section, and then move on with the discussion of the encoding.
}

%\nts{\fbox{TODO: Give an example; reuse the example from Section \ref{s:encoding}}}

We recall the two processes $P\DEF (c)(a(x).\overline{x}c.P_1)$ and $Q\DEF (d)(\overline{a}d.d(y).Q_1)$. Their composition has the folliwng transitions.
\[
\begin{array}{lcl}
P\para Q &\st{\tau}& (d)((c)(\overline{d}c.P_1\fosub{d}{x})\para d(y).Q_1) \\
&\st{\tau}& (dc)(P_1\fosub{d}{x}\para Q_1\fosub{c}{y})
\end{array}
\] 
Now the encoding $\enc{P\para Q}$ and its corresponding transitions are given below. %For clarity, we use \textbf{bold font} to indicate the evolving part during a communication.
\[
\begin{array}{lrl}
\enc{P\para Q} &\equiv& (c)(\overline{a}[\lrangle{x}\enc{\overline{x}c.P_1}]) \para (d)(a(Y).(Y\lrangle{d}\para \enc{d(y).Q_1})) \\
&\st{\tau} \equiv& (c)(d)((\lrangle{x}\enc{\overline{x}c.P_1})\lrangle{d}\para \enc{d(y).Q_1}) \\
&\equiv& (c)(d)(\enc{\overline{x}c.P_1})\fosub{d}{x} \para \enc{d(y).Q_1}) \\
&\equiv& (c)(d)((x(Y).(Y\lrangle{c}\para \enc{P_1}))\fosub{d}{x} \para \overline{d}[\lrangle{y}\enc{Q_1}]) \\
&\equiv& (c)(d)(d(Y).(Y\lrangle{c}\para \enc{P_1}\fosub{d}{x}) \para \overline{d}[\lrangle{y}\enc{Q_1}]) \\
&\st{\tau} \equiv& (c)(d)((\lrangle{y}\enc{Q_1})\lrangle{c}\para \enc{P_1}\fosub{d}{x}) \\
&\equiv& (dc)(\enc{P_1}\fosub{d}{x} \para \enc{Q_1})\fosub{c}{y}) \\
&\equiv& (dc)(\enc{P_1\fosub{d}{x}} \para \enc{Q_1}\fosub{c}{y})) \\
% (c)(a(Y).Y\lrangle{\lrangle{x}\enc{\overline{x}c.P_1}}) \,\para\, (d)(\overline{a}[\lrangle{Z}(Z\lrangle{d})].\enc{d(y).Q_1}) \\
%  &\st{\tau}& (d)\big((c)(\bm{(\lrangle{Z}(Z\lrangle{d}))\lrangle{\lrangle{x}\enc{\overline{x}c.P_1}}}) \,\para\, \enc{d(y).Q_1} \big) \\
%  &\equiv& (d)\big((c)( \bm{\enc{\overline{x}c.P_1}\fosub{d}{x}}) \,\para\, \enc{d(y).Q_1} \big) \\
%  &\equiv& (d)\big((c)( \bm{ (\overline{x}[\lrangle{Z}(Z\lrangle{c})].\enc{P_1}) \fosub{d}{x}}) \,\para\, d(Y).Y\lrangle{\lrangle{y}\enc{Q_1}}  \big) \\
%  &\equiv& (d)\big((c)( \bm{ (\overline{d}[\lrangle{Z}(Z\lrangle{c})].\enc{P_1}\fosub{d}{x}) }) \,\para\, d(Y).Y\lrangle{\lrangle{y}\enc{Q_1}}  \big) \\
%  &\st{\tau}&  (dc)\big(\enc{P_1}\fosub{d}{x} \,\para\, \bm{(\lrangle{Z}(Z\lrangle{c}))(\lrangle{\lrangle{y}\enc{Q_1}})} \big) \\
%  &\equiv&  (dc)\big(\enc{P_1}\fosub{d}{x} \,\para\, \enc{Q_1}\fosub{c}{y} \big) \\
%  &\equiv&  (dc)\big(\enc{P_1\fosub{d}{x}} \,\para\, \enc{Q_1\fosub{c}{y}} \big)
\end{array}
\]






%moved from appendix in the previous version
%\section{Proofs for Section \ref{s:conclusion}}\label{appendix:variant_encoding}
% \rc{
% The remainder of this section is devoted to a full analysis of this encoding.
% Details are in Appendix \ref{appendix:variant_encoding}.
% \nts{\fbox{move the appendix here and adjust/refactor the stuff}}
% }

%***************************************************
%NOW in the body, used by "encoding_variant.tex"

\sepp

The remainder of this section is devoted to a full analysis of the encoding above.
%Particularly, we examine the static and dynamic properties of the variant encoding given above. %in Section \ref{s:conclusion}. 
%For convenience, we reuse the encoding notation $\enc{}$ and there should be no confusion in context. 
\xxx{We first note that, similar to the encoding in Section \ref{s:encoding}, 
%Lemma \ref{l:syn-pro-encoding}, 
the encoding comes with compositionality and divergence-reflection, and preserves name substitution (i.e., name invariant) and the structural congruence.}
These properties will be used implicitly in the arguments of this section. We also point out again that since the encoding is compositional, it makes sense to extend the definition of the encoding to be over contexts, with a hole translated to a hole.

%\nts{
%\begin{itemize}
%\item[DONE] Operational correspondence (not directly true at least)
%\item[DONE] Soundness's couterexample (the original counterexample (Section \ref{s:encoding_soundness}) or its light variant seems still applicable)
%\item[DONE] Weak soundness
%\item[DONE] Completeness
%\end{itemize}
%}

\subsection{Operational correspondence}\label{s:opcoores_var_encoding}

%\FROMHERE

It is pronounced that the operational correspondence does not hold in a direct fashion, because the encoding now has two swaps: one is between input and output; the other between synchrony and asynchrony. From a certain perspective, this encoding is more of a conceptual one,
% and it is unequivocally a great step in the right direction, though 
and its analysis is more tricky than the encoding in Section \ref{ss:encoding_def}. 
\xxx{Nevertheless, as for the encoding in Section \ref{ss:encoding_def}, from the operational correspondence (Lemma \ref{l:opcor_var} and Lemma \ref{l:opcor-conv_var}), we can still see that an encoding always evolves into another one (think of $A$ in these two lemmata as an encoding abstracted on a name, i.e., $\lrangle{x}\enc{P}$). So the image of the translation yields a closed (sub)range in the target model.}
%\nts{TODO (ref. Section \ref{s:encoding_operational})}

\begin{lemma}\label{l:opcor_var}
Suppose $P$ is a \FOPi\ process and $A$ is a name abstraction. % (typically, an encoding abstracted on a name, i.e., $\lrangle{x}\enc{P}$).
\begin{enumerate}
%\item If $P \st{a(b)} P'$, then $P\SCB C[a(x).P_1]$ and $P'\SCB C[P_1\fosub{b}{x}]$ for some context $C$ and $P_1$, and \\
%$\enc{P} \st{\overline{a}[\lrangle{x}(\enc{P_1})]} T$ and $T\SCB \enc{C[0]}$;
\item If $P \st{a(b)} P'$, then $P\SGB (\ve{e})C[a(x).P_1]$ and \\
$P'\equiv (\ve{e})C[P_1\fosub{b}{x}] \equiv (\ve{e})(C[0]\para P_1\fosub{b}{x})$ for some \xiv{evaluation} context $C$, $P_1$ and $\ve{e}$ being the bound names shared between $P_1$ and its context, and \\
$\enc{P} \st{(\ve{e})\overline{a}[\lrangle{x}(\enc{P_1})]} T\SSEQV \enc{C[0]}$.
\item If $P \st{\overline{a}b} P'$, then $\enc{P} \st{a(A)} T \SSEQV A\lrangle{b} \para \enc{P'}$.
\item If $P \st{\overline{a}(b)} P'$, then $\enc{P} \st{a(A)} T \SSEQV (b)(A\lrangle{b} \para \enc{P'})$. %\nts{??}
%\item If $P \st{a(b)} P'$, then $\enc{P} \st{a(\lrangle{Z}(Z\lrangle{b}))} T$ and $T\SCB \enc{P'}$; ~
%\item If $P \st{a(b)} P'$, then $\enc{P} \st{a(\triggerD)} T$ and $(m)(T \para !m(Y).Y\lrangle{b}) \WCB \enc{P'}$; ~
%\item If $P \st{\overline{a}b} P'$, then $\enc{P} \st{\overline{a}[\lrangle{Z}(Z\lrangle{b})]} T$ and $T\SCB \enc{P'}$; ~
%\item If $P \st{\overline{a}(b)} P'$, then $\enc{P} \st{(b)\overline{a}[\lrangle{Z}(Z\lrangle{b})]} T$ and $T\SCB \enc{P'}$; ~
\item If $P \st{\tau} P'$, then $\enc{P} \st{\tau} T \SSEQV \enc{P'}$.
\end{enumerate}
\end{lemma}

The converse is as below.
\begin{lemma}\label{l:opcor-conv_var}
Suppose $P$ is a \FOPi\ process and $A$ is a name abstraction.
\begin{enumerate}
%\item If $\enc{P} \st{\overline{a}[\lrangle{x}(\enc{P_1})]} T$, then $P \st{a(b)} P'$, with $P\SCB C[a(x).P_1]$ and $P'\SCB C[P_1\fosub{b}{x}]$ for some context $C$ and $P_1$, and $T\SCB \enc{C[0]}$;
\item If $\enc{P} \st{(\ve{e})\overline{a}[\lrangle{x}(\enc{P_1})]} T$, then $P \st{a(b)} P'$, with $P\SGB (\ve{e})C[a(x).P_1]$ and \\
$P'\equiv (\ve{e})C[P_1\fosub{b}{x}] \equiv (\ve{e})(C[0]\para P_1\fosub{b}{x})$ for some \xiv{evaluation} context $C$ with $\ve{e}$ being the bound names shared between $P_1$ and its context, and $T\SSEQV \enc{C[0]}$.
\item If $\enc{P} \st{a(A)} T$, then
\begin{enumerate}
\item[(i)] $P \st{\overline{a}b} P'$, and $T\SSEQV A\lrangle{b} \para \enc{P'}$, or
\item[(ii)] $P \st{\overline{a}(b)} P'$, and $T\SSEQV (b)(A\lrangle{b} \para \enc{P'})$. %\nts{??}
\end{enumerate}
%\item If $\enc{P} \st{a(\lrangle{Z}(Z\lrangle{b}))} T$, then $P \st{a(b)} P'$ and $T\SCB \enc{P'}$; ~
%\item If $\enc{P} \st{a(\triggerD)} T$, then $P \st{a(b)} P'$ and $(m)(T \para !m(Y).Y\lrangle{b}) \WCB \enc{P'}$; ~
%\item If $\enc{P} \st{\overline{a}[\lrangle{Z}(Z\lrangle{b})]} T$, then $P \st{\overline{a}b} P'$ and $T\SCB \enc{P'}$; ~
%\item If $\enc{P} \st{(b)\overline{a}[\lrangle{Z}(Z\lrangle{b})]} T$, then $P \st{\overline{a}(b)} P'$ and $T\SCB \enc{P'}$; ~
\item If $\enc{P} \st{\tau} T$, then $P \st{\tau} P'$ and $T\SSEQV \enc{P'}$.
\end{enumerate}
\end{lemma}

Lemma \ref{l:opcor_var} and Lemma \ref{l:opcor-conv_var} can be similarly proven, \xiv{and we prove the former in Appendix \ref{a:proofs-corres-var-encoding}}. 
We note that these lemmas can be lifted to the weak case, as for the encoding in Section \ref{ss:encoding_def}.
\xxxx{
We also notice that the clause (1) of these lemmas uses $\SGB$, instead of $\equiv$, to deal with $P$ because it is needed in discussing the case for replication.
}



\subsection{Soundness}
We reuse the \ground bisimilar processes $R_1$ and $R_2$ in Section \ref{s:encoding_soundness}, and see what happens to their new encodings. The CCS-like prefixes are defines as there.
\[
\begin{array}{lcllcl}
R_1 &\DEF& (b)(a.\overline{b} \para b.\overline{c}) &\qquad\quad R_2 &\DEF& (b)(a.\overline{b} \para b.\overline{c} \para b.\overline{c})
\end{array}
\]
%(The synchronizations are defined as usual. )
Now we examine their encodings. %\overline{a}[\lrangle{x}\enc{P}]   ;      a(Y).(Y\lrangle{b}\para \enc{Q})
\[
\begin{array}{lcl}
\enc{R_1} &\equiv& (b)(\overline{a}[\lrangle{x}\enc{\overline{b}}] \para \overline{b}[\lrangle{x}\enc{\overline{c}}]) \\
\enc{R_2} &\equiv& (b)(\overline{a}[\lrangle{x}\enc{\overline{b}}] \para \overline{b}[\lrangle{x}\enc{\overline{c}}] \para \overline{b}[\lrangle{x}\enc{\overline{c}}]) \\
\mbox{in which} && \\
\enc{\overline{b}} &\equiv& (f)(b(Y).(Y\lrangle{f}\para 0)) \\
\enc{\overline{c}} &\equiv& (g)(c(Z).(Z\lrangle{g}\para 0))
\end{array}
\] Take $T\DEF a(X).(X\lrangle{d} \para X\lrangle{d})$.
Then $(a)(\enc{R_1}\para T)$ and $(a)(\enc{R_2}\para T)$ can be distinguished, as shown below. This demonstrates that $\enc{R_1}$ and $\enc{R_2}$ are not context bisimilar. Particularly, the latter can do two inputs on $c$, while the former cannot.
\[
\begin{array}{lrl}
 &(a)(\enc{R_1}\para T) \quad \st{\tau}\SCB& (ab)(\enc{\overline{b}} \para \enc{\overline{b}} \para \overline{b}[\lrangle{x}\enc{\overline{c}}]) \\
&\equiv& (ab)((f)(b(Y).(Y\lrangle{f}\para 0)) \para \enc{\overline{b}} \para \overline{b}[\lrangle{x}\enc{\overline{c}}]) \\
&\st{\tau}\SCB& (b)(\enc{\overline{c}} \para \enc{\overline{b}}) \\
&\equiv& (b)((g)(c(Z).(Z\lrangle{g}\para 0)) \para \enc{\overline{b}}) \\
&\st{c(\lrangle{z}0)}\SCB& 0 \\\\ %\\\\
%\end{array}
%\]
%\[
%\begin{array}{lrl}
 &(a)(\enc{R_2}\para T) \quad \st{\tau}\SCB& (ab)(\enc{\overline{b}} \para \enc{\overline{b}} \para \overline{b}[\lrangle{x}\enc{\overline{c}}] \para \para \overline{b}[\lrangle{x}\enc{\overline{c}}]) \\
&\equiv& (ab)((f)(b(Y).(Y\lrangle{f}\para 0)) \para (f)(b(Y).(Y\lrangle{f}\para 0))  \\
& & \qquad\qquad\qquad\qquad\qquad\,\,\,\, \para \overline{b}[\lrangle{x}\enc{\overline{c}}] \para \overline{b}[\lrangle{x}\enc{\overline{c}}]) \\
&\st{\tau}\st{\tau}\SCB& \enc{\overline{c}} \para \enc{\overline{c}} \\
&\equiv& (g)(c(Z).(Z\lrangle{g}\para 0)) \para (g)(c(Z).(Z\lrangle{g}\para 0)) \\
&\st{c(\lrangle{z}0)}\SCB& (g)(c(Z).(Z\lrangle{g}\para 0)) \\
&\st{c(\lrangle{z}0)}\SCB& 0
\end{array}
\]

\subsubsection{Weak soundness}
Short of soundness, one might expect that the encoding at least has weak soundness, {i.e., sound w.r.t. $\WWCB$, as defined in Section \ref{s:preliminary} and used in Section \ref{s:encoding_soundness}}. %(We recall that weak soundness means the soundness confined to the images of the encoding in \HOPiDd, i.e., processes in \HOPiDd\ that are the encoding of \FOPi\ processes; these processes merely communicate terms of the form $\lrangle{x}(\enc{P})$ in which $P$ is a \FOPi\ process).
We demonstrate that it is not the case. We carry out the analysis with a new counterexample below.
%(i.e., the processes in the target model that have reverse-image w.r.t. the encoding).
%\sep
%\nts{\Large See the note jpg file ``nt20170609.JPG" in the ``Long\_version" subfolder; typeset it here}
%\sep

Fix two \FOPi\ processes $R_3$ and $R_4$. That they are \ground bisimilar is in the clear. We now show that their encodings are not related by $\WWCB$. %, which is similarly defined as that in Section \ref{s:encoding_soundness}.
\[
R_3 \DEF (b)(a.b\para \overline{b}.\overline{c}) \qquad\qquad R_4 \DEF (b)(a.b\para \overline{b}.\overline{c}\para \overline{b}.\overline{c})
\]
Their encodings should have the follow-up simulation.
%\[\nsepvs{.2}
%\xymatrix{
% & P \ar@{.}[rr]|-{\mathcal{R}}\ar@{->}[d]_{\overline{a}A}  &  & Q \ar@{=>}[d]^{\overline{a}B}  &   \\
% P'\para !m.A \ar@/_1.6pc/@{.}[0,4]|{\mathcal{R}} & P'  & &  Q'  & Q'\para !m.B
%}
%\]
\[\nsepvs{.2}
\xymatrix@C=10pt{
\enc{R_3}\ar@{}[r]|-{\equiv} & (b)(\overline{a}[\lrangle{x}(\enc{b})] \para \enc{\overline{b}.\overline{c}}) \ar@{->}[d]_{(b)\overline{a}[\lrangle{x}(\enc{b})]}  &  &  (b)(\overline{a}[\lrangle{x}(\enc{b})] \para \enc{\overline{b}.\overline{c}} \para \enc{\overline{b}.\overline{c}}) \ar@{->}[d]^{(b)\overline{a}[\lrangle{x}(\enc{b})]} & \enc{R_4}\ar@{}[l]|-{\equiv}   \\
 & 0\para \enc{\overline{b}.\overline{c}}  &  &  0 \para  \enc{\overline{b}.\overline{c}} \para \enc{\overline{b}.\overline{c}} &
}
\]
In terms of context bisimulation, we choose $E[X] \DEF X\lrangle{e} \para X\lrangle{e}$ and have
\[
\begin{array}{rcl}
E[\lrangle{x}(\enc{b})] &\equiv& \enc{b}\para \enc{b} \\
\enc{b} &\equiv& \overline{b}[\lrangle{x}0] \\
\enc{\overline{b}.\overline{c}} &\equiv& (f)(b(Y).(Y\lrangle{f} \para (g)(c(Z).(Z\lrangle{g}\para 0))))
\end{array}
\]
We note that $E[\lrangle{x}(\enc{b})]$ corresponds to the \FOPi\ process $b\para b$ so it is legal for $\WWCB$.
We now argue that $(b)(E[\lrangle{x}(\enc{b})] \para \enc{\overline{b}.\overline{c}})$ and $(b)(E[\lrangle{x}(\enc{b})] \para \enc{\overline{b}.\overline{c}}\para \enc{\overline{b}.\overline{c}})$ are not bisimilar as regard to $\WWCB$, i.e.,
\[
(b)(\enc{b}\para \enc{b}\para \enc{\overline{b}.\overline{c}})  \;\;\;\;\;\bcancel{\WWCB}\;\;\;\;\; (b)(\enc{b}\para \enc{b}\para \enc{\overline{b}.\overline{c}}\para \enc{\overline{b}.\overline{c}})
\] thus violating the output requirement of the context bisimulation.
To see this, fix a \HOPiDd\ abstraction $B$, e.g., $\lrangle{x}0$. We then have the following chasing diagram showing the inequality above. 
%For the sake of conciseness, we use $\cdot$ to denote certain existent process.
\xxxx{
Here, $\cdot$ is some unspecified process.
}
\[\nsepvs{.2}
\xymatrix{
  (b)(\enc{b}\para \enc{b}\para \enc{\overline{b}.\overline{c}})  \ar@{->}[d]_{\tau}  &  &  (b)(\enc{b}\para \enc{b}\para \enc{\overline{b}.\overline{c}}\para \enc{\overline{b}.\overline{c}}) \ar@{->}[d]^{\tau}    \\
  \cdot \ar@{->}[d]_(.3){c(B)}  &  &  \cdot \ar@{->}[d]^{c(B)} \\
  {\begin{array}{c}\cdot \\[-.3cm] \rotatebox{-90}{\SCB} \\ 0 \end{array}}  &  & \cdot \ar@{->}[d]^{\tau} \\
   & & \cdot \ar@{->}[d]^(.3){c(B)} \\
   & & {\begin{array}{c}\cdot \\[-.3cm] \rotatebox{-90}{\SCB} \\ 0 \end{array}}
}
\]
We thus conclude that $\enc{R_3} \;\;\bcancel{\WWCB}\;\; \enc{R_4}$.



\sepp
%To summary, 
\xxxx{To summarize, }
we have the following lemma.
\begin{lemma}\label{l:soundness_var}
Assume $P$ and $Q$ are \FOPi\ processes. Then $P\WGB Q$ implies neither $\enc{P} \,\WCB\, \enc{Q}$ nor $\enc{P} \,\WWCB\, \enc{Q}$.
\end{lemma}
%\begin{proof}
%\nts{TODO (ref. Section \ref{s:encoding_soundness})}
%\end{proof}


\subsection{Completeness}
\xxxx{In spite of the failure of the soundness in either general or weak cases, fortuitously the encoding indeed has completeness.
} 
It is worth noting that we need Theorem \ref{factor-bigd-smalld} (Factorization) in the proof. This not only exhibits somewhat the use of that theorem, but also confers upon in a sense the synergy of the two kinds of abstractions. 
\xxrmcolor{For the sake of modular organization (viz., putting the two encodings in neighbouring sections), the statement and proof of Theorem \ref{factor-bigd-smalld} is deferred until Section \ref{s:normal}.
\xxx{Basically, as explained in Section \ref{s:introduction},} 
that theorem expresses that a sub-process of a process can be replaced with the so-called trigger, and moved to a new parallel position, which is accessible using the trigger acting as a messenger that faithfully transmits the information at the original position to the new positition, thus retaining the equivalence with the original process. Interested readers are encouraged to have a quick look at the statement of the theorem so as to understand the proof of completeness. 
More details of the theorem and its proof can be consulted later.
}
% Basically this theorem states that in \HOPiDd, there is the so-called normal bisimulation 
% %We note that the proof of completeness uses the normal bisimulation of \HOPiDd. 
% that characterizes context bisimulation but has a light requirement of checking equivalence. The definition of normal bisimulation and its coincidence with context bisimulation is given in Section \ref{s:normal}. 
% %This is actually an application of the normal bisimulation. 
% Interested readers are encouraged to have a quick look at the definition of normal bisimulation so as to understand the proof of completeness. 
% The details of the coincidence between normal bisimulation and context bisimulation can be consulted later.

\begin{lemma}\label{l:completeness_var}
Assume $P$ and $Q$ are \FOPi\ processes. Then $\enc{P} \WCB \enc{Q}$ implies $P\WGB Q$.
\end{lemma}
\xiv{The proof of Lemma \ref{l:completeness_var} is placed in Appendix \ref{a:proofs-completeness-var-encoding}.}





%---------------------------
% Local Variables:
% mode: LaTeX
% TeX-master: "main.tex"
% End:

%


\paragraph{Remark.}
Basically, the discussion of this section shows that the variant encoding is not weakly sound, let alone sound, although fortunately it is still complete. This warns us that a seemingly small difference in encoding strategy, e.g., swapping the input and output, can lead to rather unexpected results (notice also that because of this swapping, the counterexample for weak soundness in this section does not work for the encoding in Section \ref{s:encoding}).
\xxrmcolor{A closely relevant issue is whether we can use \HOPid\ instead of \HOPiDd (\HOPid\ denotes the expectable calculus, i.e., \HOPi\ with parameterization only on names). The point is that this would not bring back (weak) soundness (the counterexamples still work), rather causes us to lose the factorization property -- as far as we can see. Thus the completeness would become unclear, if not impossible.
}
Therefore it leaves open the problem of designing a (weakly) sound encoding of name-passing using parameterization on names alone.













%---------------------------
% Local Variables:
% mode: LaTeX
% TeX-master: "main.tex"
% End:

\section{Normal bisimulation for \HOPiDd}\label{s:normal}
In this section, we show that context bisimulation in \HOPiDd\ can be characterized by the much simpler normal bisimulation. % called normal bisimulation.
%In light of the result above, we have a characterization of context bisimulation in terms of normal bisimulation in calculus \HOPiDd.
%\sep
{We focus on unary parameterizations, i.e., abstractions and applications that allows only one variable or instance respectively, for the sake of simplicity and moreover because unary parameterizations appear more elementary; for example, intuitively an n-ary abstractions can be treated as a sequence of unary abstractions, and also recall that in the encoding above actually only unary parameterizations are utilized. Technically, by means of polyadic communication \cite{SW01a}, the characterization of normal bisimulation can be extended to arbitrary dimensions of parameterization. }

\tdup{
\bc{TODO (in \HOPiDd):}
\begin{itemize}
\item Factorization theorem; \bc{$\checkmark$}
\item Definition of normal bisimulation $\WNB$; \bc{$\checkmark$}
\item Coincidence between $\WNB$ and $\WCB$ (i.e., $\WNB$ implies $\WCB$ using the Factorization theorem). \bc{$\checkmark$}
\end{itemize}
\sepp\sepp

\fbox{We stipulate that $\rc{\trigger} \DEF \overline{m}$, $\rc{\triggerD} \DEF \lrangle{Z}\overline{m}Z$, and $\rc{\triggerd} \DEF \lrangle{z}\overline{m}[\lrangle{Y}(Y\lrangle{z})]$.
}
}%\tdup

\subsubsection*{The factorization theorem}
Below is the factorization theorem in presence of parameterization on names. % (and on processes as well). We recall that $\equiv$ is the structural congruence.
As explained in Section \ref{s:introduction}, the upshot of establishing the factorization theorem is to find the right small processes so-called triggers. Here we have three kinds of triggers, to tackle different kinds of parameterization. In particular, we stipulate that the triggers are as follows:  $\rrc{\triggerd} \DEF \lrangle{z}\overline{m}[\lrangle{Y}(Y\lrangle{z})]$, $\rrc{\triggerD} \DEF \lrangle{Z}\overline{m}Z$, and $\rrc{\trigger} \DEF \overline{m}$. These triggers are of somewhat a similar flavor but quite different in shape, with the aim at factorizing out respectively a name abstraction, a process abstraction and a non-abstraction process in certain context. The first trigger, i.e.,  $\triggerd$, is the main contribution of this work, whereas the other two are inherited from \cite{Xu13} and \cite{San92} respectively.

%\begin{theorem}[Factorization]\label{factor-bigd-smalld} %[Factorization]
%Given $E[X]$ of \HOPiDd, it holds for every $A$, fresh $m$ (i.e., $m\notin fn(E,A)$) that
%\begin{itemize}
%\item[(1)] if $E[X]$ is not an abstraction, then
%\begin{itemize}
%\item[(i)] if $A$ is not an abstraction, then
%$E[A] \WCB (m)(E[\trigger] \para  !m.A)
%$;
%\item[(ii)] if $A$ is an abstraction on process, then
%$E[A] \WCB (m)(E[\triggerD] \para  !m(Z).A\lrangle{Z})
%$;
%\item[(iii)] if $A$ is an abstraction on name, then
%$E[A] \WCB (m)(E[\triggerd] \para  !m(Z).Z\lrangle{A})
%$.
%\end{itemize}
%
%\item[(2)]  else if $E[X]$ is an abstraction, i.e., $E[X]\equiv \ve{\lrangle{U}}E'$ for some non-abstraction $E'$ (here $\ve{\lrangle{U}}$ denotes the abstractions prefixing $E'$), %(that is not an abstraction)
%then
%\begin{itemize}
%\item[(i)] if $A$ is not an abstraction, then
%$E[A] \WCB \ve{\lrangle{U}}((m)(E'[\trigger] \para  !m.A))
%$;
%\item[(ii)] if $A$ is an abstraction on process, then
%$E[A] \WCB \ve{\lrangle{U}}((m)(E'[\triggerD] \para  !m(Z).A\lrangle{Z}))
%$;
%\item[(iii)] if $A$ is an abstraction on name, then
%$E[A] \WCB \ve{\lrangle{U}}((m)(E'[\triggerd] \para  !m(Z).Z\lrangle{A}))
%$.
%\end{itemize}
%
%%\item[(3)]  else if $E[X]\equiv \lrangle{y_1}\cdots\lrangle{y_k}E'$ for some $k\geq 1$ and non-abstraction $E'$, %(that is not an abstraction)
%%then
%%\begin{itemize}
%%\item[(i)] if $A$ is not an abstraction, then
%%\[
%%...
%%\]
%%\item[(ii)] if $A$ is an abstraction on process, then
%%\[
%%...
%%\]
%%\item[(iii)] if $A$ is an abstraction on name, then
%%\[
%%...
%%\]
%%\end{itemize}
%
%\end{itemize}
%\end{theorem}

%\begin{theorem}[Factorization]\label{factor-bigd-smalld} %[Factorization]
%Given $E[X]$ of \HOPiDd, it holds for every $A$, fresh $m$ (i.e., $m\notin fn(E,A)$) that
%\begin{itemize}
%\item[(1)] if $A$ is not an abstraction, then
%\begin{itemize}
%\item[(i)] if $E[\trigger]$ is not an abstraction, then $E[A] \WCB (m)(E[\trigger] \para  !m.A)$;
%\item[(ii)] if $E[\trigger]$ is $\ve{\lrangle{U}}E'$ for some non-abstraction $E'$ (here and below, $\ve{\lrangle{U}}$ denotes the abstractions prefixing $E'$), then $E[A] \WCB \ve{\lrangle{U}}((m)(E' \para  !m.A))$.
%\end{itemize}
%\item[(2)] if $A$ is an abstraction on process, then
%\begin{itemize}
%\item[(i)] if $E[\triggerD]$ is not an abstraction, then $E[A] \WCB (m)(E[\triggerD] \para  !m(Z).A\lrangle{Z})$;
%\item[(ii)] if $E[\triggerD]$ is $\ve{\lrangle{U}}E'$ for some non-abstraction $E'$, then $E[A] \WCB \ve{\lrangle{U}}((m)(E' \para  !m(Z).A\lrangle{Z}))$.
%\end{itemize}
%\item[(3)] if $A$ is an abstraction on name, then
%\begin{itemize}
%\item[(i)] if $E[\triggerd]$ is not an abstraction, then $E[A] \WCB (m)(E[\triggerd] \para  !m(Z).Z\lrangle{A})$;
%\item[(ii)] if $E[\triggerd]$ is $\ve{\lrangle{U}}E'$ for some non-abstraction $E'$, then $E[A] \WCB \ve{\lrangle{U}}((m)(E' \para  !m(Z).Z\lrangle{A}))$.
%\end{itemize}
%\end{itemize}
%\end{theorem}

\begin{theorem}[Factorization]\label{factor-bigd-smalld} %[Factorization]
Given $E[X]$ of \HOPiDd, it holds for every $A$, fresh $m$ (i.e., $m\notin \fn{E,A}$) that
\begin{itemize}
\item[(1)] if $A$ is not an abstraction, then
\begin{itemize}
\item[(i)] if $E[\trigger]$ is $\ve{\lrangle{U}}E'$ for some non-abstraction $E'$ (here and below, $\ve{\lrangle{U}}$ denotes the abstractions prefixing $E'$), then $E[A] \WCB \ve{\lrangle{U}}((m)(E' \para  !m.A))$;
\item[(ii)] particularly, if $E[\trigger]$ is not an abstraction, then $E[A] \WCB (m)(E[\trigger] \para  !m.A)$.
\end{itemize}
\item[(2)] if $A$ is a process abstraction, then
\begin{itemize}
\item[(i)] if $E[\triggerD]$ is $\ve{\lrangle{U}}E'$ for some non-abstraction $E'$, then $E[A] \WCB \ve{\lrangle{U}}((m)(E' \para  !m(Z).A\lrangle{Z}))$;
\item[(ii)] particularly, if $E[\triggerD]$ is not an abstraction, then $E[A] \WCB (m)(E[\triggerD] \para  !m(Z).A\lrangle{Z})$.
\end{itemize}
\item[(3)] if $A$ is a name abstraction, then
\begin{itemize}
\item[(i)] if $E[\triggerd]$ is $\ve{\lrangle{U}}E'$ for some non-abstraction $E'$, then $E[A] \WCB \ve{\lrangle{U}}((m)(E' \para  !m(Z).Z\lrangle{A}))$;
\item[(ii)] particularly, if $E[\triggerd]$ is not an abstraction, then $E[A] \WCB (m)(E[\triggerd] \para  !m(Z).Z\lrangle{A})$.
\end{itemize}
\end{itemize}
\end{theorem}

%\begin{proof}
In Theorem \ref{factor-bigd-smalld}, clause (1) is actually Sangiorgi's seminal work \cite{San92, SW01a}. Clause (2) is shown in \cite{Xu13}. Clause (3) depicts the factorization for abstraction on names, and is the contribution of this work.   %, can be discussed through a resembling technical routine. % almost the same as (ii). %and (i)
Intuitively, it allows one to replace, in a context $E$, a specific component $A$ (being a name abstraction) with the uniform trigger $\triggerd$, move $A$ into a (distributed) new place, and in the meanwhile build up a connection between $\triggerd$ and the new place, so as to maintain the invariance of (observational) behaviour.
Moreover to this end, when the context $E$ reveals some outmost abstractions, then such connection must be upheld by positioning these abstractions outmost accordingly.
As opposed to the original process, the  process after factorization has a lighter structure to discuss, and one can concentrate on the new place where the factorized process might exhibit distinct behaviour.
%So we go no further into the details. (\bc{maybe also due to space limit?})
%See \cite{San92, SW01a, Xu13} for a reference.
%With regard to more details we refer the reader to \cite{San92, SW01a, Xu13}.
%\end{proof}

With regard to the method of \emph{trigger} (including the technical approach), the fundamental framework is well-developed in the field, due to Sangiorgi \cite{SW01a}. The key to   establishing the factorization property for processes allowing abstraction on names is the \emph{trigger}, which has been unknown for a long time in contrast to the case of abstraction on processes and that without abstractions. Once a right trigger is found, the remainder of discussion can be made through a similar (but still tricky) technical routine.
Below we give an example of the factorization concerning abstraction on names. The proof of Theorem \ref{factor-bigd-smalld} can be found in Appendix \ref{a:proofs-normal}. % with an example.
%\begin{example}

\noindent\textbf{Example} The basic idea of factorization concerning abstraction on names can be illustrated in the following example in which $m$ is fresh (i.e., not in $A\lrangle{\rrc{d}}$).
%\[
\begin{eqnarray}
%A\lrangle{d} \approx_{ct} (m)((\lrangle{z}\overline{m}[\lrangle{Y}(Y\lrangle{z})])\lrangle{d} \para m(Z).Z\lrangle{A}) \\\\
A\lrangle{\rrc{d}} &\WCB& (m)(\; (\lrangle{\rrc{z}}\overline{m}[\rbc{\lrangle{Y}(Y\lrangle{z})}])\lrangle{\rrc{d}} \,\para\, m(\rbc{Z}).\rbc{Z}\lrangle{A} \;) %\nonumber \\ %\label{eqn-fact-hopiDd}\\
%& &
\; \equiv\; (m)(\; \overline{m}[\rbc{\lrangle{Y}(Y\lrangle{\rrc{d}})}] \,\para\, m(\rbc{Z}).\rbc{Z}\lrangle{A} \;) \nonumber
\end{eqnarray}
%\]
%~~~~~~~~~~~~~~~%\rc{TO ADD (opt.): fabricate an EXAMPLE based on equation \ref{eqn-fact-hopiDd}};
\noindent For example, if $A$ is $\lrangle{x}\overline{x}b$, then $A\lrangle{d} \equiv \overline{d}b$, and
\[
\begin{array}{lclcl}
A\lrangle{d}  &\WCB& (m)(\; \overline{m}[\lrangle{Y}(Y\lrangle{d})] \,\para\, m(Z).Z\lrangle{A} \;) %\\
 &\WCB& (m)(\; (\lrangle{Y}(Y\lrangle{d}))\lrangle{A} \;) \;\equiv\; A\lrangle{d} \;\equiv\; \overline{d}b %\\
% &\equiv& A\lrangle{d} \\
% &\equiv& \overline{d}b
\end{array}
\]
%\end{example}


\subsubsection*{Normal bisimulation for \HOPiDd}
Below is the definition of normal bisimulation whose clauses are designed on top of the factorization theorem. We recall that $\rrc{\trigger} \DEF \overline{m}$, $\rrc{\triggerD} \DEF \lrangle{Z}\overline{m}Z$, and $\rrc{\triggerd} \DEF \lrangle{z}\overline{m}[\lrangle{Y}(Y\lrangle{z})]$.
\begin{definition}\label{normal-bisi-Dd} %[Normal bisimulation]
A symmetric binary relation $\mathcal{R}$ on closed processes of \HOPiDd\ is a normal bisimulation, if whenever $P\,\mathcal{R}\, Q$ the following properties hold:
\begin{enumerate}%\itemsep-.3em
\item If $P \st{a(\trigger)} P'$ ($m$ is fresh), then $Q \wt{a(\trigger)} Q'$ for some $Q'$ s.t.  $P'\,\mathcal{R}\, Q'$; %($m$ is fresh w.r.t. $P$ and $Q$)
\item If $P \st{a(\triggerD)} P'$ ($m$ is fresh), then $Q \wt{a(\triggerD)} Q'$ for some $Q'$ s.t.  $P'\,\mathcal{R}\, Q'$; %($m$ is fresh w.r.t. $P$ and $Q$)
\item If $P \st{a(\triggerd)} P'$ ($m$ is fresh), then $Q \wt{a(\triggerd)} Q'$ for some $Q'$ s.t.  $P'\,\mathcal{R}\, Q'$; %($m$ is fresh w.r.t. $P$ and $Q$)

\item If $P \st{(\ve{c})\overline{a}A} P'$ and $A$ is not an abstraction, then $Q \wt{(\ve{d})\overline{a}B} Q'$ for some $\ve{d},Q'$ and $B$ that is not an abstraction, and it holds that ($m$ is fresh) ~
$(\ve{c})(P'\para !\rbc{m.A}) \; \mathcal{R}\;  (\ve{d})(Q'\para  !\rbc{m.B})$.
\item If $P \st{(\ve{c})\overline{a}A} P'$ and $A$ is an abstraction on process, then $Q \wt{(\ve{d})\overline{a}B} Q'$ for some $\ve{d},Q'$ and $B$ that is a process abstraction, and it holds that ($m$ is fresh) ~
$(\ve{c})(P'\para !\rbc{m(Z).A\lrangle{Z}}) \; \mathcal{R}\;  (\ve{d})(Q'\para  !\rbc{m(Z).B\lrangle{Z}})$.
\item If $P \st{(\ve{c})\overline{a}A} P'$ and $A$ is a name abstraction, then $Q \wt{(\ve{d})\overline{a}B} Q'$ for some $\ve{d},Q'$ and $B$ that is a name abstraction, and it holds that ($m$ is fresh) ~
$(\ve{c})(P'\para !\rbc{m(Z).Z\lrangle{A}}) \; \mathcal{R}\;  (\ve{d})(Q'\para  !\rbc{m(Z).Z\lrangle{B}})$.

\item If $P \st{\tau} P'$, then $Q \wt{} Q'$ for some $Q'$ s.t. $P'\,\mathcal{R}\, Q'$;
\end{enumerate}
Process $P$ is normal bisimilar to $Q$, written $P\,\WNB\, Q$, if $P\,\mathcal{R}\, Q$ for some normal bisimulation $\mathcal{R}$. Relation \WNB\ is called normal bisimilarity, and is a congruence (see \cite{San92} for a reference). The strong version of \WNB\ is denoted by \SNB.
\end{definition}




\subsubsection*{Coincidence between normal bisimilarity and context bisimilarity in \HOPiDd}
Now we have the following theorem. The proof is in Appendix \ref{a:proofs-normal}.
%\iftoggle{appendixing}{%
%  %using appendixing
% The proof is in Appendix \ref{a:proofs-normal}.
%}{%
%  %no appendixing
% The detailed proof is referred to \cite{Xu16app}.
%}

\begin{theorem}\label{normal-characterization-hopiDd} %[Normal characterization]
In \HOPiDd, normal bisimilarity coincides with context bisimilarity; that is, $\WNB \,=\, \WCB$.
\end{theorem}











%---------------------------
% Local Variables:
% mode: LaTeX
% TeX-master: "main.tex"
% End:

\section{Conclusion}\label{s:conclusion}


In this paper, we have exhibited two new encodings of name-passing in the higher-order paradigm that allows parameterization, and a normal bisimulation in that setting as well. In the former, we demonstrate the conformance or inconsistency of the encoding with respect to some well-established criteria in the literature. In the latter, we prove the coincidence between normal and context bisimulation by pinpointing how to factorize an abstraction on some name. 

% The encoding of this work is inspired by a seemingly similar one proposed by Alan Schmitt during the communication concerning another work. That encoding, as given below skipping the homomorphic parts, somewhat swaps the roles of input and output and treats $a(x).P$ somehow as $a.\lrangle{x}P$ (like those calculi admitting abstractions and concretions \cite{San92}). 
% \[
% \begin{array} {rcl}
% \enc{a(x).P} & \DEF & \overline{a}[\lrangle{x}\enc{P}]\\
% \enc{\overline{a}b.Q} &\DEF & a(Y).(Y\lrangle{b}\para \enc{Q}) \\%\quad \quad (Y \mbox{ is fresh})
% \end{array}
% \]
% %\[
% %\mbox{ \begin{tabular}{l} 
% %\rc{[Optional] EXTEND the discussion of this encoding (to some extent) ? in Appendix \ref{appendix:variant_encoding}! } \\
% %\rc{(maybe without much proof) (1) operational correspodence: apparently not satisfied directly,}\\
% %\rc{(2) (counter)-example: try the original counterexample (Section \ref{s:encoding_soundness}) or its variant, } \\
% %\rc{(3) (partial) soundness: if the counterexample still applies (\bc{which seems the case}),} \\
% %\rc{then soundness is (similarly) compromised to the image,} \\
% %\rc{(4) completeness: may be still depend on the sujection of the reverse mapping of the encoding.}
% %\end{tabular}
% %}
% %\]

% From the angle of achieving first-order interaction, the encoding strategy above only employs abstraction on name and is truly interesting. However at first sight, it appears not to satisfy some usual operational correspondence (say, in \cite{Gor08a} or \cite{LPSS10}), and full abstraction is not quite clear. % with similar effort.
% Based on the results in this paper, it is tempting to expect that this encoding have some (nearly) same properties, and this is worthwhile for more investigation. 
% {We thus extend the analysis of this encoding in Appendix \ref{appendix:variant_encoding}.} 
% Basically, the discussion there shows that the encoding above is not weakly sound, let alone sound, although fortunately it is still complete. This warns us that a seemingly small difference in encoding strategy, e.g., swapping the input and output, can lead to rather unexpected results, and leaves open the problem of designing a (weakly) sound encoding of name-passing using abstraction on names alone.

The results of this paper can be dedicated to facilitate further study on the expressiveness of higher-order processes. The following questions, among others, are still open: whether \FOPi\ can be encoded in a higher-order setting only allowing parameterization on processes; whether there is a better encoding of \FOPi\ than the one in Section \ref{s:encoding_variant} or in \cite{XYL15}, using higher-order processes only capable of parameterization on names (we denote this calculus by $\HOPid$); whether \HOPid\ affords a normal-like characterization of context bisimulation. 
%For the last one, we provide some intuitive but informal account below to show the difficulty in finding a solution. 
For the last one, intuitively the major difficulty is to find a proper trigger-like apparatus (and then of course, the factorization property). 
%The crux here is that the original method of normal bisimulation does not appear to work in $\Pi^d$ by all means, probably due to the loss of a factorization property in nature.
%To understand this further, 
To have a taste, let us revisit the example in Section \ref{s:normal}, i.e., the $\Pi^d$ process $W\DEF A\lrangle{d}, \mbox{ in which } A\DEF \lrangle{x}\overline{x}$, and clearly $W\equiv \overline{d} \,\st{\overline{d}}\, 0$. We note that the concrete name $d$ can be provided by the environment dynamically, i.e., during run-time.
%However if one tries to factorize out the subprocess $A$, some contradiction arises by the examining below.
%Now with regard to (\ref{eq:normal_pid_1_faclike_prop}), 
We expect to have a factorization property like below. 
\[
(\ve{c})(T\lrangle{d} \para F[A]) \WCB A\lrangle{d} \equiv \overline{d}
\] % (here $E[X]$ is $X\lrangle{d}$), 
where $T$ is supposed to be a proper uniform-looking `trigger' (subject to the capability of the calculus), and $F$ to be the new place holding a repository of $A$, with $\ve{c}$ being the local names shared by $T$ and $F$. %and by (\ref{eq:normal_pid_3})  %
%Assuming $T\equiv \lrangle{z}T''$, we have 
%$
% (\ve{n})((\lrangle{z}T'')\lrangle{d} \para F[A]) \approx A\lrangle{d}
%$ for some $T''$ s.t. $T\equiv \lrangle{z}T''$. Then we have
%%\begin{equation}\label{eq:normal_pid_4}
%%(\ve{n}\ve{n'})(\overline{m}d\para T'\fosub{d}{z} \para F[A]) \approx \overline{d} \qquad m\in\ve{n},m\notin\ve{n'}
%%\end{equation}
%\begin{equation}\label{eq:normal_pid_4}
%(\ve{n})(T''\fosub{d}{z} \para F[A]) \WCB \overline{d}
%\end{equation}
%On the face of (\ref{eq:normal_pid_4}), 
Now the left hand side must have a transition $\wt{\overline{d}}$, which should result from the interaction between $T\lrangle{d}$ and $F[A]$, and during these interactions $d$ must be received by $F$ so as to be fed to $A$ in its own (uniform) setting such that $A$ eventually gets the right instantiation. 
However, strikingly different from the case of $\Pi^{D,d}$, there appears no hope that $T$ can finish this job of activating $A$ in $F[A]$ in need, with only name parameterization in a purely higher-order realm (i.e., no name-passing).   
To this point, resolving this conundrum may amount to exploiting further the expressiveness of parameterization on name; otherwise, showing the impossibility of characterizaing context bisimilation should rest on some precisely formulated criteria. This might take us one step further to scrutinize the discrepancy in the role of names in higher-order models.






%\subsection{Discussion: Encoding \FOPi\ with \HOPiD\ or \HOPid}
%\stress{Whether \FOPi\ can be encoded in \HOPid\ is still unknown, neither is the case with \HOPiD.}
%
%In \cite{XYL15}, we propose one such encoding which however is short of a satisfactory soundness result.
%
%Below is another trial by A. Schmitt. (\stress{A referential encoding of \FOPi\ in \HOPid\ (by A. Schmitt)})
%\[
%\begin{array} {rcl}
%\enc{a(x).P} & \DEF & \overline{a}[\lrangle{x}\enc{P}]\\
%\enc{\overline{a}b.Q} &\DEF & a(Y).(Y\lrangle{b}\para \enc{Q})\quad \quad (Y \mbox{ is fresh})
%\end{array}
%\]
%
%If we only consider the explanation of interaction (i.e., the $\tau$ action), then the above strategy would be interesting.
%However, this strategy does not respect the notion of encoding given in Section \ref{s:criteria}. It does not satisfy, for example, the (weak) operational correspondence, as for any non-nil \FOPi\ process $P$, the encoding in \HOPid\ for input (i.e., $\enc{a(x).P}$) will evolve into a nill process after performing an output action. 
%%Therefore the above schema is not precisely an encoding in the sense of our criteria. 
%%Actually to our intuition, the encoding, if any, is somewhat akin to that in \cite{Tho93, XYL14}.
%%That said,  This raises the question that in what condition we could adopt somewhat relaxed encoding criteria and still keep the study on expressiveness reasonable. This can be worthwhile for further investigation.

%\sep\sep
%\nts{\large FROM here TODO: simplify and shorten the red content below; CONSIDER use a single example (e.g., $W$ from below) to explain the core idea. Or DROP it entirely. }\sep
%\sep
%
%%------------------------------------------------------------
%%\input{normal_pid.tex}
%\oo{\scriptsize
%As shown above, the core of a normal-like characterization is the factorization property. Specifically, in order to recover the context bisimulation, one needs to work out some special form of processes to take the place of the general ones communicated during the simulation of input and output, without loss of any discriminating capability. Then by all means, the factorization property can be used to guide the design of a normal bisimulation. 
%
%%\rc{\scriptsize
%%\xx{ FMI.}\\
%%The core of a normal-like characterization is some property similar to factorization. The reason is that in order to design some special form of `small' process to {represent} the general ones during the simulation of input and output, such a process has to be endowed with the capability of retrieving the general requirement of context bisimulation. Such a kind of retrieval is, by all means, bound to attaching in parallel some `small' context containing a general process. This actually leads to some factorization-like property. Specifically,
%\begin{itemize}
%\item We recall the input clause of context bisimulation below in which $A$ is assumed to be an abstraction on a name.
%\[
%\mbox{If $P \st{a(A)} P'\DEF E[A]$, then $Q \wt{a(A)} Q'\DEF E'[A]$ for some $Q'$, and $E[A]\,\mathcal{R}\, E'[A]$}
%\] %For convenience we assume $E$ and $E'$ is the receiving environments respectively corresponding to $P$ and $Q$.
%In the very first spirit of normal bisimulation \cite{San92}, the general challenge using $A$ as the input calls for an representation by, say, a special simple term $T$. Accordingly, the requirement of relating $E[A]$ and $E'[A]$ is represented by that on $E[T]$ and $E'[T]$.
%Suppose $F$ is the context where we put the represented $A$. 
%%aimed at retrieving $E[A]$ (respectively $E'[A]$) using $(\ve{n})(E[T]\para F[A])$ (respectively $(\ve{n})(E'[T]\para F[A])$) where $\ve{m}$ are the (local) names possibly shared by $E[T]$ (respectively $E'[T]$) and $F$. 
%The desired property would be that $(\ve{n})(E[T]\para F[A])$ is equivalent with $E[A]$ w.r.t. some bisimulation congruence (at least as fine as context bisimilarity), i.e., the factorization-like property below.
%\begin{equation}\label{eq:normal_pid_1_faclike_prop}
%(\ve{n})(E[T]\para F[A]) \approx E[A] \qquad (\mbox{respectively } (\ve{n})(E'[T]\para F[A]) \approx E'[A])
%\end{equation}
%As such, the gadget $T$ is responsible for activating $A$ in $F[A]$ in need (playing a role similar to `triggers').
%%As such the factorization-like property (\ref{eq:normal_pid_1_faclike_prop}) is somewhat the core of a simpler characterization of context bisimulation.
%
%\item In the case of output, the context bisimulation requires that
%\begin{center}
%If $P \st{(\ve{c})\overline{a}A} P'$ then $Q \wt{(\ve{d})\overline{a}B} Q'$ for some $\ve{d},B,Q'$, and for every $E[X]$ ($\{\ve{c},\ve{d}\}\cap fn(E)=\emptyset$) it holds 
%$(\ve{c})(E[A]\para P') \; \approx\;  (\ve{d})(E[B]\para Q')$.
%\end{center}
%In line with the way a normal bisimulation works, %as explained at the beginning of this section, % (and the section of introduction), 
%and in accordance with the case of input, we are urged to apply (\ref{eq:normal_pid_1_faclike_prop}) to obtain the following transformation of closing statement of the output clause.
%\begin{equation}\label{eq:normal_pid_2}
%(\ve{c})((\ve{n})(E[T]\para F[A]) \para P') \; \approx\;  (\ve{d})((\ve{n})(E[T]\para F[B]) \para Q') \nonumber
%\end{equation}
%Then by isolating the different fragments on either side of (\ref{eq:normal_pid_2}), one has below the simplified requirement, which is more convenient work with since $F$ is a specific context.
%\[
%(\ve{c})(F[A]\para P') \; \approx\;  (\ve{d})(F[B]\para Q')
%\] %Since $F$ is closely related to $T$ and does not have universal quantifier before it, this clause is supposed to be more convenient to use.
%
%\item For our purposes, the crucial point is to conceive the shape of $T$ (and subsequently $F$) subject to the capability of the calculus. 
%%By analyzing the shape of $T$ based on the desired factorization-like property (\ref{eq:normal_pid_1_faclike_prop}), 
%An easy fact is that it should be a name-parameterized process, so that it can take the place of the abstraction $A$. %because otherwise the substitution would not be . 
%A tough job of $T$ is to transmit a concrete name, which it receives upon application over its parameterized name, to $F$ so as to be fed to $A$ therein such that $A$ eventually gets the right instantiation. This concrete name is provided by $E$ dynamically, i.e., during run-time.
%%(provided by $E$ dynamically, i.e., during run-time) to $A$ in the customized context $F$, otherwise the instantiation would not be fulfilled correctly, and obviously $E$ would not take care of this in general.  
%Put as a whole, $T$ should take the following form in which $m\in\ve{n}$ but $m\notin\ve{n'}$ ($\ve{n}$ is from equation (\ref{eq:normal_pid_1_faclike_prop}) and $\ve{n'}$ is some local names possibly used by $T$). %for some $\ve{n'},T'$
%\begin{equation}\label{eq:normal_pid_3}
%\lrangle{z}((\ve{n'})(\overline{m}z\para T'))
%\end{equation} 
%%This actually implies that non-parameterized processes and non-name-passing processes cannot play the role of $T$. 
%However, this is not possible in $\Pi^d$ since by no %(explicitly)
%means can a name be transmitted (at best abstraction is allowed to be communicated), so $T$ cannot be a member of $\Pi^d$. 
%Therefore, this indicates that one has to expand the search area (possibly beyond $\Pi^d$) for such $T$.
%\end{itemize}
%%}%end \scriptsize
%
%%Therefore, the method of normal bisimulation as in \cite{San92} does not appear to be directly applicable in $\Pi^d$.
%%We stress that the arguments above do not intend to rule out any possible form of (normal-like) bisimulation that simplifies context bisimulation, though that is what we believe. 
%
%}%color oo













%---------------------------
% Local Variables:
% mode: LaTeX
% TeX-master: "main.tex"
% End:



\subsection*{acknowledgements}
%If you'd like to thank anyone, place your comments here
%and remove the percent signs.
We are grateful to the referees of EXPRESS/SOS 2016 for their useful comments on this article and related works. We also thank Alan Schmitt, Qiang Yin, and the anonymous reviewers for the helpful suggestions and communications. 
%\end{acknowledgements}


%%-----------------------------
%%      your bibliography
%%-----------------------------
\bibliographystyle{spmpsci}
\bibliography{process}


%%-----------------------------
%%      your appendix
%%-----------------------------



% \sepp
% \clearpage
% \appendix

% \noindent\textbf{\Large Appendix}
% %\sepp
% \sepp

% {%\normalsize

% \noindent\textit{Remark}. For the sake of conciseness, in the arguments %in the appendix, 
% we sometimes omit the existential statement such as ``for some ..." if clear from context.

% %
% \section{\xxx{Proofs for Section \ref{s:preliminary}}}\label{appendix:up-to-context_general}

We give the proof of Theorem \ref{thm:sound_up-to_context_general}.

\begin{proof}[Proof of Theorem \ref{thm:sound_up-to_context_general}]
Assume $\R$ is as defined in Definition \ref{d:up2}. %We show that the relation $\R'$ defined below is a context bisimulation \xxx{up-to $\equiv$}, thus proving the proposition.
We define the relations $\R_n$ and $\R'$ as follows. Recall that $\mathbb{N}$ is the set of natural numbers.
\[
\begin{array}{lcl}
\R_0 &\DEF& \R \\
\R_{n+1} &\DEF& \left\{(C[M],C[N]) \,|\, M\,\R_{n}\, N \mbox{ and } C\in \mathcal{D} \right\} \\ %$\vartheta.[\cdot]$,
\R' &\DEF& \bigcup_{i\in \mathbb{N}} \R_i \\\\
\mathcal{D} &\DEF& \left\{
\begin{array}{lllll}
%[\cdot],  & 
a(X).[\cdot],\quad & \overline{a}([\cdot]).R,\quad & \overline{a}A_1.[\cdot],\quad & \lrangle{X}[\cdot], \quad & \lrangle{x}[\cdot],\\~
[\cdot]\lrangle{A_1}, & [\cdot]\lrangle{d}, & R\para [\cdot], & (d)[\cdot] &
\end{array}
\right\}
\end{array}
\]
We prove by induction on $n$ that $\R'$ is a context bisimulation up-to $\equiv$. Since $\R\subseteq \R'$, $\R\subseteq \WCB$ follows.

%-------------PROOF BODY BEGIN-------------
\xx{TODO: to fetch from `NOTES'}
%-------------PROOF BODY END-------------
\end{proof}

% \clearpage
% \input{appendix_proof_upto_enc_and_opcorres.tex}
% \clearpage
% %%\section{Proofs for Section \ref{s:encoding}}\label{a:proofs-encoding}

\begin{proof}[Proof of Lemma \ref{l:syn-pro-encoding}]
It is straightforward to check that the encoding is compositional since the designated contexts are easy to capture. For the core of the encoding, the contexts for input and output are respectively $m(Y).Y\lrangle{\lrangle{x}[\cdot]}$ and $\overline{m}[\lrangle{Z}(Z\lrangle{n})].[\cdot]$. Also the encoding is divergence-reflecting, for the reason that it does not bring about any divergence. As such, it is a simple induction, on the rules deriving $\equiv$, to show that the encoding preserves structural congruence. Below we prove by induction on the structure of $P$ that $\enc{P}\sigma \equiv \enc{P\sigma}$ in which $\sigma$ is a substitution (recall that $\sigma$ is a mapping on names).
%\oo{\large \fbox{\#\#\#\# TODO}}
\begin{itemize}
\item $P$ is $0$. This is trivial.
\item $P$ is $m(x).Q$. Then %m(Y).Y\lrangle{\lrangle{x}\enc{Q}}
\[
\begin{array}{lcll}
\enc{P}\sigma &\equiv& (m(Y).Y\lrangle{\lrangle{x}\enc{Q}})\sigma & \quad \\
 &\equiv& m'(Y).Y\lrangle{\lrangle{x}(\enc{Q}\sigma)} & \quad m' \mbox{ is } \sigma(m) \\
 &\equiv& m'(Y).Y\lrangle{\lrangle{x}(\enc{Q\sigma})} & \quad \mbox{ind. hyp. (short for induction hypothesis)}\\
 &\equiv& \enc{m'(x).Q\sigma} & \quad \\
 &\equiv& \enc{(m(x).Q)\sigma} & \quad \\ 
 &\equiv& \enc{P\sigma} & \quad
\end{array}
\]
\item $P$ is $\overline{m}n.Q$. Then %\overline{m}[\lrangle{Z}(Z\lrangle{n})].\enc{Q}
\[
\begin{array}{lcll}
\enc{P}\sigma &\equiv& (\overline{m}[\lrangle{Z}(Z\lrangle{n})].\enc{Q})\sigma & \quad \\
 &\equiv& \overline{m'}[\lrangle{Z}(Z\lrangle{n'})].(\enc{Q}\sigma) & \quad m',n' \mbox{ are respectively } \sigma(m),\sigma(n) \\
 &\equiv& \overline{m'}[\lrangle{Z}(Z\lrangle{n'})].(\enc{Q\sigma}) & \quad \mbox{ind. hyp.} \\
 &\equiv& \enc{\overline{m'}n'.(Q\sigma)} & \quad \\
 &\equiv& \enc{(\overline{m}n.Q)\sigma} & \quad \\
 &\equiv& \enc{P\sigma} & \quad
\end{array}
\]
\item $P$ is $(c)Q$. Then 
\[
\begin{array}{lcll}
\enc{P}\sigma &\equiv& ((c)\enc{Q})\sigma & \quad  \\
 &\equiv& (c)\enc{Q}\sigma & \quad  \\
 &\equiv& (c)\enc{Q\sigma} & \quad \mbox{ind. hyp.} \\
 &\equiv& \enc{(c)(Q\sigma)} & \quad  \\
 &\equiv& \enc{((c)Q)\sigma} & \quad  \\
 &\equiv& \enc{P\sigma} & \quad  
\end{array}
\]
\item $P$ is $Q\para R$. Then 
\[
\begin{array}{lcll}
\enc{P}\sigma &\equiv& (\enc{Q}\para \enc{R})\sigma  & \quad \\
 &\equiv& \enc{Q}\sigma\para \enc{R}\sigma  & \quad \\
 &\equiv& \enc{Q\sigma}\para \enc{R\sigma}  & \quad \mbox{ind. hyp.} \\ 
 &\equiv& \enc{Q\sigma \para R\sigma}  & \quad  \\ 
 &\equiv& \enc{(Q\para R)\sigma}  & \quad  \\ 
 &\equiv& \enc{P\sigma}  & \quad 
\end{array}
\]
\item $P$ is $!m(x).Q$. Then %!\enc{m(x).Q}
\[
\begin{array}{lcll}
\enc{P}\sigma &\equiv& (!\enc{m(x).Q})\sigma & \quad \\
 &\equiv& !(\enc{m(x).Q})\sigma & \quad \\
 &\equiv& !(\enc{(m(x).Q)\sigma}) & \quad \mbox{similar to the input case} \\
 &\equiv& \enc{!((m(x).Q)\sigma)} & \quad \\
 &\equiv& \enc{(!m(x).Q)\sigma} & \quad \\
 &\equiv& \enc{P\sigma} & \quad 
\end{array}
\]
\end{itemize}
\myqed
\end{proof}


\begin{proof}[Proof of Lemma \ref{l:opcor}]
We prove the clauses by induction on the structure of $P$. The case $P$ is $0$ is trivial.  %\stress{TODO (treat 1-6 together or separately)}
\begin{itemize}
\item $P$ is $a(x).P_1$.
\begin{itemize}
\item Input. $P\st{a(b)} P_1\fosub{b}{x} \equiv P'$. So $\enc{P} \equiv a(Y).Y\lrangle{\lrangle{x}\enc{P_1}} \st{a(\lrangle{Z}(Z\lrangle{b}))} \equiv \enc{P_1}\fosub{b}{x} \equiv T$. We then have $T \SSEQV \enc{P_1\fosub{b}{x}} \SSEQV \enc{P'}$. Moreover, $\enc{P} \st{a(\triggerD)} \overline{m}[\lrangle{x}\enc{P_1}] \equiv T$. We have 
\[
\begin{array}{ll}
& (m)(T \para !m(Y).Y\lrangle{b}) \\
\equiv & (m)(\overline{m}[\lrangle{x}\enc{P_1}] \para !m(Y).Y\lrangle{b}) \\
\WCB & (m)(!m(Y).Y\lrangle{b} \para \enc{P_1}\fosub{b}{x}) \\
\WCB & \enc{P_1}\fosub{b}{x}) \SSEQV \enc{P'}
\end{array}
\]
\end{itemize}

\item $P$ is $\overline{a}b.P_1$.
\begin{itemize}
\item Output. $P \st{\overline{a}b} P_1\equiv P'$. Then $\enc{P} \equiv \overline{a}[\lrangle{Z}(Z\lrangle{b})].\enc{P_1} \st{\overline{a}[\lrangle{Z}(Z\lrangle{b})]} \enc{P_1} \equiv T$, so $T\SSEQV \enc{P'}$.
\end{itemize}

\item $P$ is $P_1\para P_2$.
\begin{itemize}
\item Input. Assume $P_1\st{a(b)} P_1'$, and $P\st{a(b)} P_1'\para P_2 \equiv P'$. By ind. hyp., $\enc{P_1} \st{a(\lrangle{Z}(Z\lrangle{b}))} T_1$ and $T_1\SSEQV \enc{P_1'}$. Then $\enc{P}\equiv \enc{P_1}\para \enc{P_2} \st{a(\lrangle{Z}(Z\lrangle{b}))} T_1\para \enc{P_2} \DEF T$. So $T\SSEQV \enc{P_1'}\para \enc{P_2} \equiv \enc{P'}$. Moreover, $\enc{P_1} \st{a(\triggerD)} T_2$ and $(m)(T_2 \para !m(Y).Y\lrangle{b}) \WCB \enc{P_1'}$. Thus $\enc{P}\equiv \enc{P_1}\para \enc{P_2} \st{a(\triggerD)} T_2\para \enc{P_2} \DEF T'$. Hence $(m)(T' \para !m(Y).Y\lrangle{b}) \equiv (m)(T_2\para \enc{P_2} \para !m(Y).Y\lrangle{b}) \WCB \enc{P_1'} \para \enc{P_2} \equiv \enc{P'}$.

\item Output. Assume $P_1\st{\overline{a}b} P_1'$, and $P\st{\overline{a}b} P_1'\para P_2 \equiv P'$. By ind. hyp., $\enc{P_1} \st{\overline{a}[\lrangle{Z}(Z\lrangle{b})]} T_1$ and $T_1\SSEQV \enc{P_1'}$. Then $\enc{P}\equiv \enc{P_1}\para \enc{P_2} \st{\overline{a}[\lrangle{Z}(Z\lrangle{b})]} T_1\para \enc{P_2} \DEF T$. So $T\SSEQV \enc{P_1'}\para \enc{P_2}\equiv \enc{P'}$. 

\item Bound output. Assume $P_1\st{\overline{a}(b)} P_1'$, and $P\st{\overline{a}(b)} P_1'\para P_2 \equiv P'$. By ind. hyp., $\enc{P_1} \st{(b)\overline{a}[\lrangle{Z}(Z\lrangle{b})]} T_1$ and $T_1\SSEQV \enc{P_1'}$. Then $\enc{P}\equiv \enc{P_1}\para \enc{P_2} \st{(b)\overline{a}[\lrangle{Z}(Z\lrangle{b})]} T_1\para \enc{P_2} \DEF T$. So $T\SSEQV \enc{P_1'}\para \enc{P_2}\equiv \enc{P'}$. 

\item Internal action ($\tau$). The interesting case is when there is a communication between $P_1$ and $P_2$. Take, for example, the case $P_1\st{a(b)} P_1'$ and $P_2\st{\overline{a}(b)} P_2'$, and $P'\equiv (b)(P_1'\para P_2')$. By ind. hyp., $\enc{P_1} \st{a(\lrangle{Z}(Z\lrangle{b}))} T_1$ and $T_1\SSEQV \enc{P_1'}$, and $\enc{P_2} \st{(b)\overline{a}[\lrangle{Z}(Z\lrangle{b})]} T_2$ and $T_2\SSEQV \enc{P_2'}$. So $\enc{P}\equiv \enc{P_1}\para \enc{P_2} \st{\tau} (b)(T_1\para T_2) \DEF T$. Thus $\enc{P'} \equiv (b)(\enc{P_1'}\para \enc{P_2'}) \SSEQV (b)(T_1\para T_2) \equiv T$.
\end{itemize}

\item $P$ is $(c)P_1$.
\begin{itemize}
\item Input. Assume $P_1\st{a(b)} P_1'$, and $P\st{a(b)} (c)P_1'\equiv P'$. By ind. hyp., $\enc{P_1} \st{a(\lrangle{Z}(Z\lrangle{b}))} T_1$ and $T_1\SSEQV \enc{P_1'}$. Then $\enc{P}\equiv (c)\enc{P_1} \st{a(\lrangle{Z}(Z\lrangle{b}))} (c)T_1 \DEF T$. So $T\SSEQV (c)\enc{P_1'}\equiv \enc{P'}$. Moreover, $\enc{P_1} \st{a(\triggerD)} T_2$ and $(m)(T_2 \para !m(Y).Y\lrangle{b}) \WCB \enc{P_1'}$. Thus $\enc{P}\equiv (c)\enc{P_1} \st{a(\triggerD)} (c)T_2 \DEF T'$. Hence $(m)(T' \para !m(Y).Y\lrangle{b}) \equiv (m)((c)T_2 \para !m(Y).Y\lrangle{b}) \WCB (c)\enc{P_1'} \equiv \enc{P'}$.

\item Output. Assume $P_1\st{\overline{a}b} P_1'$, and $P\st{\overline{a}b} (c)P_1'\equiv P'$. By ind. hyp., $\enc{P_1} \st{\overline{a}[\lrangle{Z}(Z\lrangle{b})]} T_1$ and $T_1\SSEQV \enc{P_1'}$. Then $\enc{P}\equiv (c)\enc{P_1} \st{\overline{a}[\lrangle{Z}(Z\lrangle{b})]} (c)T_1 \DEF T$. So $T\SSEQV (c)\enc{P_1'}\equiv \enc{P'}$.

\item Bound output. The interesting case is $P_1\st{\overline{a}c} P_1'$, and $P\st{\overline{a}(c)} P'\equiv P_1'$. By ind. hyp., \\ $\enc{P_1} \st{\overline{a}[\lrangle{Z}(Z\lrangle{c})]} T_1$ and $T_1\SSEQV \enc{P_1'}$. So $\enc{P}\equiv (c)\enc{P_1} \st{(c)\overline{a}[\lrangle{Z}(Z\lrangle{c})]} T \equiv T_1 \SSEQV \enc{P'}$.

\item Internal action ($\tau$). Assume $P_1\st{\tau} P_1'$, and $P\st{\tau} (c)P_1'\equiv P'$. By ind. hyp., $\enc{P_1} \st{\tau} T_1$ and $T_1\SSEQV \enc{P_1'}$. Then $\enc{P}\equiv (c)\enc{P_1} \st{\tau} (c)T_1 \DEF T$. So $T\SSEQV (c)\enc{P_1'} \equiv \enc{P'}$.
\end{itemize}
\item $P$ is $!a(x).P_1$. 
\begin{itemize}
\item Input. $P\st{a(b)} P_1\fosub{b}{x}\para P \equiv P'$. So $\enc{P} \equiv !a(Y).Y\lrangle{\lrangle{x}\enc{P_1}} \st{a(\lrangle{Z}(Z\lrangle{b}))} \equiv \enc{P_1}\fosub{b}{x}\para \enc{P} \equiv T$. We then have $T \SSEQV \enc{P_1\fosub{b}{x}}\para \enc{P} \SSEQV \enc{P'}$. Moreover, $\enc{P} \st{a(\triggerD)} \overline{m}[\lrangle{x}\enc{P_1}]\para \enc{P} \equiv T$. We have 
\[
\begin{array}{ll}
& (m)(T \para !m(Y).Y\lrangle{b}) \\
\equiv & (m)(\overline{m}[\lrangle{x}\enc{P_1}]\para \enc{P} \para !m(Y).Y\lrangle{b}) \\
\WCB & (m)(!m(Y).Y\lrangle{b}\para \enc{P} \para \enc{P_1}\fosub{b}{x}) \\
\WCB & \enc{P_1}\fosub{b}{x})\para \enc{P} \SSEQV \enc{P'}
\end{array}
\]
\end{itemize}
\end{itemize}
\myqed
\end{proof}



% %\begin{proof}[Proof of Lemma \ref{l:opcor-conv}]
% \noindent\emph{Proof of Lemma \ref{l:opcor-conv}}~~ The proof proceeds by induction on the structure of $P$, %. By noticing that actions come from $\enc{P}$ instead of $P$, it 
% and is similar to the proof of Lemma \ref{l:opcor}. We thus skip the details.

%\sepp
%\noindent\greycolor{
%{Memo below; to remove: (treat 1-6 together or separately); Actually similar to Lemma \ref{l:opcor} (except that actions come from $\enc{P}$ instead of $P$); maybe simply SKIP.} \\
%The case $P$ is $0$ is trivial.  
%\begin{itemize}
%\item $P$ is $a(x).P_1$.
%\begin{itemize}
%\item Input.
%\end{itemize}
%\item $P$ is $\overline{a}b.P_1$.
%\begin{itemize}
%\item Output.
%\end{itemize}
%\item $P$ is $P_1\para P_2$.
%\begin{itemize}
%\item Input.
%\item Output.
%\item Bound output.
%\item $\tau$.
%\end{itemize}
%\item $P$ is $(c)P_1$.
%\begin{itemize}
%\item Input.
%\item Output.
%\item Bound output.
%\item $\tau$.
%\end{itemize}
%\item $P$ is $!a(x).P_1$.
%\begin{itemize}
%\item Input.
%\end{itemize}
%\end{itemize}
%}%greycolor

%\end{proof}










%---------------------------
% Local Variables:
% mode: LaTeX
% TeX-master: "main.tex"
% End:

% %proof for the operational correspondence for the variant encoding in Section \ref{s:encoding_variant}


\section{Proofs for Section \ref{s:encoding_variant}}\label{a:proofs-corres-others-var-encoding}

\subsection{Proof of the operational correspondence of the encoding in Section \ref{s:encoding_variant}}\label{a:proofs-corres-var-encoding}
We give the proof of Lemma \ref{l:opcor_var} in Section \ref{s:opcoores_var_encoding}.

\begin{proof}[Proof of Lemma \ref{l:opcor_var}]
We prove the clauses by induction on $P$. We caution that for the sake of simplicity, we always assume no name capture up-to $\alpha$-conversion.
%\nts{TODO (ref. Section \ref{s:encoding_operational})}
\begin{itemize}
\item $P$ is $0$. This is trivial.
\item $P$ is $a(x).P_1 \equiv C[a(x).P_1]$ in which $C\DEF [\cdot]$.
\begin{itemize}
\item Input. We have $P\st{a(b)} P_1\fosub{b}{x} \DEF P' \equiv C[P_1\fosub{b}{x}]$ and \\
$C[0]\para P_1\fosub{b}{x} \equiv P'$. Then $\enc{P} \equiv \overline{a}[\lrangle{x}\enc{P_1}].0 \st{\overline{a}[\lrangle{x}(\enc{P_1})]} 0 \DEF T$,  and $T\SSEQV \enc{C[0]} \equiv 0$.
\end{itemize}

\item $P$ is $\overline{a}b.P_1$.
\begin{itemize}
\item Output. We have $\enc{P} \equiv a(Y).(Y\lrangle{b}\para \enc{P_1}) \st{a(A)} A\lrangle{b}\para \enc{P_1} \DEF T$, as required.
\end{itemize}

\item $P$ is $P_1\para P_2$. %\nts{TODO (ref. Section \ref{s:encoding_operational})}
\begin{itemize}
\item Input. Suppose $P_1\st{a(b)} P_1'$ (the case $P_2$ does is similar) and \\
$P\st{a(b)} P_1'\para P_2 \equiv P'$. By ind. hyp., $P_1\SGB (\ve{e_1})C'[a(x).P_3]$ and $P_1'\equiv (\ve{e_1})C'[P_3\fosub{b}{x}] \equiv (\ve{e_1})(C'[0]\para P_3\fosub{b}{x})$ for some context $C'$, $P_3$ and $\ve{e_1}$ being the bound names shared between $P_3$ and its context within $P_1$, and $\enc{P_1} \st{(\ve{e_1})\overline{a}[\lrangle{x}(\enc{P_3})]} T_1\SSEQV \enc{C'[0]}$. Defining $C\DEF C'\para P_2$, we have $P\SGB (\ve{e_1})C[a(x).P_3]$ and $P'\equiv (\ve{e_1})C[P_3\fosub{b}{x}] \equiv (\ve{e_1})(C[0] \para P_3\fosub{b}{x})$. Moreover %with $\ve{e_2}$ being the bound names shared between $P_3$ and $P_2$. Moreover
\[
\begin{array}{lrl}
 &\enc{P}\equiv& \enc{P_1}\para \enc{P_2} \\
 &\st{(\ve{e_1})\overline{a}[\lrangle{x}(\enc{P_3})]}& T_1\para \enc{P_2} \DEF T \\
 &\SSEQV& \enc{C'[0]}\para \enc{P_2} \\
 &\equiv& \enc{C'[0]\para P_2} \equiv \enc{C[0]} %\quad \mbox{ in which }
\end{array}
\]

\item Output. Suppose $P_1 \st{\overline{a}b} P_1'$ (the case $P_2$ does is similar) and \\
$P\st{\overline{a}b} P_1'\para P_2 \equiv P'$. By ind. hyp., $\enc{P_1} \st{a(A)} T_1$ and $T_1\SSEQV A\lrangle{b} \para \enc{P_1'}$. Then we have
\[
\begin{array}{lrl}
 &\enc{P}\equiv& \enc{P_1}\para \enc{P_2} \\
 &\st{a(A)}& T_1\para \enc{P_2} \DEF T\\
 &\SSEQV& A\lrangle{b} \para \enc{P_1'} \para \enc{P_2} \\
 &\SSEQV& A\lrangle{b} \para \enc{P_1'\para P_2} \equiv A\lrangle{b} \para \enc{P'}
\end{array}
\]

\item Bound output. Suppose $P_1 \st{\overline{a}(b)} P_1'$ and $b\notin \fn{P_2}$, and $P\st{\overline{a}b} P_1'\para P_2 \equiv P'$ (the case $P_2$ does is similar). By ind. hyp., $\enc{P_1} \st{a(A)} T_1$ and $T_1\SSEQV (b)(A\lrangle{b} \para \enc{P_1'})$. Then we have
\[
\begin{array}{lrl}
 &\enc{P}\equiv& \enc{P_1}\para \enc{P_2} \\
 &\st{a(A)}& T_1\para \enc{P_2} \DEF T \\
 &\SSEQV& (b)(A\lrangle{b} \para \enc{P_1'}) \para \enc{P_2} \\
 &\SSEQV& (b)(A\lrangle{b} \para \enc{P_1'} \para \enc{P_2}) \\ %\quad \mbox{ due to } b\notin \fn{P_2} \mbox{ and name invariance } \\
 &\SSEQV& (b)(A\lrangle{b} \para \enc{P_1'\para P_2}) \\
 &\equiv& (b)(A\lrangle{b} \para \enc{P'})
\end{array}
\]

\item Interaction ($\tau$). The most interesting and hard case is when $P_1$ and $P_2$ engages a communication (of a bound name). Consider, for example, the case $P_1\st{a(b)} P_1'$ and $P_2\st{\overline{a}(b)} P_2'$ ($b\notin \fn{P_2}$ \xxxx{wolg}), and $P\st{\tau} P'\equiv (b)(P_1'\para P_2')$. By ind. hyp., $P_1\SGB (\ve{e_1})C'[a(x).P_3]$ and $P_1'\equiv (\ve{e_1})C'[P_3\fosub{b}{x}] \equiv (\ve{e_1})(C'[0] \para P_3\fosub{b}{x})$ for some context $C'$, $P_3$ and $\ve{e_1}$ being the bound names shared between $P_3$ and its context within $P_1$, and $\enc{P_1} \st{(\ve{e_1})\overline{a}[\lrangle{x}(\enc{P_3})]} T_1\SSEQV \enc{C'[0]}$; $\enc{P_2} \st{a(A)} T_2$ and $T_2\SSEQV (b)(A\lrangle{b} \para \enc{P_2'})$. Take $A$ as $\lrangle{x}(\enc{P_3})$, and we have
\[
\begin{array}{lrl}
 &\enc{P}\equiv& \enc{P_1}\para \enc{P_2} \\
 &\st{\tau}& (\ve{e_1})(T_1 \para T_2) \DEF T \\
 &\SSEQV& (\ve{e_1})(\enc{C'[0]} \para (b)((\lrangle{x}(\enc{P_3}))\lrangle{b} \para \enc{P_2'})) \\
 &\SSEQV& (\ve{e_1})(\enc{C'[0]} \para (b)(\enc{P_3}\fosub{b}{x} \para \enc{P_2'})) \\
 &\SSEQV& (\ve{e_1})(\enc{C'[0]} \para (b)(\enc{P_3\fosub{b}{x}} \para \enc{P_2'})) \\
 &\SSEQV& (b\ve{e_1})(\enc{C'[0]} \para \enc{P_3\fosub{b}{x}} \para \enc{P_2'}) \\
 &\SSEQV& (b\ve{e_1})(\enc{C'[0] \para P_3\fosub{b}{x} \para P_2'}) \\
 &\SSEQV& (b\ve{e_1})(\enc{C'[0] \para P_3\fosub{b}{x} \para P_2'}) \\
 &\SSEQV& (b)(\enc{(\ve{e_1})(C'[0] \para P_3\fosub{b}{x}) \para P_2'}) \\ %\nts{???} \\
 %&\SCB& (b)(\enc{(\ve{e_1})(C'[P_3\fosub{b}{x}]) \para P_2'}) \\
 &\SSEQV& (b)(\enc{P_1' \para P_2'})
\end{array}
\]
\end{itemize}

\item $P$ is $(c)P_1$. %\nts{TODO (ref. Section \ref{s:encoding_operational})}
\begin{itemize}
\item Input. %\nts{copy the input case of parallel composition to here and adjust...}
Suppose $P_1\st{a(b)} P_1'$  and $P\st{a(b)} (c)P_1' \equiv P'$. By ind. hyp., $P_1\SGB (\ve{e_1})C'[a(x).P_3]$ and $P_1'\equiv (\ve{e_1})C'[P_3\fosub{b}{x}] \equiv (\ve{e_1})(C'[0]\para P_3\fosub{b}{x})$ for some context $C'$, $P_3$ and $\ve{e_1}$ being the bound names shared between $P_3$ and its context within $P_1$, and $\enc{P_1} \st{(\ve{e_1})\overline{a}[\lrangle{x}(\enc{P_3})]} T_1\SSEQV \enc{C'[0]}$. There are two subcases.
\begin{itemize}
\item[(i)] If $c\in \fn{P_3}$, we set $C\DEF C'$ and have $P\SGB (\ve{e_1}c)C[a(x).P_3]$ and $P'\equiv (\ve{e_1}c)C[P_3\fosub{b}{x}]\equiv (\ve{e_1}c)(C[0] \para P_3\fosub{b}{x})$, and moreover
\[
\begin{array}{lrl}
 &\enc{P}\equiv& (c)\enc{P_1} \\
 &\st{(\ve{e_1}c)\overline{a}[\lrangle{x}(\enc{P_3})]}& T_1 \DEF T \\
 &\SSEQV& \enc{C'[0]} \equiv \enc{C[0]} %\quad \mbox{ in which }
\end{array}
\]
\item[(ii)] If $c\notin \fn{P_3}$, we set $C\DEF (c)C'$. Then we have $P\SGB (\ve{e_1})C[a(x).P_3]$ and $P'\equiv (\ve{e_1})C[P_3\fosub{b}{x}]\equiv (\ve{e_1})((c)C'[0] \para P_3\fosub{b}{x})$ \\ 
$\equiv (\ve{e_1})(C[0] \para P_3\fosub{b}{x})$, and moreover
\[
\begin{array}{lrl}
 &\enc{P}\equiv& (c)\enc{P_1} \\
 &\st{(\ve{e_1})\overline{a}[\lrangle{x}(\enc{P_3})]}& (c)T_1 \DEF T \\
 &\SSEQV& (c)\enc{C'[0]} \\
 &\equiv& \enc{(c)C'[0]} \equiv \enc{C[0]} %\quad \mbox{ in which }
\end{array}
\]
\end{itemize}

\item Output. %\nts{copy the output case of parallel composition to here and adjust...}
There are two subcases.
\begin{itemize}
\item[(i)] $P_1 \st{\overline{a}b} P_1'$ and $P\st{\overline{a}b} (c)P_1' \equiv P'$. By ind. hyp., $\enc{P_1} \st{a(A)} T_1\SSEQV A\lrangle{b} \para \enc{P_1'}$. Then we have
\[
\begin{array}{lrl}
 &\enc{P}\equiv& (c)\enc{P_1} \\
 &\st{a(A)}& (c)T_1 \DEF T\\
 &\SSEQV& (c)(A\lrangle{b} \para \enc{P_1'}) \\
 &\SSEQV& A\lrangle{b} \para (c)\enc{P_1'} \\
 &\equiv& A\lrangle{b} \para \enc{(c)P_1'} \equiv A\lrangle{b} \para \enc{P'}
\end{array}
\]
\item[(ii)] $P_1 \st{\overline{a}c} P_1'$ and $P\st{\overline{a}(c)} P_1' \equiv P'$. By ind. hyp.,  we know $\enc{P_1} \st{a(A)} T_1\SSEQV A\lrangle{c} \para \enc{P_1'}$. Then we have
\[
\begin{array}{lrl}
 &\enc{P}\equiv& (c)\enc{P_1} \\
 &\st{a(A)}& (c)T_1 \DEF T\\
 &\SSEQV& (c)(A\lrangle{c} \para \enc{P_1'}) \\
 &\equiv& (c)(A\lrangle{c} \para \enc{P'})
\end{array}
\]

\end{itemize}

\item Bound output. %\nts{copy the bound output case of parallel composition to here and adjust...}
Suppose $P_1 \st{\overline{a}(b)} P_1'$ and $P\st{\overline{a}(b)} (c)P_1' \equiv P'$. By ind. hyp., $\enc{P_1} \st{a(A)} T_1\SSEQV (b)(A\lrangle{b} \para \enc{P_1'})$. Then we have
\[
\begin{array}{lrl}
 &\enc{P}\equiv& (c)\enc{P_1} \\
 &\st{a(A)}& (c)T_1 \DEF T \\
 &\SSEQV& (c)(b)(A\lrangle{b} \para \enc{P_1'}) \\
 &\SSEQV& (b)(A\lrangle{b} \para (c)\enc{P_1'}) \\
 &\equiv& (b)(A\lrangle{b} \para \enc{(c)P_1'}) \\
 &\equiv& (b)(A\lrangle{b} \para \enc{P'})
\end{array}
\]

\item Interaction ($\tau$). Suppose $P_1\st{\tau} P_1'$ and $P\st{\tau} (c)P_1'\equiv P'$. By ind. hyp., $\enc{P_1} \st{\tau} T_1 \SSEQV \enc{P_1'}$. Then we have
\[
\begin{array}{lrl}
 &\enc{P}\equiv& (c)\enc{P_1} \\
 &\st{\tau}& (c)T_1 \DEF T \\
 &\SSEQV& (c)\enc{P_1'} \\
 &\equiv& \enc{(c)P_1'} \equiv \enc{P'}
\end{array}
\]
\end{itemize}

\item $P$ is $!a(x).P_1 \SGB C[a(x).P_1]$ in which $C\DEF [\cdot]\para !a(x).P_1$.
\begin{itemize}
\item Input. We have $P \st{a(b)} P_1\fosub{b}{x}\para !a(x).P_1 \DEF P' \equiv C[P_1\fosub{b}{x}]$ and $P' \equiv C[0]\para P_1\fosub{b}{x}$. Then $\enc{P} \equiv !\overline{a}[\lrangle{x}\enc{P_1}] \st{\overline{a}[\lrangle{x}(\enc{P_1})]} 0\para !\overline{a}[\lrangle{x}\enc{P_1}] \DEF T$, and $T\SSEQV \enc{C[0]}$.
\end{itemize}

\end{itemize}
\myqed
\end{proof}





\subsection{Proof of the completeness of the encoding in Section \ref{s:encoding_variant}}\label{a:proofs-completeness-var-encoding}


\begin{proof}[Proof of Lemma \ref{l:completeness_var}]
We show that relation $\mathcal{R}$ below is a local bisimulation up-to $\equiv$. %$\SGB$.
%>>>>>> This up-to technique for \FOPi\ processes is defined in a way similar to that in \HOPiDd. We note that up-to techniques for first-order processes are well-established in the field; more details can be referred to \cite{SW01a,PS11}.
\[
\mathcal{R} \DEF \{(P,Q) \,|\, \enc{P}\WCB \enc{Q} \} \,\cup\, \WGB
\]
Assume $P\,\mathcal{R}\, Q$ and $P\st{\lambda}P'$. We proceed by a case analysis.
%\nts{TODO (ref. Section \ref{s:encoding_completeness})}
%\nts{Standard bisimulation game arguments? }
\begin{itemize}
\item Input: $P\st{a(b)}P'$. %This is relatively the most involved case. We need to setup the context in \HOPiDd\ carefully so as to close the bisimulation game. \nts{TODO}
%By Lemma \ref{l:opcor_var},
%By Lemma \ref{l:opcor-conv_var},
By Lemma \ref{l:opcor_var}, we know that $P\SGB (\ve{e})C[a(x).P_1]$ and $P'\equiv (\ve{e})C[P_1\fosub{b}{x}] \equiv (\ve{e})(C[0]\para P_1\fosub{b}{x})$ for some context $C$, $P_1$ and $\ve{e}$ being the bound names shared between $P_1$ and its context, and $\enc{P} \st{(\ve{e})\overline{a}[\lrangle{x}(\enc{P_1})]} T\SSEQV \enc{C[0]}$. Since $\enc{P}\WCB \enc{Q}$, we know that $\enc{Q}\wt{(\ve{e'})\overline{a}[\lrangle{x}(\enc{Q_1})]} T'$ and for every $E[X]$ with $\fn{E}\cap \ve{e}\ve{e'} {=} \emptyset$, it holds that
\begin{equation}\label{eq:com_var_1}
(\ve{e})(E[\lrangle{x}(\enc{P_1})] \para T) \,\WCB\, (\ve{e'})(E[\lrangle{x}(\enc{Q_1})] \para T')
\end{equation}
By Lemma \ref{l:opcor-conv_var}, we know that $Q \wt{a(b)} Q'$ with $Q\SGB (\ve{e'})C'[a(x).Q_1]$ and $Q'\equiv (\ve{e'})C'[Q_1\fosub{b}{x}] \equiv (\ve{e'})(C'[0]\para Q_1\fosub{b}{x})$ for some context $C'$ with $\ve{e}$ being the bound names shared between $Q_1$ and its context, and
$\enc{C'[0]} \wt{} T'' \WCB T'$, where we notice that there might be a few $\tau$'s distance between $\enc{C'[0]}$ and $T'$ due to the weak transition by $\enc{Q}$. Still by Lemma \ref{l:opcor-conv_var}, we know that there exists some context $C''$ s.t. $C'[0] \wt{} C''[0]$ with $\enc{C''[0]}\WCB T'' \WCB T'$, and as such we also know that $Q' \wt{} (\ve{e'})(C''[0]\para Q_1\fosub{b}{x}) \DEF Q''$, so $Q \wt{a(b)} Q''$. We need to prove $P' \mathcal{R} Q''$, i.e., $\enc{P'} \WCB \enc{Q''}$. Through setting $E[X]$ as $X\lrangle{b}$, equation (\ref{eq:com_var_1}) turns into
\[
(\ve{e})((\lrangle{x}(\enc{P_1}))\lrangle{b} \para T) \,\WCB\, (\ve{e'})((\lrangle{x}(\enc{Q_1}))\lrangle{b} \para T')
\] which is exactly
\[
(\ve{e})(\enc{P_1\fosub{b}{x}} \para T) \,\WCB\, (\ve{e'})(\enc{Q_1\fosub{b}{x}} \para T')
\] Since we already know that $T\SSEQV \enc{C[0]}$ and $T'\WCB \enc{C''[0]}$, we have as needed
\[
\enc{P'}\;\equiv\; (\ve{e})(\enc{P_1\fosub{b}{x}} \para \enc{C[0]}) \,\WCB\, (\ve{e'})(\enc{Q_1\fosub{b}{x}} \para \enc{C''[0]}) \;\equiv\; \enc{Q''}
\]


\item Output: $P\st{\overline{a}b}P'$. %\nts{TODO}
%By Lemma \ref{l:opcor_var},
%By Lemma \ref{l:opcor-conv_var},
By Lemma \ref{l:opcor_var}, $\enc{P} \st{a(A)} T \SSEQV A\lrangle{b} \para \enc{P'}$. Since $\enc{P}\WCB \enc{Q}$, we know that $\enc{Q} \wt{a(A)} T' \WCB T$. By Lemma \ref{l:opcor-conv_var}, it must be that $Q \wt{\overline{a}b} Q'$ (otherwise $\enc{P}$ and $\enc{Q}$ would be distinguishable, \xxx{similar to the case in Lemma \ref{l:completeness}}), and 
$A\lrangle{b} \para \enc{Q'} \wt{} T'' \WCB T'$. Setting $A$ to be $\lrangle{x}0$, we have
\begin{eqnarray}
\enc{P} \st{a(\lrangle{x}0)} T \SSEQV 0 \para \enc{P'} \SSEQV \enc{P'} \nonumber \\ %\label{eq:com_var_2} \\
\enc{Q'} \equiv 0 \para \enc{Q'} \wt{} T'' \WCB T' \label{eq:com_var_3}
\end{eqnarray}
From (\ref{eq:com_var_3}) and Lemma \ref{l:opcor-conv_var}, we know that $Q'\wt{} Q''$ and $\enc{Q''} \WCB T'' \WCB T'$. So we have $Q\wt{\overline{a}b} Q''$ and need to prove $P' \mathcal{R} Q''$, i.e., $\enc{P'} \WCB \enc{Q''}$. This is immediate from the following equations.
\[
\enc{P'} \;\WCB\; T \;\WCB\; T' \;\WCB\; T'' \;\WCB\; \enc{Q''}
\]


\item Bound output: $P\st{\overline{a}(b)}P'$. %In contrast to the input case, one has to collude with the inputted process to the encoding in order to make the bisimulation closed. \nts{Use the local bisimulation's corresponding clause; TODO}
%By Lemma \ref{l:opcor_var},
%By Lemma \ref{l:opcor-conv_var},
By Lemma \ref{l:opcor_var}, $\enc{P}$ can make an input over $a$; we set the input as the trigger $\triggerd \equiv \lrangle{z}\overline{m}[\lrangle{Y}(Y\lrangle{z})]$ ($m$ fresh). Then we have $\enc{P} \st{a(\triggerd)} T \SSEQV (b)(\triggerd\lrangle{b} \para \enc{P'})$. Since $\enc{P}\WCB \enc{Q}$, we know that $\enc{Q} \wt{a(\triggerd)} T' \WCB T$. By Lemma \ref{l:opcor-conv_var}, it must be that $Q \wt{\overline{a}(b)} Q'$ (otherwise $\enc{P}$ and $\enc{Q}$ would be distinguishable, \xxx{as in the last case}), and $(b)(\triggerd\lrangle{b} \para \enc{Q'}) \wt{} T'' \WCB T'$. 
Thus we have $\enc{Q'} \wt{} T'''$ and \\
$(b)(\triggerd\lrangle{b} \para \enc{Q'})\wt{} (b)(\triggerd\lrangle{b} \para T''') \WCB T'' \WCB T'$. By Lemma \ref{l:opcor-conv_var}, we know that $Q'\wt{} Q''$ and $\enc{Q''} \WCB T'''$. %$\enc{Q''} \WCB T'''$.
So we have $Q \wt{\overline{a}(b)} Q''$ and \\
$(b)(\triggerd\lrangle{b} \para \enc{Q''}) \WCB T'' \WCB T'$. The above boils down to the following equation.
\[
(b)(\triggerd\lrangle{b} \para \enc{P'}) \;\WCB\; T\;\WCB\; T'\;\WCB\; T'' \;\WCB\; (b)(\triggerd\lrangle{b} \para \enc{Q''})
\]
Since $\WCB$ is a congruence, we can derive that
\[
\begin{array}{l}
(m)(\;(b)(\triggerd\lrangle{b} \para \enc{P'})\; \para !m(Z).Z\lrangle{A}) \\
\qquad\qquad\qquad\qquad\qquad\qquad \;\WCB\; (m)(\;(b)(\triggerd\lrangle{b} \para \enc{Q''})\; \para !m(Z).Z\lrangle{A})
\end{array}
\]
Using Theorem \ref{factor-bigd-smalld}, we have
\begin{equation}\label{eq:com_var_4}
(b)(\enc{P'}\para A\lrangle{b}) \;\WCB\; (b)(\enc{Q''} \para A\lrangle{b})
\end{equation}
We now recall that since $P\st{\overline{a}(b)}P'$ and $Q \wt{\overline{a}(b)} Q''$, our goal is to prove that for every $R$ it holds
\[
(b)(P' \para R) \;\mathcal{R}\; (b)(Q'' \para R)
\] That is,
\begin{equation}\label{eq:com_var_5}
(b)(\enc{P'} \para \enc{R}) \;\WCB\; (b)(\enc{Q''} \para \enc{R})
\end{equation}
Comparing (\ref{eq:com_var_4}) and (\ref{eq:com_var_5}), we can observe that since the image of the encoding is contained by the target model \HOPiDd\ and $A$ is arbitrary (it has the knowledge of $b$ as well), if we traverse all possibility of $A$ then $\enc{R}$ must be hit somewhere. We thus conclude that (\ref{eq:com_var_5}) is true and the simulation is closed.

\item $\tau$ action: $P\st{\tau}P'$. %\nts{TODO}
%By Lemma \ref{l:opcor_var},
%By Lemma \ref{l:opcor-conv_var},
By Lemma \ref{l:opcor_var}, $\enc{P} \st{\tau} T \SSEQV \enc{P'}$. Since $\enc{P}\WCB \enc{Q}$, we know that $\enc{Q} \wt{} T' \WCB T$. By Lemma \ref{l:opcor-conv_var}, $Q \wt{} Q'$ and $T'\WCB \enc{Q'}$. %$T'\WCB \enc{Q'}$
We need to prove $P'\,\mathcal{R}\, Q'$, i.e., $\enc{P'} \,\WCB\, \enc{Q'}$. This is straightforward because
\[
\enc{P'} \;\WCB\; T \;\WCB\; T' \;\WCB\; \enc{Q'}
\]
\end{itemize}
The proof is now completed.
\myqed
\end{proof}









%---------------------------
% Local Variables:
% mode: LaTeX
% TeX-master: "main.tex"
% End:

% \clearpage
% \section{Proofs for Section \ref{s:normal}}\label{a:proofs-normal}
We give the proof for Theorem \ref{normal-characterization-hopiDd}.
\begin{proof}[Proof of Theorem \ref{normal-characterization-hopiDd}]
We only need to prove $\WNB$ implies $\WCB$, since $\WCB$ is obviously finer than $\WNB$. 
To do this, we define the relation $\mathcal{R}$ below and show that it is a context bisimulation. % (\stress{up-to sth?}). 
\[
\mathcal{R} \DEF \{(P,Q) \,|\, P\WNB Q \}
\]
Suppose $P\mathcal{R} Q$. There are several cases (with subcases) to analyze. We focus on the main ones, and leave out the similar (and simpler) rest.

%\stress{TODO}
\begin{itemize}
\item $P\st{a(A)} P'$. There are three subcases.
\begin{itemize}
\item $A$ is an abstraction on name. %, i.e., $A\equiv \lrangle{x}B$. 
Then $P'$ must be of the form $E[A]$ for some $E[Z]$ (one can choose such an $E$ that $Z$ is instantiated by the incoming $A$; similar for the other two subcases). So $P\st{a(\triggerd)} E[\triggerd]$. Since $P\WNB Q$, we know $Q\wt{a(\triggerd)} Q'' \equiv F[\triggerd]$ for some $F[Z]$ and $E[\triggerd] \WNB F[\triggerd]$. Then $Q\wt{a(A)} Q' \equiv F[A]$. Due to the congruence property of $\WNB$, the factorization theorem (Theorem \ref{factor-bigd-smalld}), and the fact that $\WCB$ implies $\WNB$, we have
\[
P'\equiv E[A] \,\WNB\, (m)(E[\triggerd] \para  !m(Z).Z\lrangle{A}) \,\WNB\, (m)(F[\triggerd] \para  !m(Z).Z\lrangle{A}) \,\WNB\, F[A]\equiv Q'
\] So $P' \mathcal{R} Q'$.

\item $A$ is an abstraction on process. %, i.e., $A\equiv \lrangle{X}B$. 
Similar to the last subcase, except that $\triggerD$ is used instead. %Then ...\stress{TODO}
\item $A$ is not an abstraction. Similar to the first subcase, except that $\trigger$ is used. %Then ...\stress{TODO}
\end{itemize}

\item $P\st{(\ve{c})\overline{a}A} P'$. There are again three subcases.
\begin{itemize}
\item $A$ is an abstraction on name. %, i.e., $A\equiv \lrangle{x}B$. 
Because $P\WNB Q$, $Q\wt{(\ve{d})\overline{a}B} Q'$ and it holds that ($m$ is fresh)
\[(\ve{c})(P'\para !m(Z).Z\lrangle{A}) \,\WNB\,  (\ve{d})(Q'\para  !m(Z).Z\lrangle{B})
\] Using the congruence property of $\WNB$ twice and some structural manipulation, for every $E[X]$ ($\fn{E[X]}\cap \ve{c}\ve{d} = \emptyset$), we have
\[(\ve{c})(P'\para !m(Z).Z\lrangle{A} \para E[\triggerd]) \,\WNB\,  (\ve{d})(Q'\para  !m(Z).Z\lrangle{B} \para E[\triggerd])
\] and then
\[(\ve{c})(P'\para (m)(!m(Z).Z\lrangle{A} \para E[\triggerd])) \,\WNB\,  (\ve{d})(Q'\para  (m)(!m(Z).Z\lrangle{B} \para E[\triggerd]))
\] Now by the factorization theorem (Theorem \ref{factor-bigd-smalld}) and the fact that $\WCB$ implies $\WNB$, we know
\[(\ve{c})(P'\para E[A]) \,\WNB\,  (\ve{d})(Q'\para  E[B])
\] and thus 
\[(\ve{c})(P'\para E[A]) \,\mathcal{R}\,  (\ve{d})(Q'\para  E[B])
\] as required by context bisimulation.
 
%...\stress{TODO}
\item $A$ is an abstraction on process. %, i.e., $A\equiv \lrangle{X}B$. 
Similar to the last subcase, except that $\triggerD$ is used.  %Then ...\stress{TODO}
\item $A$ is not an abstraction. Similar to the first subcase, except that $\trigger$ is used. %Then ...\stress{TODO}
\end{itemize}

\item $P\st{\tau} P'$. Then we immediately have $Q\wt{} Q'$ and $P'\WNB Q'$, so $P'\mathcal{R} Q'$.
\end{itemize}



\end{proof}













%---------------------------
% Local Variables:
% mode: LaTeX
% TeX-master: "main.tex"
% End:

% \clearpage
% %}{%
%   %no appendixing
% %}
% }%normalsize

\sepp
\clearpage
\appendix

\noindent\textbf{\Large Appendix}
%\sepp
\sepp

{%\normalsize

\noindent\textit{Remark}. For the sake of conciseness, in the arguments %in the appendix, 
we sometimes omit the existential statement such as ``for some ..." if clear from context.

%
\section{\xxx{Proofs for Section \ref{s:preliminary}}}\label{appendix:up-to-context_general}

We give the proof of Theorem \ref{thm:sound_up-to_context_general}.

\begin{proof}[Proof of Theorem \ref{thm:sound_up-to_context_general}]
Assume $\R$ is as defined in Definition \ref{d:up2}. %We show that the relation $\R'$ defined below is a context bisimulation \xxx{up-to $\equiv$}, thus proving the proposition.
We define the relations $\R_n$ and $\R'$ as follows. Recall that $\mathbb{N}$ is the set of natural numbers.
\[
\begin{array}{lcl}
\R_0 &\DEF& \R \\
\R_{n+1} &\DEF& \left\{(C[M],C[N]) \,|\, M\,\R_{n}\, N \mbox{ and } C\in \mathcal{D} \right\} \\ %$\vartheta.[\cdot]$,
\R' &\DEF& \bigcup_{i\in \mathbb{N}} \R_i \\\\
\mathcal{D} &\DEF& \left\{
\begin{array}{lllll}
%[\cdot],  & 
a(X).[\cdot],\quad & \overline{a}([\cdot]).R,\quad & \overline{a}A_1.[\cdot],\quad & \lrangle{X}[\cdot], \quad & \lrangle{x}[\cdot],\\~
[\cdot]\lrangle{A_1}, & [\cdot]\lrangle{d}, & R\para [\cdot], & (d)[\cdot] &
\end{array}
\right\}
\end{array}
\]
We prove by induction on $n$ that $\R'$ is a context bisimulation up-to $\equiv$. Since $\R\subseteq \R'$, $\R\subseteq \WCB$ follows.

%-------------PROOF BODY BEGIN-------------
\xx{TODO: to fetch from `NOTES'}
%-------------PROOF BODY END-------------
\end{proof}

\clearpage
\input{appendix_proof_upto_enc_and_opcorres.tex}
\clearpage
%%\section{Proofs for Section \ref{s:encoding}}\label{a:proofs-encoding}

\begin{proof}[Proof of Lemma \ref{l:syn-pro-encoding}]
It is straightforward to check that the encoding is compositional since the designated contexts are easy to capture. For the core of the encoding, the contexts for input and output are respectively $m(Y).Y\lrangle{\lrangle{x}[\cdot]}$ and $\overline{m}[\lrangle{Z}(Z\lrangle{n})].[\cdot]$. Also the encoding is divergence-reflecting, for the reason that it does not bring about any divergence. As such, it is a simple induction, on the rules deriving $\equiv$, to show that the encoding preserves structural congruence. Below we prove by induction on the structure of $P$ that $\enc{P}\sigma \equiv \enc{P\sigma}$ in which $\sigma$ is a substitution (recall that $\sigma$ is a mapping on names).
%\oo{\large \fbox{\#\#\#\# TODO}}
\begin{itemize}
\item $P$ is $0$. This is trivial.
\item $P$ is $m(x).Q$. Then %m(Y).Y\lrangle{\lrangle{x}\enc{Q}}
\[
\begin{array}{lcll}
\enc{P}\sigma &\equiv& (m(Y).Y\lrangle{\lrangle{x}\enc{Q}})\sigma & \quad \\
 &\equiv& m'(Y).Y\lrangle{\lrangle{x}(\enc{Q}\sigma)} & \quad m' \mbox{ is } \sigma(m) \\
 &\equiv& m'(Y).Y\lrangle{\lrangle{x}(\enc{Q\sigma})} & \quad \mbox{ind. hyp. (short for induction hypothesis)}\\
 &\equiv& \enc{m'(x).Q\sigma} & \quad \\
 &\equiv& \enc{(m(x).Q)\sigma} & \quad \\ 
 &\equiv& \enc{P\sigma} & \quad
\end{array}
\]
\item $P$ is $\overline{m}n.Q$. Then %\overline{m}[\lrangle{Z}(Z\lrangle{n})].\enc{Q}
\[
\begin{array}{lcll}
\enc{P}\sigma &\equiv& (\overline{m}[\lrangle{Z}(Z\lrangle{n})].\enc{Q})\sigma & \quad \\
 &\equiv& \overline{m'}[\lrangle{Z}(Z\lrangle{n'})].(\enc{Q}\sigma) & \quad m',n' \mbox{ are respectively } \sigma(m),\sigma(n) \\
 &\equiv& \overline{m'}[\lrangle{Z}(Z\lrangle{n'})].(\enc{Q\sigma}) & \quad \mbox{ind. hyp.} \\
 &\equiv& \enc{\overline{m'}n'.(Q\sigma)} & \quad \\
 &\equiv& \enc{(\overline{m}n.Q)\sigma} & \quad \\
 &\equiv& \enc{P\sigma} & \quad
\end{array}
\]
\item $P$ is $(c)Q$. Then 
\[
\begin{array}{lcll}
\enc{P}\sigma &\equiv& ((c)\enc{Q})\sigma & \quad  \\
 &\equiv& (c)\enc{Q}\sigma & \quad  \\
 &\equiv& (c)\enc{Q\sigma} & \quad \mbox{ind. hyp.} \\
 &\equiv& \enc{(c)(Q\sigma)} & \quad  \\
 &\equiv& \enc{((c)Q)\sigma} & \quad  \\
 &\equiv& \enc{P\sigma} & \quad  
\end{array}
\]
\item $P$ is $Q\para R$. Then 
\[
\begin{array}{lcll}
\enc{P}\sigma &\equiv& (\enc{Q}\para \enc{R})\sigma  & \quad \\
 &\equiv& \enc{Q}\sigma\para \enc{R}\sigma  & \quad \\
 &\equiv& \enc{Q\sigma}\para \enc{R\sigma}  & \quad \mbox{ind. hyp.} \\ 
 &\equiv& \enc{Q\sigma \para R\sigma}  & \quad  \\ 
 &\equiv& \enc{(Q\para R)\sigma}  & \quad  \\ 
 &\equiv& \enc{P\sigma}  & \quad 
\end{array}
\]
\item $P$ is $!m(x).Q$. Then %!\enc{m(x).Q}
\[
\begin{array}{lcll}
\enc{P}\sigma &\equiv& (!\enc{m(x).Q})\sigma & \quad \\
 &\equiv& !(\enc{m(x).Q})\sigma & \quad \\
 &\equiv& !(\enc{(m(x).Q)\sigma}) & \quad \mbox{similar to the input case} \\
 &\equiv& \enc{!((m(x).Q)\sigma)} & \quad \\
 &\equiv& \enc{(!m(x).Q)\sigma} & \quad \\
 &\equiv& \enc{P\sigma} & \quad 
\end{array}
\]
\end{itemize}
\myqed
\end{proof}


\begin{proof}[Proof of Lemma \ref{l:opcor}]
We prove the clauses by induction on the structure of $P$. The case $P$ is $0$ is trivial.  %\stress{TODO (treat 1-6 together or separately)}
\begin{itemize}
\item $P$ is $a(x).P_1$.
\begin{itemize}
\item Input. $P\st{a(b)} P_1\fosub{b}{x} \equiv P'$. So $\enc{P} \equiv a(Y).Y\lrangle{\lrangle{x}\enc{P_1}} \st{a(\lrangle{Z}(Z\lrangle{b}))} \equiv \enc{P_1}\fosub{b}{x} \equiv T$. We then have $T \SSEQV \enc{P_1\fosub{b}{x}} \SSEQV \enc{P'}$. Moreover, $\enc{P} \st{a(\triggerD)} \overline{m}[\lrangle{x}\enc{P_1}] \equiv T$. We have 
\[
\begin{array}{ll}
& (m)(T \para !m(Y).Y\lrangle{b}) \\
\equiv & (m)(\overline{m}[\lrangle{x}\enc{P_1}] \para !m(Y).Y\lrangle{b}) \\
\WCB & (m)(!m(Y).Y\lrangle{b} \para \enc{P_1}\fosub{b}{x}) \\
\WCB & \enc{P_1}\fosub{b}{x}) \SSEQV \enc{P'}
\end{array}
\]
\end{itemize}

\item $P$ is $\overline{a}b.P_1$.
\begin{itemize}
\item Output. $P \st{\overline{a}b} P_1\equiv P'$. Then $\enc{P} \equiv \overline{a}[\lrangle{Z}(Z\lrangle{b})].\enc{P_1} \st{\overline{a}[\lrangle{Z}(Z\lrangle{b})]} \enc{P_1} \equiv T$, so $T\SSEQV \enc{P'}$.
\end{itemize}

\item $P$ is $P_1\para P_2$.
\begin{itemize}
\item Input. Assume $P_1\st{a(b)} P_1'$, and $P\st{a(b)} P_1'\para P_2 \equiv P'$. By ind. hyp., $\enc{P_1} \st{a(\lrangle{Z}(Z\lrangle{b}))} T_1$ and $T_1\SSEQV \enc{P_1'}$. Then $\enc{P}\equiv \enc{P_1}\para \enc{P_2} \st{a(\lrangle{Z}(Z\lrangle{b}))} T_1\para \enc{P_2} \DEF T$. So $T\SSEQV \enc{P_1'}\para \enc{P_2} \equiv \enc{P'}$. Moreover, $\enc{P_1} \st{a(\triggerD)} T_2$ and $(m)(T_2 \para !m(Y).Y\lrangle{b}) \WCB \enc{P_1'}$. Thus $\enc{P}\equiv \enc{P_1}\para \enc{P_2} \st{a(\triggerD)} T_2\para \enc{P_2} \DEF T'$. Hence $(m)(T' \para !m(Y).Y\lrangle{b}) \equiv (m)(T_2\para \enc{P_2} \para !m(Y).Y\lrangle{b}) \WCB \enc{P_1'} \para \enc{P_2} \equiv \enc{P'}$.

\item Output. Assume $P_1\st{\overline{a}b} P_1'$, and $P\st{\overline{a}b} P_1'\para P_2 \equiv P'$. By ind. hyp., $\enc{P_1} \st{\overline{a}[\lrangle{Z}(Z\lrangle{b})]} T_1$ and $T_1\SSEQV \enc{P_1'}$. Then $\enc{P}\equiv \enc{P_1}\para \enc{P_2} \st{\overline{a}[\lrangle{Z}(Z\lrangle{b})]} T_1\para \enc{P_2} \DEF T$. So $T\SSEQV \enc{P_1'}\para \enc{P_2}\equiv \enc{P'}$. 

\item Bound output. Assume $P_1\st{\overline{a}(b)} P_1'$, and $P\st{\overline{a}(b)} P_1'\para P_2 \equiv P'$. By ind. hyp., $\enc{P_1} \st{(b)\overline{a}[\lrangle{Z}(Z\lrangle{b})]} T_1$ and $T_1\SSEQV \enc{P_1'}$. Then $\enc{P}\equiv \enc{P_1}\para \enc{P_2} \st{(b)\overline{a}[\lrangle{Z}(Z\lrangle{b})]} T_1\para \enc{P_2} \DEF T$. So $T\SSEQV \enc{P_1'}\para \enc{P_2}\equiv \enc{P'}$. 

\item Internal action ($\tau$). The interesting case is when there is a communication between $P_1$ and $P_2$. Take, for example, the case $P_1\st{a(b)} P_1'$ and $P_2\st{\overline{a}(b)} P_2'$, and $P'\equiv (b)(P_1'\para P_2')$. By ind. hyp., $\enc{P_1} \st{a(\lrangle{Z}(Z\lrangle{b}))} T_1$ and $T_1\SSEQV \enc{P_1'}$, and $\enc{P_2} \st{(b)\overline{a}[\lrangle{Z}(Z\lrangle{b})]} T_2$ and $T_2\SSEQV \enc{P_2'}$. So $\enc{P}\equiv \enc{P_1}\para \enc{P_2} \st{\tau} (b)(T_1\para T_2) \DEF T$. Thus $\enc{P'} \equiv (b)(\enc{P_1'}\para \enc{P_2'}) \SSEQV (b)(T_1\para T_2) \equiv T$.
\end{itemize}

\item $P$ is $(c)P_1$.
\begin{itemize}
\item Input. Assume $P_1\st{a(b)} P_1'$, and $P\st{a(b)} (c)P_1'\equiv P'$. By ind. hyp., $\enc{P_1} \st{a(\lrangle{Z}(Z\lrangle{b}))} T_1$ and $T_1\SSEQV \enc{P_1'}$. Then $\enc{P}\equiv (c)\enc{P_1} \st{a(\lrangle{Z}(Z\lrangle{b}))} (c)T_1 \DEF T$. So $T\SSEQV (c)\enc{P_1'}\equiv \enc{P'}$. Moreover, $\enc{P_1} \st{a(\triggerD)} T_2$ and $(m)(T_2 \para !m(Y).Y\lrangle{b}) \WCB \enc{P_1'}$. Thus $\enc{P}\equiv (c)\enc{P_1} \st{a(\triggerD)} (c)T_2 \DEF T'$. Hence $(m)(T' \para !m(Y).Y\lrangle{b}) \equiv (m)((c)T_2 \para !m(Y).Y\lrangle{b}) \WCB (c)\enc{P_1'} \equiv \enc{P'}$.

\item Output. Assume $P_1\st{\overline{a}b} P_1'$, and $P\st{\overline{a}b} (c)P_1'\equiv P'$. By ind. hyp., $\enc{P_1} \st{\overline{a}[\lrangle{Z}(Z\lrangle{b})]} T_1$ and $T_1\SSEQV \enc{P_1'}$. Then $\enc{P}\equiv (c)\enc{P_1} \st{\overline{a}[\lrangle{Z}(Z\lrangle{b})]} (c)T_1 \DEF T$. So $T\SSEQV (c)\enc{P_1'}\equiv \enc{P'}$.

\item Bound output. The interesting case is $P_1\st{\overline{a}c} P_1'$, and $P\st{\overline{a}(c)} P'\equiv P_1'$. By ind. hyp., \\ $\enc{P_1} \st{\overline{a}[\lrangle{Z}(Z\lrangle{c})]} T_1$ and $T_1\SSEQV \enc{P_1'}$. So $\enc{P}\equiv (c)\enc{P_1} \st{(c)\overline{a}[\lrangle{Z}(Z\lrangle{c})]} T \equiv T_1 \SSEQV \enc{P'}$.

\item Internal action ($\tau$). Assume $P_1\st{\tau} P_1'$, and $P\st{\tau} (c)P_1'\equiv P'$. By ind. hyp., $\enc{P_1} \st{\tau} T_1$ and $T_1\SSEQV \enc{P_1'}$. Then $\enc{P}\equiv (c)\enc{P_1} \st{\tau} (c)T_1 \DEF T$. So $T\SSEQV (c)\enc{P_1'} \equiv \enc{P'}$.
\end{itemize}
\item $P$ is $!a(x).P_1$. 
\begin{itemize}
\item Input. $P\st{a(b)} P_1\fosub{b}{x}\para P \equiv P'$. So $\enc{P} \equiv !a(Y).Y\lrangle{\lrangle{x}\enc{P_1}} \st{a(\lrangle{Z}(Z\lrangle{b}))} \equiv \enc{P_1}\fosub{b}{x}\para \enc{P} \equiv T$. We then have $T \SSEQV \enc{P_1\fosub{b}{x}}\para \enc{P} \SSEQV \enc{P'}$. Moreover, $\enc{P} \st{a(\triggerD)} \overline{m}[\lrangle{x}\enc{P_1}]\para \enc{P} \equiv T$. We have 
\[
\begin{array}{ll}
& (m)(T \para !m(Y).Y\lrangle{b}) \\
\equiv & (m)(\overline{m}[\lrangle{x}\enc{P_1}]\para \enc{P} \para !m(Y).Y\lrangle{b}) \\
\WCB & (m)(!m(Y).Y\lrangle{b}\para \enc{P} \para \enc{P_1}\fosub{b}{x}) \\
\WCB & \enc{P_1}\fosub{b}{x})\para \enc{P} \SSEQV \enc{P'}
\end{array}
\]
\end{itemize}
\end{itemize}
\myqed
\end{proof}



% %\begin{proof}[Proof of Lemma \ref{l:opcor-conv}]
% \noindent\emph{Proof of Lemma \ref{l:opcor-conv}}~~ The proof proceeds by induction on the structure of $P$, %. By noticing that actions come from $\enc{P}$ instead of $P$, it 
% and is similar to the proof of Lemma \ref{l:opcor}. We thus skip the details.

%\sepp
%\noindent\greycolor{
%{Memo below; to remove: (treat 1-6 together or separately); Actually similar to Lemma \ref{l:opcor} (except that actions come from $\enc{P}$ instead of $P$); maybe simply SKIP.} \\
%The case $P$ is $0$ is trivial.  
%\begin{itemize}
%\item $P$ is $a(x).P_1$.
%\begin{itemize}
%\item Input.
%\end{itemize}
%\item $P$ is $\overline{a}b.P_1$.
%\begin{itemize}
%\item Output.
%\end{itemize}
%\item $P$ is $P_1\para P_2$.
%\begin{itemize}
%\item Input.
%\item Output.
%\item Bound output.
%\item $\tau$.
%\end{itemize}
%\item $P$ is $(c)P_1$.
%\begin{itemize}
%\item Input.
%\item Output.
%\item Bound output.
%\item $\tau$.
%\end{itemize}
%\item $P$ is $!a(x).P_1$.
%\begin{itemize}
%\item Input.
%\end{itemize}
%\end{itemize}
%}%greycolor

%\end{proof}










%---------------------------
% Local Variables:
% mode: LaTeX
% TeX-master: "main.tex"
% End:

%proof for the operational correspondence for the variant encoding in Section \ref{s:encoding_variant}


\section{Proofs for Section \ref{s:encoding_variant}}\label{a:proofs-corres-others-var-encoding}

\subsection{Proof of the operational correspondence of the encoding in Section \ref{s:encoding_variant}}\label{a:proofs-corres-var-encoding}
We give the proof of Lemma \ref{l:opcor_var} in Section \ref{s:opcoores_var_encoding}.

\begin{proof}[Proof of Lemma \ref{l:opcor_var}]
We prove the clauses by induction on $P$. We caution that for the sake of simplicity, we always assume no name capture up-to $\alpha$-conversion.
%\nts{TODO (ref. Section \ref{s:encoding_operational})}
\begin{itemize}
\item $P$ is $0$. This is trivial.
\item $P$ is $a(x).P_1 \equiv C[a(x).P_1]$ in which $C\DEF [\cdot]$.
\begin{itemize}
\item Input. We have $P\st{a(b)} P_1\fosub{b}{x} \DEF P' \equiv C[P_1\fosub{b}{x}]$ and \\
$C[0]\para P_1\fosub{b}{x} \equiv P'$. Then $\enc{P} \equiv \overline{a}[\lrangle{x}\enc{P_1}].0 \st{\overline{a}[\lrangle{x}(\enc{P_1})]} 0 \DEF T$,  and $T\SSEQV \enc{C[0]} \equiv 0$.
\end{itemize}

\item $P$ is $\overline{a}b.P_1$.
\begin{itemize}
\item Output. We have $\enc{P} \equiv a(Y).(Y\lrangle{b}\para \enc{P_1}) \st{a(A)} A\lrangle{b}\para \enc{P_1} \DEF T$, as required.
\end{itemize}

\item $P$ is $P_1\para P_2$. %\nts{TODO (ref. Section \ref{s:encoding_operational})}
\begin{itemize}
\item Input. Suppose $P_1\st{a(b)} P_1'$ (the case $P_2$ does is similar) and \\
$P\st{a(b)} P_1'\para P_2 \equiv P'$. By ind. hyp., $P_1\SGB (\ve{e_1})C'[a(x).P_3]$ and $P_1'\equiv (\ve{e_1})C'[P_3\fosub{b}{x}] \equiv (\ve{e_1})(C'[0]\para P_3\fosub{b}{x})$ for some context $C'$, $P_3$ and $\ve{e_1}$ being the bound names shared between $P_3$ and its context within $P_1$, and $\enc{P_1} \st{(\ve{e_1})\overline{a}[\lrangle{x}(\enc{P_3})]} T_1\SSEQV \enc{C'[0]}$. Defining $C\DEF C'\para P_2$, we have $P\SGB (\ve{e_1})C[a(x).P_3]$ and $P'\equiv (\ve{e_1})C[P_3\fosub{b}{x}] \equiv (\ve{e_1})(C[0] \para P_3\fosub{b}{x})$. Moreover %with $\ve{e_2}$ being the bound names shared between $P_3$ and $P_2$. Moreover
\[
\begin{array}{lrl}
 &\enc{P}\equiv& \enc{P_1}\para \enc{P_2} \\
 &\st{(\ve{e_1})\overline{a}[\lrangle{x}(\enc{P_3})]}& T_1\para \enc{P_2} \DEF T \\
 &\SSEQV& \enc{C'[0]}\para \enc{P_2} \\
 &\equiv& \enc{C'[0]\para P_2} \equiv \enc{C[0]} %\quad \mbox{ in which }
\end{array}
\]

\item Output. Suppose $P_1 \st{\overline{a}b} P_1'$ (the case $P_2$ does is similar) and \\
$P\st{\overline{a}b} P_1'\para P_2 \equiv P'$. By ind. hyp., $\enc{P_1} \st{a(A)} T_1$ and $T_1\SSEQV A\lrangle{b} \para \enc{P_1'}$. Then we have
\[
\begin{array}{lrl}
 &\enc{P}\equiv& \enc{P_1}\para \enc{P_2} \\
 &\st{a(A)}& T_1\para \enc{P_2} \DEF T\\
 &\SSEQV& A\lrangle{b} \para \enc{P_1'} \para \enc{P_2} \\
 &\SSEQV& A\lrangle{b} \para \enc{P_1'\para P_2} \equiv A\lrangle{b} \para \enc{P'}
\end{array}
\]

\item Bound output. Suppose $P_1 \st{\overline{a}(b)} P_1'$ and $b\notin \fn{P_2}$, and $P\st{\overline{a}b} P_1'\para P_2 \equiv P'$ (the case $P_2$ does is similar). By ind. hyp., $\enc{P_1} \st{a(A)} T_1$ and $T_1\SSEQV (b)(A\lrangle{b} \para \enc{P_1'})$. Then we have
\[
\begin{array}{lrl}
 &\enc{P}\equiv& \enc{P_1}\para \enc{P_2} \\
 &\st{a(A)}& T_1\para \enc{P_2} \DEF T \\
 &\SSEQV& (b)(A\lrangle{b} \para \enc{P_1'}) \para \enc{P_2} \\
 &\SSEQV& (b)(A\lrangle{b} \para \enc{P_1'} \para \enc{P_2}) \\ %\quad \mbox{ due to } b\notin \fn{P_2} \mbox{ and name invariance } \\
 &\SSEQV& (b)(A\lrangle{b} \para \enc{P_1'\para P_2}) \\
 &\equiv& (b)(A\lrangle{b} \para \enc{P'})
\end{array}
\]

\item Interaction ($\tau$). The most interesting and hard case is when $P_1$ and $P_2$ engages a communication (of a bound name). Consider, for example, the case $P_1\st{a(b)} P_1'$ and $P_2\st{\overline{a}(b)} P_2'$ ($b\notin \fn{P_2}$ \xxxx{wolg}), and $P\st{\tau} P'\equiv (b)(P_1'\para P_2')$. By ind. hyp., $P_1\SGB (\ve{e_1})C'[a(x).P_3]$ and $P_1'\equiv (\ve{e_1})C'[P_3\fosub{b}{x}] \equiv (\ve{e_1})(C'[0] \para P_3\fosub{b}{x})$ for some context $C'$, $P_3$ and $\ve{e_1}$ being the bound names shared between $P_3$ and its context within $P_1$, and $\enc{P_1} \st{(\ve{e_1})\overline{a}[\lrangle{x}(\enc{P_3})]} T_1\SSEQV \enc{C'[0]}$; $\enc{P_2} \st{a(A)} T_2$ and $T_2\SSEQV (b)(A\lrangle{b} \para \enc{P_2'})$. Take $A$ as $\lrangle{x}(\enc{P_3})$, and we have
\[
\begin{array}{lrl}
 &\enc{P}\equiv& \enc{P_1}\para \enc{P_2} \\
 &\st{\tau}& (\ve{e_1})(T_1 \para T_2) \DEF T \\
 &\SSEQV& (\ve{e_1})(\enc{C'[0]} \para (b)((\lrangle{x}(\enc{P_3}))\lrangle{b} \para \enc{P_2'})) \\
 &\SSEQV& (\ve{e_1})(\enc{C'[0]} \para (b)(\enc{P_3}\fosub{b}{x} \para \enc{P_2'})) \\
 &\SSEQV& (\ve{e_1})(\enc{C'[0]} \para (b)(\enc{P_3\fosub{b}{x}} \para \enc{P_2'})) \\
 &\SSEQV& (b\ve{e_1})(\enc{C'[0]} \para \enc{P_3\fosub{b}{x}} \para \enc{P_2'}) \\
 &\SSEQV& (b\ve{e_1})(\enc{C'[0] \para P_3\fosub{b}{x} \para P_2'}) \\
 &\SSEQV& (b\ve{e_1})(\enc{C'[0] \para P_3\fosub{b}{x} \para P_2'}) \\
 &\SSEQV& (b)(\enc{(\ve{e_1})(C'[0] \para P_3\fosub{b}{x}) \para P_2'}) \\ %\nts{???} \\
 %&\SCB& (b)(\enc{(\ve{e_1})(C'[P_3\fosub{b}{x}]) \para P_2'}) \\
 &\SSEQV& (b)(\enc{P_1' \para P_2'})
\end{array}
\]
\end{itemize}

\item $P$ is $(c)P_1$. %\nts{TODO (ref. Section \ref{s:encoding_operational})}
\begin{itemize}
\item Input. %\nts{copy the input case of parallel composition to here and adjust...}
Suppose $P_1\st{a(b)} P_1'$  and $P\st{a(b)} (c)P_1' \equiv P'$. By ind. hyp., $P_1\SGB (\ve{e_1})C'[a(x).P_3]$ and $P_1'\equiv (\ve{e_1})C'[P_3\fosub{b}{x}] \equiv (\ve{e_1})(C'[0]\para P_3\fosub{b}{x})$ for some context $C'$, $P_3$ and $\ve{e_1}$ being the bound names shared between $P_3$ and its context within $P_1$, and $\enc{P_1} \st{(\ve{e_1})\overline{a}[\lrangle{x}(\enc{P_3})]} T_1\SSEQV \enc{C'[0]}$. There are two subcases.
\begin{itemize}
\item[(i)] If $c\in \fn{P_3}$, we set $C\DEF C'$ and have $P\SGB (\ve{e_1}c)C[a(x).P_3]$ and $P'\equiv (\ve{e_1}c)C[P_3\fosub{b}{x}]\equiv (\ve{e_1}c)(C[0] \para P_3\fosub{b}{x})$, and moreover
\[
\begin{array}{lrl}
 &\enc{P}\equiv& (c)\enc{P_1} \\
 &\st{(\ve{e_1}c)\overline{a}[\lrangle{x}(\enc{P_3})]}& T_1 \DEF T \\
 &\SSEQV& \enc{C'[0]} \equiv \enc{C[0]} %\quad \mbox{ in which }
\end{array}
\]
\item[(ii)] If $c\notin \fn{P_3}$, we set $C\DEF (c)C'$. Then we have $P\SGB (\ve{e_1})C[a(x).P_3]$ and $P'\equiv (\ve{e_1})C[P_3\fosub{b}{x}]\equiv (\ve{e_1})((c)C'[0] \para P_3\fosub{b}{x})$ \\ 
$\equiv (\ve{e_1})(C[0] \para P_3\fosub{b}{x})$, and moreover
\[
\begin{array}{lrl}
 &\enc{P}\equiv& (c)\enc{P_1} \\
 &\st{(\ve{e_1})\overline{a}[\lrangle{x}(\enc{P_3})]}& (c)T_1 \DEF T \\
 &\SSEQV& (c)\enc{C'[0]} \\
 &\equiv& \enc{(c)C'[0]} \equiv \enc{C[0]} %\quad \mbox{ in which }
\end{array}
\]
\end{itemize}

\item Output. %\nts{copy the output case of parallel composition to here and adjust...}
There are two subcases.
\begin{itemize}
\item[(i)] $P_1 \st{\overline{a}b} P_1'$ and $P\st{\overline{a}b} (c)P_1' \equiv P'$. By ind. hyp., $\enc{P_1} \st{a(A)} T_1\SSEQV A\lrangle{b} \para \enc{P_1'}$. Then we have
\[
\begin{array}{lrl}
 &\enc{P}\equiv& (c)\enc{P_1} \\
 &\st{a(A)}& (c)T_1 \DEF T\\
 &\SSEQV& (c)(A\lrangle{b} \para \enc{P_1'}) \\
 &\SSEQV& A\lrangle{b} \para (c)\enc{P_1'} \\
 &\equiv& A\lrangle{b} \para \enc{(c)P_1'} \equiv A\lrangle{b} \para \enc{P'}
\end{array}
\]
\item[(ii)] $P_1 \st{\overline{a}c} P_1'$ and $P\st{\overline{a}(c)} P_1' \equiv P'$. By ind. hyp.,  we know $\enc{P_1} \st{a(A)} T_1\SSEQV A\lrangle{c} \para \enc{P_1'}$. Then we have
\[
\begin{array}{lrl}
 &\enc{P}\equiv& (c)\enc{P_1} \\
 &\st{a(A)}& (c)T_1 \DEF T\\
 &\SSEQV& (c)(A\lrangle{c} \para \enc{P_1'}) \\
 &\equiv& (c)(A\lrangle{c} \para \enc{P'})
\end{array}
\]

\end{itemize}

\item Bound output. %\nts{copy the bound output case of parallel composition to here and adjust...}
Suppose $P_1 \st{\overline{a}(b)} P_1'$ and $P\st{\overline{a}(b)} (c)P_1' \equiv P'$. By ind. hyp., $\enc{P_1} \st{a(A)} T_1\SSEQV (b)(A\lrangle{b} \para \enc{P_1'})$. Then we have
\[
\begin{array}{lrl}
 &\enc{P}\equiv& (c)\enc{P_1} \\
 &\st{a(A)}& (c)T_1 \DEF T \\
 &\SSEQV& (c)(b)(A\lrangle{b} \para \enc{P_1'}) \\
 &\SSEQV& (b)(A\lrangle{b} \para (c)\enc{P_1'}) \\
 &\equiv& (b)(A\lrangle{b} \para \enc{(c)P_1'}) \\
 &\equiv& (b)(A\lrangle{b} \para \enc{P'})
\end{array}
\]

\item Interaction ($\tau$). Suppose $P_1\st{\tau} P_1'$ and $P\st{\tau} (c)P_1'\equiv P'$. By ind. hyp., $\enc{P_1} \st{\tau} T_1 \SSEQV \enc{P_1'}$. Then we have
\[
\begin{array}{lrl}
 &\enc{P}\equiv& (c)\enc{P_1} \\
 &\st{\tau}& (c)T_1 \DEF T \\
 &\SSEQV& (c)\enc{P_1'} \\
 &\equiv& \enc{(c)P_1'} \equiv \enc{P'}
\end{array}
\]
\end{itemize}

\item $P$ is $!a(x).P_1 \SGB C[a(x).P_1]$ in which $C\DEF [\cdot]\para !a(x).P_1$.
\begin{itemize}
\item Input. We have $P \st{a(b)} P_1\fosub{b}{x}\para !a(x).P_1 \DEF P' \equiv C[P_1\fosub{b}{x}]$ and $P' \equiv C[0]\para P_1\fosub{b}{x}$. Then $\enc{P} \equiv !\overline{a}[\lrangle{x}\enc{P_1}] \st{\overline{a}[\lrangle{x}(\enc{P_1})]} 0\para !\overline{a}[\lrangle{x}\enc{P_1}] \DEF T$, and $T\SSEQV \enc{C[0]}$.
\end{itemize}

\end{itemize}
\myqed
\end{proof}





\subsection{Proof of the completeness of the encoding in Section \ref{s:encoding_variant}}\label{a:proofs-completeness-var-encoding}


\begin{proof}[Proof of Lemma \ref{l:completeness_var}]
We show that relation $\mathcal{R}$ below is a local bisimulation up-to $\equiv$. %$\SGB$.
%>>>>>> This up-to technique for \FOPi\ processes is defined in a way similar to that in \HOPiDd. We note that up-to techniques for first-order processes are well-established in the field; more details can be referred to \cite{SW01a,PS11}.
\[
\mathcal{R} \DEF \{(P,Q) \,|\, \enc{P}\WCB \enc{Q} \} \,\cup\, \WGB
\]
Assume $P\,\mathcal{R}\, Q$ and $P\st{\lambda}P'$. We proceed by a case analysis.
%\nts{TODO (ref. Section \ref{s:encoding_completeness})}
%\nts{Standard bisimulation game arguments? }
\begin{itemize}
\item Input: $P\st{a(b)}P'$. %This is relatively the most involved case. We need to setup the context in \HOPiDd\ carefully so as to close the bisimulation game. \nts{TODO}
%By Lemma \ref{l:opcor_var},
%By Lemma \ref{l:opcor-conv_var},
By Lemma \ref{l:opcor_var}, we know that $P\SGB (\ve{e})C[a(x).P_1]$ and $P'\equiv (\ve{e})C[P_1\fosub{b}{x}] \equiv (\ve{e})(C[0]\para P_1\fosub{b}{x})$ for some context $C$, $P_1$ and $\ve{e}$ being the bound names shared between $P_1$ and its context, and $\enc{P} \st{(\ve{e})\overline{a}[\lrangle{x}(\enc{P_1})]} T\SSEQV \enc{C[0]}$. Since $\enc{P}\WCB \enc{Q}$, we know that $\enc{Q}\wt{(\ve{e'})\overline{a}[\lrangle{x}(\enc{Q_1})]} T'$ and for every $E[X]$ with $\fn{E}\cap \ve{e}\ve{e'} {=} \emptyset$, it holds that
\begin{equation}\label{eq:com_var_1}
(\ve{e})(E[\lrangle{x}(\enc{P_1})] \para T) \,\WCB\, (\ve{e'})(E[\lrangle{x}(\enc{Q_1})] \para T')
\end{equation}
By Lemma \ref{l:opcor-conv_var}, we know that $Q \wt{a(b)} Q'$ with $Q\SGB (\ve{e'})C'[a(x).Q_1]$ and $Q'\equiv (\ve{e'})C'[Q_1\fosub{b}{x}] \equiv (\ve{e'})(C'[0]\para Q_1\fosub{b}{x})$ for some context $C'$ with $\ve{e}$ being the bound names shared between $Q_1$ and its context, and
$\enc{C'[0]} \wt{} T'' \WCB T'$, where we notice that there might be a few $\tau$'s distance between $\enc{C'[0]}$ and $T'$ due to the weak transition by $\enc{Q}$. Still by Lemma \ref{l:opcor-conv_var}, we know that there exists some context $C''$ s.t. $C'[0] \wt{} C''[0]$ with $\enc{C''[0]}\WCB T'' \WCB T'$, and as such we also know that $Q' \wt{} (\ve{e'})(C''[0]\para Q_1\fosub{b}{x}) \DEF Q''$, so $Q \wt{a(b)} Q''$. We need to prove $P' \mathcal{R} Q''$, i.e., $\enc{P'} \WCB \enc{Q''}$. Through setting $E[X]$ as $X\lrangle{b}$, equation (\ref{eq:com_var_1}) turns into
\[
(\ve{e})((\lrangle{x}(\enc{P_1}))\lrangle{b} \para T) \,\WCB\, (\ve{e'})((\lrangle{x}(\enc{Q_1}))\lrangle{b} \para T')
\] which is exactly
\[
(\ve{e})(\enc{P_1\fosub{b}{x}} \para T) \,\WCB\, (\ve{e'})(\enc{Q_1\fosub{b}{x}} \para T')
\] Since we already know that $T\SSEQV \enc{C[0]}$ and $T'\WCB \enc{C''[0]}$, we have as needed
\[
\enc{P'}\;\equiv\; (\ve{e})(\enc{P_1\fosub{b}{x}} \para \enc{C[0]}) \,\WCB\, (\ve{e'})(\enc{Q_1\fosub{b}{x}} \para \enc{C''[0]}) \;\equiv\; \enc{Q''}
\]


\item Output: $P\st{\overline{a}b}P'$. %\nts{TODO}
%By Lemma \ref{l:opcor_var},
%By Lemma \ref{l:opcor-conv_var},
By Lemma \ref{l:opcor_var}, $\enc{P} \st{a(A)} T \SSEQV A\lrangle{b} \para \enc{P'}$. Since $\enc{P}\WCB \enc{Q}$, we know that $\enc{Q} \wt{a(A)} T' \WCB T$. By Lemma \ref{l:opcor-conv_var}, it must be that $Q \wt{\overline{a}b} Q'$ (otherwise $\enc{P}$ and $\enc{Q}$ would be distinguishable, \xxx{similar to the case in Lemma \ref{l:completeness}}), and 
$A\lrangle{b} \para \enc{Q'} \wt{} T'' \WCB T'$. Setting $A$ to be $\lrangle{x}0$, we have
\begin{eqnarray}
\enc{P} \st{a(\lrangle{x}0)} T \SSEQV 0 \para \enc{P'} \SSEQV \enc{P'} \nonumber \\ %\label{eq:com_var_2} \\
\enc{Q'} \equiv 0 \para \enc{Q'} \wt{} T'' \WCB T' \label{eq:com_var_3}
\end{eqnarray}
From (\ref{eq:com_var_3}) and Lemma \ref{l:opcor-conv_var}, we know that $Q'\wt{} Q''$ and $\enc{Q''} \WCB T'' \WCB T'$. So we have $Q\wt{\overline{a}b} Q''$ and need to prove $P' \mathcal{R} Q''$, i.e., $\enc{P'} \WCB \enc{Q''}$. This is immediate from the following equations.
\[
\enc{P'} \;\WCB\; T \;\WCB\; T' \;\WCB\; T'' \;\WCB\; \enc{Q''}
\]


\item Bound output: $P\st{\overline{a}(b)}P'$. %In contrast to the input case, one has to collude with the inputted process to the encoding in order to make the bisimulation closed. \nts{Use the local bisimulation's corresponding clause; TODO}
%By Lemma \ref{l:opcor_var},
%By Lemma \ref{l:opcor-conv_var},
By Lemma \ref{l:opcor_var}, $\enc{P}$ can make an input over $a$; we set the input as the trigger $\triggerd \equiv \lrangle{z}\overline{m}[\lrangle{Y}(Y\lrangle{z})]$ ($m$ fresh). Then we have $\enc{P} \st{a(\triggerd)} T \SSEQV (b)(\triggerd\lrangle{b} \para \enc{P'})$. Since $\enc{P}\WCB \enc{Q}$, we know that $\enc{Q} \wt{a(\triggerd)} T' \WCB T$. By Lemma \ref{l:opcor-conv_var}, it must be that $Q \wt{\overline{a}(b)} Q'$ (otherwise $\enc{P}$ and $\enc{Q}$ would be distinguishable, \xxx{as in the last case}), and $(b)(\triggerd\lrangle{b} \para \enc{Q'}) \wt{} T'' \WCB T'$. 
Thus we have $\enc{Q'} \wt{} T'''$ and \\
$(b)(\triggerd\lrangle{b} \para \enc{Q'})\wt{} (b)(\triggerd\lrangle{b} \para T''') \WCB T'' \WCB T'$. By Lemma \ref{l:opcor-conv_var}, we know that $Q'\wt{} Q''$ and $\enc{Q''} \WCB T'''$. %$\enc{Q''} \WCB T'''$.
So we have $Q \wt{\overline{a}(b)} Q''$ and \\
$(b)(\triggerd\lrangle{b} \para \enc{Q''}) \WCB T'' \WCB T'$. The above boils down to the following equation.
\[
(b)(\triggerd\lrangle{b} \para \enc{P'}) \;\WCB\; T\;\WCB\; T'\;\WCB\; T'' \;\WCB\; (b)(\triggerd\lrangle{b} \para \enc{Q''})
\]
Since $\WCB$ is a congruence, we can derive that
\[
\begin{array}{l}
(m)(\;(b)(\triggerd\lrangle{b} \para \enc{P'})\; \para !m(Z).Z\lrangle{A}) \\
\qquad\qquad\qquad\qquad\qquad\qquad \;\WCB\; (m)(\;(b)(\triggerd\lrangle{b} \para \enc{Q''})\; \para !m(Z).Z\lrangle{A})
\end{array}
\]
Using Theorem \ref{factor-bigd-smalld}, we have
\begin{equation}\label{eq:com_var_4}
(b)(\enc{P'}\para A\lrangle{b}) \;\WCB\; (b)(\enc{Q''} \para A\lrangle{b})
\end{equation}
We now recall that since $P\st{\overline{a}(b)}P'$ and $Q \wt{\overline{a}(b)} Q''$, our goal is to prove that for every $R$ it holds
\[
(b)(P' \para R) \;\mathcal{R}\; (b)(Q'' \para R)
\] That is,
\begin{equation}\label{eq:com_var_5}
(b)(\enc{P'} \para \enc{R}) \;\WCB\; (b)(\enc{Q''} \para \enc{R})
\end{equation}
Comparing (\ref{eq:com_var_4}) and (\ref{eq:com_var_5}), we can observe that since the image of the encoding is contained by the target model \HOPiDd\ and $A$ is arbitrary (it has the knowledge of $b$ as well), if we traverse all possibility of $A$ then $\enc{R}$ must be hit somewhere. We thus conclude that (\ref{eq:com_var_5}) is true and the simulation is closed.

\item $\tau$ action: $P\st{\tau}P'$. %\nts{TODO}
%By Lemma \ref{l:opcor_var},
%By Lemma \ref{l:opcor-conv_var},
By Lemma \ref{l:opcor_var}, $\enc{P} \st{\tau} T \SSEQV \enc{P'}$. Since $\enc{P}\WCB \enc{Q}$, we know that $\enc{Q} \wt{} T' \WCB T$. By Lemma \ref{l:opcor-conv_var}, $Q \wt{} Q'$ and $T'\WCB \enc{Q'}$. %$T'\WCB \enc{Q'}$
We need to prove $P'\,\mathcal{R}\, Q'$, i.e., $\enc{P'} \,\WCB\, \enc{Q'}$. This is straightforward because
\[
\enc{P'} \;\WCB\; T \;\WCB\; T' \;\WCB\; \enc{Q'}
\]
\end{itemize}
The proof is now completed.
\myqed
\end{proof}









%---------------------------
% Local Variables:
% mode: LaTeX
% TeX-master: "main.tex"
% End:

\clearpage
\section{Proofs for Section \ref{s:normal}}\label{a:proofs-normal}
We give the proof for Theorem \ref{normal-characterization-hopiDd}.
\begin{proof}[Proof of Theorem \ref{normal-characterization-hopiDd}]
We only need to prove $\WNB$ implies $\WCB$, since $\WCB$ is obviously finer than $\WNB$. 
To do this, we define the relation $\mathcal{R}$ below and show that it is a context bisimulation. % (\stress{up-to sth?}). 
\[
\mathcal{R} \DEF \{(P,Q) \,|\, P\WNB Q \}
\]
Suppose $P\mathcal{R} Q$. There are several cases (with subcases) to analyze. We focus on the main ones, and leave out the similar (and simpler) rest.

%\stress{TODO}
\begin{itemize}
\item $P\st{a(A)} P'$. There are three subcases.
\begin{itemize}
\item $A$ is an abstraction on name. %, i.e., $A\equiv \lrangle{x}B$. 
Then $P'$ must be of the form $E[A]$ for some $E[Z]$ (one can choose such an $E$ that $Z$ is instantiated by the incoming $A$; similar for the other two subcases). So $P\st{a(\triggerd)} E[\triggerd]$. Since $P\WNB Q$, we know $Q\wt{a(\triggerd)} Q'' \equiv F[\triggerd]$ for some $F[Z]$ and $E[\triggerd] \WNB F[\triggerd]$. Then $Q\wt{a(A)} Q' \equiv F[A]$. Due to the congruence property of $\WNB$, the factorization theorem (Theorem \ref{factor-bigd-smalld}), and the fact that $\WCB$ implies $\WNB$, we have
\[
P'\equiv E[A] \,\WNB\, (m)(E[\triggerd] \para  !m(Z).Z\lrangle{A}) \,\WNB\, (m)(F[\triggerd] \para  !m(Z).Z\lrangle{A}) \,\WNB\, F[A]\equiv Q'
\] So $P' \mathcal{R} Q'$.

\item $A$ is an abstraction on process. %, i.e., $A\equiv \lrangle{X}B$. 
Similar to the last subcase, except that $\triggerD$ is used instead. %Then ...\stress{TODO}
\item $A$ is not an abstraction. Similar to the first subcase, except that $\trigger$ is used. %Then ...\stress{TODO}
\end{itemize}

\item $P\st{(\ve{c})\overline{a}A} P'$. There are again three subcases.
\begin{itemize}
\item $A$ is an abstraction on name. %, i.e., $A\equiv \lrangle{x}B$. 
Because $P\WNB Q$, $Q\wt{(\ve{d})\overline{a}B} Q'$ and it holds that ($m$ is fresh)
\[(\ve{c})(P'\para !m(Z).Z\lrangle{A}) \,\WNB\,  (\ve{d})(Q'\para  !m(Z).Z\lrangle{B})
\] Using the congruence property of $\WNB$ twice and some structural manipulation, for every $E[X]$ ($\fn{E[X]}\cap \ve{c}\ve{d} = \emptyset$), we have
\[(\ve{c})(P'\para !m(Z).Z\lrangle{A} \para E[\triggerd]) \,\WNB\,  (\ve{d})(Q'\para  !m(Z).Z\lrangle{B} \para E[\triggerd])
\] and then
\[(\ve{c})(P'\para (m)(!m(Z).Z\lrangle{A} \para E[\triggerd])) \,\WNB\,  (\ve{d})(Q'\para  (m)(!m(Z).Z\lrangle{B} \para E[\triggerd]))
\] Now by the factorization theorem (Theorem \ref{factor-bigd-smalld}) and the fact that $\WCB$ implies $\WNB$, we know
\[(\ve{c})(P'\para E[A]) \,\WNB\,  (\ve{d})(Q'\para  E[B])
\] and thus 
\[(\ve{c})(P'\para E[A]) \,\mathcal{R}\,  (\ve{d})(Q'\para  E[B])
\] as required by context bisimulation.
 
%...\stress{TODO}
\item $A$ is an abstraction on process. %, i.e., $A\equiv \lrangle{X}B$. 
Similar to the last subcase, except that $\triggerD$ is used.  %Then ...\stress{TODO}
\item $A$ is not an abstraction. Similar to the first subcase, except that $\trigger$ is used. %Then ...\stress{TODO}
\end{itemize}

\item $P\st{\tau} P'$. Then we immediately have $Q\wt{} Q'$ and $P'\WNB Q'$, so $P'\mathcal{R} Q'$.
\end{itemize}



\end{proof}













%---------------------------
% Local Variables:
% mode: LaTeX
% TeX-master: "main.tex"
% End:

\clearpage
%}{%
  %no appendixing
%}
}%normalsize



\end{document}


\section{\xxx{Proofs for Section \ref{s:encoding}}}\label{appendix:up-to-context_encodings}

We give the proof of Proposition \ref{prop:up-to-context_encodings}.
For convenience, we reproduce the definition of a relation that is a context bisimulation up-to context. 
By saying some term is an encoding, we mean that it is the encoding of certain ($\pi$) process. Recall that the encoding of a context is a conext put through the encoding with the hole translated into a hole. 
In the output clause of Definition \ref{d:up3_encoding}, $Q$ is required to make the same output as $P$. This corresponds to the requirement of bound output in the bisimulation of $\pi$-calculus, and suffices for our purpose.
\begin{definition}[up-to context]\label{d:up3_encoding}
%modify from up-to \SCB above
%P'\equiv C[P''],  Q'\equiv C[Q''], and P''\R Q''
A symmetric relation $\mathcal{R}$ on the image of the encoding is a context bisimulation up-to context, if whenever $P\,\mathcal{R}\, Q$ the following properties hold.
\begin{itemize}
\item if $P \st{\alpha} P'$ and $\alpha$ is $\tau$ or $a(A)$ in which $A$ is $\lrangle{Z}(Z\lrangle{b})$, then $Q \wt{\widehat{\alpha}} Q'$ for some $Q'$, $P'\equiv C[P'']$ and $Q'\equiv C[Q'']$ for some $P'', Q''$ and context $C$ that is an encoding, and $P''\,\mathcal{R}\, Q''$;

\item if $P \st{(\ve{c})\overline{a}A} P'$ in which $A$ is $\lrangle{Z}(Z\lrangle{b})$ and $\ve{c}$ is either empty or $\{b\}$, 
%then $Q \wt{(\ve{d})\overline{a}B} Q'$ for some $B$ that is an enoding and of the same type as $A$, 
then $Q \wt{(\ve{c})\overline{a}A} Q'$ 
and for every $E[\cdot]$ (\xx{or $E[X]$?}) that is an encoding and satisfies 
%$\{\ve{c},\ve{d}\}\cap \fn{E}=\emptyset$, 
$\{\ve{c}\}\cap \fn{E}=\emptyset$, 
%it holds that $(\ve{c})(E[A]\para P') \equiv C[P'']$ and $(\ve{d})(E[B]\para Q')\equiv C[Q'']$
it holds that $(\ve{c})(E[A]\para P') \equiv C[P'']$ and $(\ve{c})(E[A]\para Q')\equiv C[Q'']$  
for some $P'', Q''$ and context $C$ that is an encoding, and $P''\,\mathcal{R}\, Q''$.
\end{itemize}
\end{definition}


\begin{proof}[Proof of Proposition \ref{prop:up-to-context_encodings}]
Assume $\R$ is as defined in Definition \ref{d:up3_encoding}. 
We show that the relation $\R'$ defined below is a context bisimulation \xxx{up-to $\equiv$}, thus proving the proposition.
\[
\R' \DEF \{(C[P], C[Q]) \,|\, P \,\R\, Q, \mbox{ and $C$ is a context that is an encoding}\}
\]
We make a case analysis. Suppose $P\R' Q$ and notice that $P$ and $Q$ are not parameterized.

%-------------PROOF BODY BEGIN-------------
\xx{TODO: to fetch from `NOTES'}
%-------------PROOF BODY END-------------

\begin{itemize}
\item 
\item 
\item 
\item 
\item 
\item 
\item 
\end{itemize}
\qed
\end{proof}

\noindent\textit{Remark} ~ 
\xx{We remark on the proof above.}
\begin{itemize}
\item 
\xx{\small Perhaps we can also use the way of proving congruence \cite{Fu07,MH02,San92,SW01a}. We denote by $\mathbb{N}$ the set of natural numbers. Define the relations $\R_n$ and $\R'$ as follows.
% \[
% \begin{array}{lcl}
% \R_0 &\DEF& \R \\
% \R_{n+1} &\DEF& \left\{(C[M],C[N]) \,|\, M\,\R_{n}\, N \mbox{ and } C \mbox{ is $R \para [\cdot]$, or $(d)[\cdot]$} \right\} \\ %$\vartheta.[\cdot]$,
% \R' &\DEF& \bigcup_{i\in \mathbb{N}} \R_i
% \end{array}
% \]
\[
\begin{array}{lcl}
\R_0 &\DEF& \R \\
\R_{n+1} &\DEF& \left\{(C[M],C[N]) \,|\, M\,\R_{n}\, N \mbox{ and } C C\in \mathcal{D} \right\} \\ %$\vartheta.[\cdot]$,
\R' &\DEF& \bigcup_{i\in \mathbb{N}} \R_i\\\\
\mathcal{D} &\DEF& \left\{
\begin{array}{lllll}
%[\cdot],  & 
a(X).[\cdot],\quad & \overline{a}([\cdot]).R,\quad & \overline{a}A_1.[\cdot],\quad & \lrangle{X}[\cdot], \quad & \lrangle{x}[\cdot],\\~
[\cdot]\lrangle{A_1}, & [\cdot]\lrangle{d}, & R\para [\cdot], & (d)[\cdot] &
\end{array}
\right\}
\end{array}
\]
In somewhat a similar way, we can prove by induction on $n$ that $\R'$ is a bisimulation up-to $\equiv$. In this approach, the procedure of induction is like the proof above, and takes the usual pattern \cite{Fu07,MH02,San92,SW01a}. This would actually even complicates the arguments, so rather the current proof we provide as above offers technically a somewhat simplified way in a particular seeting of encodings.
}

\item 
\end{itemize}
























%---------------------------
% Local Variables:
% mode: LaTeX
% TeX-master: "main.tex"
% End:



%\section{Proofs for Section \ref{s:encoding}}\label{a:proofs-encoding}

\begin{proof}[Proof of Lemma \ref{l:syn-pro-encoding}]
It is straightforward to check that the encoding is compositional since the designated contexts are easy to capture. For the core of the encoding, the contexts for input and output are respectively $m(Y).Y\lrangle{\lrangle{x}[\cdot]}$ and $\overline{m}[\lrangle{Z}(Z\lrangle{n})].[\cdot]$. Also the encoding is divergence-reflecting, for the reason that it does not bring about any divergence. As such, it is a simple induction, on the rules deriving $\equiv$, to show that the encoding preserves structural congruence. Below we prove by induction on the structure of $P$ that $\enc{P}\sigma \equiv \enc{P\sigma}$ in which $\sigma$ is a substitution (recall that $\sigma$ is a mapping on names).
%\oo{\large \fbox{\#\#\#\# TODO}}
\begin{itemize}
\item $P$ is $0$. This is trivial.
\item $P$ is $m(x).Q$. Then %m(Y).Y\lrangle{\lrangle{x}\enc{Q}}
\[
\begin{array}{lcll}
\enc{P}\sigma &\equiv& (m(Y).Y\lrangle{\lrangle{x}\enc{Q}})\sigma & \quad \\
 &\equiv& m'(Y).Y\lrangle{\lrangle{x}(\enc{Q}\sigma)} & \quad m' \mbox{ is } \sigma(m) \\
 &\equiv& m'(Y).Y\lrangle{\lrangle{x}(\enc{Q\sigma})} & \quad \mbox{ind. hyp. (short for induction hypothesis)}\\
 &\equiv& \enc{m'(x).Q\sigma} & \quad \\
 &\equiv& \enc{(m(x).Q)\sigma} & \quad \\ 
 &\equiv& \enc{P\sigma} & \quad
\end{array}
\]
\item $P$ is $\overline{m}n.Q$. Then %\overline{m}[\lrangle{Z}(Z\lrangle{n})].\enc{Q}
\[
\begin{array}{lcll}
\enc{P}\sigma &\equiv& (\overline{m}[\lrangle{Z}(Z\lrangle{n})].\enc{Q})\sigma & \quad \\
 &\equiv& \overline{m'}[\lrangle{Z}(Z\lrangle{n'})].(\enc{Q}\sigma) & \quad m',n' \mbox{ are respectively } \sigma(m),\sigma(n) \\
 &\equiv& \overline{m'}[\lrangle{Z}(Z\lrangle{n'})].(\enc{Q\sigma}) & \quad \mbox{ind. hyp.} \\
 &\equiv& \enc{\overline{m'}n'.(Q\sigma)} & \quad \\
 &\equiv& \enc{(\overline{m}n.Q)\sigma} & \quad \\
 &\equiv& \enc{P\sigma} & \quad
\end{array}
\]
\item $P$ is $(c)Q$. Then 
\[
\begin{array}{lcll}
\enc{P}\sigma &\equiv& ((c)\enc{Q})\sigma & \quad  \\
 &\equiv& (c)\enc{Q}\sigma & \quad  \\
 &\equiv& (c)\enc{Q\sigma} & \quad \mbox{ind. hyp.} \\
 &\equiv& \enc{(c)(Q\sigma)} & \quad  \\
 &\equiv& \enc{((c)Q)\sigma} & \quad  \\
 &\equiv& \enc{P\sigma} & \quad  
\end{array}
\]
\item $P$ is $Q\para R$. Then 
\[
\begin{array}{lcll}
\enc{P}\sigma &\equiv& (\enc{Q}\para \enc{R})\sigma  & \quad \\
 &\equiv& \enc{Q}\sigma\para \enc{R}\sigma  & \quad \\
 &\equiv& \enc{Q\sigma}\para \enc{R\sigma}  & \quad \mbox{ind. hyp.} \\ 
 &\equiv& \enc{Q\sigma \para R\sigma}  & \quad  \\ 
 &\equiv& \enc{(Q\para R)\sigma}  & \quad  \\ 
 &\equiv& \enc{P\sigma}  & \quad 
\end{array}
\]
\item $P$ is $!m(x).Q$. Then %!\enc{m(x).Q}
\[
\begin{array}{lcll}
\enc{P}\sigma &\equiv& (!\enc{m(x).Q})\sigma & \quad \\
 &\equiv& !(\enc{m(x).Q})\sigma & \quad \\
 &\equiv& !(\enc{(m(x).Q)\sigma}) & \quad \mbox{similar to the input case} \\
 &\equiv& \enc{!((m(x).Q)\sigma)} & \quad \\
 &\equiv& \enc{(!m(x).Q)\sigma} & \quad \\
 &\equiv& \enc{P\sigma} & \quad 
\end{array}
\]
\end{itemize}
\myqed
\end{proof}


\begin{proof}[Proof of Lemma \ref{l:opcor}]
We prove the clauses by induction on the structure of $P$. The case $P$ is $0$ is trivial.  %\stress{TODO (treat 1-6 together or separately)}
\begin{itemize}
\item $P$ is $a(x).P_1$.
\begin{itemize}
\item Input. $P\st{a(b)} P_1\fosub{b}{x} \equiv P'$. So $\enc{P} \equiv a(Y).Y\lrangle{\lrangle{x}\enc{P_1}} \st{a(\lrangle{Z}(Z\lrangle{b}))} \equiv \enc{P_1}\fosub{b}{x} \equiv T$. We then have $T \SSEQV \enc{P_1\fosub{b}{x}} \SSEQV \enc{P'}$. Moreover, $\enc{P} \st{a(\triggerD)} \overline{m}[\lrangle{x}\enc{P_1}] \equiv T$. We have 
\[
\begin{array}{ll}
& (m)(T \para !m(Y).Y\lrangle{b}) \\
\equiv & (m)(\overline{m}[\lrangle{x}\enc{P_1}] \para !m(Y).Y\lrangle{b}) \\
\WCB & (m)(!m(Y).Y\lrangle{b} \para \enc{P_1}\fosub{b}{x}) \\
\WCB & \enc{P_1}\fosub{b}{x}) \SSEQV \enc{P'}
\end{array}
\]
\end{itemize}

\item $P$ is $\overline{a}b.P_1$.
\begin{itemize}
\item Output. $P \st{\overline{a}b} P_1\equiv P'$. Then $\enc{P} \equiv \overline{a}[\lrangle{Z}(Z\lrangle{b})].\enc{P_1} \st{\overline{a}[\lrangle{Z}(Z\lrangle{b})]} \enc{P_1} \equiv T$, so $T\SSEQV \enc{P'}$.
\end{itemize}

\item $P$ is $P_1\para P_2$.
\begin{itemize}
\item Input. Assume $P_1\st{a(b)} P_1'$, and $P\st{a(b)} P_1'\para P_2 \equiv P'$. By ind. hyp., $\enc{P_1} \st{a(\lrangle{Z}(Z\lrangle{b}))} T_1$ and $T_1\SSEQV \enc{P_1'}$. Then $\enc{P}\equiv \enc{P_1}\para \enc{P_2} \st{a(\lrangle{Z}(Z\lrangle{b}))} T_1\para \enc{P_2} \DEF T$. So $T\SSEQV \enc{P_1'}\para \enc{P_2} \equiv \enc{P'}$. Moreover, $\enc{P_1} \st{a(\triggerD)} T_2$ and $(m)(T_2 \para !m(Y).Y\lrangle{b}) \WCB \enc{P_1'}$. Thus $\enc{P}\equiv \enc{P_1}\para \enc{P_2} \st{a(\triggerD)} T_2\para \enc{P_2} \DEF T'$. Hence $(m)(T' \para !m(Y).Y\lrangle{b}) \equiv (m)(T_2\para \enc{P_2} \para !m(Y).Y\lrangle{b}) \WCB \enc{P_1'} \para \enc{P_2} \equiv \enc{P'}$.

\item Output. Assume $P_1\st{\overline{a}b} P_1'$, and $P\st{\overline{a}b} P_1'\para P_2 \equiv P'$. By ind. hyp., $\enc{P_1} \st{\overline{a}[\lrangle{Z}(Z\lrangle{b})]} T_1$ and $T_1\SSEQV \enc{P_1'}$. Then $\enc{P}\equiv \enc{P_1}\para \enc{P_2} \st{\overline{a}[\lrangle{Z}(Z\lrangle{b})]} T_1\para \enc{P_2} \DEF T$. So $T\SSEQV \enc{P_1'}\para \enc{P_2}\equiv \enc{P'}$. 

\item Bound output. Assume $P_1\st{\overline{a}(b)} P_1'$, and $P\st{\overline{a}(b)} P_1'\para P_2 \equiv P'$. By ind. hyp., $\enc{P_1} \st{(b)\overline{a}[\lrangle{Z}(Z\lrangle{b})]} T_1$ and $T_1\SSEQV \enc{P_1'}$. Then $\enc{P}\equiv \enc{P_1}\para \enc{P_2} \st{(b)\overline{a}[\lrangle{Z}(Z\lrangle{b})]} T_1\para \enc{P_2} \DEF T$. So $T\SSEQV \enc{P_1'}\para \enc{P_2}\equiv \enc{P'}$. 

\item Internal action ($\tau$). The interesting case is when there is a communication between $P_1$ and $P_2$. Take, for example, the case $P_1\st{a(b)} P_1'$ and $P_2\st{\overline{a}(b)} P_2'$, and $P'\equiv (b)(P_1'\para P_2')$. By ind. hyp., $\enc{P_1} \st{a(\lrangle{Z}(Z\lrangle{b}))} T_1$ and $T_1\SSEQV \enc{P_1'}$, and $\enc{P_2} \st{(b)\overline{a}[\lrangle{Z}(Z\lrangle{b})]} T_2$ and $T_2\SSEQV \enc{P_2'}$. So $\enc{P}\equiv \enc{P_1}\para \enc{P_2} \st{\tau} (b)(T_1\para T_2) \DEF T$. Thus $\enc{P'} \equiv (b)(\enc{P_1'}\para \enc{P_2'}) \SSEQV (b)(T_1\para T_2) \equiv T$.
\end{itemize}

\item $P$ is $(c)P_1$.
\begin{itemize}
\item Input. Assume $P_1\st{a(b)} P_1'$, and $P\st{a(b)} (c)P_1'\equiv P'$. By ind. hyp., $\enc{P_1} \st{a(\lrangle{Z}(Z\lrangle{b}))} T_1$ and $T_1\SSEQV \enc{P_1'}$. Then $\enc{P}\equiv (c)\enc{P_1} \st{a(\lrangle{Z}(Z\lrangle{b}))} (c)T_1 \DEF T$. So $T\SSEQV (c)\enc{P_1'}\equiv \enc{P'}$. Moreover, $\enc{P_1} \st{a(\triggerD)} T_2$ and $(m)(T_2 \para !m(Y).Y\lrangle{b}) \WCB \enc{P_1'}$. Thus $\enc{P}\equiv (c)\enc{P_1} \st{a(\triggerD)} (c)T_2 \DEF T'$. Hence $(m)(T' \para !m(Y).Y\lrangle{b}) \equiv (m)((c)T_2 \para !m(Y).Y\lrangle{b}) \WCB (c)\enc{P_1'} \equiv \enc{P'}$.

\item Output. Assume $P_1\st{\overline{a}b} P_1'$, and $P\st{\overline{a}b} (c)P_1'\equiv P'$. By ind. hyp., $\enc{P_1} \st{\overline{a}[\lrangle{Z}(Z\lrangle{b})]} T_1$ and $T_1\SSEQV \enc{P_1'}$. Then $\enc{P}\equiv (c)\enc{P_1} \st{\overline{a}[\lrangle{Z}(Z\lrangle{b})]} (c)T_1 \DEF T$. So $T\SSEQV (c)\enc{P_1'}\equiv \enc{P'}$.

\item Bound output. The interesting case is $P_1\st{\overline{a}c} P_1'$, and $P\st{\overline{a}(c)} P'\equiv P_1'$. By ind. hyp., \\ $\enc{P_1} \st{\overline{a}[\lrangle{Z}(Z\lrangle{c})]} T_1$ and $T_1\SSEQV \enc{P_1'}$. So $\enc{P}\equiv (c)\enc{P_1} \st{(c)\overline{a}[\lrangle{Z}(Z\lrangle{c})]} T \equiv T_1 \SSEQV \enc{P'}$.

\item Internal action ($\tau$). Assume $P_1\st{\tau} P_1'$, and $P\st{\tau} (c)P_1'\equiv P'$. By ind. hyp., $\enc{P_1} \st{\tau} T_1$ and $T_1\SSEQV \enc{P_1'}$. Then $\enc{P}\equiv (c)\enc{P_1} \st{\tau} (c)T_1 \DEF T$. So $T\SSEQV (c)\enc{P_1'} \equiv \enc{P'}$.
\end{itemize}
\item $P$ is $!a(x).P_1$. 
\begin{itemize}
\item Input. $P\st{a(b)} P_1\fosub{b}{x}\para P \equiv P'$. So $\enc{P} \equiv !a(Y).Y\lrangle{\lrangle{x}\enc{P_1}} \st{a(\lrangle{Z}(Z\lrangle{b}))} \equiv \enc{P_1}\fosub{b}{x}\para \enc{P} \equiv T$. We then have $T \SSEQV \enc{P_1\fosub{b}{x}}\para \enc{P} \SSEQV \enc{P'}$. Moreover, $\enc{P} \st{a(\triggerD)} \overline{m}[\lrangle{x}\enc{P_1}]\para \enc{P} \equiv T$. We have 
\[
\begin{array}{ll}
& (m)(T \para !m(Y).Y\lrangle{b}) \\
\equiv & (m)(\overline{m}[\lrangle{x}\enc{P_1}]\para \enc{P} \para !m(Y).Y\lrangle{b}) \\
\WCB & (m)(!m(Y).Y\lrangle{b}\para \enc{P} \para \enc{P_1}\fosub{b}{x}) \\
\WCB & \enc{P_1}\fosub{b}{x})\para \enc{P} \SSEQV \enc{P'}
\end{array}
\]
\end{itemize}
\end{itemize}
\myqed
\end{proof}



% %\begin{proof}[Proof of Lemma \ref{l:opcor-conv}]
% \noindent\emph{Proof of Lemma \ref{l:opcor-conv}}~~ The proof proceeds by induction on the structure of $P$, %. By noticing that actions come from $\enc{P}$ instead of $P$, it 
% and is similar to the proof of Lemma \ref{l:opcor}. We thus skip the details.

%\sepp
%\noindent\greycolor{
%{Memo below; to remove: (treat 1-6 together or separately); Actually similar to Lemma \ref{l:opcor} (except that actions come from $\enc{P}$ instead of $P$); maybe simply SKIP.} \\
%The case $P$ is $0$ is trivial.  
%\begin{itemize}
%\item $P$ is $a(x).P_1$.
%\begin{itemize}
%\item Input.
%\end{itemize}
%\item $P$ is $\overline{a}b.P_1$.
%\begin{itemize}
%\item Output.
%\end{itemize}
%\item $P$ is $P_1\para P_2$.
%\begin{itemize}
%\item Input.
%\item Output.
%\item Bound output.
%\item $\tau$.
%\end{itemize}
%\item $P$ is $(c)P_1$.
%\begin{itemize}
%\item Input.
%\item Output.
%\item Bound output.
%\item $\tau$.
%\end{itemize}
%\item $P$ is $!a(x).P_1$.
%\begin{itemize}
%\item Input.
%\end{itemize}
%\end{itemize}
%}%greycolor

%\end{proof}










%---------------------------
% Local Variables:
% mode: LaTeX
% TeX-master: "main.tex"
% End:

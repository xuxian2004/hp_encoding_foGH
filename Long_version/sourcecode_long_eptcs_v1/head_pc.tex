%---------------------------------------------------------------------
%the definitions used by xx
%---------------------------------------------------------------------


%---------------------------packages begin---------------------------------------------------------------
\usepackage{etex}
\usepackage[usenames]{color}
%\usepackage{xcolor}
%\usepackage{inference}
%\usepackage{vmacros,cmacros}
\usepackage{makeidx}
\usepackage{verbatim}
\usepackage{fancyhdr}
\usepackage{fancybox}
\usepackage{stmaryrd}
\usepackage[reqno]{amsmath}
%\usepackage[notref,notcite]{showkeys}
%\usepackage{layout}
%\usepackage{graphicx}
\usepackage{amsthm}
\usepackage{amssymb}
\usepackage{amsfonts}
\usepackage{bm}
\usepackage{amsbsy}
\usepackage[all]{xypic}
\usepackage {indentfirst}
\usepackage{graphicx}
\DeclareGraphicsRule{.jpg}{eps}{.bb}{}
\DeclareGraphicsRule{.bmp}{eps}{.bb}{}
%\usepackage{palatino}
\usepackage[hang]{subfigure}
\usepackage{cite}
\usepackage{proof}
\usepackage{bbding}
\usepackage{pgf}
\usepackage{tikz}
\usetikzlibrary{positioning,arrows,automata}
\usepackage{cancel}
\usepackage{semantic}
\usepackage{extarrows}
\usepackage{etoolbox} %for defining conditions
%\usepackage{hyperref}
%\usepackage[left=1in,right=1in,top=1in,bottom=1in,bindingoffset=1cm,a4paper]{geometry}
%\usepackage[a4paper,tmargin=0.8in,bmargin=0.85in,lmargin=1.4in,rmargin=1.4in]{geometry}
%\usepackage{hyperref} % ×îºÃ±£Ö¤ hyperref ÊÇ×îºó¼ÓÔصĺê°ü
%\hypersetup{%
%  dvipdfmx,% É趨ҪʹÓÃµÄ driver Ϊ dvipdfmx
%  unicode={true},% ʹÓà unicode À´±àÂë PDF ×Ö·û´®
%  pdfstartview={FitH},% Îĵµ³õʼÊÓͼΪƥÅä¿í¶È
%  bookmarksnumbered={true},% ÊéÇ©¸½ÉÏÕ½ڱàºÅ
%  bookmarksopen={true},% Õ¹¿ªÊéÇ©
%  pdfborder={0 0 0},% Á´½ÓÎÞ¿ò
%  pdftitle={±êÌâ},
%  pdfauthor={×÷Õß},
%  pdfsubject={Ö÷Ìâ},
%  pdfkeywords={¹Ø¼ü´Ê},
%  pdfcreator={Ó¦ÓóÌÐò},
%  pdfproducer={PDF ÖÆ×÷³ÌÐò},% Õâ¸öºÃÏñûÆð×÷Óã¿
%}
%---------------------------packages end----------------------------------------------------------------


%---------------------------thms begin--------------------------------------------------------------------
 \newtheorem{theorem}{Theorem}%[chapter]
 \newtheorem{proposition}[theorem]{Proposition}
 \newtheorem{lemma}[theorem]{Lemma}
 \newtheorem{corollary}[theorem]{Corollary}
 \newtheorem{conjecture}[theorem]{Conjecture}
 \theoremstyle{definition}
 \newtheorem{definition}[theorem]{Definition}
 \newtheorem{example}[theorem]{Example}
 \newtheorem{exercise}[theorem]{Exercise}
 \theoremstyle{remark}
 \newtheorem{remark}[theorem]{Remark}
 \newtheorem{notes}[theorem]{Notes}
 \newtheorem{convention}[theorem]{Convention}
%---------------------------thms end---------------------------------------------------------------------


%---------------------------common 1 begin-------------------------------------------------------------------
%\newcommand{\HOPi}{HOPi}
\newcommand{\HOPi}{\ensuremath{\Pi}}
\newcommand{\LHOPi}{LHOPi}
%\newcommand{\FOPi}{FOPi}
\newcommand{\FOPi}{\ensuremath{\pi}}
\newcommand{\HOCCS}{HOCCS}
\newcommand{\CHOCS}{CHOCS}
\newcommand{\PC}{Plain CHOCS}
\newcommand{\HOPid}{\ensuremath{\Pi^d}}
\newcommand{\HOPiD}{\ensuremath{\Pi^D}}
\newcommand{\HOPiDd}{\ensuremath{\Pi^{D,d}}}
%
\newcommand{\NAT}{\mathcal{N}}
%
\newcommand{\para}{\,|\,}
%\newcommand{\para}{|}
\newcommand{\cho}{\;{+}\;}
\newcommand{\mat}{{=}}
\newcommand{\mmat}{{\neq}}

%
\newcommand{\fn}[1]{\mbox{\rm fn($#1$)}} %free names
\newcommand{\bn}[1]{\mbox{\rm bn($#1$)}} %bound names
\newcommand{\n}[1]{\mbox{\rm n($#1$)}} %names
\newcommand{\fnv}[1]{\mbox{\rm fnv($#1$)}} %free name variables
\newcommand{\bnv}[1]{\mbox{\rm bnv($#1$)}} %bound name variables
\newcommand{\nv}[1]{\mbox{\rm nv($#1$)}} %name variables
\newcommand{\fpv}[1]{\mbox{\rm fpv($#1$)}} %free process variables
\newcommand{\bpv}[1]{\mbox{\rm bpv($#1$)}} %bound process variables
\newcommand{\pv}[1]{\mbox{\rm pv($#1$)}} %process variables
\newcommand{\vv}[1]{\mbox{\rm vv($#1$)}} %variables


%
\newcommand{\encode}[3]{ \lds #1 \rds^{#2}_{#3}}
\newcommand{\dpara}{\;|\!|\, }
\newcommand{\nber}[1]{|#1|}
%\newcommand{\lrangle}[1]{\langle #1 \rangle}
%
\newcommand{\fosub}[2]{\{#1/#2\} }
%\newcommand{\hosub}[2]{[#1/#2] }
\newcommand{\hosub}[2]{\{#1/#2\} }
\newcommand{\hosubd}[2]{[#1/#2] }%dynamic substitution
%
\newcommand{\vect}[1]{{#1_1,#1_2,...,#1_n} }
\newcommand{\ve}[1]{\widetilde{#1}}
\newcommand{\vet}[1]{{#1}}
%\newcommand{\vet}[1]{\widetilde{#1}}
%
\newcommand{\size}[1]{|#1|}
%strong transition
\newcommand{\st}[1]{\,{\xrightarrow{#1}}\, }
%strong transition star
\newcommand{\sts}[1]{\,{\xrightarrow{#1}}^*\, }
%weak transition
\newcommand{\wt}[1]{{\xLongrightarrow{#1}} }
%\newcommand{\wt}[1]{{\stackrel{#1}{\Longrightarrow}} }
%\newcommand{\rc}[1]{{\color{Red} #1}}
\newcommand{\rc}[1]{{\color{red} #1}}
\newcommand{\rrc}[1]{{#1}} %deprecated color
%\newcommand{\bc}[1]{{\color{Blue} #1}}
\newcommand{\bc}[1]{{\color{blue} #1}}
\newcommand{\rbc}[1]{{#1}} %deprecated color
%
\newcommand{\da}{\!\!\downarrow}
\newcommand{\Da}{\!\!\Downarrow}
\newcommand{\R}{\mathcal{R} }
%
%\newcommand{\sep}{\vspace*{0.7cm}}
\newcommand{\sepp}{\vspace*{0.4cm}}
%
\newcommand{\itsep}{\itemsep -0.2em}
%
%\newcommand{\nsep}{\vspace{-0.13cm}}
%\newcommand{\nsepv}[1]{\vspace{-#1cm}}
\newcommand{\nsep}{\vspace{0mm}}
\newcommand{\nsepv}[1]{\vspace{0mm}}
\newcommand{\nsepvs}[1]{\vspace{-#1cm}} %negative vertical space
\newcommand{\psepvs}[1]{\vspace{#1cm}} %positive vertical space

%
\newcommand{\hfilldots}{$\cdots\cdots\cdots\cdots\cdots\cdots\cdots\cdots\cdots\cdots\cdots\cdots\cdots\cdots\cdots\cdots\cdots\cdots$ }
%
%\usepackage{mathabx}
\newcommand{\FROMHERE}{\rc{$\blacktriangleright\blacktriangleright\blacktriangleright\blacktriangleright\blacktriangleright\blacktriangleright\blacktriangleright\blacktriangleright\blacktriangleright\blacktriangleright\blacktriangleright\blacktriangleright\blacktriangleright\blacktriangleright\blacktriangleright\blacktriangleright\blacktriangleright\blacktriangleright\blacktriangleright\blacktriangleright\blacktriangleright\blacktriangleright$ \textsc{\large from here!!!!!!....}} }
%
\newcommand{\TODO}{$\blacktriangleright\blacktriangleright\blacktriangleright\blacktriangleright\blacktriangleright\blacktriangleright\blacktriangleright\blacktriangleright\blacktriangleright\blacktriangleright\blacktriangleright\blacktriangleright\blacktriangleright\blacktriangleright\blacktriangleright\blacktriangleright\blacktriangleright\blacktriangleright\blacktriangleright\blacktriangleright\blacktriangleright\blacktriangleright$ \textsc{\large TODO!!!!!!....} }
%---------------------------common 1 end-------------------------------------------------------------------




%---------------------------common 2(bisimulation) begin-------------------------------------------------------------------
%CCS
\newcommand{\CCSSB}{\sim }
\newcommand{\CCSWB}{\approx }
\newcommand{\CCSWC}{= }
\newcommand{\CCSIMP}{\vdash }
%first-order barbed bisimulation
\newcommand{\FOBB}{\approx_{bf}}
%higher-order barbed bisimulation
\newcommand{\HOBB}{\approx_{bh}}
%first-order barbed congruence
\newcommand{\FOBC}{\approx_{bf}^c}
%higher-order barbed congruence
\newcommand{\HOBC}{\approx_{bh}^c}
%first-order barbed equivalence
\newcommand{\FOBE}{\approx_{bf}^e}
%higher-order barbed equivalence
\newcommand{\HOBE}{\approx_{bh}^e}
%strong higher-order context bisimulation
\newcommand{\SHOCB}{\sim_{ct}}
%higher-order context bisimulation
\newcommand{\HOCB}{\approx_{ct}}
%higher-order context congruence
\newcommand{\HOCC}{\approx_{Ct}^c}
%higher-order normal bisimulation
\newcommand{\HONB}{\approx_{nr}}
%higher-order triggered bisimulation
\newcommand{\HOTB}{\approx_{Tr}}
%index operation in quasi open bisimulations
\newcommand{\qidx}[2]{{#1}^{#2} }
%higher-order bisimulation defined in [San92]
\newcommand{\HB}{\approx_H }
%
\newcommand{\CHB}{\approx_H^{CHOCS} }
%a symbol for barbed bisimulation in [San92]
\newcommand{\BB}{\approx }
%a symbol for early bisimulation in [San92]
\newcommand{\EB}{\approx_e }
%local bisimilarity
\newcommand{\LB}{\approx_l }
%structural equivalence
%\newcommand{\SE}{\sim_s }
%\newcommand{\SE}{\sim }
\newcommand{\SE}{\equiv }
%structural equivalence
\newcommand{\HOSE}{\sim }
%---------------------------common 2(bisimulation) end-------------------------------------------------------------------


%---------------------------mHO begin-------------------------------------------------------------------
%higher-order pi-calculus with mismatch
%\newcommand{\mHOPi}{mHOPi }
\newcommand{\mHOPi}{mHOPi}
%encoding higher-order to first-order
\newcommand{\hofo}[1]{{\mathcal{C}[\![ #1]\!]} }
%open strong HO bisimilarity
\newcommand{\HOOSB}{\sim_{oh} }
%open weak HO bisimilarity
\newcommand{\HOOWB}{\approx_{oh} }
%open weak HO congruence
\newcommand{\HOOWC}{\simeq_{oh} }
%quasi open (weak) higher-order bisimilarity
\newcommand{\QOHOB}[1]{\approx_{qh}^{#1} }
%quasi open (weak) higher-order congruence
\newcommand{\QOHOC}{\simeq_{qh} }
%local higher-order bisimilarity
\newcommand{\LHOB}{\approx_{lh} }
%local higher-order congruence
\newcommand{\LHOC}{\simeq_{lh} }
%the axiom system for open weak HO bisimilarity
\newcommand{\ASLM}{\mathcal{AS}_{LM} }
%the axiom system for quasi open weak HO bisimilarity or local HO bisimilarity
\newcommand{\ASLMQ}{\mathcal{AS}_{LM_q} }
%the axiom system for quasi open weak HO bisimilarity or local HO bisimilarity
\newcommand{\ASLML}{\mathcal{AS}_{LM_l} }
%---------------------------mHO end-------------------------------------------------------------------


%---------------------------encoding begin-------------------------------------------------------------------
%
\newcommand{\lds}{[\![}
\newcommand{\rds}{]\!]}
%
\newcommand{\encoding}[3]{ \lds #1 \rds^{#2}_{#3}}
\newcommand{\iencoding}[3]{ \lds #1 \rds^{#2}_{#3}}
\newcommand{\enc}[1]{\llbracket #1 \rrbracket}
%
\newcommand{\wm}[1]{ \mathcal{W}[\![ #1 ]\!]}
\newcommand{\iwm}[1]{ \mathcal{W}^n[\![ #1 ]\!]}
\newcommand{\chan}[1]{#1{-}chan}
\newcommand{\ichan}[1]{#1{-}ichan}
\newcommand{\backs}[1]{\!\setminus\!#1\!}
\newcommand{\dmd}[2]{\{\diamond_{#1}\; #2\} }
%\newcommand{\dmd}[2]{\{\diamond\; #2\} }
\newcommand{\idmd}[2]{\{\diamond_{#1}^i\; #2\} }
%ground bisimulation
\newcommand{\GB}{\stackrel{.}{\approx}}
%strong applicative higher-order bisimulation
\newcommand{\SAHOB}{\stackrel{:}{\sim}}
%applicative higher-order bisimulation
\newcommand{\AHOB}{\stackrel{:}{\approx}}
%indexed applicative higher-order bisimulation
\newcommand{\IAHOB}{\stackrel{:}{\approx}_S}
%indexed higher-order barbed bisimulation
\newcommand{\IHOBB}{\approx_{bh}^S}
%indexed higher-order barbed congruence
\newcommand{\IHOBC}{^c\approx_{bh}^S}
%indexed higher-order barbed equivalence
\newcommand{\IHOBE}{^e\approx_{bh}^S}
%indexed strong higher-order context bisimulation
\newcommand{\ISHOCB}{\sim_{Ct}^S}
%indexed higher-order context bisimulation
\newcommand{\IHOCB}{\approx_{Ct}^S}
%indexed higher-order context congruence
\newcommand{\IHOCC}{\simeq_{Ct}^S}
%indexed higher-order normal bisimulation
\newcommand{\IHONB}{\approx_{Nr}^S}
%wired bisimulation
\newcommand{\WB}{\approx_{Wr}}
%wired congruence
\newcommand{\WC}{\simeq_{Wr}}
%indexed wired bisimulation
\newcommand{\IWB}{\approx_{Wr}^S}
%indexed wired congruence
\newcommand{\IWC}{\simeq_{Wr}^S}
%
\newcommand{\OHOCB}{\approx_{Ct}}
%quasi open bisimulation
\newcommand{\QOB}[1]{\approx_q^{#1} }
%local open bisimilarity
\newcommand{\LOB}{\approx_{lo}}
%local wired bisimulation
\newcommand{\LWB}{\approx_{lw}}
%---------------------------encoding end-------------------------------------------------------------------


%---------------------------logic begin-------------------------------------------------------------------
%first-order q-open bisimilarity
\newcommand{\FOQOB}{\approx_{qo}}
%higher-order q-open bisimilarity
\newcommand{\HOQOB}{\approx_{qoh}}
%higher-order q-open linear bisimilarity
\newcommand{\HOQOLB}{\approx_{qoh}^{l}}
%higher-order local bisimilarity
\newcommand{\HOLB}{\approx_{l} }
%higher-order local congruence
\newcommand{\HOLC}{\simeq_{l} }
%higher-order local linear bisimilarity
\newcommand{\HOLLB}{\approx_{ll} }
%higher-order local linear congruence
\newcommand{\HOLLC}{\simeq_{ll} }
%higher-order local linear variant bisimilarity
\newcommand{\HOLLVB}{\approx_{ll}^{v} }
%higher-order local linear variant bisimilarity: hierarchy
\newcommand{\HOLLVBH}[1]{\approx^{#1} }
%higher-order general bisimilarity
\newcommand{\HOGNLB}{\approx }
%linear higher-order logic
\newcommand{\LHOL}{LHOL }
%on logic characterization
\newcommand{\dias}[1]{\langle {#1} \rangle }
\newcommand{\boxs}[1]{[ {#1} ] }
\newcommand{\conjuct}{\wedge }
\newcommand{\disjuct}{\vee }
%semantically imply or not imply
\newcommand{\semimply}{\vDash }
\newcommand{\nsemimply}{\not\vDash }
%syntactically imply or not imply
\newcommand{\synimply}{\vdash }
\newcommand{\nsynimply}{\not\vdash }
%(): [] or <>
\newcommand{\diaobox}[1]{( {#1} ) }
%constructive implication
\newcommand{\ci}{\Rightarrow }
%depth of a formula
\newcommand{\depth}[1]{| {#1} | }
%characteristic formula
\newcommand{\cformula}[2]{C^{#1}(#2) }
%
\newcommand{\HOLE}{\approx_{L} } %higher-order logical equivalence in the weak case
\newcommand{\HOLEH}[1]{\approx^{#1}_{L} } %higher-order logical equivalence in hierarchy  in the weak case
\newcommand{\NHOLEH}[1]{\not\approx^{#1}_{L} } %negation of higher-order logical equivalence in hierarchy  in the weak case
%---------------------------logic end-------------------------------------------------------------------


%---------------------------common 3() begin-------------------------------------------------------------------
%more
\newcommand{\DEF}{\stackrel{\textrm{def}}{=}} %definition
\newcommand{\myxcancel}[1]{\xcancel{#1}} %cancel notation
%add to original head_pc
\newcommand{\lrangle}[1]{\langle #1 \rangle} % for abstraction and etc.
\newcommand{\CCSSLE}{\sim_{L} } %higher-order logical equivalence in the weak case
\newcommand{\CCSSLEH}[1]{\sim_{#1} } %higher-order logical equivalence in hierarchy  in the weak case
%consider redefine \encoding{}{}{} as ``\llbracket #1 \rrbracket^{#2}_{#3}"
\newcommand{\piBiCo}{\approx_\downarrow^\pi}
%pipe definition
\newcommand{\PIPE}[4]{\mathfrak{#1}{-}\mathfrak{pipe}(#2,#3,#4) }
\newcommand{\PIPEA}[1]{\mathbf{#1}{\textendash}\mathfrak{pipe} }

%---------------------------common 3() end-------------------------------------------------------------------


%---------------------------expressiveness begin-------------------------------------------------------------------
%\makeindex
\newcommand{\AEHOPi}{\ensuremath{=_{\Pi}} } %absolute equality of HOPi
\newcommand{\AEHOPid}{\ensuremath{=_{\Pi^d}} } %absolute equality of HOPid
\newcommand{\AEHOPiddiv}{\ensuremath{=_{\scriptscriptstyle \Pi^d}^{\scriptscriptstyle \Uparrow}} } %absolute equality of HOPid with divergence-sensitivity
\newcommand{\AEFOPi}{\ensuremath{=_{\pi}} } %absolute equality of FOPi
%
\newcommand{\EWLB}{\ensuremath{\approx_{l}} } %external weak local bisimilarity
%\newcommand{\EWLB}{\approx_{l}^{\pi} } %external weak local bisimilarity
\newcommand{\EWBFOPi}{\ensuremath{\approx_{w}^{\pi}} }%external weak bisimilarity of FOPi
\newcommand{\SGB}{\ensuremath{\sim_{g}} }%strong ground bisimilarity of FOPi
\newcommand{\WGB}{\ensuremath{\approx_{g}} }%ground bisimilarity of FOPi

%\newcommand{\ground}{ground }%the name of "ground": can be omitted
\newcommand{\ground}{ }%the name of "ground": can be omitted
%\newcommand{\Ground}{Ground }%the name of "ground": can be omitted
\newcommand{\Ground}{ }%the name of "ground": can be omitted

\newcommand{\CBHOPi}{\approx_{Ct}^{\Pi} }%weak context bisimilarity of HOPi
\newcommand{\WCB}{\ensuremath{\approx_{ct}} }%weak context bisimilarity for general HO
\newcommand{\WWCB}{\ensuremath{\dot\approx_{ct}} }%weak context bisimilarity restricted to the image of an encoding
\newcommand{\SCB}{\ensuremath{\sim_{ct}} }%strong context bisimilarity for general HO
\newcommand{\WSCB}{\ensuremath{\dot\sim_{ct}} }%strong context bisimilarity restricted to the image of an encoding
\newcommand{\WNB}{\ensuremath{\approx_{nr}} }%weak normal bisimilarity for general HO
\newcommand{\SNB}{\ensuremath{\sim_{nr}} }%strong normal bisimilarity for general HO
\newcommand{\diverge}[1]{#1^{\Uparrow}}
%
\newcommand{\trigger}{\ensuremath{Tr_m} } %trigger of HOPi
\newcommand{\triggerD}{\ensuremath{Tr_m^D} } %trigger of HOPiD
\newcommand{\triggerd}{\ensuremath{Tr_m^d} } %trigger of HOPiDd
\newcommand{\triggerExp}{\ensuremath{\overline{m}} } %trigger of HOPi expanded
\newcommand{\triggerDExp}{\ensuremath{\lrangle{Z}\overline{m}Z} } %trigger of HOPiD expanded
\newcommand{\triggerdExp}{\ensuremath{\lrangle{z}\overline{m}[\lrangle{Y}(Y\lrangle{z})]} } %trigger of HOPiDd expanded
%---------------------------expressiveness end-------------------------------------------------------------------


%---------------------------theory of computation begin-------------------------------------------------------------------
%\makeindex
%\makeatletter
\makeatletter
\@ifundefined{newslide}{%
\newcommand{\slidesepa}{}
}{%
\newcommand{\slidesepa}{\newslide} %defined as \newslide when used in slides making
}
%whether to make a new input file
\makeatletter
\@ifundefined{endinput}{%
\newcommand{\infilesepa}{}
}{%
\newcommand{\infilesepa}{\endinput} %defined as \newslide when used in slides making
}

%
\newcommand{\stress}[1]{\rc{#1}} %emphasize
\newcommand{\strs}[1]{{#1}} %emphasize
\newcommand{\strss}[1]{\rc{#1}} %emphasize
%\newcommand{\stress}[1]{\textbf{#1}} %emphasize
\newcommand{\length}[1]{|#1|} %length/size of #1
\newcommand{\lang}[1]{\mathcal{L}(#1)} %language of #1
\newcommand{\choice}{\quad|\quad} %separating items in a grammar definition
\newcommand{\showspace}{\raisebox{0.45mm}{\texttt{\char32}}} %explicit space symbol \raisebox{-0.7ex}{aa}
%\newcommand{\showspace}{\textvisiblespace}
%\newcommand{\showspace}{\sqcup}
%---------------------------theory of computation end-------------------------------------------------------------------


%---------------------------others begin-------------------------------------------------------------------
%
%color for comments
\newcommand{\lh}[1]{{\textcolor[RGB]{178,58,238}{#1}}}
\newcommand{\yq}[1]{{\textcolor[RGB]{255,215,0}{#1}}}
\newcommand{\xx}[1]{{\textcolor[RGB]{0,200,0}{#1}}}
\newcommand{\oo}[1]{{\color{red} #1}}
\newcommand{\nts}[1]{{\textcolor[RGB]{0,221,0}{#1}}}

\newcommand{\greycolor}[1]{{\textcolor[RGB]{112,128,144}{#1}}}
\newcommand{\greycolora}[1]{{\textcolor[RGB]{119,136,153}{#1}}}

%switch for comments to tiny up
%\newcommand{\tdup}[1]{{\tiny #1}}%keep the related comments temporarily
\newcommand{\tdup}[1]{}%remove the related comments

%---------------------------others end-------------------------------------------------------------------


%---------------------------
% Local Variables:
% mode: LaTeX
% TeX-master: "main.tex"
% End:
% End:

\section{\xxxx{Proofs for Section \ref{s:encoding}}}\label{appendix:up-to-context_encodings}

We give the proof of Proposition \ref{prop:up-to-context_encodings}.
For convenience, we reproduce the definition of the context bisimulation up-to context. 
By saying some term is an encoding, we mean that it is the encoding of certain ($\pi$) process. Recall that the encoding of a context is a context put through the encoding with the hole translated into a hole. 
In the output clause of Definition \ref{d:up3_encoding}, $Q$ is required to make the same output as $P$. This corresponds to the requirement of bound output in the bisimulation of $\pi$-calculus, and suffices for our purpose.
\begin{definition}[up-to context]\label{d:up3_encoding}
%modify from up-to \SCB above
%P'\equiv C[P''],  Q'\equiv C[Q''], and P''\R Q''
A symmetric relation $\mathcal{R}$ on the image of the encoding is a context bisimulation up-to context, if whenever $P\,\mathcal{R}\, Q$ the following properties hold.
\begin{itemize}
\item if $P \st{\alpha} P'$ and $\alpha$ is $\tau$ or $a(A)$ in which $A$ is $\lrangle{Z}(Z\lrangle{b})$, then $Q \wt{\widehat{\alpha}} Q'$ for some $Q'$, $P'\equiv C[P'']$ and $Q'\equiv C[Q'']$ for some $P'', Q''$ and context $C$ that is an encoding, and $P''\,\mathcal{R}\, Q''$;

\item if $P \st{(\ve{c})\overline{a}A} P'$ in which $A$ is $\lrangle{Z}(Z\lrangle{b})$ and $\ve{c}$ is either empty or $\{b\}$, 
%then $Q \wt{(\ve{d})\overline{a}B} Q'$ for some $B$ that is an enoding and of the same type as $A$, 
then $Q \wt{(\ve{c})\overline{a}A} Q'$ 
%and for every $E[\cdot]$ (\xx{or $E[X]$?}) that is an encoding and satisfies 
and for every $E[X]$ s.t.  
%$\{\ve{c},\ve{d}\}\cap \fn{E}=\emptyset$, 
$\{\ve{c}\}\cap \fn{E}=\emptyset$, 
%it holds that $(\ve{c})(E[A]\para P') \equiv C[P'']$ and $(\ve{d})(E[B]\para Q')\equiv C[Q'']$
it holds that \\
$(\ve{c})(E[A]\para P') \equiv C[P'']$ and $(\ve{c})(E[A]\para Q')\equiv C[Q'']$  
for some $P'', Q''$ and context $C$ that are encodings, and $P''\,\mathcal{R}\, Q''$.
\end{itemize}
\end{definition}

The framework of the proof is akin to proving the congruence of a bisimilarity in a higher-order setting \cite{SW01a,DHS18,Fu07,MH02,San92} (think of $\R$ as the context bisimilarity in the definition of $\R'$ below and the argument is similar). 
Indeed this observation ensures that the context bisimilarity on top of the encodings is a congruence. 

\begin{proof}[Proof of Proposition \ref{prop:up-to-context_encodings}]
Assume $\R$ is as defined in Definition \ref{d:up3_encoding}. 
We show that the relation $\R'$ defined below is a context bisimulation \xxx{up-to $\equiv$}, thus proving the proposition.
\[
\R' \DEF \{(C[P], C[Q]) \,|\, P \,\R\, Q, \mbox{ and $C$ is a context that is an encoding}\}
\]
We make a case analysis. Suppose $P\,\R'\, Q$ where $P$ and $Q$ are processes. % (not abstractions). 
We notice that we may sometimes %write context and $C[\cdot]$ and often 
omit the $\cdot$ in $C[\cdot]$ for the sake of conciseness. 

%-------------PROOF BODY BEGIN-------------
%\xx{TODO: to fetch from `NOTES'  --- DONE!}

\begin{enumerate}

\item $C[P]\st{a(A)} T_1$ in which $A\equiv \lrangle{X}X\lrangle{b}$.
\begin{itemize}

\item The case when $C$ is $[\cdot]$ is immediate because $C[P]$ and $C[Q]$ are $P$ and $Q$ respectively, and $P\,\R\, Q$ from premise. The cases when $C$ is $\overline{a}(C_1[\cdot]).R$, $\overline{a}A_1.C_1[\cdot]$, $\lrangle{X}C_1[\cdot]$, $\lrangle{x}C_1[\cdot]$ are not possible. The cases when $C$ is $C_1[\cdot]\lrangle{A_1}$, $C_1[\cdot]\lrangle{d}$ fall onto the other cases after possible application.


%\item $C$ is $[\cdot]$. This is immediate because $C[P]$ and $C[Q]$ are $P$ and $Q$ respectively, and $P\,\R\, Q$ from premise. 

\item $C$ is $a(X).C_1[\cdot]$. In this case, $C[P]\st{a(A)} T_1\equiv C_1[P]\hosub{A}{X}\equiv C_2[P]$ for $C_2\DEF C_1\hosub{A}{X}$ since $P$ is closed. Then $C[Q]$ simulates by \\
$C[Q] \st{a(A)}  C_1[Q]\hosub{A}{X}\equiv C_2[Q]$. So we have $C_2[P] \,\R'\, C_2[Q]$ with \\ 
$P\,\mathcal{R}\, Q$.

% \item $C$ is $\overline{a}(C_1[\cdot]).R$. This is not possible.
% \item $C$ is $\overline{a}A_1.C_1[\cdot]$. This is not possible.
% \item $C$ is $\lrangle{X}C_1[\cdot]$. This is not possible.
% \item $C$ is $\lrangle{x}C_1[\cdot]$. This is not possible.
% \item $C$ is $C_1[\cdot]\lrangle{A_1}$. This actually falls onto the other cases after application. 
% \item $C$ is $C_1[\cdot]\lrangle{d}$. This actually falls onto the other cases after application. 

\item $C$ is $R\para C_1[\cdot]$. There are three possibilities. 
\begin{itemize}
\item The action is from $P$. That is, $P\st{a(A)} P'$, and $C[P] \st{a(A)} R\para C_1[P']$. Since $P\,\R\, Q$, we know that $Q \wt{a(A)} Q'$ for some $Q'$, $P'\equiv C'[P'']$ and $Q'\equiv C'[Q'']$ for some $P'', Q''$ and context $C$ as stipulated, and $P''\,\mathcal{R}\, Q''$. So $C[Q]$ simulates by $C[Q] \wt{a(A)} R\para C_1[Q']$. Now we have for context $C_2\DEF R\para C_1[C'[]]$ 
%(because $P$ is on a firable position)
\[
\begin{array}{lclcl}
R\para C_1[P'] &\equiv& R\para C_1[C'[P'']] &\equiv& C_2[P'']  \\
R\para C_1[Q'] &\equiv& R\para C_1[C'[Q'']] &\equiv& C_2[Q'']
\end{array}
\]
Hence we obtain $C_2[P''] \,\R'\, C_2[Q'']$ with $P''\,\mathcal{R}\, Q''$. 

\item The action is from $C_1$. That is,  $C_1[P]\st{a(A)} C_1'[P]$, and \\
$C[P] \st{a(A)} R\para C_1'[P]$. So $C[Q]$ simulates by $C[Q] \st{a(A)} R\para C_1'[Q]$. \\
Now we have for context $C_2\DEF R\para C_1'[]$ 
%(because $P$ is on a firable position)
\[
\begin{array}{lcl}
R\para C_1'[P] &\equiv& C_2[P]  \\
R\para C_1'[Q] &\equiv& C_2[Q]
\end{array}
\]
Hence we obtain $C_2[P] \,\R'\, C_2[Q]$ with $P\,\mathcal{R}\, Q$. 

\item The action is from $R$. That is, $R\st{a(A)} R'$, and $C[P] \st{a(A)} R'\para C_1[P]$. So $C[Q]$ simulates by $C[Q] \st{a(A)} R'\para C_1[Q]$. Now we have for context $C_2\DEF R'\para C_1[]$ 
%(because $P$ is on a firable position)
\[
\begin{array}{lcl}
R'\para C_1[P] &\equiv& C_2[P]  \\
R'\para C_1[Q] &\equiv& C_2[Q]
\end{array}
\]
Hence we obtain $C_2[P] \,\R'\, C_2[Q]$ with $P\,\mathcal{R}\, Q$. 
\end{itemize}

\item $C$ is $(d)C_1[\cdot]$. There are two possibilities. 
\begin{itemize}
\item The action is from $P$. That is, $P\st{a(A)} P'$, and $C[P] \st{a(A)} (d)C_1[P']$. Since $P\,\R\, Q$, we know that $Q \wt{a(A)} Q'$ for some $Q'$, $P'\equiv C'[P'']$ and $Q'\equiv C'[Q'']$ for some $P'', Q''$ and context $C$ as stipulated, and $P''\,\mathcal{R}\, Q''$. So $C[Q]$ simulates by 
 $C[Q] \wt{a(A)} (d)C_1[Q']$. Now for some context $C_2\DEF (d)C_1[C'[]]$ we have
%(because $P$ is on a firable position)
\[
\begin{array}{lclcl}
(d)C_1[P'] &\equiv& (d)C_1[C'[P'']] &\equiv& C_2[P'']  \\
(d)C_1[Q'] &\equiv& (d)C_1[C'[Q'']] &\equiv& C_2[Q'']
\end{array}
\]
Hence we obtain $C_2[P''] \,\R'\, C_2[Q'']$ with $P''\,\mathcal{R}\, Q''$. 

\item The action is from $C_1$. That is, $C_1[P]\st{a(A)} C_1'[P]$, and \\
$C[P] \st{a(A)} (d)C_1'[P]$. So $C[Q]$ simulates by $C[Q] \st{a(A)} (d)C_1'[Q]$. \\
Now we have for context $C_2\DEF (d)C_1'[]$ 
%(because $P$ is on a firable position)
\[
\begin{array}{lcl}
(d)C_1'[P] &\equiv& C_2[P]  \\
(d)C_1'[Q] &\equiv& C_2[Q]
\end{array}
\]
Hence we obtain $C_2[P] \,\R'\, C_2[Q]$ with $P\,\mathcal{R}\, Q$. 

\end{itemize}
\end{itemize}




\item  $C[P]\st{(\ve{c})\overline{a}A} T_1$ in which $A\equiv \lrangle{X}X\lrangle{b}$ and $\ve{c}$ is $\{b\}$ or empty.
\begin{itemize}

\item The case when $C$ is $[\cdot]$ is immediate. The cases when $C$ is $a(X).C_1[\cdot]$, $\lrangle{X}C_1[\cdot]$, $\lrangle{x}C_1[\cdot]$ are not possible. The cases when $C$ is $C_1[\cdot]\lrangle{A_1}$, $C_1[\cdot]\lrangle{d}$ fall onto the other cases after possible application.

% \item $C$ is $[\cdot]$. This is immediate. 
% \item $C$ is $a(X).C_1[\cdot]$. This is not possible.

\item $C$ is $\overline{a}(C_1[\cdot]).R$. This case does not make sense because in the encodings the term allowed to be on the object position of an output can only be of the form $\lrangle{X}X\lrangle{b}$. 

\item $C$ is $\overline{a}A_1.C_1[\cdot]$. Then here $\ve{c}$ is empty, $A_1$ is $A$, and $T_1$ is $C_1[P]$. 
So $C[Q]$ simulates by $C[Q]\st{\overline{a}A} C_1[Q]$. Now for every $E[\cdot]$ as specified in the definition, 
consider $E[A]\para C_1[P] \equiv C'[P]$ and $E[A]\para  C_1[Q] \equiv C'[Q]$ in which $C'\DEF E[A]\para C_1[]$. Thus we have $C'[P] \,\R'\, C'[Q]$ with $P\,\mathcal{R}\, Q$.


% \item $C$ is $\lrangle{X}C_1[\cdot]$. This is not possible.
% \item $C$ is $\lrangle{x}C_1[\cdot]$. This is not possible.
% \item $C$ is $C_1[\cdot]\lrangle{A_1}$. This falls onto the other cases after application. 
% \item $C$ is $C_1[\cdot]\lrangle{d}$. This falls onto the other cases after application. 

\item $C$ is $R\para C_1[\cdot]$. There are four  possibilities.
\begin{itemize}
\item The action is from $R$. That is, $R\st{(\ve{c})\overline{a}A} R'$ and \\
$C[P]\equiv R\para C_1[P]\st{(\ve{c})\overline{a}A} T_1\equiv R'\para C_1[P]$. In this case, $C[Q]$ simulates by $C[Q]\st{(\ve{c})\overline{a}A} R'\para C_1[Q]$. Now we have  for every $E$ as demanded, and $C_2\DEF (\ve{c})(E[A]\para R'\para C_1[])$.
\[
\begin{array}{lcl}
(\ve{c})(E[A]\para R'\para C_1[P]) &\equiv& C_2[P]  \\
(\ve{c})(E[A]\para R'\para C_1[Q]) &\equiv& C_2[Q]
\end{array}
\]
Hence we obtain $C_2[P] \,\R'\, C_2[Q]$ with $P\,\mathcal{R}\, Q$. 

\item The action is from $C_1$. That is, $C_1[P]\st{(\ve{c})\overline{a}A} C_1'[P]$ and $C[P]\equiv R\para C_1[P]\st{(\ve{c})\overline{a}A} T_1\equiv R\para C_1'[P]$. In this case, $C[Q]$ simulates by $C[Q]\st{(\ve{c})\overline{a}A} R\para C_1'[Q]$. Now we have  for every $E$ as demanded, and $C_2\DEF (\ve{c})(E[A]\para R\para C_1'[])$.
\[
\begin{array}{lcl}
(\ve{c})(E[A]\para R\para C_1'[P]) &\equiv& C_2[P]  \\
(\ve{c})(E[A]\para R\para C_1'[Q]) &\equiv& C_2[Q]
\end{array}
\]
Hence we obtain $C_2[P] \,\R'\, C_2[Q]$ with $P\,\mathcal{R}\, Q$. 

\item The action is from $P$. That is, $P\st{(\ve{c})\overline{a}A} P'$ and \\
$C[P]\equiv R\para C_1[P]\st{(\ve{c})\overline{a}A} T_1\equiv R\para C_1[P']$. In this case, since $P\,\R\, Q$, we know that $Q \wt{(\ve{c})\overline{a}A} Q'$, 
and for every $E$ as described in the definition, 
it holds that $(\ve{c})(E[A]\para P') \equiv C'[P'']$, $(\ve{c})(E[A]\para Q')\equiv C'[Q'']$, and $P''\,\mathcal{R}\, Q''$. So $C[Q]$ simulates by $C[Q] \wt{(\ve{c})\overline{a}A} R\para C_1[Q']$. Now, we have   for every $E$, and some $C_2$ (which exists because $P$ is in a firable position) and $C_3\DEF R\para C_2[C'[]]$. 
\[
\begin{array}{lcl}
(\ve{c})(E[A]\para R\para C_1[P']) &\equiv& R\para  (\ve{c})(E[A]\para C_1[P']) \\
&\equiv&  R\para C_2[(\ve{c})(E[A]\para P')] \\
&\equiv& R\para C_2[C'[P'']] \\
&\equiv& C_3[P''] \\\\
(\ve{c})(E[A]\para R\para C_1[Q']) &\equiv& R\para (\ve{c})(E[A]\para C_1[Q']) \\
&\equiv& R\para C_2[(\ve{c})(E[A]\para Q')] \\
&\equiv&  R\para C_2[C'[Q'']] \\
&\equiv& C_3[Q'']
\end{array}
\]
Hence we obtain $C_3[P''] \,\R'\, C_3[Q'']$ with $P''\,\mathcal{R}\, Q''$. 

\item The action is from $P$ and $C_1$. That is, $P\st{\overline{a}A} P'$ and $C[P]\equiv R\para C_1[P]\st{(b)\overline{a}A} T_1\equiv R\para C_1'[P']$. 
%This case is similar to the last one. 
Since $P\,\R\, Q$, we know that $Q \wt{\overline{a}A} Q'$, 
and for every $E$ as described in the definition, 
it holds that \\
$E[A]\para P' \equiv C'[P'']$, $E[A]\para Q'\equiv C'[Q'']$, and $P''\,\mathcal{R}\, Q''$. So $C[Q]$ simulates by $C[Q] \wt{(b)\overline{a}A} R\para C_1'[Q']$. Now, we have  for every $E$, and some $C_2$ (which exists because $P$ is in a firable position) and $C_3\DEF R\para (b)C_2[C'[]]$. 
\[
\begin{array}{lcl}
(b)(E[A]\para R\para C_1'[P']) &\equiv& R\para  (b)(E[A]\para C_1'[P']) \\
&\equiv&  R\para (b)C_2[E[A]\para P'] \\
&\equiv& R\para (b)C_2[C'[P'']] \\
&\equiv& C_3[P''] \\\\
(b)(E[A]\para R\para C_1'[Q']) &\equiv& R\para (b)(E[A]\para C_1'[Q']) \\
&\equiv& R\para (b)C_2[E[A]\para Q'] \\
&\equiv&  R\para (b)C_2[C'[Q'']] \\
&\equiv& C_3[Q'']
\end{array}
\]
Hence we obtain $C_3[P''] \,\R'\, C_3[Q'']$ with $P''\,\mathcal{R}\, Q''$. 
\end{itemize}


\item $C$ is $(d)C_1[\cdot]$. There are seven possibilities.  
\begin{itemize}
\item The action is from $P$, $\ve{c}$ is empty and $d$ is $b$. That is, $P\st{\overline{a}A} P'$ and $C[P]\st{(b)\overline{a}A} C_1[P']$. Then since $P\,\R\, Q$, 
we know that $Q \wt{\overline{a}A} Q'$, 
and for every $E$ as described in the definition, 
it holds that $E[A]\para P' \equiv C'[P'']$, $E[A]\para Q'\equiv C'[Q'']$, and $P''\,\mathcal{R}\, Q''$. So $C[Q]$ simulates by $C[Q] \wt{(b)\overline{a}A} C_1[Q']$. Now, we have  for every $E$, and some $C_2$ (which exists because $P$ is in a firable position) and $C_3\DEF (b)C_2[C'[]]$. 
\[
\begin{array}{lclclcl}
(b)(E[A]\para C_1[P']) &\equiv&  (b)C_2[E[A]\para P'] &\equiv& (b)C_2[C'[P'']] &\equiv& C_3[P''] \\
(b)(E[A]\para C_1[Q']) &\equiv& (b)C_2[E[A]\para Q'] &\equiv&  (b)C_2[C'[Q'']] &\equiv& C_3[Q'']
\end{array}
\]
Hence we obtain $C_3[P''] \,\R'\, C_3[Q'']$ with $P''\,\mathcal{R}\, Q''$. 

\item The action is from $P$, $\ve{c}$ is $\{b\}$ and $d$ is not $b$. That is, $P\st{(b)\overline{a}A} P'$ and $C[P]\st{(b)\overline{a}A} (d)C_1[P']$. Then since $P\,\R\, Q$, 
we know that $Q \wt{(b)\overline{a}A} Q'$, 
and for every $E$ as described in the definition, 
it holds that \\
$(b)(E[A]\para P') \equiv C'[P'']$, $(b)(E[A]\para Q') \equiv C'[Q'']$, and $P''\,\mathcal{R}\, Q''$. So $C[Q]$ simulates by $C[Q] \wt{(b)\overline{a}A} (d)C_1[Q']$. Now, we have  for every $E$, and some $C_2$ (which exists because $P$ is in a firable position) and $C_3\DEF (d)C_2[C'[]]$. 
\[
\begin{array}{lcl}
(b)(E[A]\para (d)C_1[P']) &\equiv&  (d)C_2[(b)(E[A]\para P')] \\
&\equiv& (d)C_2[C'[P'']] \\
&\equiv& C_3[P''] \\\\
(b)(E[A]\para (d)C_1[Q']) &\equiv& (d)C_2[(b)(E[A]\para Q')] \\
&\equiv&  (d)C_2[C'[Q'']] \\
&\equiv& C_3[Q'']
\end{array}
\]
Hence we obtain $C_3[P''] \,\R'\, C_3[Q'']$ with $P''\,\mathcal{R}\, Q''$. 

\item The action is from $P$ and $C_1$, $\ve{c}$ is $\{b\}$ and $d$ is not $b$. That is, $P\st{\overline{a}A} P'$ and $C[P]\st{(b)\overline{a}A} (d)C_1'[P']$. 
%This case is similar to the last one. 
Then since $P\,\R\, Q$, 
we know that $Q \wt{\overline{a}A} Q'$, 
and for every $E$ as described in the definition, 
it holds that $E[A]\para P' \equiv C'[P'']$, $E[A]\para Q' \equiv C'[Q'']$, and $P''\,\mathcal{R}\, Q''$. So $C[Q]$ simulates by $C[Q] \wt{(b)\overline{a}A} (d)C_1'[Q']$. Now, we have  for every $E$, and some $C_2$ (which exists because $P$ is in a firable position) and $C_3\DEF (bd)C_2[C'[]]$. 
\[
\begin{array}{lcl}
(b)(E[A]\para (d)C_1'[P']) &\equiv&  (bd)C_2[E[A]\para P'] \\
&\equiv& (bd)C_2[C'[P'']] \\
&\equiv& C_3[P''] \\\\
(b)(E[A]\para (d)C_1'[Q']) &\equiv& (bd)C_2[E[A]\para Q'] \\ 
&\equiv&  (bd)C_2[C'[Q'']] \\
&\equiv& C_3[Q'']
\end{array}
\]
Hence we obtain $C_3[P''] \,\R'\, C_3[Q'']$ with $P''\,\mathcal{R}\, Q''$. 

\item The action is from $P$, $\ve{c}$ is empty and $d$ is not $b$. That is, $P\st{\overline{a}A} P'$ and $C[P]\st{\overline{a}A} (d)C_1[P']$. Then since $P\,\R\, Q$, 
we know that $Q \wt{\overline{a}A} Q'$, 
and for every $E$ as stipulated in the definition, 
it holds that \\
$E[A]\para P' \equiv C'[P'']$, $E[A]\para Q' \equiv C'[Q'']$, and $P''\,\mathcal{R}\, Q''$. So $C[Q]$ simulates by $C[Q] \wt{\overline{a}A} (d)C_1[Q']$. Now, we have  for every $E$, and some $C_2$ (which exists because $P$ is in a firable position) and $C_3\DEF (d)C_2[C'[]]$. 
\[
\begin{array}{lclclcl}
E[A]\para (d)C_1[P'] &\equiv&  (d)C_2[E[A]\para P'] &\equiv& (d)C_2[C'[P'']] &\equiv& C_3[P''] \\
E[A]\para (d)C_1[Q'] &\equiv& (d)C_2[E[A]\para Q'] &\equiv&  (d)C_2[C'[Q'']] &\equiv& C_3[Q'']
\end{array}
\]
Hence we obtain $C_3[P''] \,\R'\, C_3[Q'']$ with $P''\,\mathcal{R}\, Q''$. 

\item The action is from $C_1$, $\ve{c}$ is empty and $d$ is $b$. That is, we have $C_1[P]\st{\overline{a}A} C_1'[P]$ and $C[P]\st{(b)\overline{a}A} C_1'[P]$. Then $C[Q]$ simulates by $C[Q] \st{(b)\overline{a}A} C_1'[Q]$. Now for every $E$ and $C_2\DEF (b)(E[A]\para C_1'[])$  we have
\[
\begin{array}{lcl}
(b)(E[A]\para C_1'[P]) &\equiv&  C_2[P] \\
(b)(E[A]\para C_1'[Q]) &\equiv&  C_2[Q]
\end{array}
\]
Hence we obtain $C_2[P] \,\R'\, C_2[Q]$ with $P\,\mathcal{R}\, Q$. 

\item The action is from $C_1$, $\ve{c}$ is $\{b\}$ and $d$ is not $b$. That is, we have $C_1[P]\st{(b)\overline{a}A} C_1'[P]$ and $C[P]\st{(b)\overline{a}A} (d)C_1'[P]$. Then $C[Q]$ simulates by $C[Q] \st{(b)\overline{a}A} (d)C_1'[Q]$. Now, we have  for every $E$, and $C_2\DEF (b)(E[A]\para (d)C_1'[])$.
\[
\begin{array}{lcl}
(b)(E[A]\para (d)C_1'[P]) &\equiv&  C_2[P] \\
(b)(E[A]\para (d)C_1'[Q]) &\equiv&  C_2[Q]
\end{array}
\]
Hence we obtain $C_2[P] \,\R'\, C_2[Q]$ with $P\,\mathcal{R}\, Q$. 

\item The action is from $C_1$, $\ve{c}$ is empty and $d$ is not $b$. That is, we have $C_1[P]\st{\overline{a}A} C_1'[P]$ and $C[P]\st{\overline{a}A} (d)C_1'[P]$. Then $C[Q]$ simulates by $C[Q] \st{\overline{a}A} (d)C_1'[Q]$. Now for every $E$ and $C_2\DEF E[A]\para (d)C_1'[]$, we have 
\[
\begin{array}{lcl}
E[A]\para (d)C_1'[P] &\equiv&  C_2[P] \\
E[A]\para (d)C_1'[Q] &\equiv&  C_2[Q]
\end{array}
\]
Hence we obtain $C_2[P] \,\R'\, C_2[Q]$ with $P\,\mathcal{R}\, Q$. 
\end{itemize}
\end{itemize}



% \item  $C[P]\st{\overline{a}A} T_1$ in which $A\equiv \lrangle{X}X\lrangle{b}$. \xx{Similar to the last case; MERGE and SKIP? seems yes!}
% \begin{itemize}
% \item $C$ is $[\cdot]$. 
% \item $C$ is $a(X).C_1[\cdot]$. 
% \item $C$ is $\overline{a}(C_1[\cdot]).R$. 
% \item $C$ is $\overline{a}A_1.C_1[\cdot]$. 
% \item $C$ is $\lrangle{X}C_1[\cdot]$. 
% \item $C$ is $\lrangle{x}C_1[\cdot]$. 
% \item $C$ is $C_1[\cdot]\lrangle{A_1}$.
% \item $C$ is $C_1[\cdot]\lrangle{d}$. 
% \item $C$ is $R\para C_1[\cdot]$. 
% \item $C$ is $(d)C_1[\cdot]$. 
% \end{itemize}



\item  $C[P]\st{\tau} T_1$. 
\begin{itemize}
\item The case when $C$ is $[\cdot]$ is immediate. The cases when $C$ is $a(X).C_1[\cdot]$, $\overline{a}(C_1[\cdot]).R$, $\overline{a}A_1.C_1[\cdot]$, $\lrangle{X}C_1[\cdot]$, $\lrangle{x}C_1[\cdot]$ are not possible. The cases when $C$ is $C_1[\cdot]\lrangle{A_1}$, $C_1[\cdot]\lrangle{d}$ fall onto the other cases after possible application.

% \item $C$ is $[\cdot]$. This is immediate. 
% \item $C$ is $a(X).C_1[\cdot]$. This is not possible. 
% \item $C$ is $\overline{a}(C_1[\cdot]).R$. This is not possible. 
% \item $C$ is $\overline{a}A_1.C_1[\cdot]$. This is not possible. 
% \item $C$ is $\lrangle{X}C_1[\cdot]$. This is not possible. 
% \item $C$ is $\lrangle{x}C_1[\cdot]$. This is not possible. 
% \item $C$ is $C_1[\cdot]\lrangle{A_1}$. This falls onto the other cases after application. 
% \item $C$ is $C_1[\cdot]\lrangle{d}$. This falls onto the other cases after application. 

\item $C$ is $R\para C_1[\cdot]$.  Several possibilities. For two-part communication, we only consider one subcase in which one part makes input and the other makes output, and omit the other similar (and actually simpler) symmetric subcase in which their roles are switched. 
\begin{itemize}
\item $\tau$ is from $R$. Suppose $R\st{\tau} R'$, and $C[P]\st{\tau} R'\para C_1[P]$. Then $C[Q]$ simulates by $C[Q]\st{\tau} R'\para C_1[Q]$. So we set $C'\DEF R'\para C_1[]$, and have $C'[P]\,\R'\, C'[Q]$ in which $P\,\R\, Q$.

\item $\tau$ is from $C_1$. Suppose $C_1[P] \st{\tau} C_1'[P]$, and $C[P]\st{\tau} R\para C_1'[P]$. Then $C[Q]$ simulates by $C[Q]\st{\tau} R\para C_1'[Q]$. So we set $C'\DEF R\para C_1'[]$, and have $C'[P] \,\R'\, C'[Q]$ in which $P\,\R\, Q$.

\item $\tau$ is from $P$. Suppose $P\st{\tau} P'$, and  $C[P] \st{\tau} R\para C_1[P']$. Since $P\,\R\, Q$, we know $Q\wt{} Q'$ and $P'\equiv C_2[P'']$ and $Q'\equiv C_2[Q'']$ for some $P'', Q''$ and context $C_2$ that is an encoding, and $P''\,\R\, Q''$. So $C[Q]$ simulates by $C[Q] \wt{} R\para C_1[Q']$. Now consider the following pair:
\[
\begin{array}{lcl}
R\para C_1[P'] &\equiv&  R\para C_1[C_2[P'']]  \\
R\para C_1[Q']  &\equiv&  R\para C_1[C_2[Q'']]
\end{array}
\] Now setting $C_3\DEF R\para C_1[C_2[]]$ to be the context closes this case as required, because $C_3[P''] \,\R'\, C_3[Q'']$ with $P''\,\R\, Q''$.

\item $\tau$ is from interaction between $R$ and $C_1$. 
Suppose $C_1[P] \st{(\ve{c})\overline{a}A} C_1'[P]$ for some $C_1$, in which $A$ is $\lrangle{Z}(Z\lrangle{b})$ and $\ve{c}$ is either empty or $\{b\}$, $R \st{a(A)} R'$, and $C[P] \st{\tau} (\ve{c})(R'\para C_1'[P])$. Then  $C[Q] \st{\tau} (\ve{c})(R'\para C_1'[Q])$. We close this case by setting the context to be $C_2\DEF (\ve{c})(R'\para C_1'[])$, and having $C_2[P] \,\R'\, C_2[Q]$ with $P\,\R\, Q$.

\item $\tau$ is from interaction between $R$ and $P$. Suppose $P \st{(\ve{c})\overline{a}A} P'$ in which $A$ is $\lrangle{Z}(Z\lrangle{b})$ and $\ve{c}$ is either empty or $\{b\}$, $R \st{a(A)} R'$, and $C[P] \st{\tau} (\ve{c})(R'\para C_1[P']) \equiv C_2[(\ve{c})(R'\para P')]$ for some $C_2$ since $P$ is in a firable position. Then since $P\R Q$, we know that $Q \wt{(\ve{c})\overline{a}A} Q'$, 
and for every $E$ as described in the definition, 
it holds that \\
$(\ve{c})(E[A]\para P') \equiv C'[P'']$, $(\ve{c})(E[A]\para Q')\equiv C'[Q'']$, and $P''\,\mathcal{R}\, Q''$. So $C[Q]$ simulates by $C[Q] \wt{\tau} (\ve{c})(R'\para C_1[Q']) \equiv C_2[(\ve{c})(R'\para Q')]$.
It meets the bisimulation requirement because one can take $E[A]$ to be $R'$ and obtain the following pair for some $C_3,P''',Q'''$ with $P'''\,\R\, Q'''$. 
\[
\begin{array}{lclcl}
(\ve{c})(R'\para C_1[P']) &\equiv& C_2[(\ve{c})(R'\para P')] &\equiv& C_2[C_3[P''']] \\
 (\ve{c})(R'\para C_1[Q']) &\equiv& C_2[(\ve{c})(R'\para Q')] &\equiv& C_2[C_3[Q''']
\end{array}
\] Now we set $C_4\DEF C_2[C_3[]]$ to be the context, and have as required $C_4[P'''] \,\R'\, C_4[Q''']$ with $P'''\,\R\, Q'''$.

\item $\tau$ is from interaction between $R$ and $P$ together with $C_1$. Suppose $P \st{\overline{a}A} P'$ in which $A$ is $\lrangle{Z}(Z\lrangle{b})$ and $\ve{c}$ is $\{b\}$, $R \st{a(A)} R'$, and $C[P] \st{\tau} (b)(R'\para C_1'[P']) \equiv (b)C_2[R'\para P']$ for some $C_2$ since $P$ is in a firable position. 
%This case is similar to the last one. 
Then since $P\R Q$, we know that $Q \wt{\overline{a}A} Q'$, 
and for every $E$ as described in the definition, 
it holds that $E[A]\para P' \equiv C'[P'']$, $E[A]\para Q'\equiv C'[Q'']$, and $P''\,\mathcal{R}\, Q''$. So $C[Q]$ simulates by $C[Q] \wt{\tau} (b)(R'\para C_1'[Q']) \equiv (b)C_2[R'\para Q']$.
It meets the bisimulation requirement because one can take $E[A]$ to be $R'$ and obtain the following pair for some $C_3,P''',Q'''$ with $P'''\,\R\, Q'''$. 
\[
\begin{array}{lclcl}
(b)(R'\para C_1'[P']) &\equiv& (b)C_2[R'\para P'] &\equiv& (b)C_2[C_3[P''']] \\
 (b)(R'\para C_1'[Q']) &\equiv& (b)C_2[R'\para Q'] &\equiv& (b)C_2[C_3[Q''']
\end{array}
\] Now we set $C_4\DEF (b)C_2[C_3[]]$ to be the context, and have as required $C_4[P'''] \,\R'\, C_4[Q''']$ with $P'''\,\R\, Q'''$.

\item $\tau$ is from interaction between $P$ and $C_1$.    This case is similar to the last one. Notice that $C_1$ here has a role similar to $C$ in the last case. 
%\xx{ok to skip? seems yes but ... will do still; done}
Suppose $P \st{(\ve{c})\overline{a}A} P'$ in which $A$ is $\lrangle{Z}(Z\lrangle{b})$ and $\ve{c}$ is either empty or $\{b\}$, $C_1[P] \st{a(A)} C_1'[P]$, and $C[P] \st{\tau} R\para (\ve{c})(C_1'[P']) \equiv R\para C_2[(\ve{c})(T\para P')]$ for some $C_2$ and $T$ (which possibly has $A$ occurrence) since $P$ is in a firable position. Then since $P\R Q$, we know that $Q \wt{(\ve{c})\overline{a}A} Q'$, 
and for every $E$ as described in the definition, 
it holds that $(\ve{c})(E[A]\para P') \equiv C'[P'']$, $(\ve{c})(E[A]\para Q')\equiv C'[Q'']$, and $P''\,\mathcal{R}\, Q''$. So $C[Q]$ simulates by \\
$C[Q] \wt{\tau} R\para (\ve{c})(C_1'[Q']) \equiv R\para C_2[(\ve{c})(T\para Q')]$.
It meets the bisimulation requirement because one can take $E[A]$ to be $T$ and obtain the following pair for some $C_3,P''',Q'''$ with $P'''\,\R\, Q'''$. 
\[
\begin{array}{lclcl}
R\para (\ve{c})(C_1'[P']) &\equiv& R\para C_2[(\ve{c})(T\para P')] &\equiv& R\para C_2[C_3[P''']] \\
R\para (\ve{c})(C_1'[Q']) &\equiv& R\para C_2[(\ve{c})(T\para Q')] &\equiv& R\para C_2[C_3[Q''']
\end{array}
\] Now we set $C_4\DEF R\para C_2[C_3[]]$ to be the context, and have as required $C_4[P'''] \,\R'\, C_4[Q''']$ with $P'''\,\R\, Q'''$.
\end{itemize}



\item $C$ is $(d)C_1[\cdot]$. The action $C[P]\equiv (d)C_1[P] \st{\tau} T_1$ must come from \\
$C_1[P] \st{\tau} T_2$ such that $(d)T_2$. There are several subcases.
\begin{itemize}
\item  $\tau$ is from $C_1$. Suppose $C_1[P] \st{\tau} C_1'[P]$, and $C[P]\st{\tau} (d)C_1'[P]$. Then $C[Q]$ simulates by $C[Q]\st{\tau} (d)C_1'[Q]$. So we set $C'\DEF (d)C_1'[]$, and have $C'[P] \,\R'\, C'[Q]$ in which $P\,\R\, Q$.

\item $\tau$ is from $P$. Suppose $P\st{\tau} P'$, and  $C[P] \st{\tau} (d)C_1[P']$. Since $P\,\R\, Q$, we know $Q\wt{} Q'$ and $P'\equiv C_2[P'']$ and $Q'\equiv C_2[Q'']$ for some $P'', Q''$ and context $C_2$ that is an encoding, and $P''\,\R\, Q''$. So $C[Q]$ simulates by $C[Q] \wt{} (d)C_1[Q']$. Now consider the following pair:
\[
\begin{array}{lcl}
(d)C_1[P'] &\equiv&  (d)C_1[C_2[P'']]  \\
(d)C_1[Q']  &\equiv&  (d)C_1[C_2[Q'']]
\end{array}
\] Now setting $C_3\DEF (d)C_1[C_2[]]$ to be the context closes this case, because $C_3[P''] \,\R'\, C_3[Q'']$ with $P''\,\R\, Q''$.

\item $\tau$ is from interaction between $P$ and$C_1$. Suppose $P \st{(\ve{c})\overline{a}A} P'$ in which $A$ is $\lrangle{Z}(Z\lrangle{b})$ and $\ve{c}$ is either empty or $\{b\}$, $C_1[P] \st{a(A)} C_1'[P]$, and $C[P] \st{\tau} (d)((\ve{c})(C_1'[P'])) \equiv (d)C_2[(\ve{c})(T\para P')]$ for some $C_2$ and $T$ (which possibly has $A$ occurrence) since $P$ is in a firable position. Then since $P\R Q$, we know that $Q \wt{(\ve{c})\overline{a}A} Q'$, 
and for every $E$ as described in the definition, 
it holds that $(\ve{c})(E[A]\para P') \equiv C'[P'']$, $(\ve{c})(E[A]\para Q')\equiv C'[Q'']$, and $P''\,\mathcal{R}\, Q''$. So $C[Q]$ simulates by $C[Q] \wt{\tau} (d)((\ve{c})(C_1'[Q'])) \equiv (d)C_2[(\ve{c})(T\para Q')]$.
It meets the bisimulation requirement because one can take $E[A]$ to be $T$ and obtain the following pair for some $C_3,P''',Q'''$ with $P'''\,\R\, Q'''$. 
\[
\begin{array}{lclcl}
(d)((\ve{c})(C_1'[P'])) &\equiv& (d)C_2[(\ve{c})(T\para P')] &\equiv& (d)C_2[C_3[P''']] \\
(d)((\ve{c})(C_1'[Q'])) &\equiv& (d)C_2[(\ve{c})(T\para Q')] &\equiv& (d)C_2[C_3[Q''']
\end{array}
\] Now we set $C_4\DEF (d)C_2[C_3[]]$ to be the context as required, and have $C_4[P'''] \,\R'\, C_4[Q''']$ with $P'''\,\R\, Q'''$.
\end{itemize}

\end{itemize}

\end{enumerate}
%-------------PROOF BODY END-------------
% \begin{itemize}
% \item 
% \item 
% \item 
% \item 
% \item 
% \item 
% \item 
% \end{itemize}
\qed
\end{proof}

% \noindent\textit{Remark} ~ 
% \xx{We remark on the proof above.}
% \begin{itemize}
% \item 
% \xx{\tiny Perhaps we can also use the way of proving congruence \cite{Fu07,MH02,San92,SW01a}. We denote by $\mathbb{N}$ the set of natural numbers. Define the relations $\R_n$ and $\R'$ as follows.
% % \[
% % \begin{array}{lcl}
% % \R_0 &\DEF& \R \\
% % \R_{n+1} &\DEF& \left\{(C[M],C[N]) \,|\, M\,\R_{n}\, N \mbox{ and } C \mbox{ is $R \para [\cdot]$, or $(d)[\cdot]$} \right\} \\ %$\vartheta.[\cdot]$,
% % \R' &\DEF& \bigcup_{i\in \mathbb{N}} \R_i
% % \end{array}
% % \]
% \[
% \begin{array}{lcl}
% \R_0 &\DEF& \R \\
% \R_{n+1} &\DEF& \left\{(C[M],C[N]) \,|\, M\,\R_{n}\, N \mbox{ and } C C\in \mathcal{D} \right\} \\ %$\vartheta.[\cdot]$,
% \R' &\DEF& \bigcup_{i\in \mathbb{N}} \R_i\\\\
% \mathcal{D} &\DEF& \left\{
% \begin{array}{lllll}
% %[\cdot],  & 
% a(X).[\cdot],\quad & \overline{a}([\cdot]).R,\quad & \overline{a}A_1.[\cdot],\quad & \lrangle{X}[\cdot], \quad & \lrangle{x}[\cdot],\\~
% [\cdot]\lrangle{A_1}, & [\cdot]\lrangle{d}, & R\para [\cdot], & (d)[\cdot] &
% \end{array}
% \right\}
% \end{array}
% \]
% In somewhat a similar way, we can prove by induction on $n$ that $\R'$ is a bisimulation up-to $\equiv$. In this approach, the procedure of induction is like the proof above, and takes the usual pattern \cite{Fu07,MH02,San92,SW01a}. This would actually even complicates the arguments, so rather the current proof we provide as above offers technically a somewhat simplified way in a particular seeting of encodings.
% }

% %\item The proof can be easily adapted to the proof of the correctness of the up-to context technique of general higher-order processes, for context bisimulation and even normal bisimulation. 

% % \item The main thrust of the proof is somewhat akin to proving the congruence of a bisimilarity in a higher-order setting (actually it can be somehow used to prove congruence properties) \cite{SW01a, DHS18}; (maybe also Howe's method, but essentially the same). So this indeed ensures that the context bisimulation on top of encodings (as well as on general higher-order processes) is a congruence. 

% \item The framework of the proof is akin to proving the congruence of a bisimilarity in a higher-order setting \cite{SW01a,DHS18,Fu07,MH02,San92}. This indeed ensures that the context bisimilarity on top of the encodings is a congruence. 

% \end{itemize}














%---------------------------
% Local Variables:
% mode: LaTeX
% TeX-master: "main.tex"
% End:



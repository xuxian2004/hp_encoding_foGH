\section{Encoding \FOPi\ into \HOPiDd}\label{s:encoding}
We show that \FOPi\ can be encoded in \HOPiDd, by presenting an encoding that satisfies all the criteria in Definition \ref{gorla-like-cond} except adequacy. % (1a).

%\bc{\scriptsize We first recall the notations:
%\[
%\begin{array}{ll}
%\lrangle{X}P \,,\, A\lrangle{Q} \quad & \mbox{abstraction and application on processes} \\
%\lrangle{x}P \,,\, A\lrangle{b} \quad & \mbox{abstraction and application on names}
%\end{array}
%\]
%}

\subsection{The encoding}\label{ss:encoding_def}
%In light of related work in the field \cite{San92, Tho93, LPSS10, XYL14},
%It is still not clear whether the encoding can be adapted (or rewritten) into one that only uses name parameterization.

%\FROMHERE


The encoding has its core defined as follows, and is homomorphic on the other operators. %, and we note that the encoding of input guarded replication is defined as $\enc{!m(x).P}\DEF !\enc{m(x).P}$.
\[
\begin{array} {lrcll}
 & \enc{m(x).P} & \DEF & m(Y).Y\lrangle{\lrangle{x}\enc{P}} & \\ %\quad \quad (Y \mbox{ is fresh}) \\
 & \enc{\overline{m}n.Q} &\DEF & \overline{m}[\lrangle{Z}(Z\lrangle{n})].\enc{Q} &  \\ %\quad \quad (Z \mbox{ is fresh}) %\\
%(\mbox{\rc{asynchronous}}) & \enc{\overline{m}n} &\DEF & \overline{m}[\lrangle{Z}(Z\lrangle{n})] & \quad \quad (Z \mbox{ is fresh})
\end{array}
\]
The encoding appears laconic, and uses both name parameterization and process parameterization. %, and seems correct. 
%We note that for input guarded replication, the encoding is. 
Typically one can assume that $Y$ and $Z$ are fresh for simplicity, but this is not essential, because these variables are bound and can be $\alpha$-converted whenever necessary, and moreover the encoded first-order process does not have higher-order variables.
Specifically, the encoding of an output `transmits' the name to be sent (i.e., $n$) in terms of a process parameterization (i.e., $\lrangle{Z}(Z\lrangle{n})$, sometimes called an encapsulation of $n$ in effect) that, once being received by the encoding of an input, is instantiated by a name-parameterized term (i.e., $\lrangle{x}\enc{P}$), which then can apply $n$ on $x$ in the encoding of $P$, thus fulfilling `name-passing'. Below we give an example. 
\begin{example}\label{eg:fo2ho1}
Suppose $P\DEF (c)(a(x).\overline{x}c.P_1)$ and $Q\DEF (d)(\overline{a}d.d(y).Q_1)$. So
\[
\begin{array}{lcl}
P\para Q &\st{\tau}& (d)((c)(\overline{d}c.P_1\fosub{d}{x})\para d(y).Q_1) \\
&\st{\tau}& (dc)(P_1\fosub{d}{x}\para Q_1\fosub{c}{y})
\end{array}
\] The encoding and interactions of $\enc{P\para Q}$ are as below. 
%For clarity, we use \textbf{bold font} to indicate the evolving part during a communication. 
The last equivalence is due to the name invariance property to be given shortly in Lemma \ref{l:syn-pro-encoding}.
% \[
% \begin{array}{lcl}
% \enc{P\para Q} &\equiv& (c)(a(Y).Y\lrangle{\lrangle{x}\enc{\overline{x}c.P_1}}) \,\para\, (d)(\overline{a}[\lrangle{Z}(Z\lrangle{d})].\enc{d(y).Q_1}) \\
%  &\st{\tau}& (d)\big((c)(\bm{(\lrangle{Z}(Z\lrangle{d}))\lrangle{\lrangle{x}\enc{\overline{x}c.P_1}}}) \,\para\, \enc{d(y).Q_1} \big) \\
%  &\equiv& (d)\big((c)( \bm{\enc{\overline{x}c.P_1}\fosub{d}{x}}) \,\para\, \enc{d(y).Q_1} \big) \\
%  &\equiv& (d)\big((c)( \bm{ (\overline{x}[\lrangle{Z}(Z\lrangle{c})].\enc{P_1}) \fosub{d}{x}}) \,\para\, d(Y).Y\lrangle{\lrangle{y}\enc{Q_1}}  \big) \\
%  &\equiv& (d)\big((c)( \bm{ (\overline{d}[\lrangle{Z}(Z\lrangle{c})].\enc{P_1}\fosub{d}{x}) }) \,\para\, d(Y).Y\lrangle{\lrangle{y}\enc{Q_1}}  \big) \\
%  &\st{\tau}&  (dc)\big(\enc{P_1}\fosub{d}{x} \,\para\, \bm{(\lrangle{Z}(Z\lrangle{c}))(\lrangle{\lrangle{y}\enc{Q_1}})} \big) \\
%  &\equiv&  (dc)\big(\enc{P_1}\fosub{d}{x} \,\para\, \enc{Q_1}\fosub{c}{y} \big) \\
%  &\equiv&  (dc)\big(\enc{P_1\fosub{d}{x}} \,\para\, \enc{Q_1\fosub{c}{y}} \big)
% \end{array}
% \]
\[
\begin{array}{lcl}
\enc{P\para Q} &\equiv& (c)(a(Y).Y\lrangle{\lrangle{x}\enc{\overline{x}c.P_1}}) \,\para\, (d)(\overline{a}[\lrangle{Z}(Z\lrangle{d})].\enc{d(y).Q_1}) \\
 &\st{\tau}& (d)\big((c)({(\lrangle{Z}(Z\lrangle{d}))\lrangle{\lrangle{x}\enc{\overline{x}c.P_1}}}) \,\para\, \enc{d(y).Q_1} \big) \\
 &\equiv& (d)\big((c)( {\enc{\overline{x}c.P_1}\fosub{d}{x}}) \,\para\, \enc{d(y).Q_1} \big) \\
 &\equiv& (d)\big((c)( { (\overline{x}[\lrangle{Z}(Z\lrangle{c})].\enc{P_1}) \fosub{d}{x}}) \,\para\, d(Y).Y\lrangle{\lrangle{y}\enc{Q_1}}  \big) \\
 &\equiv& (d)\big((c)( { (\overline{d}[\lrangle{Z}(Z\lrangle{c})].\enc{P_1}\fosub{d}{x}) }) \,\para\, d(Y).Y\lrangle{\lrangle{y}\enc{Q_1}}  \big) \\
 &\st{\tau}&  (dc)\big(\enc{P_1}\fosub{d}{x} \,\para\, {(\lrangle{Z}(Z\lrangle{c}))(\lrangle{\lrangle{y}\enc{Q_1}})} \big) \\
 &\equiv&  (dc)\big(\enc{P_1}\fosub{d}{x} \,\para\, \enc{Q_1}\fosub{c}{y} \big) \\
 &\equiv&  (dc)\big(\enc{P_1\fosub{d}{x}} \,\para\, \enc{Q_1\fosub{c}{y}} \big)
\end{array}
\]
\end{example}


\tdup{
\bc{TODO (across \HOPid\ and \FOPi):}
\begin{itemize}
\item \bc{Operational correspondence}. \bc{$\checkmark$}

\item \bc{Soundness/weak adequacy}.  \bc{$\checkmark$}
(\Ground bisimilarity: $\WGB$; Strong \ground bisimilarity: $\SGB$; Context bisimilarity: $\WCB$; Strong context bisimilarity: $\SCB$)
\[
P\WGB Q \mbox{~~ implies ~~} \enc{P} \WCB \enc{Q}
\]
\stress{(Possibly use the result in Section \ref{s:normal} on normal bisimulation.)}
\end{itemize}
}

\subsection{Static properties and operational correspondence}\label{s:encoding_operational}
%Apparently 
The encoding is compositional and divergence-reflecting (since the encoding does not introduce any extra internal action), preserves the (free) names and the structural congruence, as stated in the follow-up lemma whose proof is standard induction. %We stress that the side conditions in the encoding is not essential.
\begin{lemma}\label{l:syn-pro-encoding}
Assume $P,Q$ are \FOPi\ processes. The encoding above from \FOPi\ to \HOPiDd\ is compositional; % and divergence-reflecting; 
moreover $\enc{P}\fosub{n}{m} \equiv \enc{P\fosub{n}{m}}$ (called name invariance), and $\enc{P}\equiv \enc{Q}$ iff $P\equiv Q$.
\end{lemma}
%\sep
%moved from appendix in the previous version**********************************
%\section{Proofs for Section \ref{s:encoding}}\label{a:proofs-encoding}
%***************************************************
%NOW in the body, used by "encoding.tex"



\begin{proof}[Proof of Lemma \ref{l:syn-pro-encoding}]
It is straightforward to check that the encoding is compositional since the designated contexts are easy to capture. For the core of the encoding, the contexts for input and output are respectively $m(Y).Y\lrangle{\lrangle{x}[\cdot]}$ and $\overline{m}[\lrangle{Z}(Z\lrangle{n})].[\cdot]$. 
%Also the encoding is divergence-reflecting, for the reason that it does not bring about any divergence. 
As such, it is a simple induction, on the rules deriving $\equiv$, to show that the encoding preserves structural congruence. Below we prove by induction on the structure of $P$ that $\enc{P}\sigma \equiv \enc{P\sigma}$ in which $\sigma$ is a substitution (recall that $\sigma$ is a mapping on names).
%\oo{\large \fbox{\#\#\#\# TODO}}
\begin{itemize}
\item $P$ is $0$. This is trivial.
\item $P$ is $m(x).Q$. Then %m(Y).Y\lrangle{\lrangle{x}\enc{Q}}
\[
\begin{array}{lcll}
\enc{P}\sigma &\equiv& (m(Y).Y\lrangle{\lrangle{x}\enc{Q}})\sigma & \quad \\
 &\equiv& m'(Y).Y\lrangle{\lrangle{x}(\enc{Q}\sigma)} & \quad m' \mbox{ is } \sigma(m) \\
 &\equiv& m'(Y).Y\lrangle{\lrangle{x}(\enc{Q\sigma})} & \quad \mbox{ind. hyp. (short for induction hypothesis)}\\
 &\equiv& \enc{m'(x).Q\sigma} & \quad \\
 &\equiv& \enc{(m(x).Q)\sigma} & \quad \\ 
 &\equiv& \enc{P\sigma} & \quad
\end{array}
\]
\item $P$ is $\overline{m}n.Q$. Then %\overline{m}[\lrangle{Z}(Z\lrangle{n})].\enc{Q}
\[
\begin{array}{lcll}
\enc{P}\sigma &\equiv& (\overline{m}[\lrangle{Z}(Z\lrangle{n})].\enc{Q})\sigma & \quad \\
 &\equiv& \overline{m'}[\lrangle{Z}(Z\lrangle{n'})].(\enc{Q}\sigma) & \quad m',n' \mbox{ are respectively } \sigma(m),\sigma(n) \\
 &\equiv& \overline{m'}[\lrangle{Z}(Z\lrangle{n'})].(\enc{Q\sigma}) & \quad \mbox{ind. hyp.} \\
 &\equiv& \enc{\overline{m'}n'.(Q\sigma)} & \quad \\
 &\equiv& \enc{(\overline{m}n.Q)\sigma} & \quad \\
 &\equiv& \enc{P\sigma} & \quad
\end{array}
\]
\item $P$ is $(c)Q$. Then 
\[
\begin{array}{lcll}
\enc{P}\sigma &\equiv& ((c)\enc{Q})\sigma & \quad  \\
 &\equiv& (c)\enc{Q}\sigma & \quad  \\
 &\equiv& (c)\enc{Q\sigma} & \quad \mbox{ind. hyp.} \\
 &\equiv& \enc{(c)(Q\sigma)} & \quad  \\
 &\equiv& \enc{((c)Q)\sigma} & \quad  \\
 &\equiv& \enc{P\sigma} & \quad  
\end{array}
\]
\item $P$ is $Q\para R$. Then 
\[
\begin{array}{lcll}
\enc{P}\sigma &\equiv& (\enc{Q}\para \enc{R})\sigma  & \quad \\
 &\equiv& \enc{Q}\sigma\para \enc{R}\sigma  & \quad \\
 &\equiv& \enc{Q\sigma}\para \enc{R\sigma}  & \quad \mbox{ind. hyp.} \\ 
 &\equiv& \enc{Q\sigma \para R\sigma}  & \quad  \\ 
 &\equiv& \enc{(Q\para R)\sigma}  & \quad  \\ 
 &\equiv& \enc{P\sigma}  & \quad 
\end{array}
\]
\item $P$ is $!m(x).Q$. Then %!\enc{m(x).Q}
\[
\begin{array}{lcll}
\enc{P}\sigma &\equiv& (!\enc{m(x).Q})\sigma & \quad \\
 &\equiv& !(\enc{m(x).Q})\sigma & \quad \\
 &\equiv& !(\enc{(m(x).Q)\sigma}) & \quad \mbox{similar to the input case} \\
 &\equiv& \enc{!((m(x).Q)\sigma)} & \quad \\
 &\equiv& \enc{(!m(x).Q)\sigma} & \quad \\
 &\equiv& \enc{P\sigma} & \quad 
\end{array}
\]
\end{itemize}
\myqed
\end{proof}










%---------------------------
% Local Variables:
% mode: LaTeX
% TeX-master: "main.tex"
% End:



We now give the correspondence of actions before and after the encoding.
To delineate some case of the operational correspondence in terms of certain special input, i.e., a trigger, we define $\triggerD \DEF \lrangle{Z}\overline{m}Z$ in which $m$ is assumed to be fresh (it will also be used in Section \ref{s:normal}, but here simply allows for more flexible characterization of the operational correspondence). We note that sometimes existential quantification is omitted when it is clear from context.
%Literally in the meanwhile, 
The operational correspondence actually says that the encoding of a process always evolves into the encoding of another, thus forming a somewhat closed image domain.


\xxx{
\xxxx{Lemma \ref{l:opcor} gives the forward operational correspondence while Lemma \ref{l:opcor-conv} gives the backward operational correspondence.} 
In these lemmas, 
the input cases are not stated for any process inputted on $\enc{P}$, 
because the action correspondence would become unclear if arbitrary input other than those used in the lemmas is permitted.
This is attributed to the richer environment of \HOPiDd, and actually is one reason why we argue for relinquishing uniform requirement on operational semantics in the notion of encoding (see Section \ref{s:criteria}). 
We note that, as mentioned above, clause (2) of the two lemmas is based on an input that does not come from an encoding; it is provided for the sake of enabling observation of input from a different angle.
These two lemmas can be proven in a similar fashion and we only give the proof of the forward operation correspondence. %(we give details in Appendix \ref{a:proofs-encoding} \nts{\fbox{MOVE the proof here?!}}),
%\iftoggle{appendixing}{%
%  %using appendixing
% (we give details in Appendix \ref{a:proofs-encoding}),
%}{%
%  %no appendixing
% (details can be found in \cite{Xu16app}),
%}
Moreover, they can be lifted to the weak situation. That is, if one replaces strong transitions (single arrows) with weak transitions (double arrows), % and $\SCB$ with $\WCB$,
the results still hold ($\SSEQV$ retains because the encoding does not bring any extra internal action); see \cite{San92,SW01a} for a reference.
We will however simply refer to these two lemmas in related discussions.
%The rest of this subsection is dedicated to the proofs of Lemma \ref{l:syn-pro-encoding} and Lemma \ref{l:opcor}.  As said above, the proof of Lemma \ref{l:opcor-conv} is similar to that of Lemma \ref{l:opcor} and thus skipped.
}


\begin{lemma}\label{l:opcor}
Suppose $P$ is a \FOPi\ process.
\begin{enumerate}
\item[(1)] If $P \st{a(b)} P'$, then $\enc{P} \st{a(\lrangle{Z}(Z\lrangle{b}))} T$ and $T\SSEQV \enc{P'}$; ~ %\xx{$\SSEQV$}
\tdup{
\item (\rc{useful?seems not! to remove!}) If $P \st{a(b)} P'$, then for some fresh $m$, $\enc{P} \st{a(\lrangle{Z}\overline{m}[Z\lrangle{b}])} T$ and $(m)(T \para !m(Y).Y) \WCB \enc{P'}$;
}
%\rc{Use equation (\ref{eqn-fact-hopiDd}) to deal with $Z\lrangle{b}$ in $\lrangle{Z}\overline{m}[Z\lrangle{b}]$ ??}
\item[(2)] If $P \st{a(b)} P'$, then $\enc{P} \st{a(\triggerD)} T$ and $(m)(T \para !m(Y).Y\lrangle{b}) \WCB \enc{P'}$; ~
%\stress{ (relating $T$ and $\enc{P'}$) };\\
\item[(3)] If $P \st{\overline{a}b} P'$, then $\enc{P} \st{\overline{a}[\lrangle{Z}(Z\lrangle{b})]} T$ and $T\SSEQV \enc{P'}$; ~
\item[(4)] If $P \st{\overline{a}(b)} P'$, then $\enc{P} \st{(b)\overline{a}[\lrangle{Z}(Z\lrangle{b})]} T$ and $T\SSEQV \enc{P'}$; ~
\item[(5)] If $P \st{\tau} P'$, then $\enc{P} \st{\tau} T$ and $T\SSEQV \enc{P'}$.
\end{enumerate}
\end{lemma}

The converse is as below.
\begin{lemma}\label{l:opcor-conv}
Suppose $P$ is a \FOPi\ process.
\begin{enumerate}
\item[(1)] If $\enc{P} \st{a(\lrangle{Z}(Z\lrangle{b}))} T$, then $P \st{a(b)} P'$ and $T\SSEQV \enc{P'}$; ~
\tdup{
\item (\rc{useful? seems not! to remove!}) If for some fresh $m$, $\enc{P} \st{a(\lrangle{Z}\overline{m}[Z\lrangle{b}])} T$, then $P \st{a(b)} P'$ and $(m)(T \para !m(Y).Y) \WCB \enc{P'}$;
}
\item[(2)] If $\enc{P} \st{a(\triggerD)} T$, then $P \st{a(b)} P'$ and $(m)(T \para !m(Y).Y\lrangle{b}) \WCB \enc{P'}$; ~
%\stress{ (relating $T$ and $\enc{P'}$) };\\
\item[(3)] If $\enc{P} \st{\overline{a}[\lrangle{Z}(Z\lrangle{b})]} T$, then $P \st{\overline{a}b} P'$ and $T\SSEQV \enc{P'}$; ~
\item[(4)] If $\enc{P} \st{(b)\overline{a}[\lrangle{Z}(Z\lrangle{b})]} T$, then $P \st{\overline{a}(b)} P'$ and $T\SSEQV \enc{P'}$; ~
\item[(5)] If $\enc{P} \st{\tau} T$, then $P \st{\tau} P'$ and $T\SSEQV \enc{P'}$.
\end{enumerate}
\end{lemma}


%moved from appendix in the previous version**********************************
\input{appendix_proof_encoding_opcor.tex}
%

\xxxx{
An immediate corollary of the operational correspondence is that the encoding is divergence-reflecting, for the reason that it does not bring about any divergence (Lemma \ref{l:opcor} and Lemma \ref{l:opcor-conv}). 
\begin{corollary}\label{cor:syn-encoding-div-refle}
%Assume $P,Q$ are \FOPi\ processes. 
The encoding from \FOPi\ to \HOPiDd\ is divergence-reflecting.
\end{corollary}
}




\subsection{Soundness}\label{s:encoding_soundness}
In this section, we discuss the soundness of the encoding.
First of all, it is unfortunate that 
%the soundness of the encoding is not true. 
\xxxx{the encoding is not sound in general.}
To see this, take the processes $R_1$ and $R_2$ below. We recall that the CCS-like prefixes are defined as usual, i.e., $a.P\DEF a(x).P$ ($x\notin \n{P}$), $\overline{a}.P\DEF (c)\overline{a}c.P$ ($c\notin \n{P}$); sometimes we trim the trailing $0$, e.g., $a$ stands for $a.0$ and $\overline{a}$ for $\overline{a}.0$.
\[
\begin{array}{lcllcl}
R_1 &\DEF& (b)(a.\overline{b} \para b.\overline{c}) &\qquad\quad R_2 &\DEF& (b)(a.\overline{b} \para b.\overline{c} \para b.\overline{c})
\end{array}
\]
Obviously, $R_1$ and $R_2$ are \ground bisimilar.
%(The synchronizations are defined as usual. )
Now we examine their encodings. %m(Y).Y\lrangle{\lrangle{x}\enc{P}}   ;      \overline{m}[\lrangle{Z}(Z\lrangle{n})].\enc{Q}
\[
\begin{array}{lcl}
\enc{R_1} &\equiv& (b)(a(Y).Y\lrangle{\lrangle{x}\enc{\overline{b}}} \para b(Y).Y\lrangle{\lrangle{x}\enc{\overline{c}}}) \\
\enc{R_2} &\equiv& (b)(a(Y).Y\lrangle{\lrangle{x}\enc{\overline{b}}} \para b(Y).Y\lrangle{\lrangle{x}\enc{\overline{c}}} \para b(Y).Y\lrangle{\lrangle{x}\enc{\overline{c}}})
\end{array}
\]

We show that $\enc{R_1}$ and $\enc{R_2}$ are not context bisimilar. Define 
\[T\DEF (m)(\overline{a}[\lrangle{Z}\overline{m}Z] \para m(X).(X\lrangle{d} \para X\lrangle{d})
\]
Then $(a)(\enc{R_1}\para T)$ and $(a)(\enc{R_2}\para T)$ can be distinguished. The latter can fire two outputs on $c$, whereas the former cannot, as shown below.
\[
\begin{array}{lrl}
 &(a)(\enc{R_1}\para T) \quad \st{\tau}\SCB& (m)((b)(\overline{m}[\lrangle{x}\enc{\overline{b}}] \para b(Y).Y\lrangle{\lrangle{x}\enc{\overline{c}}}) \\
 & & \qquad\qquad\qquad\quad\,\,\,\, \para m(X).(X\lrangle{d} \para X\lrangle{d})) \\
&\st{\tau}\SCB& (b)(b(Y).Y\lrangle{\lrangle{x}\enc{\overline{c}}} \para \enc{\overline{b}} \para \enc{\overline{b}}) \\
&\equiv& (b)(b(Y).Y\lrangle{\lrangle{x}\enc{\overline{c}}} \para (e)\overline{b}[\lrangle{Z}(Z\lrangle{e})] \para \enc{\overline{b}}) \\
&\st{\tau}\SCB& (b)(\enc{\overline{c}} \para \enc{\overline{b}}) \\
&\equiv& (b)((f)\overline{c}[\lrangle{Z}(Z\lrangle{f})] \para \enc{\overline{b}}) \\
&\st{(f)\overline{c}[\lrangle{Z}(Z\lrangle{f})]}\SCB& 0 %\\\\
\end{array}
\]
\[
\begin{array}{lrl}
 &(a)(\enc{R_2}\para T) \quad \st{\tau}\SCB& (m)((b)(\overline{m}[\lrangle{x}\enc{\overline{b}}] \para b(Y).Y\lrangle{\lrangle{x}\enc{\overline{c}}} \para b(Y).Y\lrangle{\lrangle{x}\enc{\overline{c}}}) \\
 & & \qquad\qquad\qquad\quad\,\,\,\, \para m(X).(X\lrangle{d} \para X\lrangle{d})) \\
&\st{\tau}\SCB& (b)(b(Y).Y\lrangle{\lrangle{x}\enc{\overline{c}}}\para b(Y).Y\lrangle{\lrangle{x}\enc{\overline{c}}} \para \enc{\overline{b}} \para \enc{\overline{b}}) \\
&\equiv& (b)(b(Y).Y\lrangle{\lrangle{x}\enc{\overline{c}}}\para b(Y).Y\lrangle{\lrangle{x}\enc{\overline{c}}} \para (e)\overline{b}[\lrangle{Z}(Z\lrangle{e})] \\
& & \qquad\qquad\qquad\quad\,\,\,\,\,\, \para (e)\overline{b}[\lrangle{Z}(Z\lrangle{e})]) \\
&\st{\tau}\st{\tau}\SCB& \enc{\overline{c}} \para \enc{\overline{c}} \\
&\equiv& (f)\overline{c}[\lrangle{Z}(Z\lrangle{f})] \para (f)\overline{c}[\lrangle{Z}(Z\lrangle{f})] \\
&\st{(f)\overline{c}[\lrangle{Z}(Z\lrangle{f})]}\SCB& (f)\overline{c}[\lrangle{Z}(Z\lrangle{f})] \\
&\st{(f)\overline{c}[\lrangle{Z}(Z\lrangle{f})]}\SCB& 0
\end{array}
\]


Intuitively, the reason general soundness does not hold is that context \xxxx{bisimilarity} is somewhat more discriminating in the target higher-order calculus, which can have more flexibility when dealing with blocks of processes in presence of parameterization (e.g., some subprocess of interest can be sent as a whole as needed). This is however beyond the capability of a first-order process.
\xxxx{
In particular, the example above fails the general soundness because it is possible to capture the process corresponding to $\overline{b}$ then to duplicate it, by harnessing the capability of parameterization. This is similar to some behaviour that can be seen in presence of passivation (see \cite{LSS11}, Section 2.3).	
}


In spite of the falsity of soundness in general, we can have a somewhat weaker yet still sensible soundness. Remember that our main goal is to achieve first-order concurrency in the higher-order model, so maybe we do not need to be so demanding when coping with the encodings of  first-order processes, that is, when testing an encoding process with an input, one can focus on those representing a name (say, its encapsulation) instead of a general one. Then it is expected that soundness will hold under this assumption. Fortunately, this is indeed true.

%The choice of this variant soundness is further motivated below.(opt.) .

%\sepp

%The soundness of the encoding is stated in the followig lemma.
\xxx{
We have Lemma \ref{l:soundness} stating the weak soundness of the encoding. 
We recall that $\WWCB$ is the $\WCB$ restricted to the image of the encoding as defined in Section \ref{s:preliminary}; or in other words, context bisimilarity defined over the image of the encoding. Formally we have the following definition.
}

\begin{definition}\label{def:image_contxtbisi} 
\xxxx{We define %$\enc{\pi}$ , i.e., \\
$
\enc{\pi}\DEF \{\encoding{P}{}{} \,|\, P \mbox{ is a $\pi$ process} \}. 
$ to be the image of the encoding. 
Since the encoding is compositional, this can be extended to contexts, with a hole mapped to a hole.
}

\xxxx{We denote by $\WWCB$ the context bisimilarity over $\enc{\pi}$. That is, it is defined as in Definition \ref{context-bisimulation}, except that the domain is now the image of the encoding, for both the processes and contexts.
}
\end{definition}

\xxx{It should be clear that $\WWCB$ is well defined, due to the operational correspondence and static properties of the encoding. A crucial point here, owing to the tight operational correspondence (Lemma \ref{l:opcor} and Lemma \ref{l:opcor-conv}), is that the contexts to be considered in the bisimulation are those processes in the target model that have reverse-image w.r.t. the encoding. By `tight', we mean that an encoding \HOPiDd\ process always evolves into the encoding of another \FOPi\ process, and thus rendering the restriction to the encodings sensible.
}

\xxrmcolor{
\xxxx{We note that the proof of Lemma \ref{l:soundness} uses the up-to context technique to establish the bisimulation \cite{San94, Mil89, SW01a,PS11,San11}.}
%We recall that if a relation $\mathcal{R}$ is a bisimulation up-to, say, $\equiv$, then in this relation the two simulating processes after making some action need not be directly related by $\mathcal{R}$, but (respectively) structural congruent to a pair of processes that belongs to $\mathcal{R}$. 
%Such a relation is not necessarily a context bisimulation itself, but it must be contained by the context bisimilarity.
% (for the case of up-to $\SCB$, this is achieved by showing the relation $\SCB \R \SCB$ is a context bisimulation). 
%See \cite{SW01a} for comprehensive discussion. 
}
%\xx{
%This up-to technique and similar ones (for example, up-to context used in the soundness proof of the encoding in Section \ref{s:encoding}) are standard for establishing bisimulation relations, so we omit the definition here and in what follows and advise the reader to consult \cite{San94, Mil89, SW01a} for further details.
%}
\xxxx{
The up-to technique for $\WWCB$ is defined in the same way as that in Definitions \ref{d:up1} and \ref{d:up2}, except that the communicated processes and contexts are all from images of the encoding. % (see Definition \ref{d:up3_encoding}). 
The following proposition shows the correctness of the up-to context technique, and its proof 
%together with the definition of the bisimulation up-to context 
is in Appendix \ref{appendix:up-to-context_encodings}.
This up-to context technique is sufficient for our purpose, though the general up-to context technique is also true in the broader situation (see Theorem \ref{thm:sound_up-to_context_general}). 
\begin{proposition}\label{prop:up-to-context_encodings}
If a relation $\mathcal{R}$ on the images of the encoding is a context bisimulation up-to context, then $\mathcal{R} \subseteq \WWCB$.
\end{proposition}
}
\sepp

We now prove the weak soundness of the encoding.
\begin{lemma}\label{l:soundness}
Suppose $P$ is a \FOPi\ process. Then $P\WGB Q$ implies $\enc{P} \,\WWCB\, \enc{Q}$.
\end{lemma}

\tdup{
$\myxcancel{
\fbox{
\begin{minipage}{8cm}
\rc{Use normal bisimulation for \HOPiDd\ to prove soundness, and original context bisimulation for completeness (?); Notice normal bisimulation for \HOPiDd\ inherits that for \HOPiD\ (type of input can be name-parameterization or process-parameterization.)}
\end{minipage}
}}$
}

\begin{proof}
\tdup{
\stress{ \scriptsize DONE! $\bcancel{\mbox{TODO: ,}}$
to deal with input and output, USE ``up-to context" technique (\cite{SW01a}, page 80-92; to confirm that it can be extended to higher-order paradigm (e.g., the case a context hole appears beneath an input or name-abstraction (seems ok)) !!; maybe also notice (e.g.) \cite{BPPR15}). }

\stress{ \scriptsize NOTICE that in input/output bisimulation (to prove $\enc{P}\WNB \enc{Q}$ or $\enc{P}\WCB \enc{Q}$):
\begin{itemize}
\item (by going through the FO processes $P,Q$) $\enc{P}\st{a(\triggerD)}$ must be able to be matched by $\enc{Q}\st{a(\triggerD)}$;
\item (by going through the FO processes $P,Q$)  $\enc{P}\st{\overline{a}[\lrangle{Z}(Z\lrangle{b})]}$ must be able to be matched by $\enc{Q}\st{\overline{a}[\lrangle{Z}(Z\lrangle{b})]}$.
\end{itemize}}
}

%\sep

We show that $\mathcal{R}\DEF \{(\enc{P},\enc{Q}) \,|\, P\WGB Q\} \cup \WWCB$ is a context bisimulation up-to context and $\equiv$. % $\SCB$. 
%We note that using $\equiv$ here is sufficient (one can also uses $\dot{\equiv}$, i.e., $\equiv$ restricted to the image of the encoding). 
%we refer the reader to, for example,  \cite{SW01a,BPPR15} and the references therein for the up-to proof technique for establishing bisimulations; 
%We note that using $\SCB$ here is sufficient %since it is strong enough 
%(one can also uses $\WSCB$, i.e., $\SCB$ restricted to the image of the encoding). 
%\stress{\large ***About the ``up-to context" technique (\cite{SW01a}, page 80-92; to confirm that it can be extended to higher-order paradigm (e.g., the case a context hole appears beneath an input or name-abstraction. (seems ok)) !!; maybe also notice (e.g.) \cite{BPPR15}).***}
Suppose $\enc{P}\,\mathcal{R}\, \enc{Q}$. We make a case analysis. %, where  Lemma \ref{l:opcor} and Lemma \ref{l:opcor-conv} %(as well as their weak versions)
%play an important part.
\xxxx{We note that in the input and out, we do not consider general processes, but instead those sendable by an encoding, because we confine to the image of the encoding. This is where, particularly the input, this proof cannot be extended to the general case, otherwise the counterexample above would become in vain.}
\begin{itemize}
\item $\enc{P}\wt{a(\lrangle{Z}(Z\lrangle{b}))} T$.
By Lemma \ref{l:opcor-conv}, $P \wt{a(b)} P'$ and $T\SSEQV \enc{P'}$. Because $P\WGB Q$, we know that $Q \wt{a(b)} Q'$ ~ $\WGB P'$ and thus $\enc{P'} \,\mathcal{R}\, \enc{Q'}$. Then by Lemma \ref{l:opcor}, $\enc{Q} \wt{a(\lrangle{Z}(Z\lrangle{b}))} T'$ and $T'\SSEQV \enc{Q'}$. So we have $T \SSEQV \enc{P'} \,\mathcal{R}\, \enc{Q'} \SSEQV T'$.

\tdup{
$\myxcancel{
\fbox{
\begin{minipage}{14cm}
$\enc{P}\wt{a(\triggerD)} T$. \stress{todo ~~ (\& see NOTICE just above: use Lemma \ref{l:opcor},~\ref{l:opcor-conv}(3) and up-to context}) \\
By Lemma \ref{l:opcor-conv}, $P \wt{a(b)} P'$ and $(m)(T \para !m(Y).Y\lrangle{b}) \WCB \enc{P'}$. Because $P\WGB Q$, we know that $Q \wt{a(b)} Q' \WGB P'$ and thus $\enc{P'} \,\mathcal{R}\, \enc{Q'}$. Then by Lemma \ref{l:opcor}, $\enc{Q} \wt{a(\triggerD)} T'$ and $(m)(T' \para !m(Y).Y\lrangle{b}) \WCB \enc{Q'}$. So  {\large \stress{??? (some informal discussion in ``q.txt")}} \\ %$\alpha$
\stress{Input is the crux!! Try ...
A compromise is to confine to $\enc{}(\FOPi)$
(i.e., the image of the encoding that communicate only $\lrangle{Z}(Z\lrangle{b})$)}; \\
\stress{under this constraint the argument for input would be straightforward (see the box on the right).}\\
Three possible motiv: 1) encoding used to achieve correct FO interactions in HO, so limited contexts may be ok; 2) too demanding to require all kinds of input because the target (HO) model has somewhat more powerful computation ability;  3) in practice usually not all possible objects are communicated (rather some typical ones).
\end{minipage}
}}$
}%\tdup
%\fbox{
%\begin{minipage}{7cm}
%\begin{itemize}
%\item $\enc{P}\wt{a(\lrangle{Z}(Z\lrangle{b}))} T$.
%By Lemma \ref{l:opcor-conv}, $P \wt{a(b)} P'$ and $T\SCB \enc{P'}$. Because $P\WGB Q$, we know that $Q \wt{a(b)} Q' \WGB P'$ and thus $\enc{P'} \,\mathcal{R}\, \enc{Q'}$. Then by Lemma \ref{l:opcor}, $\enc{Q} \wt{a(\lrangle{Z}(Z\lrangle{b}))} T'$ and $T'\SCB \enc{Q'}$. So we have $T \SCB \enc{P'} \,\mathcal{R}\, \enc{Q'} \SCB T'$.
%\end{itemize}
%\end{minipage}
%}
%\sep\sep

\tdup{
$\myxcancel{
\fbox{
\begin{minipage}{14cm}
\rc{Some more discussion: tackle general input directly?}\\
{First let us consider a special case, i.e., the input is $\lrangle{Z}Z\lrangle{b}$, somewhat the encoding of a transmitted name.
%(replace the current discussion with one on this special case)
By Lemma \ref{l:opcor-conv}, that $\enc{P}\wt{a(\lrangle{Z}(Z\lrangle{b}))} T_1$ implies that $P \wt{a(b)} P_1$ and $T_1\SCB \enc{P_1}$. Because $P\WGB Q$, we know that $Q \wt{a(b)} Q_1 \WGB P_1$ and thus $\enc{P_1} \,\mathcal{R}\, \enc{Q_1}$. Then by Lemma \ref{l:opcor}, $\enc{Q} \wt{a(\lrangle{Z}(Z\lrangle{b}))} T_2$ and $T_2\SCB \enc{Q_1}$. So we have $T_1 \SCB \enc{P_1} \,\mathcal{R}\, \enc{Q_1} \SCB T_2$. \\
Now consider the general input $A$, which basically should take the form $\lrangle{Z}F[Z\lrangle{b}]$ for some context $F$, so as to make the applications happen in a correct manner (otherwise the discussion would be similar, e.g., $Z$ does not appear in $F$ or is not fed with a name).
%(now do the input and use the special case and up-to context to finish the simulation)
Say $\enc{P}\wt{a(\lrangle{Z}F[Z\lrangle{b}])} T$. Then we know from the special case above that $\enc{Q}\wt{a(\lrangle{Z}F[Z\lrangle{b}])} T'$.
\bc{The problem here is how to relate $T$ with $T_1$ (and $T'$ with $T_2$)}.
Specifically, we know $T_1\equiv G[(\lrangle{x}\enc{R})\lrangle{b}] \equiv G[\enc{R}\fosub{b}{x}]$ for some context $G$ and $\lrangle{x}R$. Then $T\equiv G[F[(\lrangle{x}\enc{R})\lrangle{b}]] \equiv G[F[\enc{R}\fosub{b}{x}]]$. Similarly, we have $T_2\equiv H[(\lrangle{x}\enc{R'})\lrangle{b}] \equiv H[\enc{R'}\fosub{b}{x}]$ for some context $H$ and $\lrangle{x}R'$, and $T'\equiv H[F[(\lrangle{x}\enc{R'})\lrangle{b}]] \equiv H[F[\enc{R'}\fosub{b}{x}]]$. The situation is depicted below.
\[
\xymatrix{
  T \ar@{}[r]|-{\equiv} & G[F[\enc{R}\fosub{b}{x}]] \ar@{.}[rr]|-{?}\ar@{.}[d]|-{?}  &  & H[F[\enc{R'}\fosub{b}{x}]] \ar@{.}[d]|-{?} \ar@{}[r]|-{\equiv}  & T'  \\
  T_1 \ar@{}[r]|-{\equiv} & G[\enc{R}\fosub{b}{x}] \ar@{}[rr]|-{ \SCB\, \enc{P_1}\,\mathcal{R}\,\enc{Q_1}\,\SCB}  & &  H[\enc{R'}\fosub{b}{x}] \ar@{}[r]|-{\equiv} & T_2
}
\]
\rc{How can we proceed? Use normal bisimulation, we can set $F$ as $\overline{m}[\cdot]$. But then how? }
}
\end{minipage}
}}$
}%\tdup


%\sepp\sepp

\item $\enc{P}\wt{(b)\overline{a}[\lrangle{Z}(Z\lrangle{b})]} T$. %\stress{\scriptsize done! ~~(\& note NOTICE just above: use Lemma \ref{l:opcor},~\ref{l:opcor-conv}(5) and up-to context}) \\
By Lemma \ref{l:opcor-conv}, $P \wt{\overline{a}(b)} P'$ and $T\SSEQV \enc{P'}$. Because $P\WGB Q$, we know that $Q \wt{\overline{a}(b)} Q' \WGB P'$ and thus $\enc{P'} \,\mathcal{R}\, \enc{Q'}$. Then by Lemma \ref{l:opcor}, $\enc{Q} \st{(b)\overline{a}[\lrangle{Z}(Z\lrangle{b})]} T'$ and $T'\SSEQV \enc{Q'}$. Consider the following pair
\[
(b)(T\para E[A]) \;\quad,\quad\;  (b)(T'\para E[A])
\] in which $b\notin \fn{E[X]}$ %$E[X]\DEF !\bc{m(Z).X\lrangle{Z}}$ (\stress{\small do not set this if using context bisimulation instead of normal bisimulation})
and $A\DEF \lrangle{Z}(Z\lrangle{b})$. So
\[
(b)(T\para E[A]) \SSEQV (b)(\enc{P'}\para E[A]) \;\quad,\quad\; (b)(\enc{Q'}\para E[A]) \SSEQV (b)(T'\para E[A])
\] By setting a context $C\DEF (b)([\cdot]\para E[A])$, we have the following pair in which $\enc{P'} \,\mathcal{R}\, \enc{Q'}$.
\[
C[\enc{P'}] \;\quad,\quad\; C[\enc{Q'}]
\] This suffices to close this case in terms of the up-to context requirement.

\item $\enc{P}\wt{\overline{a}[\lrangle{Z}(Z\lrangle{b})]} T$. %\stress{\scriptsize done! ~~ (\& note NOTICE just above: use Lemma \ref{l:opcor},~\ref{l:opcor-conv}(4) and up-to context}) \\
This case is similar to the last case.

\item $\enc{P}\wt{\tau} T$. By Lemma \ref{l:opcor-conv}, $P \wt{\tau} P'$ and $T\SSEQV \enc{P'}$. From $P\WGB Q$, we know $Q \wt{} Q'\WGB P'$ and thus $\enc{P'} \,\mathcal{R}\, \enc{Q'}$. Then by Lemma \ref{l:opcor}, $\enc{Q} \wt{} T'$ and $T'\SSEQV \enc{Q'}$. So we have $T\SSEQV \enc{P'} \,\mathcal{R}\, \enc{Q'}\SSEQV T'$.
\end{itemize}
\myqed
\end{proof}
\xxxx{
\paragraph{Remark.} We caution again that Lemma \ref{l:soundness} cannot be extend to general soundness mainly because the input case would become incorrect. Notice that the current clause only allows the input of the `encapsulation of a name' from encodings. Moving to general process would fail the lemma, 
because two processes bisimilar in the image of the encodings may not be so in the general case, when we allow more contexts that can be non-encodings.
This is actually what the counterexample above suggests.
}


\subsection{Completeness}\label{s:encoding_completeness}
The completeness of the encoding is given in Lemma \ref{l:completeness}. We note that completeness is true even if we do not constrain the domain to be the image of the encoded \FOPi\ processes. 
%In terms of the notion of encoding, this is a more desired result. 
\begin{lemma}\label{l:completeness}
Suppose $P$ is a \FOPi\ process. Then $\enc{P} \WCB \enc{Q}$ implies $P\WGB Q$.
\end{lemma}
\begin{proof}
\tdup{
\stress{ \scriptsize DONE! $\bcancel{\mbox{TODO: if necessary,}}$ \\
(1) (for bound output in particular) maybe Use ``local bisimulation" on the \FOPi\ side and context bisimulation on \HOPiDd\ side, to analyze using ``context surjection" (e.g., \cite{Fu05b,Xu12}). \\
(2) maybe use certain up-to technique on the FO side.}
}

\tdup{
\stress{\scriptsize NOTICE that in input/output bisimulation (to prove $P\WGB Q$):
\begin{itemize}
%\item to prove $P$ and $Q$ bisimulate on input: (by going through the HO processes $\enc{P},\enc{Q}$) $\enc{P}\st{a(\triggerD)}$ must be able to be matched by $\enc{Q}\st{a(\triggerD)}$;
\item to prove $P$ and $Q$ bisimulate on input: (by going through the HO processes $\enc{P},\enc{Q}$) $\enc{P}\st{a(\lrangle{Z}(Z\lrangle{b}))}$ must be able to be matched by $\enc{Q}\st{a(\lrangle{Z}(Z\lrangle{b}))}$;
\item to prove $P$ and $Q$ bisimulate on free output:(by going through the HO processes $\enc{P},\enc{Q}$)  $\enc{P}\st{\overline{a}[\lrangle{Z}(Z\lrangle{b})]}$ must be able to be matched by $\enc{Q}\st{\overline{a}[\lrangle{Z}(Z\lrangle{b})]}$, because $\enc{Q}$ can only emit such form of processes, and moreover if the matching is $\enc{Q}\st{\overline{a}[\lrangle{Z}(Z\lrangle{c})]}$ then one can design a context to distinguish $\enc{P}$ and $\enc{Q}$;
\item to prove $P$ and $Q$ bisimulate on bound output:(by going through the HO processes $\enc{P},\enc{Q}$)  $\enc{P}\st{\overline{a}[(b)\lrangle{Z}(Z\lrangle{b})]}$ must be able to be matched by $\enc{Q}\st{(b)\overline{a}[\lrangle{Z}(Z\lrangle{b})]}$ (apply $\alpha$-conversion if needed), because $\enc{Q}$ can only emit such form of processes, and moreover if the matching is $\enc{Q}\st{(c)\overline{a}[\lrangle{Z}(Z\lrangle{c})]}$ (or $\enc{Q}\st{\overline{a}[\lrangle{Z}(Z\lrangle{c})]}$) then one can design a context to distinguish between $\enc{P}$ and $\enc{Q}$;
\end{itemize}}
}%\tdup
%\sep

We show that $\mathcal{R}\DEF \{(P,Q) \,|\, \enc{P} \WCB \enc{Q}\} \,\cup\, \WGB$ is a local bisimulation. %a (ground)/(local) bisimulation up-to $\SGB$ and \stress{??}. 
Suppose $P\,\mathcal{R}\, Q$. We make a case analysis below. %, where again Lemma \ref{l:opcor} and Lemma \ref{l:opcor-conv} are frequently used.
\begin{itemize}
\item $P\wt{a(b)} P'$. %\stress{see NOTICE just above \& todo: use input clause of context bisimulation, apply Lemma \ref{l:opcor},~\ref{l:opcor-conv}(3)} \\
By Lemma \ref{l:opcor}, $\enc{P} \wt{a(\lrangle{Z}(Z\lrangle{b}))} T$ and $T\SSEQV \enc{P'}$. Because $\enc{P} \WCB \enc{Q}$, we know that $\enc{Q} \wt{a(\lrangle{Z}(Z\lrangle{b}))} T'\WCB T$. By Lemma \ref{l:opcor-conv}, $Q \wt{a(b)} Q'$ and $T'\SSEQV \enc{Q'}$. Thus we have $\enc{P'} \SSEQV T \WCB T'\SSEQV \enc{Q'}$, so $P'\,\mathcal{R}\, Q'$, which fulfills this case.

\item $P\wt{\overline{a}b} P'$. %\stress{see NOTICE just above \& todo: use free output clause of context bisimulation, apply Lemma \ref{l:opcor},~\ref{l:opcor-conv}(3)} \\
By Lemma \ref{l:opcor}, $\enc{P} \wt{\overline{a}[\lrangle{Z}(Z\lrangle{b})]} T$ and $T\SSEQV \enc{P'}$. Since $\enc{P} \WCB \enc{Q}$, we know that $\enc{P}$ must be matched by $\enc{Q}\wt{\overline{a}[\lrangle{Z}(Z\lrangle{b})]} T'$, because $\enc{Q}$ can only output such shape of processes.\\
\xxxx{We claim that if the matching is, say $\enc{Q}\wt{\overline{a}[\lrangle{Z}(Z\lrangle{c})]} T''$ \\
(similar for $\enc{Q}\wt{(b)\overline{a}[\lrangle{Z}(Z\lrangle{b})]}$), then $\enc{P}$ and $\enc{Q}$ would be immediately distinguishable.   
\xxrmcolor{
In particular, a %parallel 
context can be designed to distinguish between $\enc{P}$ and $\enc{Q}$ by harnessing $c$ to exhibit different behaviour with some singular process; say, $H\DEF a(X).(X\lrangle{A})$ in which $A\DEF \lrangle{x}\overline{x}[\overline{d}]$ where $d$ is fresh. When putting $H$ in parallel and counteracting some interfering factors up-to structural equivalence (if necessary), we can see that $\enc{P}$ and $\enc{Q}$ would trigger output of fresh $d$ on different names, causing neither of them to be able to match the other.
}
%%\ntsrm{FOR EXAMPLE, $c$ can be used to send a singular process that cannot be matched anyhow
%%}.
%Similarly the matching cannot be $\enc{Q}\wt{(b)\overline{a}[\lrangle{Z}(Z\lrangle{b})]}$ either. 
}
%\xxx{\tiny -  [note, may not be for response; attached as a anntation to the proof of completeness] About In the proof of completeness of the encoding, the (bound) output must be able to be matched by a (bound) output with the same name in the outputted object (say $b$ in $\lrangle{Z}(Z\lrangle{b})$): we have given some argument concerning this, see the paper. Also notice that to force a contradiction if assumed otherwise, one may counteract / offset interfering factors from environment by consuming them using bisimulation between $\enc{P}$ and $\enc{Q}$. For example,to have a situation that one part has action $\alpha$ but the other part does not, one may eliminate the possible simulation from the remaining part by consuming all of  them before starting the argument. 
%BUT this note may not be so relevant. JUST stick to the arguments in the paper. 
%}

Therefore for every $E[X]$, we have $T \para E[\lrangle{Z}(Z\lrangle{b})] \WCB T' \para E[\lrangle{Z}(Z\lrangle{b})]$. By Lemma \ref{l:opcor-conv}, $Q \wt{\overline{a}b} Q'$ and $T'\SSEQV \enc{Q'}$. So we know
\begin{equation}\label{eq:comp4}
\enc{P'} \para E[\lrangle{Z}(Z\lrangle{b})) \WCB \enc{Q'} \para E[\lrangle{Z}(Z\lrangle{b})]
\end{equation}
We want to show
\begin{equation}\label{eq:comp5}
P' \mathcal{R} Q' \quad \mbox{ that is, }\; \enc{P'} \WCB \enc{Q'}
\end{equation}
By setting $E$ to be $0$ in (\ref{eq:comp4}), we obtain (\ref{eq:comp5}), and thus close this case.

\item $P\wt{\overline{a}(b)} P'$. %\stress{see NOTICE just above \& todo: use bound output clause of context bisimulation, and ``context surjection", apply Lemma \ref{l:opcor},~\ref{l:opcor-conv}(3)} \\
By Lemma \ref{l:opcor}, $\enc{P} \wt{(b)\overline{a}[\lrangle{Z}(Z\lrangle{b})]} T$ and $T\SSEQV \enc{P'}$. Since $\enc{P} \WCB \enc{Q}$, we know that $\enc{P}$  must be able to be matched by $\enc{Q}\wt{(b)\overline{a}[\lrangle{Z}(Z\lrangle{b})]} T'$ (apply $\alpha$-conversion if needed). This is because $\enc{Q}$ can only emit such form of processes, and moreover if the matching does not have a bound name (say $\enc{Q}\wt{\overline{a}[\lrangle{Z}(Z\lrangle{c})]} T''$) then $\enc{P}$ and $\enc{Q}$ would be immediately distinguishable \xxx{for the reason similar to that of the last case}. %one can design a context to distinguish $\enc{P}$ and $\enc{Q}$. 
So for all $E[X]$ s.t. $b\notin \mbox{fn}(E)$, we have $(b)(T \para E[\lrangle{Z}(Z\lrangle{b})]) \WCB (b)(T' \para E[\lrangle{Z}(Z\lrangle{b})])$. By Lemma \ref{l:opcor-conv}, $Q \wt{\overline{a}(b)} Q'$ and $T'\SSEQV \enc{Q'}$. So we know
\begin{equation}\label{eq:comp1}
(b)(\enc{P'} \para E[\lrangle{Z}(Z\lrangle{b})]) \WCB (b)(\enc{Q'} \para E[\lrangle{Z}(Z\lrangle{b})])
\end{equation}
In terms of local bisimulation \cite{Fu05b,Xu12}, for every \FOPi\ process $R$, we need to show
\begin{equation}\label{eq:comp2}
(b)(P'\para R) \,\mathcal{R}\, (b)(Q'\para R) \quad\mbox{ i.e., }\quad (b)(\enc{P'}\para \enc{R}) \WCB (b)(\enc{Q'} \para \enc{R})
\end{equation}
%That is,
%\begin{equation}\label{eq:comp3}
%(b)(\enc{P'}\para \enc{R}) \WCB (b)(\enc{Q'} \para \enc{R})
%\end{equation}
Comparing equations (\ref{eq:comp1}) and (\ref{eq:comp2}), one can see that the different part is $E[\lrangle{Z}(Z\lrangle{b})]$ and $\enc{R}$. Since the inverse of the encoding is a surjection, if all possible forms of $E$ is itinerated 
\xxx{(notice that $E$ may have the knowledge of $b$ from $\lrangle{Z}(Z\lrangle{b})$)}, 
$\enc{R}$ must be hit somewhere (i.e., some choice of $E$ makes $E[\lrangle{Z}(Z\lrangle{b})]$ and $\enc{R}$ equal). Therefore we infer that (\ref{eq:comp2}) %(and thus (\ref{eq:comp2})) 
is true and thus complete this case.


\item $P\wt{\tau} P'$. By Lemma \ref{l:opcor}, $\enc{P} \wt{\tau} T$ and $T\SSEQV \enc{P'}$. Because $\enc{P} \WCB \enc{Q}$, we know $\enc{Q} \wt{} T' \WCB T$. Then by Lemma \ref{l:opcor-conv}, $Q \wt{} Q'$ and $T'\SSEQV \enc{Q'}$. % (notice that the case $\enc{Q}$ is exactly $T'$ is trivial).
So we have $P'\,\mathcal{R}\, Q'$ because $\enc{P'}\SSEQV T \WCB T' \SSEQV \enc{Q'}$.
\end{itemize}
\myqed
\end{proof}






%---------------------------
% Local Variables:
% mode: LaTeX
% TeX-master: "main.tex"
% End:

\subsection{Calculus $\pi$}
The first-order pi-calculus (notation $\pi$) of Milner, Parrow and Walker \cite{MPW92} is one of the most developed process calculi in the literature. Here we use a lightweight variant \cite{EN86a}\cite{EN00}, in which names (ranged over by $m,n,u,v,w$) are classified into two categories: name constants (ranged over by $a,b,c,d,e$) and name variables (ranged over by $x,y,z$). Below is the grammar whose constructs have their standard meaning. We use guarded replication instead of general replication without loss of expressiveness \cite{San98}\cite{FL09a}.
\[P,Q := 0 \,\Big{|}\, m(x).P \,\Big{|}\, \overline{m}n.P \,\Big{|}\,  (c)P \,\Big{|}\, P\para Q \,\Big{|}\, !m(x).P \,\Big{|}\, !\overline{m}n.P
\]
A name constant $a$ is bound (local or restricted) in $(a)P$, otherwise it is free (global). A name variable $x$ is bound in $a(x).P$ and free otherwise.
We use respectively $fn(\cdot), bn(\cdot), n(\cdot), fv(\cdot), bv(\cdot), v(\cdot)$ to denote free constants, bound constants, constants, free variables, bound variables, and variables in a set of processes. A name is fresh if it does not appear in any of the processes under consideration. A process $P$ is closed if $fv(P){=}\emptyset$. We consider closed processes by default. We have a few derived operators: $\overline{a}(d).P \DEF (d)\overline{a}d.P$, $a.P \DEF  a(x).P \;(x\notin fv(P))$, $\overline{a}.P \DEF \overline{a}(d).P\;(d\notin fn(P))$. A trailing $0$ process is usually trimmed. We use a tilde to stand for tuples. For $\ve{n}$, $\size{\ve{n}}$ stands for its length; $m\in \ve{n}$ means $m$ is its element; $m\ve{n}$ abbreviates adding $m$ to the tuple; $(c_1)(c_2)\cdots (c_k)E$ is written as $(c_1c_2\cdots c_k)E$ or simply $(\ve{c})E$. Substitution $\fosub{n}{m}$, used in postfix form $P\fosub{n}{m}$, means a mapping that replaces $m$ with $n$ (in $P$) while keeping the others intact. %are ranged over by $\sigma$. 
Sometimes substituting a constant for another is called renaming, and assignment if for a variable. A context $C$ is a process with some subprocess replaced by the hole $[\cdot]$; then $C[A]$ is the process resulting from filling in the hole by process $A$.
%The LTS is recalled in appendix \ref{appendix:semantics}.
The LTS (Labelled Transition System) comprises the rules as below (symmetric rules are skipped).
\[
\begin{array}{lll}
\infer{a(x).P\st{a(b)} P\fosub{b}{x}}{} \quad &
\infer{\overline{a}b.P\st{\overline{a}b} P}{} \quad &
\infer[{\scriptstyle bn(\lambda)\cap fn(Q)=\emptyset}]{P\para Q\st{\lambda} P'\para Q}{P\st{\lambda} P'}
\end{array}
\]
\[
\begin{array}{lll}
\infer{P\para Q\st{\tau}P'\para Q'}{P\st{a(b)}P', Q\st{\overline{a}b} Q'}\quad &
\infer{P\para Q\st{\tau}(b)(P'\para Q')}{P\st{a(b)} P', Q\st{\overline{a}(b)} Q'} \quad&
\infer[{\scriptstyle c\not\in n(\lambda)}]{(c)P\st{\lambda} P'}{P\st{\lambda} P'}
\end{array}
\]
\[
\begin{array}{lll}
\infer[{\scriptstyle c\neq a}]{(c)P\st{\overline{a}(c)} P'}{P\st{\overline{a}c} P'} \quad &
\infer{!a(x).P \st{a(b)} P\fosub{b}{x}\para !a(x).P}{}\quad  &
\infer{!\overline{a}b.P \st{\overline{a}b} P\para !\overline{a}b.P}{}
\end{array}
\]
%\[Structural:
%\begin{array}{c}
%\frac{\displaystyle Q\equiv P,\; P\st{\lambda} P', \; P'\equiv Q'}{\displaystyle Q\st{\lambda} Q'}
%\end{array}
%\]
Actions (ranged over by $\lambda$) include $\tau$, and visible ones: input ($a(b)$), output ($\overline{a}b$) and bound output ($\overline{a}(c)$) which occur on constant names, and the outputted name is a constant (i.e. no variable can be transmitted).
%Define $\overline{\lambda}$ as $\overline{a}b$ if $\lambda$ is $a(b)$, $a(b)$ if $\lambda$ is $\overline{a}b \mbox{ or } \overline{a}(b)$, and $\tau$ if $\lambda$ is $\tau$.
As usual, $\equiv$ denotes the standard structural congruence \cite{MPW92}\cite{SW01a}, which is the smallest relation satisfying the monoid laws for parallel composition, commutative laws for both composition and restriction, and a distributive law $(c)(P\para Q) \SE (c)P\para Q$ (if $c\notin fn(Q)$).
We write $\wt{}$ for the reflexive transitive closure of $\st{\tau}$, and $\wt{\lambda}$ for $\wt{}\st{\lambda}\wt{}$; then $\wt{\widehat{\lambda}}$ is $\wt{\lambda}$ if $\lambda$ is not $\tau$, and $\wt{}$ otherwise. A process $P$ is divergent, denoted $\diverge{P}$, if it has an infinite computation (i.e. $\tau$ sequence). \strs{A relation $\R$ is divergence-sensitive if whenever $P\,\R\, Q$ then $\diverge{Q}$ implies $\diverge{P}$}.
%Let $\equiv$ denote the standard structural congruence, as defined below.
%\[
%\begin{array}{l}
%P\equiv Q, \mbox{ if they are $\alpha$-convertible to each other} \\
%P\para 0 \equiv P, \, P\para (Q\para R) \equiv (P\para Q)\para R, \, P \para Q  \equiv Q\para P \\
%(c)(d)P \equiv (d)(c)P, \,(c)0\equiv 0 \\
% (c)(P\para Q) \equiv (c)P\para Q \mbox{ whenever } c\notin fn(Q) %\\
%\end{array}
%\]
In the standard way, the bisimulation on $\pi$ is defined as below and is a congruence relation \cite{MPW92}\cite{EN00}. %\cite{FZ09}.
\begin{definition}[Bisimulation]\label{ex-w-bisi}
Weak bisimilarity $\AEFOPi$ is the largest symmetric bisimulation relation $\mathcal{R}$ on $\pi$ processes such that whenever $P \,\mathcal{R}\, Q$ and $P\st{\lambda} P'$ then $Q\wt{\widehat{\lambda}} Q'$ and $P' \,\mathcal{R}\, Q'$.
\end{definition}
We use $\SAEFOPi$ to denote the strong version of the bisimulation (i.e., the one obtained by replacing $\wt{\widehat{\lambda}}$ with  $\st{\lambda}$ in the definition). Notice that an alternative way to define weak bisimulation %(similar for other bisimulation relations) 
is to use $\wt{\lambda}$ instead of $\st{\lambda}$ in Definition \ref{ex-w-bisi} (see \cite{Mil89, SW01a}).

The local bisimilarity ($\EWLB$) characterizes weak bisimilarity, i.e. $\EWLB$ coincides with $\AEFOPi$ (see \cite{Fu05b} for a proof), and will be used when discussing the encodings. 
\begin{definition}\label{ext-local-bisi}
Local bisimilarity $\EWLB$ is the largest symmetric local bisimulation relation $\mathcal{R}$ on $\pi$ processes such that:
\begin{itemize}
\item if $P\st{\lambda} P'$, $\lambda$ is not bound output, then $Q\wt{\widehat{\lambda}} Q'$ and $P'\,\mathcal{R}\, Q'$;
\item if $P\st{\overline{a}(b)} P'$, then $Q\wt{\overline{a}(b)} Q'$, and for every $R$, $(b)(P'\para R)\,\mathcal{R}\, (b)(Q'\para R)$.
\end{itemize}
\end{definition}
% $\EWLB$ indeed characterizes $\AEFOPi$ (see \cite{Fu05b} for a proof).
% \begin{lemma}\label{ext-local-bisi-prop}
% $\EWLB$ coincides with $\AEFOPi$.
% \end{lemma}


%---------------------------------------------------------------------------------------------------------------------------------------------------------------

\subsection{Calculus $\Pi$}}
 $\Pi$ processes, denoted by uppercase letters ($T,P,Q,R,A,B,E,F...$), are defined by the following grammar. Lowercase letters stand for channel names. Letters $X,Y,Z$ represent process variables.
We use $\Pi_{seg}$ for convenience.
\[
\begin{array}{l}
\Pi_{seg} ::= 0 \;|\; X \;|\;  u(X).T \;|\; \overline{u}T'.T \;|\; T\para T' \;|\; (c)T  
\end{array}
\]
 The operators are: prefix:$u(X).T, \overline{u}T'.T$; composition: $T\para T'$; restriction: $(c)T$ in which $c$ is bound. 
  Parallel composition has the least precedence.
 As usual some notations are: $a$ for $a(X).0$; $\overline{a}$ for $\overline{a}0.0$; $\overline{m}A$ for $\overline{m}A.0$; $\tau.P$ for $(a)(a.P\para \overline{a})$; sometimes $\overline{a}[A].T$ for $\overline{a}A.T$; $\ve{\cdot}$ for a finite sequence of something.
 By standard definition, $fn(\ve{T})$, $bn(\ve{T})$, $n(\ve{T})$; $fv(\ve{T})$, $bv(\ve{T})$,
 $v(\ve{T})$ respectively denote free names, bound names, all names; free variables, bound variables and all variables in $\ve{T}$.
  Closed processes contain no free variables.
  A fresh name or variable is one that does not occur in the processes under consideration.
 Name substitution $T\fosub{\ve{n}}{\ve{m}}$ and higher-order substitution $T\hosub{\ve{A}}{\ve{X}}$ are defined structurally in a standard way.
$E[\ve{X}]$ denotes $E$ with (at most) variables $\ve{X}$, and $E[\ve{A}]$ stands for $E\hosub{\ve{A}}{\ve{X}}$.
We work up-to $\alpha$-conversion and always assume no capture.
We use input-guarded replication as a derived operator \cite{Tho93,LPSS08}:
$ !\phi.P \DEF (c)(Q_{\scriptscriptstyle c,\phi,P} \para \overline{c}Q_{\scriptscriptstyle c,\phi,P})$,
  $Q_{\scriptscriptstyle c,\phi,P} \DEF c(X).(\phi.(X\para P) \para \overline{c}X)$, where $\phi$ is a prefix.

The operational semantics is given in Figure \ref{lts-Pi}. Symmetric rules are omitted.
The rules are mostly self-explanatory. For instance,
in higher-order input $a(A)$, the received process $A$ becomes part of the receiving environment through a substitution.
 $\alpha,\beta,\lambda,...$ denote action, whose subject indicates the channel name on which it happens.
Operations $fn(), bn(), n()$ can be similarly defined on actions.
$\wt{}$ is the reflexive transitive closure of internal actions $\tau$, and $\wt{\lambda}$ is $\wt{}\xrightarrow{\lambda}\wt{}$.  $\wt{\hat{\lambda}}$ is $\wt{}$ when $\lambda$ is $\tau$ and $\wt{\lambda}$ otherwise. $\st{\tau}_k$ means $k$ consecutive $\tau$'s.
``$P\wt{\hat{\lambda}} P'$" abbreviates ``there is a process $P'$ such that $P\wt{\hat{\lambda}} P'$".
 $P\wt{}\cdot \mathcal{R}\, Q$ means $P\wt{} Q'$ and $Q' \,\mathcal{R}\, Q$ (i.e. $(Q',Q)\in \mathcal{R}$), where $\mathcal{R}$ is a binary relation.
We say relation $\mathcal{R}$ is closed under (variable) substitution if $(E\hosub{A}{X},F\hosub{A}{X})\in \mathcal{R}$ for any $A$ whenever $(E,F)\in \mathcal{R}$. 
A process diverges if it can perform an infinite $\tau$ sequence.
%*************************
%\vspace*{-.9cm}
%*************************
\begin{figure}[htbp]
\[\begin{array}{llll}
\frac{\displaystyle }{\displaystyle a(X).T\st{a(A)} T\hosub{A}{X}}  &    
 \frac{}{\displaystyle \overline{a}A.T\st{\overline{a}A} T}  &      
\frac{\displaystyle T\st{\lambda} T'}{\displaystyle (c)T\st{\lambda} (c)T'}{\scriptstyle c\not\in n(\lambda)} &
\frac{\displaystyle T\st{\lambda} T'}{\displaystyle T\para T_1\st{\lambda} T'\para T_1} {\scriptstyle bn(\lambda)\cap fn(T_1)=\emptyset}
\end{array}
\]
\[\begin{array}{ll}
\frac{\displaystyle T\st{(\ve{c})\overline{a}[A]} T'}{\displaystyle (d)T\st{(d)(\ve{c})\overline{a}[A]} T'} {\scriptstyle d \in fn(A){-}\{\ve{c},a\}} \quad &
 \frac{\displaystyle T_1\st{a(A)} T_1', T_2\st{(\ve{c})\overline{a}[A]} T_2'}{\displaystyle T_1\para T_2 \st{\tau}(\ve{c})(T_1'\,|\,T_2')}~ {\scriptstyle \ve{c}\cap fn(T_1) = \emptyset}
\end{array}
\]
%*************************
%\vspace*{-.5cm}
%*************************
\caption{Semantics of $\Pi$}\label{lts-Pi}
\end{figure}



%---------------------------------------------------------------------------------------------------------------------------------------------------------------
\subsection{Calculi $\Pi^D_n$ and $\Pi^d_n$}}

Parameterization extends $\Pi$ with the syntax and semantics in Figure \ref{PiDn}.
$\lrangle{\vect{U}} T$ is n-ary parameterization where $\vect{U}$ are the formal parameters to be instantiated by the n-ary  application $T\lrangle{\vect{K}}$ where the parameters are instantiated by concrete objects $\vect{K}$.
In the rule, $\size{\ve{U}}{=}\size{\ve{K}}{=}n$ requires the parameters and the instantiating objects should be equally sized (equal to $n$). It also expresses that the parameterized process can do an action only after the application happens.
Calculus $\Pi^D_n$, which has process parameterization (or higher-order abstraction),
  is defined by taking $\ve{U},\ve{K}$ as $\ve{X},\ve{T'}$ respectively.
Calculus $\Pi^d_n$, which has name parameterization (or first-order abstraction),  is defined by taking $\ve{U},\ve{K}$ as $\ve{x},\ve{u}$ respectively.
For convenience, names (ranged over by $u,v,w$) are handled dichotomically: name constants (ranged over by $a,b,c,...,m,n$); name variables (ranged over by $x,y,z$).
%*************************
%\vspace*{-.4cm}
%*************************
\begin{figure}[htbp]
\[\begin{array}{l}
T::= \Pi_{seg}  \;|\; \lrangle{\vect{U}} T \;|\;  T\lrangle{\vect{K}}  \\ 
\frac{\displaystyle T\hosub{\ve{K}}{\ve{U}} \stackrel{\lambda}{\longrightarrow } T'}{\displaystyle F\lrangle{\ve{K}}\stackrel{\lambda}{\longrightarrow } T'}\;\;\mbox{ if } F\DEF \lrangle{\ve{U}}T \;\;(\size{\ve{U}}=\size{\ve{K}}=n)
\end{array}
\]
%*************************
%\vspace*{-.5cm}
%*************************
\caption{$\Pi$ with parameterization}\label{PiDn}
\end{figure}
%*************************
%\vspace*{-.3cm}
%*************************




%remark  about the calculi

Process expressions (or terms) of the form $\lrangle{\ve{X}} P$ or $\lrangle{\ve{x}} P$, in which $\ve{X}$ or $\ve{x}$ is not empty, are \emph{parameterized processes}. 
Terms without outmost parameterization are \emph{non-parameterized processes}, or simply processes. 
We mainly focus on processes in the encodings to be presented. Sometimes related notations are slightly abused if no confusion is caused.
Only free  variables can be effectively parameterized; they become \emph{bound} after parameterization. In the syntax, null-parameterizations, for example $\lrangle{X_1,X_2}P$ in which $X_2\notin fv(P)$, are allowed.
A $\Pi^D_n$($n\geq 1$) process is definable in $\Pi^D_{n+1}$ (similar for $\Pi^d_n$) by making use of fresh dummy variable. 
The semantics of parameterization renders it somewhat natural to deem application as extra rule of structural congruence,
denoted by $\equiv$, which is defined in the standard way \cite{Mil92,San94}. In addition to the standard rules (e.g. monoid rules for composition), it includes the application rules:
$(\lrangle{\ve{X}}P)\lrangle{\ve{A}}\equiv P\hosub{\ve{A}}{\ve{X}}$ and $(\lrangle{\ve{x}}P)\lrangle{\ve{u}}\equiv P\fosub{\ve{u}}{\ve{x}}$.

Type systems for $\Pi^D_n$ and $\Pi^d_n$ (also $\Pi^{D,d}_n$) can be routinely defined in a similar way to that in \cite{San92}, where the concept of \emph{order} is also formalized.
As mentioned, here we require that only second-order processes be admitted. 
So parameterizing does not occur on parameterized processes, and application does not use parameterized processes as instance.
That is, we stipulate that in $\lrangle{\ve{U}}T'$ and $T\lrangle{\ve{T'}}$, $T'$ are not parameterized processes.
This prohibits processes such as $\lrangle{X}(\lrangle{Y}T)$ and $(\lrangle{X}(X\lrangle{A}))\lrangle{B}$ in which $B\equiv \lrangle{Y}Y$; instead, in the former one could use $\lrangle{X,Y}T$, and in the latter the instance $A$ could be sent to a service containing $B$ before retrieving the result.
The main reason we restrict to second-order processes is that it serves well in related study (e.g. the encodings in this paper and work from \cite{San92}) and can also simplify type consistency in programming. Technically this rules out a trivial encoding in section \ref{hierarchy} (some more remark is given in its end), simplifies the encoding in section \ref{relationship-ho-fo-abstraction} because it is currently not obvious how to extend to the general case.





